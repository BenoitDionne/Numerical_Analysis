\nonumsection{Chapter~\ref{chaptSeqA} : Iterative Methods for Nonlinear Equations of One Variable}

\solution{\SOL}{\ref{solvAQ1}}{
We will roughly sketch the graph of $f(x) = 4x^2 - e^x$.  This is not
easy because even the critical points are hard to find.  However, it
is easier to determine the intervals of concavity of the function.
Since $f''(x) = 8 -e^x$, a potential point of inflection is
$x=\ln(8)$.  Since $f''(x) > 0$ for $x < \ln(8)$ and $f''(x)<0$ for
$x>\ln(8)$, we have that $\ln(8)$ is a point of inflection, $f$ is
concave up for $x<\ln(8)$ and concave down for $x>\ln(8)$.  To sketch
the graph of $f$ below, we have computed the sign of $f$ at $-1$,
$0$, $1$, $4$ and $5$.

\pdfbox{solve_equ_A/root_quest1}

There is a unique root in each of the intervals $[-1,0]$, $[0,1]$
and $[4,5]$.  There is no other root. 
}

\solution{\SOL}{\ref{solvAQ2}}{
$\sqrt[3]{25}$ is the unique root of $f(x) = x^3 -25$.

Since $f$ is a continuous function and
$f(2) = -17 < 0 < 2 = f(3)$, it follows from the Intermediate Value
Theorem that $\sqrt[3]{25}$, the only root of $f$, is between $2$ and
$3$.

According to Corollary~\ref{bisectCor}, to get an approximation of
$\sqrt[3]{25}$ with an accuracy of $10^{-4}$, we need to select the
number of iterations $n$ such that
\[
\frac{3-2}{2^n} < 10^{-4}
\Rightarrow 10^4 < 2^n
\Rightarrow 4\ln(10) < n \ln(2)
\Rightarrow \frac{4\ln(10)}{\ln(2)} = 13.28771237\ldots < n \ .
\]
Since $n$ is an integer, we may take $n=14$.

We get the following results from the Bisection Method
{\small
\[
\begin{array}{c|rrrcc}
n & x_n & a_n & b_n & \text{sign of } f(x_n) & \text{sign of } f(a_{n-1}) \\
\hline
0 & & 2 & 3 & & \\
1 & 2.5 & 2.5 & 3 & - & - \\
2 & 2.75 & 2.75 & 3 & - & - \\
3 & 2.875 & 2.875 & 3 & - & - \\
4 & 2.9375 & 2.875 & 2.9375 &  + & - \\
5 & 2.90625 & 2.90625 & 2.9375 & - & - \\
6 & 2.921875 & 2.921875 & 2.9375 & - & - \\
7 & 2.9296875 & 2.921875 & 2.9296875 &  + & - \\
8 & 2.9257812 & 2.921875 & 2.9257812 &  + & - \\
9 & 2.9238281 & 2.9238281 & 2.9257812 & - & - \\
10 & 2.9248047 & 2.9238281 & 2.9248047 &  + & - \\
11 & 2.9243164 & 2.9238281 & 2.9243164 &  + & - \\
12 & 2.9240723 & 2.9238281 & 2.9240723 &  + & - \\
13 & 2.9239502 & 2.9239502 & 2.9240723 & - & - \\
14 & 2.9240112 & 2.9240112 & 2.9240723 & - & -
\end{array}
\]
}
After 14 iterations, we find $\sqrt[3]{25} \approx 2.9240112$\ .
}

\solution{\SOL}{\ref{solvAQ3}}{
We have seen in Corollary~\ref{bisectCor} that
\[
  |x_n - r| < |b_n - a_n| = \frac{b_0-a_0}{2^n} \ .
\]
Hence, the relative error satisfies
\begin{equation}\label{second}
\frac{|x_n-r|}{|r|} < \frac{(b_0-a_0)}{2^n\,|r|}
\end{equation}
If $n$ satisfies (\ref{first}), we have that
\[
\ln(2^n) = n \ln(2) \geq \ln(b_0-a_0)-\ln(\epsilon)-\ln(a_0)
= \ln\left(\frac{b_0-a_0}{\epsilon \, a_0}\right)
\Rightarrow 
2^n \geq \frac{b_0-a_0}{\epsilon \, a_0}
\Rightarrow
\epsilon\, a_0 \geq \frac{b_0-a_0}{ 2^n} \ .
\]
Hence, from (\ref{second}), we get
\[
\frac{|x_n-r|}{|r|} < \frac{\epsilon\,a_0}{|r|} \leq \epsilon
\]
for $n$ satisfying (\ref{first}) because $r>a_0>0$ implies that
$a_0/r < 1$.
}

\solution{\SOL}{\ref{solvAQ4}}{
Since $x_{n+1}$ is the middle point of the interval
$[a_n,b_n]$, where one of the endpoints is $x_n$, we get
\[
|x_{n+1}-x_n| = \frac{1}{2}\, (b_n-a_n)
= \frac{1}{2} \left(\frac{b_0-a_0}{2^n}\right) = \frac{b_0-a_0}{2^{n+1}} \ ,
\]
where we have used the property that the length of the interval
$[a_n,b_n]$ is $(b_0-a_0)/2^n$.
}

\solution{\SOL}{\ref{solvAQ5}}{
As it is given, the bisection algorithm can never have a sequence
$a_0 <a_1 < a_2 < \ldots$.

Let $r$ be the root in $[a_0,b_0]$ approximated by the bisection
algorithm.  We suppose that $a_0 <a_1 < a_2 < \ldots$ and
prove by contradiction that this is not possible.

Since we have an infinite sequence of distinct values
$\displaystyle \{a_n\}_{n=0}^\infty$, none of $a_n$ or $b_n$ for
$n \geq 0$ can be a root of $f$ because of (\ref{BisectAlgo2}) and
(\ref{BisectAlgo4}) in Algorithm~\ref{BisectAlgo}.
Moreover, since only $a_n$ varies, we must have $b_n = b_0$ and
$a_n < r$ for all $n$.  But then $b_n - a_n \geq b_0-r$ for all $n$.  So
$\displaystyle \{b_n - a_n \|_{n=0}^\infty$ cannot converge to $0$.
This is our contradiction.
}

\solution{\SOL}{\ref{solvAQ8}}{
We first find all points $r>0$ such that
\[
-r = r - \frac{f(r)}{f'(r)} = r - (1+r^2)\arctan(r) \ ;
\]
namely, such that
\[
\arctan(r) = \frac{2r}{1+r^2} \ .
\]
In the figure below, we draw the curves $y=\arctan(x)$ (in black) and
$y=2x/(1+x^2)$ (in blue) on the same coordinate system.  Note that the
two functions are odd.  There is only one intersection for $x>0$
between these two curves.  The $x$-coordinate of this point is the
value of interest.
\pdfbox{solve_equ_A/arctanNewton}
For $0<x_0<r$, we have $-r < x_0-f(x_0)/f'(x_0) <0$ and the
Newton's method will converge to the root $0$ of $\arctan(x)$.

For $x_0>r$, we have $x_0-f(x_0)/f'(x_0) < -r$ and the Newton's method
will not converge to the root $0$ of $\arctan(x)$.

For $x_0=r$, we have $x_1 = x_0-f(x_0)/f'(x_0) = -r$,
$x_2 = x_1-f(x_1)/f'(x_1) = r = x_0$, and so on.

We may use Newton's method to approximate $r$.  Let
\[
g(x) = \arctan(x) - \frac{2x}{1+x^2} \ .
\]
Then
\[
g'(x) = \frac{1}{1+x^2} - \frac{2}{1+x^2} + \frac{4x^2}{(1+x^2)^2}
= \frac{-1+3x^2}{(1+x^2)^2} \ .
\]
With $x_0=1$, the formula
\[
x_{n+1} = x_n - \frac{g(x_n)}{g'(x_n)}
\]
yields $r \approx 1.3917452002707$ after $5$ iterations with an
accuracy of at least $10^{-12}$.
}

\solution{\SOL}{\ref{solvAQ9}}{
We seek a function $f$ such that
\[
2x_n - C x_n^2 = x_n - \frac{f(x_n)}{f'(x_n)} \ .
\]
Thus,
\[
\frac{f(x)}{f'(x)} = -x + C x^2 \ .
\]
This is the separable differential equation
\[
\frac{f'(x)}{f(x)} = \frac{1}{-x+ C x^2 } \ .
\]
If we integrate both sides with respect to $x$, we get
\begin{align*}
\ln|f(x)| & = \int \frac{f'(x)}{f(x)} \dx{x}
= \int \frac{1}{-x+ C x^2} \dx{x} = \int \frac{1}{x(C x - 1)} \dx{x} \\
&= \int \left( -\frac{1}{x} + \frac{C}{C x - 1}\right) \dx{x}
= - \ln|x| + \ln|Cx-1| + D = \ln\left(\frac{|Cx-1|}{|x|}\right) + D \ .
\end{align*}
Taking the exponential on both sides yields
\[
|f(x)| =  e^D \, \frac{|Cx-1|}{|x|}
\Rightarrow f(x) = E\, \frac{Cx-1}{x}
\]
for $E \neq 0$ in $\RR$.
}

\solution{\SOL}{\ref{solvAQ10}}{
This is the formula for the Newton's method applied to
$f(x) = \tan(x) -1$.  Since $x_0=0$, the limit of the sequence above is
a root of $f$ between $-\pi/2$ and $\pi/2$.  The easiest way to
convince us that this is true is to use the graphical interpretation
of the Newton's method as the intersection of the tangent line to the
graph of $f$ at $(x_n,f(x_n))$ with the $x$-axis to get $x_{n+1}$.

All tangents to the graph of $f$ at $(x,f(x))$ for
$-\pi/2 < x < \pi/2$ intersect the $x$-axis between $-\pi/2$ and $\pi/2$
because:
\begin{enumerate}
\item $f'(x) = \sec^2(x) > 1$ for $-\pi/2 < x < \pi/2$ with $x\neq 0$,
and $f'(0) = 1$:
\item $f''(x) = 2\sec^2(x)\tan(x) \geq 0$ for $0 \leq x < \pi/2$,
  and $f''(x) \leq 0$ for $-\pi/2<x\leq 0$;
\item $\displaystyle \lim_{x\to -pi/2^-} f(x) = -\infty$
and $\displaystyle \lim_{x\to pi/2^+} f(x) = \infty$.
\end{enumerate}
Using curve sketching as seen in calculus, we get the following graph.
\pdfbox{solve_equ_A/tanmone}
The solution of $f(x)=0$ between $-\pi/2$ and $\pi/2$ is the value $x$
between $-\pi/2$ and $\pi/2$ such that $\tan(x) =1$; namely,
$x = \pi/4$.  The limit of the sequence above is therefore $\pi/4$.
}

\solution{\SOL}{\ref{solvAQ11}}{
The formula for the Newton's method is
\[
x_{n+1} = x_n - \frac{\tan(x_n)}{\sec^2(x_n)}
= x_n - \sin(x_n)\cos(x_n)
\]
for $n=0$, $1$, $2$, \ldots\ We get
{
\small
\[
\begin{array}{c|c|c|c}
\hline
n & x_{n-1} & x_n & |x_n - x_{n-1}| \\
\hline
1 & 5.000000000000 & 5.272010555445 & 0.272010555445 \not< 10^{-8} \\
2 & 5.272010555445 & 5.721895774546 & 0.449885219101 \not< 10^{-8} \\
3 & 5.721895774546 & 6.172506324972 & 0.450610550427 \not< 10^{-8} \\
4 & 6.172506324972 & 6.282283652706 & 0.109777327733 \not< 10^{-8} \\
5 & 6.282283652706 & 6.283185306691 & 0.000901653985 \not< 10^{-8} \\
6 & 6.283185306691 & 6.283185307180 & 0.000000000489 < 10^{-8} \\
\hline
\end{array}
\]
}
Starting with $x_0 = 5$, it took $6$ iterations to get the
approximation $x_6 = 6.283185307180$ of the root of $f$ with the
requested accuracy.
}

\solution{\SOL}{\ref{solvAQ12}}{
The formula for the secant method is
\[
x_{n+1} = x_n - \frac{e^{x_n}-\tan(x_n)}
{\big((e^{x_n}-\tan(x_n))-(e^{x_{n-1}}-\tan(x_{n-1}))\big)/(x_n-x_{n-1})}
\]
for $n=1$, $2$, $3$, \ldots\ We have
{
\small
\[
\begin{array}{c|c|c|c}
\hline
n & x_n & x_{n+1} & |x_{n+1} - x_n| \\
\hline
0 & 1.300000000000 & 1.350000000000 & 0.050000000000 \not< 10^{-8} \\
1 & 1.350000000000 & 1.305052269533 & 0.044947730467 \not< 10^{-8} \\
2 & 1.305052269533 & 1.306071050733 & 0.001018781201 \not< 10^{-8} \\
3 & 1.306071050733 & 1.306328498317 & 0.000257447584 \not< 10^{-8} \\
4 & 1.306328498317 & 1.306326938521 & 0.000001559796 \not< 10^{-8} \\
5 & 1.306326938521 & 1.306326940423 & 0.000000001902 < 10^{-8} \\
\hline
\end{array}
\]
}
The secant algorithm must be started with $x_0$ and $x_1$ really close
to the first positive root of $f$, the root that we want to approximate,
otherwise the secant algorithm will likely converge to a totally
different root of $f$.   The first positive root of $f$ is the first
intersection of $\tan(x)$ (in black) and $e^x$ (in blue) in the
following figure.
\figbox{solve_equ_A/exptanSecant}{8cm}
Starting with $x_0 = 1.3$ and $x_1=1.35$, it took $5$ iterations to
get the approximation $x_6 = 1.306326940423$ of the root of $f$ with the
requested accuracy.

To illustrate how unpredictable Newton's method can be, if we start
with $x_0 = 4$ and $x_1=5$, it takes $13$ iterations
to get the approximation $x_{14} = -3.096412304914$ of the first
negative root of $f$ instead of the first positive root.
}

\solution{\SOL}{\ref{solvAQ13}}{
\subQ{a} The Taylor polynomial of $f$ of degree one about $x_n$ is
$p(x) = f(x_n) + f'(x_n)(x-x_n)$.  We have that
$\displaystyle f(x) = p(x) + \frac{1}{2}\,f''(\xi_n)(x-x_n)^2$
for some $\xi_n$ between $x_n$ and $x$.  If $x=r$, the root of $f$ in
the interval $[0,1]$, we get
\begin{align*}
&0 = f(x_n) + f'(x_n)(r-x_n) + \frac{1}{2}\, f''(\xi_n)(r-x_n)^2 \\
&\Rightarrow -f(x_n) = f'(x_n)(r-x_n) + \frac{1}{2}\, f''(\xi_n)(r-x_n)^2
\Rightarrow -\frac{f(x_n)}{f'(x_n)} = r-x_n +
\frac{f''(\xi_n)}{2f'(x_n)}(r-x_n)^2 \\
&\Rightarrow \underbrace{x_n -\frac{f(x_n)}{f'(x_n)}}_{=x_{n+1}}
= r + \frac{f''(\xi_n)}{2f'(x_n)}(r-x_n)^2
\Rightarrow \underbrace{x_{n+1} - r}_{=e_{n+1}} =
\frac{f''(\xi_n)}{2f'(x_n)}\underbrace{(r-x_n)^2}_{=e_n^2} \ .
\end{align*}
Thus,
\begin{equation}\label{NMerrorQuest}
e_{n+1} = \frac{f''(\xi_n)}{2f'(x_n)} \, e^2_n
\end{equation}
for some $\xi_n$ between $x_n$ and $x$.

\subQ{b} You may assume that $x_n\geq 0$ for all $n$ if $x_0\geq 0$,
because
\[
x_{n+1} = x_n - \frac{f(x_n)}{f'(x_n)}
= x_n - \frac{x_n-e^{-x_n}}{1+e^{x_n}}
= \frac{(x_n+1)e^{x_n}}{1+e^{x_n}} > 0
\]
for $n=1$, $2$, $3$, \ldots\ by induction.  Moreover, we have
$f'(x) = 1 + e^{-x}$ and $f''(x) = -e^{-x}$.  Hence,
\[
\left| \frac{f''(\xi_n)}{f'(x_n)} \right|
 = \frac{e^{-\xi_n}}{1+e^{-x_n}} \leq 1
\]
for all non-negative numbers $\xi_n$ and $x_n$.

We use induction to prove (\ref{NMconvQuest}).
From (\ref{NMerrorQuest}) with $n=0$, we get
\[
|e_1| = \left|\frac{f''(\xi_0)}{2f'(x_0)}\right|
e_0^2 \leq \frac{e_0^2}{2} \ .
\]
This is (\ref{NMconvQuest}) for $n=1$.  Suppose that
(\ref{NMconvQuest}) is true for $n=k$.  Then,
\[
|e_{k+1}| = \left|\frac{f''(\xi_k)}{2f'(k_n)}\right|\, e_k^2
\leq \frac{e_k^2}{2} \leq \frac{1}{2} \left(2
\left(\frac{x_0}{2}\right)^{2^k} \right)^2
= 2 \left(\frac{x_0}{2}\right)^{2^{k+1}} \ .
\]
The first equality comes from (\ref{NMerrorQuest}) with $n=k$ and the
second inequality comes from the hypothesis of induction.  We get
that (\ref{NMconvQuest}) is true for $n=k+1$.  This complete the proof
by induction.

\subQ{c} According to (b), we need to find $n$ such that
\[
|e_n| \leq 2\left(\frac{1-r}{2}\right)^{2^n} < 10^{-5} \ .
\]
We do not know $r$ but we know that $r$ is between $0$ and $1$, so
$|1-r|<1$.  It is therefore enough to find $n$ such that
$\displaystyle 2\left(1/2\right)^{2^n} < 10^{-5}$.
Thus, $n$ satisfies
\[
2^{2^n-1} > 10^5
\Rightarrow (2^n-1)\ln(2) > 5 \ln(10)  
\Rightarrow
n > \frac{1}{\ln(2)}\ln\left(\frac{5\,\ln(10)}{\ln(2)}+1\right)
\approx 4.1383 \ .
\]
We choose $n=5$.
}

\solution{\SOL}{\ref{solvAQ16}}{
\subQ{a} Let $f(x) = g(x) -x$.  A fixed point of $g$ is a root of $f$
and vice-versa.  Since $f$ is continuous and
$f(0) = 1 > 0 > -1/2 = f(1)$, it follows from the Mean Value Theorem 
that $f(x) = 0$ for some $x \in ]0,1[$.  Since
$\displaystyle f'(x) = g'(x) - 1 = -2x/(1+x^2)^2 -1 < 0$
for $x \geq 0$, the function $f$ is strictly decreasing on
$[0,1]$.  Thus, $f(x) = 0$ for a unique value of $x \in [0,1]$ and,
therefore, $g(x) = x$ for a unique value of $x \in [0,1]$.

\subQ{b} We show that the hypotheses of the Fixed Point Theorem are
satisfied; namely, we show that $g([0,1]) \subset [0,1]$ and
$|g(x) - g(y)| \leq K |x-y|$ for some $K<1$ and all $x,y \in [0,1]$.

Since $\displaystyle g'(x) = -2x/(1+x^2)^2 < 0$ for $x>0$, 
the function $g$ is strictly decreasing on $]0,\infty[$.  Hence,
$1 = g(0) \geq g(x) \geq g(1) = 1/2$ for all $x \in [0,1]$.  Thus,
$g:[0,1] \rightarrow [0, 1/2] \subset [0,1]$..

To prove that there exists a positive constant $K<1$ such that
$|g(x)-g(y)| \leq K |x-y|$ for $x,y \in [0,1]$, we show that the
maximum of $G(x) = |g'(x)| = 2x/(1+x^2)^2$ on $[0,1]$ is less than $1$
and use this maximum as our constant $K$ as explained in
Remark~\ref{FxPtThDer}.

We use the maximum principle to find the maximum of $G$ on $[0,1]$.
Namely, since $G$ is differentiable on $[0,1]$, the maximum of $G$ is
either at the endpoints of the interval or at one of the critical
points of $G$ in $[0,1]$ if there is one.  The critical
points of $G$ are given by
$\displaystyle G'(x) = 2(1-3x^2)/(1+x^2)^3 = 0$.
There is only one critical point in $[0,1]$.  It is $x= 1/\sqrt{3}$.
Since $G(0) = 0$, $G(1) = 1/2$ and
$G(1/\sqrt{3}) = 3\sqrt{3}/8 <1$, we have that
$G(x) = |g'(x)| \leq K = 3\sqrt{3}/8 <1$ for all $x\in [0,1]$.

\subQ{c} Let $p$ be the fixed point of $g$ in $[0,1]$.  Since $p >0$
and $g'(x) \neq 0$ for $x>0$, we get that 
$g'(p) \neq 0$.  Hence, we have only linear convergence according to
Theorem~\ref{FPorder}.
}

\solution{\SOL}{\ref{solvAQ17}}{
\subQ{a} We show that $g$ satisfies all the hypothesis of the Fixed
Point Theorem on the interval $[1/3,1]$.  Hence, from the Fixed Point
Theorem, we will have that the sequence $\{x_n\}_{n=0}^\infty$
generated by $x_{n+1} = g(x_n)$ for $n\geq 0$ and $x_0\in [1/3,1]$
converges to the unique fixed point $p$ of $g$ in the interval $[1/3,1]$.
\begin{enumerate}
\item Since $g'(x) = -2^{-x}\ln(2)<0$ for all $x$, the function $g$ is
strictly decreasing on $[1/3,1]$.  Thus,
\[
1 > 2^{-1/3} = g(1/3) \geq g(x) \geq g(1) = 0.5 
\]
for all $x \in [1/3,1]$.  Hence, $g:[1/3,1]\rightarrow [1/3,1]$.
\item Since $|g'(x)| = 2^{-x}\ln(2) < 2^{-1/3}\ln(2)$ for
$x \in [1/3,1]$.  We have that 
$|g(x)-g(y)| \leq K |x-y|$ for all $x,y\in [1/3,1]$ with
$K= 2^{-1/3}\ln(2) < 1$ according to Remark~\ref{FxPtThDer}.
\end{enumerate}

\subQ{b} Since $\displaystyle |x_n - p | \leq \frac{K^n}{1-K} |x_1-x_0|$,
we need to find $n$ such that
\[
  \frac{K^n}{1-K} |x_1-x_0| < 10^{-4} \ ;
\]
namely,
\[
\frac{1}{-\ln(K)}\,\ln\left( \frac{10^4|x_1-x_0|}{1-K}\right) < n \ .
\]
If $x_0=0.5$, we have $x_1 = 1/\sqrt{2}$.  Thus, $n$ must satisfy
\[
n > \frac{1}{-\ln(2^{-1/3}\ln(2))}\,
\ln\left( \frac{10^4|1/2 - 1/\sqrt{2}|}{1-2^{-1/3}\ln(2)}\right)
\approx 14.115 \ .
\]
Hence, $n=15$ iterations is enough to reach the accuracy requested. 

\subQ{c} Starting with $x_0 = 0.5$, we compute $x_{n+1} = g(x_n)$
until $|x_{n+1} - x_n| < 10^{-4}$.  The first time that this happen is
for $n=10$.  We get $x_{11} \approx 0.64120525$ as an
approximation of the fixed point of $g$ in the interval $[1/3,1]$.
}

\solution{\SOL}{\ref{solvAQ18}}{
\subQ{a} For $x\neq 0$, we have
\[
g(x) = 12 - \frac{20}{x} = x \iff
x^2 -12 x +20 = (x-10)(x-2) = 0 \iff x=2 \quad \text{or} \quad x=10 \ .
\]

\subQ{b} We show that $g$ satisfies all the hypothesis of the Fixed
Point Theorem on the interval $[9.5,11.5]$.  Hence, from the Fixed
Point Theorem, we will have that the sequence $\{x_n\}_{n=0}^\infty$
generated by $x_{n+1} = g(x_n)$ for $n\geq 0$ and $x_0\in [9.5,11.5]$
converges to the unique fixed point of $g$ in the interval $[9.5,11.5]$.
\begin{enumerate}
\item Since $g'(x) = 20/x^2>0$ for all $x\in[9.5,11.5]$, the function $g$ is
strictly increasing.  Thus,
\[
9.5 < \frac{188}{19} = g(9.5) \leq g(x) \leq g(11.5) =
\frac{236}{23} < 10.5
\]
for all $x \in [9.5,10.5]$.  Hence, $g:[9.5,11.5]\rightarrow [9.5,11.5]$.
\item Since $|g'(x)| = 20/x^2 < 20/9.5^2 = 80/361$ for
$x\in[9.5,11.5]$.  We have that 
$|g(x)-g(y)| \leq K |x-y|$ for all $x,y\in [9.5,11.5]$ with $K=80/361 < 1$
according to Remark~\ref{FxPtThDer}.
\end{enumerate}

\subQ{c} Since $\displaystyle |x_n - p | \leq \frac{K^n}{1-K} |x_1-x_0|$,
we need to find $n$ such that
\[
  \frac{K^n}{1-K} |x_1-x_0| < 10^{-7} \ ;
\]
namely,
\[
\frac{1}{-\ln(K)}\,\ln\left( \frac{10^7|x_1-x_0|}{1-K}\right) < n \ .
\]
If $x_0=9.5$, we have $x_1 = 188/19$.  Thus, $n$ must satisfy
\[
n > \frac{1}{-\ln(80/361)}\,
\ln\left( \frac{10^7|9.5 - 188/19|}{1-80/361}\right)
\approx 10.2459 \ .
\]
Hence, $n=11$ iterations is enough to reach the accuracy requested. 

\subQ{d} Since $g'(x) = 20/x^2 > 0$ for $x\in [9.5,11.5]$, we
have that $g'(p)\neq 0$ at the fixed point $p \in [9.5,11.5]$.  Thus,
the method is of order one; the order of the first non-null derivative
of $g$ at $p$.

\subQ{e} We use the Steffensen's algorithm given in
Algorithm~\ref{SteffAlgo}.   Let $\hat{x}_{-1} = x_0 = 9.5$ and
\[
\hat{x}_{n+1} = \hat{x}_n - \frac{(g(\hat{x}_n) -
  \hat{x}_n)^2}{g(g(\hat{x}_n)) -2 g(\hat{x}_n) + \hat{x}_n}
\]
for $n=-1$, $0$, $1$, \ldots until $|\hat{x}_{n+1} - \hat{x}_n|< 10^{-7}$.
We get $|\hat{x}_{n+1} - \hat{x}_n|<10^{-7}$ for the first time with
$n=1$.  We have $\hat{x}_2 = 10$ to 16 significant digits.

The Steffensen's Algorithm converge faster than the simple Fixed Point
Method applied to $g$ because it is of order two.
}

\solution{\SOL}{\ref{solvAQ19}}{
\subQ{a} Since $f$ is a continuous function and
$f(1) = e - 3 < 0 < f(2) = e^2-5$, it follows from the Intermediate Value
Theorem that $f$ has at least one root in the interval $[1,2]$.  To
prove that it is unique, we note that $f'(x) = e^x-2>0$ for all
$x\in[1,2]$.  So the function is strictly increasing and can therefore cross
the $x$-axis only once.

\subQ{b} We have
\[
f(x) = 0 \iff e^x-2x-1=0 \iff e^x = 2x+1 \iff x = \ln(2x+1) = g(x)
\]
for $x>-1/2$.

\subQ{c} We show that $g$ satisfies all the hypothesis of the Fixed
Point Theorem on the interval $[1,2]$.  Hence, from the Fixed Point
Theorem, we will have that the sequence $\{x_n\}_{n=0}^\infty$
generated by $x_{n+1} = g(x_n)$ for $n\geq 0$ and $x_0\in [1,2]$
converges to the unique fixed point of $g$ in the interval $[1,2]$.
\begin{enumerate}
\item Since $g'(x) = 2/(1+2x)>0$ for all $x\in[1,2]$, the function $g$ is
strictly increasing.  Thus,
\[
1 < \ln(3) = g(1) \leq g(x) \leq g(2) = \ln(5) <2
\]
for all $x\in [1,2]$.  Hence, $g:[1,2]\rightarrow [1,2]$.
\item Since $|g'(x)| = 2/(2x+1) < 2/3$ for $x\in[1,2]$.  We have that
$|g(x)-g(y)| \leq K |x-y|$ for all $x,y\in [1,2]$ with $K=2/3 < 1$
according to Remark~\ref{FxPtThDer}.
\end{enumerate}

\subQ{d} Since $g'(x) \geq g'(2) = 2/5$ for $x\in [1,2]$, we certainly have
that $g'(p)\neq 0$ at the fixed point $p$.  So the method is of order
one; the order of the first non-null derivative of $g$ at $p$.
}

\solution{\SOL}{\ref{solvAQ20}}{
\subQ{a} $\sqrt[3]{25}$ is obviously the unique root of $f(x) = x^3 -25$
because
\[
  x^3 - 25 = 0 \iff x^3 = 25 \iff x = \sqrt[3]{25} \ .
\]

\subQ{b} If $p>0$, we have
\[
  f(p) = p^3 - 25 = 0 \iff p^3 = 25 \iff p^2 = \frac{25}{p} 
  \iff p = \frac{5}{\sqrt{p}} = g(p) \ .
\]
Since $\sqrt[3]{25}$ is the only (positive) root of $f$, it is the
fixed point $p>0$ of $g$.

\subQ{c} The graph of $g$ (in blue) between $1$ and $4$ is sketched
in the figure below.  We have also included the line $y=x$ (in red).
\figbox{solve_equ_A/fixedpoint_quest1}{7cm}
The fixed point of $g$ is given by the point of intersection of the
graph of $g$ with the line $y=x$.  The fixed point
$p = \sqrt[3]{25}$ is closed to $3$.

We consider the interval $[2,4]$ that contains $p$.  We show that $g$
satisfies all the hypothesis of the Fixed Point Theorem on the
interval $[2,4]$.  Hence, from the Fixed Point Theorem, we will have
that the sequence $\{x_n\}_{n=0}^\infty$ generated by
$x_{n+1} = g(x_n)$ for $n\geq 0$ and $x_0\in [2,4]$ converges to the
unique fixed point $p=\sqrt[3]{25}$ of $g$ in the interval $[2,4]$. 
\begin{enumerate}
\item Since $\sqrt{x}$ is strictly increasing and positive for $x>0$,
we have that $g$ is strictly decreasing for $x>0$.  Thus, 
\[
4 > 5/\sqrt{2} = g(2) \geq g(x) \geq g(4) = 5/2 > 2
\]
for $x \in [2,4]$.  Hence, $g:[2,4]\rightarrow [2,4]$.
\item Since $\displaystyle |g'(x)| = \frac{5}{2x^{3/2}} \leq \frac{5}{2^{5/2}}$
for $x\in[2,4]$.  We have that $|g(x)-g(y)| \leq K |x-y|$ for all
$x,y\in [1,2]$ with $\displaystyle K= \frac{5}{2^{5/2}} < 1$
according to Remark~\ref{FxPtThDer}.
\end{enumerate}

\subQ{d} Since $\displaystyle |x_n - p | \leq \frac{K^n}{1-K} |x_1-x_0|$,
we need to find $n$ such that
\[
\frac{K^n}{1-K} |x_1-x_0| < 10^{-5} \ ;
\]
namely,
\[
\frac{1}{-\ln(K)}\,\ln\left( \frac{10^5|x_1-x_0|}{1-K}\right) < n \ .
\]
If $x_0=3$, we have $x_1 = 5/\sqrt{3} \approx 2.8867513$.
Thus, $n$ must satisfy
\[
n > \frac{1}{-\ln(5/2^{5/2})}\,
\ln\left( \frac{10^5|3 - 5/\sqrt{3}|}{1-5/2^{5/2}}\right)
\approx 105.6598 \ .
\]
Hence, $n=106$ iterations is enough to reach the accuracy requested. 

\subQ{e} Starting with $x_0 = 3$, we compute $x_n = g(x_{n-1})$
until $|x_n - x_{n-1}| < 10^{-5}$.  The first time that this happen is
for $n=15$.  We get $x_{15} \approx 2.924015$ as an
approximation of the fixed point of $g$ in the interval $[2,4]$.

As we can see, the value of $n$ estimated in (d) is a very large
overestimation of the number of iterations needed to reach an accuracy
of $10^{-5}$ if we start with $x_0 = 3$.  The formula used in (d) to
estimate $n$ will generally give a large overestimation.  However, we
have to keep in mind that the value $n$ obtained in (d) is valid for
all $x_0$, not just for $x_0 = 3$.
}

\solution{\SOL}{\ref{solvAQ22}}{
\subQ{a} The figure below shows the graph of $\tan(x)$ and $g$ (red
curve in the graph on the right).

\pdfbox{solve_equ_A/arctan_quest1a}

\subQ{b} We have $g'(x) = 1/(1+x^2)$.  Hence $g$ is strictly
increasing on $[\pi,3\pi/2]$ and
$|g'(x)| \leq g'(\pi) = 1/(1+\pi^2) < 1$ on
$[\pi,3\pi/2]$.  We can then say that:

\begin{enumerate}
\item $g:[\pi,3\pi/2] \rightarrow [\pi,3\pi/2]$ because
\[
\pi < \pi + \arctan(\pi) = g(\pi) \leq g(x) \leq
g(3\pi/2) = \pi + \arctan(3\pi/2) < 3\pi/2
\]
for all $x\in[\pi,3\pi/2]$ since $g$ is increasing.  We have used the fact that
$0<\arctan(x)<\pi/2$ for all $x>0$.
\item With $K=1/(1+\pi^2)$, we have
\[
|g'(x)| \leq K <1
\]
for all $x \in [\pi.3\pi/2]$.  Thus,
$|g(x)-g(y)| \leq K |x-y|$ for all $x,y \in [\pi,3\pi/2]$
according to Remark~\ref{FxPtThDer}.
\end{enumerate}

\subQ{c} From the Fixed Point Theorem, we have that the sequence
$\{x_i\}_{i=0}^\infty$ defined by $x_0\in [\pi,3\pi/2]$ and
$x_{i+1} = g(x_i)$ converges to $p$.  Since
\[
|x_n - p | \leq \frac{K^n}{1-K} |x_1-x_0|
= \frac{1}{\pi^2(1+\pi^2)^{n-1}} |x_1-x_0| \ ,
\]
we need to find $n$ such that
\[
\frac{1}{\pi^2(1+\pi^2)^{n-1}} |x_1-x_0| < 10^{-5} \ ;
\]
namely,
\[
\frac{\ln\left(10^5|x_1-x_0|/\pi^2\right)}{\ln(1+\pi^2)}  < n - 1 \ .
\]
If $x_0=4$, we have $x_1 = g(x_0) = \pi + \arctan(4)$. 
Thus, $n$ must satisfy
\[
n > \frac{\ln\left(10^5|\pi+\arctan(4)-4|/\pi^2\right)}{\ln(1+\pi^2)} + 1
\approx 4.54695 \ .
\]
Hence, $n=5$ iterations is enough to reach the accuracy requested.
}

\solution{\SOL}{\ref{solvAQ23}}{
\subQ{a} If $p>0$ is the fixed point of $g$, then
\[
p = \frac{p}{2} + \frac{a}{2p} \ .
\]
If we multiply both sides of the equality by $2$ and subtract $p$ from
both sides, we get $p = a/p$.  Thus $p^2=a$ or $p=\sqrt{a}$ since we
assume that $a>0$.

\subQ{b} Suppose that $x>0$.  Since $(x-\sqrt{a})^2 >0$, we get
$x^2 -2\sqrt{a} x + a >0$.  Thus $x^2 + a > 2\sqrt{a} x$ and,
after division by $2x$ on both side of the inequality, we have
\[
g(x) = \frac{x}{2} + \frac{a}{2x} > \sqrt{a} \ .
\]
Therefore, $x_i = g(x_0) \geq \sqrt{a}$ if $x_0>0$.  We may assume
that $x_0 \geq \sqrt{a}$.

Since
\[
g'(x) = \frac{1}{2} - \frac{a}{2x^2} > 0
\]
for $x>\sqrt{a}$, we have that $g$ is strictly increasing on
$]\sqrt{a},\infty[$.

We now show that $g$ satisfies all the hypothesis of the Fixed Point
Theorem on $[\sqrt{a},m]$ for $m >\sqrt{a}$ arbitrary.

\begin{enumerate}
\item We have $g:[\sqrt{a},m]\rightarrow [\sqrt{a},m]$.  Since $g$
is strictly increasing and $m > \sqrt{a}$, we have that
\[
\sqrt{a} = g(\sqrt{a}) \leq g(x) \leq g(m) = 
\frac{m}{2} + \frac{a}{2m^2} \leq \frac{m}{2} + \frac{m}{2} = m
\]
for all $x\in [\sqrt{a},m]$.
\item For all $x\in [\sqrt{a},m]$, we have that
\[
|g'(x)| = \left| \frac{1}{2} - \frac{a}{2x^2} \right|
= \frac{1}{2} - \frac{a}{2x^2} \leq \frac{1}{2}
\]
because $x^2 \geq a$.  Hence,
$|f(x) - f(y) \leq K |x-y|$ for all $x,y \in [\sqrt{a},m]$ with
$K = 1/2 <1$ according to Remark~\ref{FxPtThDer}.
\end{enumerate}

For any $m>\sqrt{a}$, we have from the Fixed Point Theorem that the
function $g$ has an unique fixed point in $[\sqrt{a},m]$ and the
sequence $\displaystyle \{x_n\}_{n=0}^{\infty}$ converge to this fixed
point for any $x_0 \in [\sqrt{a},m]$.   Thus, since $\sqrt{a}$ is a
unique fixed point for $x>0$, the sequence
$\displaystyle \{x_n\}_{n=0}^{\infty}$
converges to the fixed point $\sqrt{a}$ whatever $x_0\geq \sqrt{a}$.

Since $x_1 = g(x_0) \geq \sqrt{a}$ for all $x_0>0$,
the sequence $\displaystyle \{x_n\}_{n=0}^{\infty}$
converges to the fixed point $\sqrt{a}$ whatever $x_0>0$
because the sequences
$\displaystyle \{x_n\}_{n=0}^{\infty}$ and
$\displaystyle \{x_n\}_{n=1}^{\infty}$ have the same limit.
}

\solution{\SOL}{\ref{solvAQ24}}{
\subQ{a} The function $f$ is continuous on $[a,b]$ because it is
differentiable on $[a,b]$.  Moreover, $f$ has opposite signs at $a$ and 
$b$ because $f(a)f(b)<0$.  It follows from the Intermediate Value
Theorem that $f$ must be null at a point in the interval $[a,b]$.

Since $f'(x) >0$ for all $x \in [a,b]$, we have that $f$ is a
strictly increasing function on $[a,b]$.  Thus, $f(a) < 0 < f(b)$ and
$f$ cannot intersect the $x$ axis more than once.

\subQ{b} Since $f'$ is continuous on the closed interval $[a,b]$, it
reaches its absolute maximum and absolute minimum at some points of
the interval $[a,b]$.  Let $x_M$ and $x_m$ in $[a,b]$ be such that
\[
M = f(x_M) = \max\{ f'(x) : a\leq x \leq b \} \quad
\text{and} \quad
m = f(x_m) = \min\{ f'(x) : a\leq x \leq b \} \ .
\]
We have that $0 < m < M$ since $f'(x)>0$ for all $x\in [a,b]$.

We claim that the function $F(x) = x+\lambda f(x)$ with
$\lambda = -1/M$ satisfies the Fixed Point Theorem on $[a,b]$.

\begin{enumerate}
\item Since
\[
  F'(x) = 1 + \lambda f'(x) = 1 - \frac{f'(x)}{M} \geq 0
\]
for all $x\in[a,b]$, we have that $F$ is never decreasing on $[a,b]$.
Thus
\[
a < a - f(a)/M = F(a) \leq F(x) \leq F(b) = b - f(b)/M < b
\]
for all $x\in[a,b]$.  Hence, $F:[a,b]\rightarrow [a,b]$.  Recall that
$f(a)<0<f(b)$.
\item Since
\[
0 \leq F'(x) = 1 + \lambda f'(x) = 1 - \frac{f'(x)}{M}
\leq 1 - \frac{m}{M}
\]
for all $x \in [a,b]$, we have that $|F'(x)| \leq K = 1- m/M < 1$
for all $x \in [a,b]$.  It follows that
$|F(x)-F(y)| \leq K|x-y|$ for all $x \in [a,b]$ with $K<1$
according to Remark~\ref{FxPtThDer}.
\end{enumerate}

The hypotheses of the Fixed Point Theorem are satisfied by $F$ on
$[a,b]$.  Obviously, if $F(p) = p$, then $p + \lambda f(p) = p$.
Hence, $f(p) = 0$.
}

\solution{\SOL}{\ref{solvAQ25}}{
We prove that $g$ satisfies the hypothesis of the Fixed Point
Theorem on $[a,b]$.

\begin{enumerate}
\item Choose any $x\in [a,b]$.  From the Mean Value Theorem, there
exists $\xi$ between $x$ and $m$ (and so in $[a,b]$) such that 
$g(x)-g(m) = g'(\xi)(x-m)$.  Since $|g'(\xi)|<1$, we have
\[
|g(x)-m| = |g(x)-g(m)| = |g'(\xi)(x-m)| = |g'(\xi)|\,|x-m| <|x-m|
\]
for all $x\in[a,b]$.  From $|x-m| \leq (b-a)/2$, we get
$|g(x)-m| \leq (b-a)/2$ for all $x\in[a,b]$.  This proves that
$g(x) \in [a,b]$ for all $x \in [a,b]$. 
\item Since $|g'|$ is a continuous function on the closed set $[a,b]$,
$|g'|$ reaches its absolute maximum at a point $\nu \in [a,b]$.  Hence
$K = |g'(\nu)| < 1$ will satisfies $|g'(x)|\leq K < 1$ for all
$x \in [a,b]$.  Therefore,
$|g(x)-g(y)| \leq K |x-y|$ for all $x,y \in [a,b]$ with $K<1$
according to Remark~\ref{FxPtThDer}.
\end{enumerate}
}

\solution{\SOL}{\ref{solvAQ26}}{
It is clear that if $\{x_n\}_{n=0}^\infty$ is a sequence
defined by $x_{n+1} = g(x_n)$ for $n \geq 0$ such that
$x_N = p$ for some $N$, then $x_n = p$ for all $n \geq N$ because
$g(p) =p$.  Thus, $\{x_n\}_{n=0}^\infty$ converges to $p$ in a finite
number of iterations.

Are there other sequences $\{x_n\}_{n=0}^\infty$ that converge to $p$?
To show that there are no other sequence converging to $p$ requires a
little bit of work.

Choose $K$ between $1$ and $|g'(p)|$.  Since $|g'|$ is continuous and
$|g'(p)|>K$, there exists $\delta >0$ such that
$1< K \leq |g'(x)|$ for all $x \in [p-\delta, p+\delta]$.

Since $g$ is a continuous function and $|g'(x)|>1$ for
$x \in [p-\delta, p+\delta]$, then $p$ is the unique fixed point of
$g$ in $[p-\delta, p+\delta]$

Given any $x\in [p-\delta,p[ \cup ]p,p+\delta]$, it follows from the
Mean Value Theorem that there exists $\xi$ between $p$ and $x$, and so in
$[p-\delta, p+\delta]$, such that
\[
|g(x) - p| = |g(x)-g(p)| = |g'(\xi)(x-p)| = |g'(\xi)|\,|x-p|
\geq K|x-p| > |x-p| \ .
\]

Suppose that $\{x_n\}_{n=0}^\infty$ is a sequence defined by
$x_{n+1} = g(x_n)$ for $n \geq 0$ such that $x_n \neq p$ for all $n$.
Suppose also that this sequence also converges to the fixed point $p$
of $g$.  By definition of convergence, there exists $N>0$ such that
$|x_n-p|<\delta$ for all $n>N$.  However,
$x_n \in [p-\delta, p+\delta]$ for all $n>N$ implies that
\[
|x_{n+1} - p| = |g(x_n)-p| > |x_n-p|
\]
for all $n>N$ as we have shown above.  It follows by induction that
$|x_n - p| \geq |x_{N+1}-p| > 0$ for $n>N$.   The distance between $x_n$
and $p$ does not go to zero.  This is a contradiction that
$\{x_n\}_{n=0}^\infty$ is a sequence converging to the fixed point $p$.
}

\solution{\SOL}{\ref{solvAQ27}}{
We already have one of the two hypotheses required by the Fixed
Point Theorem; namely, $|g'(x)| \leq \lambda < 1$ for all
$x \in [x_0-\rho,x_0+\rho]$.  It is left to show that
$g:[x_0-\rho,x_0+\rho] \rightarrow [x_0-\rho,x_0+\rho]$.

Choose $x \in [x_0-\rho,x_0+\rho]$.  Then,
\begin{align*}
|g(x) - x_0| &= |g(x) - g(x_0) + g(x_0) - x_0|
\leq |g(x)-g(x_0)| + |g(x_0)-x_0| \\
&= |g'(\mu)(x-x_0)| + (1-\lambda)\rho \ ,
\end{align*}
where we have used the Mean Value Theorem to find $\mu$ between $x$ and
$x_0$ such that $g(x)-g(x_0) = g'(\mu)(x-x_0)$.  We have also used the
definition of $\rho$.  Hence, since $|g'(x)| \leq \lambda < 1$ for all
$x \in [x_0-\rho, x_0+\rho]$, we have
\[
|g(x) - x_0| \leq \lambda |x-x_0| + (1-\lambda)\rho
\leq \lambda\rho + (1-\lambda)\rho = \rho
\]
for all $x \in [x_0-\rho, x_0+\rho]$.  Hence,
$g(x) \in [x_0 -\rho, x_0+\rho]$ for all $x \in [x_0-\rho, x_0+\rho]$.
}

\solution{\SOL}{\ref{solvAQ28}}{
If $[a,b] = [0,1]$ and $f(x) = 0.5 x + 1$, then $f'(x) = 0.5 <1$ for
all $x \in [0,1]$ but there is no fixed point of $f$ in $[a,b]$.

A contraction satisfies all hypothesis of the Fixed Point Theorem but
one.  The condition $f:[a,b]\rightarrow [a,b]$ is not required for a
contraction.  The function $F$ of the previous paragraph satisfies
$f:[0,1]\rightarrow [1, 1.5]$.  The interval $[0,1]$ has been
contracted but not mapped into itself.

$f(x)=0.5x+1$ for $x\in[1,3]$ does satisfy all the hypotheses of
the Fixed Point Theorem (verify this).  The fixed point is $2\in[1,3]$.
}

\solution{\SOL}{\ref{solvAQ29}}{
\stage{$\mathbf{\Leftarrow}$} From Taylor's Theorem,
Theorem~\ref{TaylorTheo}, we have that
$f(x) = p_{m-1}(x) + r_{m-1}(x)$, where
\[
p_{m-1}(x) = \sum_{j=0}^{m-1} \frac{1}{j!}\,f^{(j)}(p)\,(x-p)^j
\quad \text{and} \quad
r_{m-1}(x) = \frac{1}{m!}\,f^{(m)}(\xi)\,(x-p)^m
\]
for $\xi(x)$ in the interval with endpoints $x$ and $p$.  Since,
$f^{(j)}(p) = 0$ for $0 \leq j < m$, we have that $p_{m-1}(x) = 0$ for
all $x$.  Thus $f(x) = r_{m-1}(x) = q(x) \, (x-p)^m$ with
$\displaystyle q(x) = \frac{1}{m!}\,f^{(m)}(\xi(x))$.

Since $\xi(x)$ is between $x$ and $p$, we have
that $\displaystyle \lim_{x\to p} \xi(x) = p$.  Therefore, since
$f^{(m)}$ is continuous, we have
\[
  \lim_{x\to p} q(x) = \frac{1}{m!}\,f^{(m)}(p) \neq 0 \ .
\]
Note that $q$ is a continuous function on $\RR\setminus \{p\}$
because $\displaystyle q(x) = \frac{f(x)}{(x-p)^m}$ for $x \neq p$,
where $f$ and $(x-p)^m$ are continuous functions of $x$.  The function
$q$ is also continuous at $p$ because
$\displaystyle \lim_{x\to p} \frac{f(x)}{(x-p)^m} =
\frac{1}{m!}\, f^{(m)}(p) = q(p)$ according to l'Hospital Rule. 

\stage{$\mathbf{\Rightarrow}$}
From $f(x) = (x-p)^m q(x)$, we have by induction that
\begin{equation} \label{solveA29eq1} 
f^{(k)}(x) = \sum_{j=0}^k \binom{k}{j} C_j (x-p)^{m-j} q^{(k-j)}(x)
\end{equation}
for $0 \leq k \leq m$, where
\[
C_j = \begin{cases}
1 & \qquad \text{if} \quad j = 0 \\
m(m-1)\ldots(m-j+1) & \qquad \text{if} \quad j>0
\end{cases}
\]
Hence, $f^{(k)}(p) = 0$ for $0\leq k <m$ because each term in
(\ref{solveA29eq1}) has a factor $(x-p)$.  For $k=m$, we get
\[
f^{(m)}(p) = \binom{m}{m} C_m q(p)
= m!\, q(p) \neq 0 \ .
\]
}

\solution{\SOL}{\ref{solvAQ30}}{
According to Theorem~\ref{FPorder}, to get a convergence of order
three, we need $F(r) = r$, $F'(r) = F''(r) =0$ and $F'''(r)\neq 0$.

Since $f(r)=0$, we get $F(r) = r -f(r)f'(r) = r$.

From $F'(r) = 0$, we get
\[
0 = 1 - (f'(r))^2 - f(r)f''(r) = 1 - (f'(r))^2
\]
because $f(r)=0$.  Hence $f'(r) = \pm 1$.   From $F''(r) = 0$, we get
\[
0 = -3 f'(r)f''(r) - f(r)f'''(r) = -3 f'(r) f''(r)
\]
because $f(r)=0$.  Since $f'(r) \neq 0$, we get $f''(r) = 0$.  From
$F'''(r) \neq 0$, we get
\[
0 \neq  -3 (f''(r))^2 - 4 f'(r)f'''(r) - f(r)f^{(4)}(r)
= - 4 f'(r) f'''(r)
\]
because $f(r) = f''(r) = 0$.  Since $f'(r) \neq 0$, we get
$f'''(r) \neq 0$.

The conditions on $f$ are $f(r)=f''(r)= 0$, $|f'(r)| = 1$ and
$f'''(r)\neq 0$.
}

\solution{\SOL}{\ref{solvAQ31}}{
To get a convergence of order exactly three, we need $F(r) = r$,
$F'(r) = F''(r) =0$ and $F'''(r)\neq 0$.

Since $f(r)=0$, we get $F(r) = r + f(r)g(r) = r$.

From $F'(r) = 0$, we get
\[
0 = 1 +f'(r)g(r) + f(r)g'(r) = 1 + f'(r)g(r)
\]
because $f(r)=0$.  Hence $\displaystyle g(r) = -\frac{1}{f'(r)}$.
From $F''(r) = 0$, we get
\[
0 = f''(r)g(r) + 2f'(r)g'(r) + f(r)g''(r)
= -\frac{f''(r)}{f'(r)} + 2f'(r)g'(r)
\]
because $f(r)=0$ and $\displaystyle g(r) = -\frac{1}{f'(r)}$.  Hence
$\displaystyle g'(r) = \frac{f''(r)}{2(f'(r))^2}$.  From $F'''(r) \neq 0$,
we get
\begin{align*}
0 &\neq f'''(r)g(r) + 3 f''(r)g'(r) + 3 f'(r)g''(r) + f(r)g'''(r) \\
&= -\frac{f'''(x)}{f'(r)} + \frac{3(f''(r))^2}{2(f'(r))^2}
+ 3 f'(r) g''(r)
\end{align*}
because $f(r)=0$, $\displaystyle g(r) = -\frac{1}{f'(r)}$ and
$\displaystyle g'(r) = \frac{f''(r)}{2(f'(r))^2}$.  Hence
$\displaystyle g''(r) \neq
\frac{f'''(x)}{3(f'(r))^2} - \frac{(f''(r))^2}{2(f'(r))^3}$.

The conditions on $g$ are 
$\displaystyle g(r) = -\frac{1}{f'(r)}$,
$\displaystyle g'(r) = \frac{f''(r)}{2(f'(r))^2}$ and
$\displaystyle g''(r) \neq
\frac{f'''(x)}{3(f'(r))^2} - \frac{(f''(r))^2}{2(f'(r))^3}$.
}

\solution{\SOL}{\ref{solvAQ32}}{
We have to find for which sequence $\{x_n\}$ above the following statement
is true.
\[
\lim_{n\rightarrow \infty}\frac{|x_{n+1}|}{|x_n|^2} = \lambda \neq 0
\text{ or } \infty \ .
\]
Note that all sequences converge to $0$.  Therefore, the error $e_n$ is
$e_n = |x_n - 0| = |x_n|$ for all $n$.

We have:\\
\subQ{a}
\[
\lim_{n\rightarrow \infty}\frac{1/(n+1)^2}{(1/n^2)^2}
= \lim_{n\rightarrow \infty}\frac{n^4}{(n+1)^2}
= \lim_{n\rightarrow \infty}\frac{n^4}{n^2+2n+1}
= \lim_{n\rightarrow \infty}\frac{n^2}{1 + 2/n +1/n^2}
=\infty \ .
\]
\subQ{b}
\[
\lim_{n\rightarrow \infty}\frac{1/2^{2^{(n+1)}}}{\left(1/2^{2^n}\right)^2}
=\lim_{n\rightarrow \infty} \frac{2^{2^{(n+1)}}}{2^{2^{(n+1)}}} = 1 \ .
\]
\subQ{c}
\[
\lim_{n\rightarrow \infty}\frac{1/\sqrt{n+1}}{(1/\sqrt{n})^2}
= \lim_{n\rightarrow \infty}\frac{n}{\sqrt{n+1}}
= \infty
\]
because
\[
\frac{n}{\sqrt{n+1}} \geq \frac{n}{\sqrt{2n}} =
\frac{\sqrt{n}}{\sqrt{2}} \rightarrow \infty
\]
as $n\rightarrow \infty$.\\
\subQ{d}
\[
\lim_{n\rightarrow \infty}\frac{1/(e^{n+1})}{(1/e^n)^2}
= \lim_{n\rightarrow \infty}\frac{e^{2n}}{e^{n+1}}
= \lim_{n\rightarrow \infty} e^{n-1} = \infty \ .
\]
\subQ{e}  We have
\[
\lim_{n\rightarrow \infty}\frac{1/(n+1)^{n+1}}{(1/n^n)^2}
= \lim_{n\rightarrow \infty}\frac{n^{2n}}{(n+1)^{n+1}}
= \infty
\]
because
\[
\frac{n^{2n}}{(n+1)^{n+1}} \geq \frac{n^{2n}}{(2n)^{n+1}}
= \frac{n^{n-1}}{2^{n+1}}
= \frac{1}{2^2} \, \left(\frac{n}{2}\right)^{n-1} \rightarrow \infty
\]
as $n\rightarrow \infty$.

Only the sequence in (b) converges quadratically.
}

\solution{\SOL}{\ref{solvAQ33}}{
\subQ{a} Let $\displaystyle e_n = 10^{-k^n} - 0$.  We have
\[
\frac{|e_{n+1}|}{|e_n|^\alpha}
= \frac{10^{-k^{n+1}}}{\left(10^{-k^n}\right)^\alpha}
= 10^{-k^{n+1} +\alpha\, k^n} = 10^{(\alpha-k)k^n} \ .
\]
Hence,
\[
\lim_{n\rightarrow \infty} \frac{|e_{n+1}|}{|e_n|^\alpha} =
\begin{cases}
1 & \quad \text{if} \quad \alpha = k \\
+\infty & \quad \text{if} \quad \alpha > k \\
0 & \quad \text{if} \quad \alpha < k
\end{cases}
\]
The order of convergence is $\alpha = k$.

\subQ{b} Let $\displaystyle e_n = 10^{-n^k}-0$.  We have
\[
\frac{|e_{n+1}|}{|e_n|^\alpha}
= \frac{10^{-(n+1)^k}}{\left(10^{-n^k}\right)^\alpha}
= 10^{-(n+1)^k +\alpha\, n^k} = 10^{(\alpha-1)n^k - k n^{k-1} -\, \text{l.o.t.}}
\ ,
\]
where \mbox{l.o.t.} stands for the terms in $n^j$ with $0 \leq j < n-1$.
Hence,
\[
\lim_{n\rightarrow \infty} \frac{|e_{n+1}|}{|e_n|^\alpha} =
\begin{cases}
0 & \quad \text{if} \quad \alpha \leq 1 \\
\infty & \quad \text{if} \quad \alpha >1
\end{cases}
\]
There is nothing in between.  Recall that according to the binomial theorem,
\[
(n+1)^k = \sum_{i=0}^k \binom{k}{i}\, n^i = n^k + k\,n^{k-1} +
\ldots + 1 \ ,\ \text{where} \quad
\binom{k}{i} = \frac{k!}{(k-i)!\,i!} \ .
\]
}

\solution{\SOL}{\ref{solvAQ35}}{
All the results of the computations below will be displayed using
15-digit rounding accuracy to compare with the exact solutions at
the end.

Using Newton's Method with Horner's Algorithm, Code~\ref{NRHorner},
with $x_0=1$ and a tolerance of $10^{-10}$, we find
$r_0 = 0.333333333308985$ as an approximation of a root of $p$.

The deflated polynomial $q_1$ such that $p(x) = (x-r_0) q_1(x)$ is\\
$q_1(x) = x^2 -5.641592653691014 x + 7.853981634573692$\ .

Using Newton's Method with Horner's Algorithm, Code~\ref{NRHorner},
with $x_0=1$ and a tolerance of $10^{-10}$, we find
$r_1 = 2.500000000539526$ as an approximation of a root of $q_1$.

Using Newton's Method with Horner's Algorithm, Code~\ref{NRHorner},
with $x_0=r_1$ and a tolerance of $10^{-10}$, we find
$c_1 = 2.500000000539525$ as an approximation of a root of $p$.

The deflated polynomial $q_2$ such that $q_1(x) = (x-r_1) q_2(x)$ is\\
$q_2(x) =  x -3.141592653151490$\ .

we have that $r_2 = 3.141592653151490$ as an approximation of a
root of $q_2$.

Finally, using Newton's Method with Horner's Algorithm, Code~\ref{NRHorner},
with $x_0=r_2$ and a tolerance of $10^{-10}$, we find
$c_2 = 3.141592653151491$ as an approximation of a root of $p$.

If $c_0 = r_0$, the approximations of the roots of $p$ are
$c_i$ for $i =0$, $1$ and $2$.  If we use $10$-digit rounding accuracy
for the $c_i$, we have that all the $10$ digits of $c_0$ are correct,
only the last digit of $c_1$ and $c_2$ is wrong.
}

\solution{\SOL}{\ref{solvAQ36}}{
All the results of the computations below will be displayed using
15-digit rounding accuracy to compare with the exact solutions at
the end.

Using Newton's Method with Horner's Algorithm, Code~\ref{NRHorner},
with $x_0=1$ and a tolerance of $10^{-9}$, we find
$r_0 = -3.548232897979703$ as an approximation of a root of $p$.

The deflated polynomial $q_1$ such that $p(x) = (x-r_0) q_1(x)$ is\\
$q_1(x) = x^3 -5.548232897979703 x^2 + 7.686422494264843 x
-11.273217161921718$\ .

Using Newton's Method with Horner's Algorithm, Code~\ref{NRHorner},
with $x_0=1$ and a tolerance of $10^{-9}$, we find
$r_1 = 4.381113440995944$ as an approximation of a root of $q_1$.

Using Newton's Method with Horner's Algorithm, Code~\ref{NRHorner},
with $x_0=r_1$ and a tolerance of $10^{-9}$, we find
$c_1 = 4.381113440995943$ as an approximation of a root of $p$.

The deflated polynomial $q_2$ such that $q_1(x) = (x-r_1) q_2(x)$ is\\
$q_2(x) =  x^2 -1.167119456983760 x + 2.573139754025407$\ .

The polynomial $q_2$ has two complex roots that can be found with the
formula to find the roots of a quadratic polynomial.  They are
$r_2 = 0.583559728491880 + 1.494188006011255\, i$ and
$r_3 = 0.583559728491880 - 1.494188006011255\, i$.
Using Newton's Method with Horner's Algorithm, Code~\ref{NRHorner},
with $x_0=r_2$ and a tolerance of $10^{-9}$, we find
$c_2 = 0.583559728491880 + 1.494188006011255\,i$ as an approximation of
a root of $p$.  Similarly, with $x_0 = r_3$, we find
$c_3 = 0.583559728491880 - 1.494188006011255\,i$ as an approximation of
a root of $p$ as expected since complex roots of a polynomial with real
coefficients come in pairs of complex conjugate values.

If $c_0 = r_0$, the approximations of the roots of $p$ are
$c_i$ for $0 \leq i \leq 3$.  If we use $9$-digit rounding accuracy
for the $c_i$, we have that all the $9$ digits are right.
}

%%% Local Variables: 
%%% mode: latex
%%% TeX-master: "notes"
%%% End: 
