\nonumsection{Chapter~\ref{chaptInterB} : Splines}

\solution{\SOL}{\ref{intBQ1}}{
We have
\[
p_i(x) = \left(\left(\alpha_i(x-x_i) + \beta_i\right)(x-x_i)
+\gamma_i\right)(x-x_i) + \delta_i
\]
on $[x_i,x_{i+1}]$, where $x_0 = 0$, $x_1 = 1$, $x_2=3$. $x_3 = 4$,
$x_4 =5$ and $x_5 = 5.5$.

The solution of $A\VEC{z} = \VEC{b}$, where
\begin{align*}
A &= \begin{pmatrix}
2 \Delta x_0 & \Delta x_0 & 0 & 0 & 0 & 0 \\
\Delta x_0 & 2(\Delta x_1 - \Delta x_0) & \Delta x_1 & 0 & 0 & 0 \\
0 & \Delta x_1 & 2(\Delta x_2 - \Delta x_1) & \Delta x_2 & 0 & 0 \\
0 & 0 & \Delta x_2 & 2(\Delta x_3 - \Delta x_2) & \Delta x_3 & 0 \\
0 & 0 & 0 & \Delta x_3 & 2(\Delta x_4 - \Delta x_3) & \Delta x_4 \\
0 & 0 & 0 & 0 & \Delta x_4 & 2 \Delta x_4
\end{pmatrix} \\
&= \begin{pmatrix}
2 & 1 & 0 & 0 & 0 & 0 \\
1 & 6 & 2 & 0 & 0 & 0 \\
0 & 2 & 6 & 1 & 0 & 0 \\
0 & 0 & 1 & 4 & 1 & 0 \\
0 & 0 & 0 & 1 & 3 & 0.5 \\
0 & 0 & 0 & 0 & 0.5 & 1
\end{pmatrix}
\end{align*}
and
\[
\VEC{b} = \begin{pmatrix}
\displaystyle - 6f'(x_0) + 6\, \frac{f(x_1) - f(x_0)}{x_1-x_0}\\[0.8em]
\displaystyle 6\,\frac{f(x_2) -f(x_1)}{x_2-x_1}
- 6\,\frac{f(x_1) - f(x_0)}{x_1-x_0}\\[0.8em]
\displaystyle 6\,\frac{f(x_3) -f(x_2)}{x_3-x_2}
- 6\,\frac{f(x_3) - f(x_1)}{x_2-x_1}\\[0.8em]
\displaystyle 6\,\frac{f(x_4) -f(x_3)}{x_4-x_3}
- 6\,\frac{f(x_3) - f(x_2)}{x_3-x_2}\\[0.8em]
\displaystyle 6\,\frac{f(x_5) -f(x_4)}{x_5-x_4}
- 6\,\frac{f(x_4) - f(x_3)}{x_4-x_3}\\[0.8em]
\displaystyle 6f'(x_5) - 6\,\frac{f(x_5)-f(x_4)}{x_5-x_4}
\end{pmatrix}
= \begin{pmatrix}
6 \\
-6 \\
-6 \\
6 \\
-12 \\
0
\end{pmatrix} \ ,
\]
is
\[
\VEC{z} = \begin{pmatrix}
3.615279672578445 \\
-1.230559345156889 \\
-1.115961800818554 \\
3.156889495225103 \\
-5.511596180081855 \\
2.755798090040928
\end{pmatrix} \ .
\]

The coefficients of $p_i$ are given by
\[
\delta_i = f(x_i)  \ , \quad
\gamma_i = -\frac{z_i \Delta x_i}{3} - \frac{z_{i+1} \Delta x_i}{6} +
\frac{f(x_{i+1})-f(x_i)}{\Delta x_i}  \ , \quad
\beta_i = \frac{z_i}{2} \quad \text{and} \quad
\alpha_i = \frac{z_{i+1}-z_i}{6 \Delta x_i}
\]
for $i=0$, $2$, \ldots, $5$.

The following table lists the values of the coefficients of
$p_i$.
\[
\begin{array}{c|cccc}
i & \alpha_i & \beta_i & \gamma_i & \delta_i \\
\hline
0 & -0.807639836289222 & 1.807639836289222 & 0 & 2 \\
1 & 0.009549795361528 & -0.615279672578445 & 1.192360163710777 & 3 \\
2 & 0.712141882673943 & -0.557980900409277 & -1.154160982264666 & 3 \\
3 & -1.444747612551160 & 1.578444747612551 & -0.133697135061392 & 2 \\
4 & 2.755798090040928 & -2.755798090040928 & -1.311050477489768 & 2
\end{array}
\]
The graph of the clamped cubic spline polynomial is given below.
\figbox{interpolation_B/spline_quest1}{7cm}
}

\solution{\SOL}{\ref{intBQ2}}{
The MATLAB code to generate the system $A \VEC{z} = \VEC{b}$ for the natural
cubic spline interpolation is given in Code~\ref{codeNCSI}.

The MATLAB code used to produce the figure below is given in
Code~\ref{codeQuest2}.

\figbox{interpolation_B/spline_quest2}{7cm}
}
  
\begin{code}[Natural Cubic Spline Interpolant - System] \label{codeNCSI}
This program computes the tridiagonal matrix $A$ and the right hand
side column vector $\VEC{b}$ associated to the natural cubic spline
interpolation.\\
\subI{Input} The nodes $x_i$ for $0 \leq i \leq n$ (x(i+1) in the code
below).\\
The values $f(x_i)$ for $0 \leq i \leq n$ (f(i+1) in the code
below).\\
\subI{Output} The lower diagonal $L$, the diagonal $D$ and the upper
diagonal $U$ of the tridiagonal matrix $A$.\\
The right hand side $\VEC{b}$ of $A\VEC{x} = \VEC{b}$.
\small
\begin{verbatim}
%   [L,D,U,b] = naturalsplinematrix(f,p)

function [L,D,U,b] = naturalsplinematrix(f,fp,p)
  N = length(p);
  L = repmat(NaN,1,N-3);
  U = repmat(NaN,1,N-3);
  D = repmat(NaN,1,N-2);
  b = repmat(NaN,1,N-2);

  dp = p(2)-p(1);
  if (dp == 0)
    return;
  end
  ratio = (f(2)-f(1))/dp;

  for n=1:N-2
    prevdp = dp;
    dp = p(n+2)-p(n+1);
    if (dp == 0)
      return;
    end
    prevratio = ratio;
    ratio = (f(n+2)-f(n+1))/dp;
    D(n) = 2*(dp+prevdp);
    if ( n < N - 2 )
        U(n) = dp;
        L(n) = dp;
    end
    b(n) = 6*(ratio - prevratio);
  end
end
\end{verbatim}
\end{code}

\begin{code}[Code for Question~\ref{intBQ2}] \label{codeQuest2}
\small
\begin{verbatim}
p = [ 0 , 1 , 3 , 4 , 5 , 5.5 ];
f = [ 2 , 3 , 3 , 2 , 2 , 1 ];
[L,D,U,b] = naturalsplinematrix(f,p)

z = tridmatrix(L,D,U,b);
z = [0,z,0]

x = 0:0.01:5.5;
[y,coeffs] = splinepoly(z,f,p,x);
coeffs
plot(x,y,'b')
grid on
xlabel('x')
ylabel('y')
\end{verbatim}
\end{code}


%%% Local Variables: 
%%% mode: latex
%%% TeX-master: "notes"
%%% End: 
