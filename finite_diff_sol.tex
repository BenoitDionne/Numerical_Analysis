\nonumsection{Chapter~\ref{FiniteDiffMeth} : Finite Difference Methods}

\solution{\SOL}{\ref{fdmQ1}}{
Consider a function $v:R_\Delta \to \RR$.  Let $v_{i,j} = v(x_i,t_j)$
for all $(x_i,t_j) \in R_\Delta$ and let\\
$\displaystyle \frac{1}{2}\left( f(x_i,y_j) + f(x_i,y_{j+1})\right)
= P_\Delta(v_{i,j},v_{i,j+1},v_{i+1,j},\ldots)$.

We have
\[
\frac{v_{i,j+1}- v_{i,j}}{\dtx{t}} - \frac{c^2}{2} \,
\left( \frac{v_{i+1,j} - 2v_{i,j} + v_{i-1,j}}{(\dtx{x})^2}
+ \frac{v_{i+1,j+1} - 2v_{i,j+1} + v_{i-1,j+1}}{(\dtx{x})^2} \right)
= \frac{1}{2}\left( f(x_i,y_j) + f(x_i,y_{j+1})\right)
\]
for $0<i<N$ and $0\leq j < M$.  Thus
\[
(1 + \alpha) v_{i,j+1} = (1-\alpha) v_{i,j}
+ \frac{\alpha}{2} \left( v_{i+1,j} + v_{i-1,j} \right)
+ \frac{\alpha}{2} \left( v_{i+1,j+1} + v_{i-1,j+1} \right)
+ \frac{1}{2}\left( f(x_i,y_j) + f(x_i,y_{j+1})\right) \dtx{t}
\]
for $0<i<N$ and $0\leq j < M$, where
$\displaystyle \alpha = \frac{c^2\dtx{t}}{(\dtx{x})^2}$.

Let $\displaystyle v_j = \max_{0<i<N} |v_{i,j}|$ and
$\displaystyle F = \max_{\substack{0<i<N\\0\leq j<N}} \frac{1}{2}
  |f(x_i,y_j) + f(x_i,y_{j+1})|$.
If $\alpha \leq 1$, we get
\begin{align*}
(1 + \alpha) |v_{i,j+1}| &\leq (1-\alpha) |v_{i,j}|
+ \frac{\alpha}{2} \left( |v_{i+1,j}| + |v_{i-1,j}| \right)
+ \frac{\alpha}{2} \left( |v_{i+1,j+1}| + |v_{i-1,j+1}| \right) \\
& \qquad\quad + \frac{1}{2}\left| f(x_i,y_j) + f(x_i,y_{j+1})\right| \dtx{t} \\
& \leq (1-\alpha) v_j + v_j + v_{j+1} + F \dtx{t}
\leq v_j + v_{j+1} + F \dtx{t}
\end{align*}
for $0<i<N$ and $0\leq j < M$.  Thus
\[
(1 + \alpha) |v_{j+1}| \leq v_j + v_{j+1}  + F \dtx{t}
\Rightarrow
v_{j+1} \leq v_j + F \dtx{t}
\]
for $0\leq j < M$.  By induction, we get
\[
  v_j \leq v_0 + (j \dtx{t}) F \leq v_0 + T F 
\]
for $0\leq j \leq M$.  Hence,
\[
  |v_{i,j}| \leq v_j \leq v_0 + T F
\]
for $0\leq j \leq M$ and $0<i<N$.  Since
$\displaystyle \frac{1}{2}\left( f(x_i,y_j) + f(x_i,y_{j+1})\right)
= P_\Delta(v_{i,j},v_{i,j+1},v_{i+1,j},\ldots)$
for $(i,j)$ such that $(x_i,t_j) \in R_\Delta^o$
and $B_\Delta(v_{i,j},v_{i,j+1},v_{i+1,j},\ldots) = v_{i,j}$
for $(i,j)$ such that $(x_i,t_j) \in \partial R_\Delta$.  We can
rewrite the previous inequality as
\begin{align*}
|v_{i,j}| &\leq \max_{0<i<N}|B_\Delta(v_{i,0},v_{i,1},v_{i+1,0},\ldots)|
+ T \max_{\substack{0<i<N\\0<j\leq M}}|P_\Delta(v_{i,j},v_{i,j+1},v_{i+1,j},\ldots)|
\\
&\leq \max_{\substack{(i,j)\ \text{such that}\\(x_i,t_j) \in \partial R_\Delta}}
|B_\Delta(v_{i,0},v_{i,1},v_{i+1,0},\ldots)|
+ T \max_{\substack{(i,j)\ \text{such that}\\(x_i,t_j) \in R_\Delta^o}}
|P_\Delta(v_{i,j},v_{i,j+1},v_{i+1,j},\ldots)|
\end{align*}
for $0\leq j \leq M$ and $0<i<N$.  Since
$B_\Delta(v_{i,j},v_{i,j+1},v_{i+1,j},\ldots) = v_{i,j}$ for $(i,j)$
such that $(x_i,t_j) \in \partial R_\Delta$, we get 
(\ref{stabCondFDM}) with $C = \max \{1, T\}$.
}

\solution{\SOL}{\ref{fdmQ2}}{
\subQ{a} Since $P P^\ast = P^\ast P$, we have that
\[
\ps{P^2 x}{P^2x} = \ps{Px}{P^\ast P P x} = \ps{Px}{P P^\ast Px}
= \ps{P^\ast P x}{P^\ast P x} \ .
\]
Thus
\[
\|P^2\| = \sup_{\|x|=1} \left\|P^2 x\right\|
= \sup_{\|x|=1} \sqrt{\ps{P^2 x}{P^2x}}  
= \sup_{\|x|=1} \sqrt{\ps{P^\ast P x}{P^\ast P x}}  
= \sup_{\|x|=1} \left\| P^\ast P x \right\| = \left\|P^\ast P\right\| \ .
\]

\subQ{b}  We first prove that
\begin{equation} \label{fdmQ2eq1}
  \left\|P^\ast P \right\| = \sup_{\|x|=\|y\|=1}\ps{x}{P^\ast P y} \ .
\end{equation}
Using Schwartz's inequality, we have
\[
\ps{x}{P^\ast P y} \leq \|x\|\, \left\|P^\ast P y \right\|
\leq \|x\| \, \left\|P^\ast P \right\| \, \|y\| = \left\|P^\ast P \right\|
\]
for all $x$ and $y$ such that $\|x|=\|y\|=1$.  Thus
\begin{equation} \label{fdmQ2eq2}
\sup_{\|x|=\|y\|=1}\ps{x}{P^\ast P y} \leq \left\| P^\ast P \right\| \ .
\end{equation}
Moreover
\begin{equation} \label{fdmQ2eq3}
\sup_{\|x|=\|y\|=1}\ps{x}{P^\ast P y}
\geq \sup_{\substack{\|y\|=1\\x = \left\|P^\ast P y \right\|^{-1} P^\ast P y}}
\ps{x}{P^\ast P y} = 
= \sup_{\|y\|=1} \left\|P^\ast P y \right\| = \left\|P^\ast P\right\| \ .
\end{equation}
because $\|x\|=1$ for
$x = \left\|P^\ast P y \right\|^{-1} P^\ast P y$.  Thus
(\ref{fdmQ2eq1}) follows from (\ref{fdmQ2eq2}) and (\ref{fdmQ2eq3}).

We have that
\[
  \left\|P^\ast P \right\| = \sup_{\|x|=\|y\|=1}\ps{x}{P^\ast P y}
= \sup_{\|x|=\|y\|=1}\ps{P x}{P y} \geq \sup_{\|y\|=1}\ps{P y}{P y}
= \|P\|^2 \ .
\]
We get from (a) that $\displaystyle \left\|P^2\right\| \geq \|P\|^2$.
But we already know that $\displaystyle \left\|P^2\right\| \leq \|P\|^2$
by a property of bounded linear operators.  Thus
$\displaystyle \left\|P^2\right\| = \|P\|^2$.
}

\solution{\SOL}{\ref{fdmQ3}}{
We use Proposition~\ref{MatrixStabProp} to answer this question.

As we have seen in Example~\ref{Moreell2StabEgg}, the finite difference
scheme given by Algorithm~\ref{fdm_sch1S} can be expressed as
$\VEC{w}_{j+1} = Q\VEC{w}_j + \VEC{B}_j$ for $j\geq 0$, where
$Q = -K$ for $K$ given in (\ref{fdm_K}) and
\[
\VEC{B}_j = \begin{pmatrix}
\alpha w_{0,j} + f(x_1,t_j) \dtx{t} \\
f(x_2,t_j) \dtx{t} \\
\vdots \\
f(x_{N-2},t_j) \dtx{t} \\
\alpha w_{N,j} + f(x_{N-1},t_j) \dtx{t}
\end{pmatrix} \ .
\]
The matrix $Q$ can be written as $Q = \Id + \alpha A$, where
\[
A = \begin{pmatrix}
-2 & 1 & 0 & 0 & 0 & \ldots & 0 & 0 \\
1 & -2 & 1 & 0 & 0 & \ldots & 0 & 0  \\
0 & 1 & -2 & 1 & 0 & \ldots & 0 & 0 \\
0 & 0 & 1 & -2 & 1 & \ldots & 0 & 0 \\
\vdots & \vdots & \vdots & \vdots & \vdots & \ddots & \vdots & \vdots \\
0 & 0 & 0 & 0 & 0 & \cdots & 1 & -2
\end{pmatrix}
\]
is a \nm{(N-1)}{(N-1)} matrix.  It follows from
Proposition~\ref{fdm_eigR} that the eigenvalues of $A$ are
\[
\lambda_i = -2 + 2 \cos\left(k\,\pi/N\right)
= -4 \sin^2\left(k \,\pi/(2N)\right) \quad , \quad
0 < k < N \ .
\]
Thus, the eigenvalues of $Q$ are $1+\alpha \lambda_k$ for $0 < k < N$.
To get eigenvalues that are smaller or equal to $1$ in absolute value,
we need to have $\left|1+\alpha \lambda_k \right| \leq 1$ for
$0 < k < N$; namely, we need to have
\[
-1 \leq 1 - 4 \alpha \sin^2\left(k\,\pi/(2N)\right) \leq 1
\quad , \quad 0 < k < N \ .
\]
The second inequality is always satisfied, so $\alpha$ must satisfy
\[
-1 \leq 1 - 4 \alpha \sin^2\left(k\,\pi/(2N)\right) \quad ,
\quad  0 < k < N  \ .
\]
This is equivalent to
\[
0 < \alpha \leq \frac{1}{2 \sin^2\left(k\,\pi/(2N)\right)}
\quad , \quad  0 < k < N  \ .
\]
However, 
\[
  \frac{1}{2 \sin^2\left(k\,\pi/(2N)\right)} > \frac{1}{2}
\]
for $0 < k < N$ and converges to $1/2$ for $k=N-1$ and $N \to \infty$.

Thus, the finite difference scheme is $\ell^2$-stable if $0 < \alpha \leq 1/2$.
}

%%% Local Variables:
%%% mode: latex
%%% TeX-master: "notes"
%%% End:
