\nonumchapter{Preface}

This book cover the material normally presented in a two-term course on
numerical analysis.  It starts with the basic concept normally presented
in a first course on numerical analysis.  It ends with topics that are
more appropriated for a first course in numerical analysis for
differential equations.

This book can be used by two different groups of students.  If the
focus is on the algorithms and the theory is ignored, then the book
can be used for an introduction to numerical analysis for engineering
and applied science students.  The book can also be used as an
introduction to numerical analysis for students in mathematics, or
students who plan to study more advanced topics in numerical analysis, if
the theory is covered.  We do not think that there is a need to
emphasize the importance of the theory in numerical analysis.  No
serious progress in numerical analysis is possible without it.
Most of the numerical methods presented in this book is accompanied by
a code in MATLAB.

The background for this book is a two-term course in linear algebra,
a course in real analysis (often called advanced calculus to make
the subject less scary), and a course in ordinary differential
equation for the last part of the book.

This book is divided into several parts.

After a brief introduction to the arithmetic on computers, the
{\em first part} on {\bfseries solving equations} is composed of
Chapter~\ref{chaptSeqA} on iterative methods to solve nonlinear
equations of one unknown variable, Chapter~\ref{chaptSeqB} on
iterative methods to solve systems of linear equations,
Chapter~\ref{chaptSeqC} on algebraic methods to solve systems of
linear equations, and Chapter~\ref{chaptSeqD} on iterative methods to
solve system of nonlinear equations.

The {\em second part} of the book on {\bfseries polynomial interpolation}
is composed of Chapter~\ref{chaptInterA} on polynomial interpolation of
real valued functions and Chapter~\ref{chaptInterB} on spline
interpolation; in particular, cubic splines and Bézier curves.

The {\em third part} on the {\bfseries approximation of functions} is
composed of three short chapters: Chapter~\ref{chaptApproxA} on
continuous least square approximation (i.e.\ in $L^2$), 
Chapter~\ref{chaptApproxB} on uniform approximation of real
valued functions and Chapter~\ref{ChaptApproxC} on discrete least square
approximation (i.e.\ in $\ell^2$).

The {\em fourth part} on finding {\bfseries eigenvalues} of matrices
is composed of only one chapter, Chapter~\ref{chapEigVal}, on
numerical methods to compute eigenvalues of \nn matrices.

The {\em last part} on {\bfseries differential equations} is
composed of Chapter~\ref{chaptDiffInt} on the numerical
differentiation and integration of real valued functions,
Chapter~\ref{chaptInitVal} on the numerical methods to solve initial
value problems for ordinary differential equations,
Chapter~\ref{chapBoundValProbl} on the numerical methods to solve
boundary value problems for ordinary differential equations, and
Chapter~\ref{FiniteDiffMeth} on finite difference methods to solve
partial differential equations.

There is no chapter on finite element methods to solve partial
differential equations.  This topic requires some knowledge of
functional analysis to be properly covered.  To keep the book
accessible the undergraduate students (as much as possible), no
knowledge of functional analysis is assumed. 

The second and third part of the book are related and even
intertwine in some cases.  There are many ways to approximate a
function $f:[a,b] \rightarrow \RR$ by polynomials.  The major
approaches are:

\begin{enumerate}
\item Given any small $\epsilon$, we could find a polynomial
$p_\epsilon$ such that
\[
\max_{a\leq x \leq b} \,|f(x) - p_\epsilon(x)| < \epsilon \ .
\]
Stone-Weierstrass Theorem, Theorem~\ref{SWtheorem}, states that
$p_\epsilon$ can always be found if $f$ is a continuous function on
$[a,b]$.  The function $f$ is
{\bfseries uniformly approximate}\index{Functions!Uniformly Approximate}
by the polynomial $p_\epsilon$.  This will
be studied in Chapter~\ref{chaptApproxB}.  This very short
chapter is more theoretical.  Nevertheless, it is important to
understand the limitations of polynomial interpolation and splines
presented in Chapters~\ref{chaptInterA} and
\ref{chaptInterB}.
\item Given any small $\epsilon$, we could find a polynomial
$p_\epsilon$ such that
\[
\int_a^b \,|f(x) - p_\epsilon(x)|^2 \dx{x} < \epsilon \ .
\]
The polynomial $p_\epsilon$ is a
{\bfseries quadratic approximation}\index{Functions!Quadratic Approximation}
of the function $f$.  This will be studied in Chapter~\ref{chaptApproxB}.
Again, this chapter is more theoretical but essential to understand
discrete least square approximation in Chapter~\ref{ChaptApproxC}.
This material is also fundamental in the study of numerical analysis;
in particular, to develop methods to solve partial differential
equations.
\item Given any small $\epsilon$ and
$a \leq x_0 < x_1 < x_2 < \ldots < x_n \leq b$, we could find a polynomial
$p_\epsilon$ such that
\[
\sum_{i=0}^n \,|f(x_i) - p_\epsilon(x_i)|^2 < \epsilon \ .
\]
This is a {\bfseries discrete least square
approximation}\index{Functions!Discrete Least Square Approximation}
of the data set
\[
\{(x_i,f(x_i)) : i=0,1,2,\ldots, n\}
\]
by a polynomial.  This will be the subject of Chapter~\ref{ChaptApproxC}.
\item We could also find a polynomial $p$ of degree at most
$n$ such that $p(x_i) = f(x_i)$ for $0 \leq i \leq n$.  This is the
subject of Chapter~\ref{chaptInterA}.
\item Instead of looking for a polynomial of degree $n$, where $n$ may
be large, we could find polynomials $p_i$ of small degrees (usually of
degree $3$) such that $p_i$ is an approximation of $f$ on the small
interval $[x_i,x_{i+1}]$.  The polynomials $p_i$ are determined from
conditions at the endpoints $x_i$ that provide some degree of
smoothness for the piecewise polynomial $p$ defined by $p(x) = p_i(x)$
for $x\in [x_i, x_{i+1}]$.  The polynomial $p$ is called a
{\bfseries spline}\index{Spline}.   We will present the
{\bfseries cubic splines}\index{Functions!Cubic Spline
Interpolation}, {\bfseries B\'ezier curves}\index{B\'ezier Curves} and
{\bfseries B-Splines}\index{Functions!B-Spline Interpolation}
in Sections~\ref{CSI}, \ref{BEZIER} and \ref{BSI} of
Chapter~\ref{chaptInterB}.  These piecewise polynomial approximations
are superior to the simple polynomial interpolation mentioned in the
previous item.  Cubic splines, B\'ezier curves, ... are used in some
of the major software for drawing.
\end{enumerate}

There is a strong emphasis in this book on differential equations.
This is only a reflect of the principal interest of the author.
Contrary to must introductory textbooks in numerical analysis, 
there is an extensive chapter, Chapter~\ref{chaptInitVal}, on the
numerical methods to solve initial value problems for ordinary
differential equations.  There is also a full chapter,
Chapter~\ref{chapBoundValProbl}, on the numerical methods to solve
boundary value problems for ordinary differential equations, and 
a full chapter, Chapter~\ref{FiniteDiffMeth}, on finite difference
methods.

There are many solved exercises at the end of several chapters.  Most
of the exercises are to reinforce the concepts presented in the text.
We have kept the number of theoretical questions to the minimum.
This was mainly motivated by the groups of students who took the
numerical analysis courses.  They were more interested in the
applications of numerical analysis than in the theory.  Sadly, there
are not real life applications of numerical analysis in this book.  It
would be nice (in the future) to add some realistic projects to
illustrate each topics.

The examples should be treated as problems to be solved by the reader.
The reader should try to answer each problem before looking at its
solution (if it is available).

In this book, we use the following notation for some standard
sets of numbers.

\begin{defn*}
The following well known sets are frequently used in this document.
\begin{itemize}
\item $\NN = \{0,1,2,3,\ldots \}$ is the set of natural numbers.
\item $\NNp = \{1,2,3,\ldots \}$ is the set of positive natural numbers.
\item $\ZZ = \{0,1,-1,2,-2,3,-3, \ldots \}$ is the set of integers.
\item $\QQ$ is the set of rational numbers.
\item $\RR$ is the set of real numbers.
\item $\CC$ is the set of complex numbers.
\end{itemize}
\end{defn*}

We will also often use the following definition when approximating
functions.

\begin{defn*} Let $f:\RR^n \rightarrow \RR$ and
$g:\RR^n \rightarrow \RR$ be two functions.  We write
$f\left(\VEC{x}\right) = O\left(g\left(\VEC{x}\right)\right)$ near
the origin if there exists a positive constant $K$ such that
\[
\left| f\left(\VEC{x}\right) \right| < K \left| g\left(\VEC{x}\right) \right|
\]
for $\VEC{x}$ in a neighbourhood of the origin.  We write
$f\left(\VEC{x}\right) = o\left(g\left(\VEC{x}\right)\right)$ near
the origin if
\[
\lim_{\VEC{x}\rightarrow \VEC{0}} \frac{f\left(\VEC{x}\right)}
{g\left(\VEC{x}\right)} = 0 \ .
\]
\end{defn*}

%%% Local Variables: 
%%% mode: latex
%%% TeX-master: "notes"
%%% End: 
