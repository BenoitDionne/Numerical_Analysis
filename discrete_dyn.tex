\section{Appendix}

This section is to illustrate how complex a simple
{\bfseries discrete dynamical system}\label{Discrete Dynamical System}
of the form
\[
x_{i+1} = f(x_i)  \; ,
\]
where $f:\RR\rightarrow \RR$ is a continuous function, can be.  The
complexity is even greater if $f:\RR^k \rightarrow \RR^k$ with $k>1$.
Discrete dynamical systems show up in many numerical algorithms.
For instance, the Newton's Method to find zeros of functions yields
discrete dynamical systems, some numerical methods to solve ordinary
differential equations or partial differential equations are
discrete dynamical systems, etc.  It is therefore important to
understand the behaviour of discrete dynamical systems, or at least to
be aware of the complex behaviour of these systems. 

A good introduction to the subject of this appendix is \cite{D}.  It
is also a good reference for the proofs of most of results stated in this
appendix.

\subsection{Elementary Concepts of Discrete Dynamical Systems}

\begin{defn}
Consider a continuous function $f:\RR \rightarrow \RR$ and $x \in \RR$.
The {\bfseries forward orbit}\index{Orbit!Forward} of $x$ is the set
\[
\OO_x^+ = \left\{ x, f(x), f^2(x) = f(f(x)), f^3(x)= f(f(f(x))), \ldots
\right\} \  .
\]
If $f$ has an inverse $f^{-1}:\RR\rightarrow \RR$, the
{\bfseries backward orbit}\index{Orbit!Backward} of $x$ is the set
\[
\OO_x^- = \left\{ x, f^{-1}(x), f^{-2}(x) = f^{-1}(f^{-1}(x)),
f^{-3}(x)= f^{-1}(f^{-1}(f^{-1}(x))), \ldots \right\}
\]
and the {\bfseries orbit}\index{Orbit} of $x$ is
$\OO_x = \OO_x^+ \cup \OO_x^-$.
\end{defn}

\begin{defn}
Consider a continuous function $f:\RR \rightarrow \RR$.  The point
$p \in \RR$ is a {\bfseries periodic point}\index{Periodic Point} of
$f$ if there exists a positive integer $n$ such that $f^n(p) = p$.  If
$n$ is the smallest positive integer such that $f^n(p)=p$, we say that
$p$ is of {\bfseries period}\index{Period} $n$.  When $n=1$, $p$ is a
{\bfseries fixed point}\index{Fixed Point}.
We denote by $\Per_n(f)$ the set of all periodic point of $f$ of
period $n$.  In particular, $\Fix(f) = \Per_1(f)$ is the set of fixed
points of $f$.  If $p$ is a periodic point of $f$, $\OO_p$ is a
{\bfseries periodic orbit}\index{Periodic Orbit}.
\end{defn}

\begin{egg}
Consider the {\bfseries logistic map}\index{Logistic Map}
\[
f_\mu(x) = \mu x(1-x)
\]
for $0\leq x \leq 1$.  For $0 \leq \mu \leq 4$, we have
$f_\mu:[0,1]\rightarrow [0,1]$.  Moreover, $f_\mu$ has two fixed
points: $p_0 = 0$ and
$\displaystyle p_{\mu} = \frac{\mu-1}{\mu}$ for $\mu > 0$.

For $\mu=3.4$, $p_{3.4} = 0.45195878844045\ldots$ is a periodic point of
$f_\mu$ of period $2$.  The orbit of period two is
\[
\left\{ 0.45195878844045, 0.84215476876273, 0.45195878844045,
0.84215476876273, \ldots \right\} \; ,
\]
where the values have been chopped
to $14$ digits after the decimal point.  This can be easily seen from the
{\bfseries staircase diagram}\index{Staircase Diagram} or
{\bfseries cobweb}\index{Cobweb} of $f_\mu$ shown in Figure~\ref{figCW}.

\mathF{discrete_dyn/logistic1}{7cm}{Cobweb}{Cobweb}{figCW}

Another way to illustrate the behaviour of $f_\mu$ is with the
{\bfseries phase portrait}\index{Phase Portrait} of $f_\mu$ shown in
Figure~\ref{figPP}.

\pdfF{discrete_dyn/phase_portrait}{Phase Portrait}{Phase Portrait}{figPP}

Finally, one can plot the histogram of $f_\mu$.  Namely, we divide
the interval $[0,1]$ into a large number of subintervals of equal
lengths and we compute the percentage of iterations that enter each
subinterval.  For $\mu=3.4$, the histogram with $200$ subintervals of
$[0,1]$ and $10,000$ iterations is given in Figure~\ref{figHG}.

\mathF{discrete_dyn/histogram1}{7cm}{Histogram for $f_{3.4}$}
{Histogram for $f_{3.4}$}{figHG}

For $\mu=3.5$, $p_{3.5} = 0.87499726360246\ldots$ is a periodic point of
$f_\mu$ of period $4$.  The orbit of period four is
\[
\left\{0.87499726360246, 0.38281968301732, 0.82694070659144,
0.50088421030722, \ldots \right\} \; ,
\]
where the values have been chopped
to $14$ digits after the decimal point (see Figure~\ref{figPF}).

\mathF{discrete_dyn/logistic2}{7cm}{Period Four}{Period Four}{figPF}
\label{EggLogMap1}
\end{egg}

\begin{defn}
Consider a continuous function $f:\RR \rightarrow \RR$.  The points
$p$ and $q$ in $\RR$ are
{\bfseries forward asymptotic}\index{Forward Asymptotic} if
\[
\lim_{j\rightarrow \infty} \left| f^j(p) - f^j(q) \right| = 0 \ .
\]
In particular, if $p \in \RR$ is a periodic point of period $n$, then a
point $q$ is
{\bfseries forward asymptotic to $p$}\index{Forward Asymptotic to a Point} if
\[
p = \lim_{j\rightarrow \infty} f^{jn}(q) \  .
\]
The set of all points forward asymptotic to $p$ is denoted by $W^s(p)$.
There are similar definitions for
{\bfseries backward asymptotic}\index{Backward Asymptotic}.
\end{defn}

\begin{defn}
Consider a continuous function $f:\RR \rightarrow \RR$.
A fixed point $p$ of $f$ is {\bfseries stable}\index{Stable} if, for any open
neighbourhood $U$ of $p$, there exists an open neighbourhood $V$ of $p$
such that $f^i(V) \subset U$ for all $i > 0$.  Fixed points that are
not stable are called {\bfseries unstable}\index{Unstable}.   A fixed
point $p$ of
$f$ is {\bfseries asymptotically stable}\index{Asymptotically Stable} if it is
stable and there exists an open neighbourhood $W$ of $p$ such
$\displaystyle \lim_{i\rightarrow \infty} f^{i}(x) = p$ for all
$x\in W$.

If $p$ is a period point of period $n$ for $f$, we say that the
periodic orbit $\OO_p$ is {\bfseries stable}\index{Stable} if $p$ is a stable
fixed point of $f^n$.  We say that the periodic orbit is
{\bfseries asymptotically stable}\index{Asymptotically Stable} if $p$
is an asymptotically stable fixed point of $f^n$.
\end{defn}

\begin{rmk}
The previous definition of stability and asymptotic stability for a
period orbit is independent of the point $p$ of the orbit used to
determine the stability or the asymptotic stability.

Suppose that $p$ is a periodic point of period $n>1$ for a continuously
invertible function $f:\RR\to \RR$.  Suppose that $p$ is stable
for $f^n$ and let $q=f^k(p)$ be another point on the orbit $\OO_p$.
If $U_q$ is an open neighbourhood of $q$, then $f^{-k}(U_q)$ is an
open neighbourhood of $p$.  Since $p$ is a stable fixed point for
$f^n$, there exists an open neighbourhood $V_p$ of $p$ such that
$f^{ni}(V_q) \subset f^{-k}(U_q)$ for all $i>0$. 
Hence $f^{ni}(f^k(V_p)) = f^{ni+k}(V_q) \subset U_q$ for all $i>0$,
where $f^k(V_p)$ is an open neighbourhood of $q$.  This proves that
$q$ is a stable fixed point of $f^n$.

Suppose furthermore that $p$ is an asymptotically stable fixed point
of $f^n$.  Then there exists an open neighbourhood $W_p$ of $p$ such
that $\displaystyle \lim_{i\to \infty}f^{ni}(x) = p$ for all $x \in W_p$.
Thus $\displaystyle \lim_{i\to \infty}f^{ni}(f^k(x)) = f^{k}(p) = q$
for all $x \in W_p$; namely,
$\displaystyle \lim_{i\to \infty}f^{ni}(y) = q$ for all
$y$ in the open neighbourhood $f^k(W_p)$ of $q$.  This proves that
$q$ is an asymptotically stable fixed point of $f^n$.
\end{rmk}

\subsection{Qualitative Study}

Consider the discrete dynamical system
\begin{equation} \label{DDS}
x_{i+1} = f(x_i) \; .
\end{equation}
For a qualitative study of this system, we would like to find all the
fixed points, periodic orbits, \ldots.  We would also like to find the
sets of points forward asymptotic to these objects.

We first study the fixed points of (\ref{DDS}).

\begin{defn}
A fixed point $p$ of $f$ is
{\bfseries hyperbolic}\index{Hyperbolic Fixed Point} if
$\left|f'(p)\right| \neq 1$.
\end{defn}

A proof similar to the proof of the Fixed Point Theorem yields the next
theorem.

\begin{prop}
Let $p$ be an hyperbolic fixed point of $f$.  If
$\left|f'(p)\right|<1$, then $p$ is asymptotically stable
However, if $\left|f'(p)\right|>1$, then there exists an open
interval $I$ containing $p$ such that for each $x$ in $I$,
$x \neq p$, one can find $j \in \NN$ such that $f^j(x) \not\in I$.
This implies that $f^j(x) \not\in I$ for infinitely many values of $j$.
\label{fp_stability}
\end{prop}

\begin{defn}
If $p$ is an hyperbolic fixed point of $f$ such that
$\left|f'(p)\right|<1$, then $p$ is called an
{\bfseries attracting}\index{Attracting} fixed point or a
{\bfseries sink}\index{Sink}.
If $p$ is an hyperbolic fixed point of $f$ such that
$\left|f'(p)\right|>1$, then $p$ is called a
{\bfseries repelling}\index{Repelling} fixed point or a
{\bfseries source}\index{source}.
\end{defn}

\begin{egg}
For the logistic map $f_\mu(x) = \mu x(1-x)$, the origin is a source
if $\mu>1$, and $\displaystyle p_\mu  = \frac{\mu-1}{\mu}$ is a sink
if $1<\mu<3$.  This follows from
$\displaystyle \pdfdx{f_\mu(x)}{x}\bigg|_{x=0} = \mu$ and
$\displaystyle \pdfdx{f_\mu(x)}{x}\bigg|_{x=p_\mu} = 2-\mu$.
\label{EggLogMap2}
\end{egg}

We can expand the notion of hyperbolicity to periodic points.

\begin{defn}
A periodic point $p$ of period $n$ for $f$ is
{\bfseries hyperbolic}\index{Hyperbolic Period Point}
if $\displaystyle \left| \dydx{f^n}{x}(p) \right| \neq 1$.
\end{defn}

A periodic point $p$ of $f$ of period $n$ is a fixed point of $f^n$.
The stability of the periodic point $p$ of $f$ is determined by the
stability of the fixed point $p$ of $f^n$.  Proposition
\ref{fp_stability} holds for a periodic point $p$ of period $n$ if
$f$ is replaced by $f^n$.

\begin{egg}
To study the stability of the periodic points of period $2$ for the
logistic map $f_\mu$ with $\mu = 3.4$, we consider the iterative system
$x_{i+1} = f_\mu^2(x_i) = f_\mu(f_\mu(x_i))$ for $i=0$, $1$, $2$, \ldots

\mathF{discrete_dyn/logistic3}{7cm}{Period Two}{Period Two}{figPT}

From the graph of $f_\mu^2$ given in Figure~\ref{figPT}, we see that
the periodic point\\ $0.45195878844045\ldots$ of period $2$ is a sink.
\label{EggLogMap3}
\end{egg}

\subsection{Bifurcation}

Consider a nice function $f:\RR^2 \rightarrow \RR$.  It defines a
one-parameter family of functions $f_\mu(x) = f(x,\mu)$.

We say that $\mu = \mu_0$ is a
{\bfseries bifurcation point}\index{Bifurcation Point} of the
discrete dynamical system
\[
x_{i+1} = f_\mu(x_i)
\]
if the qualitative behaviour of the phase portrait changes as $\mu$
goes through $\mu_0$.  For instance, the number of fixed points
change, new periodic solutions appear, etc.

There are three major results related to bifurcation.  The first
result is a simple consequence of the Implicit Function Theorem
applied to the function $g(x,\mu) = f(x,\mu) - x$.

\begin{theorem}
Let $f:\RR^2 \rightarrow \RR$ be a smooth function and let
$f_\mu(x)= f(x,\mu)$.  Suppose that $f_{\mu_0}(x_0) = x_0$ and
\[
\pdfdx{f_{\mu}(x)}{x} \bigg|_{\mu=\mu_0,x=x_0}
= \pdydx{f}{x}(x,\mu) \bigg|_{\mu=\mu_0,x=x_0} \neq 1  \ ,
\]
then there exist an open interval $I$ about $x_0$, an open interval
$J$ about $\mu_0$, and a mapping $p:J\rightarrow I$ such that
$p(\mu_0)=x_0$ and $f_\mu(p(\mu))=p(\mu)$ for all $\mu \in U$.
Moreover, $f_\mu$ has no other fixed point in $I$ (Figure~\ref{figIFT}).
\label{IFTforDS}
\end{theorem}

\pdfF{discrete_dyn/impl_funct_th}{The Implicit Function Theorem}{The Implicit
Function Theorem}{figIFT}

The next theorems describe two ``generic'' types of bifurcation.  We
have bifurcation only when
$\displaystyle \left| \pdfdx{f_{\mu}(x)}{x}\right| = 1$.  We use the
word generic because the other conditions to classify these types of
bifurcation require only that some derivatives be non-null.

\begin{theorem}[Saddle-node, tangent or fold bifurcation]
Let $f:\RR^2 \rightarrow \RR$ be a smooth function and let
$f_\mu(x)= f(x,\mu)$.  Suppose that
\begin{enumerate}
\item $\displaystyle f_{\mu_0}(0)=0$\ ,
\item $\displaystyle \pdfdx{f_{\mu}(x)}{x}\bigg|_{\mu=\mu_0,x=0} = 1$ \ ,
\item $\displaystyle \pdfdxn{f_{\mu}(x)}{x}{2}\bigg|_{\mu=\mu_0,x=0} \neq 0$
\quad and
\item $\displaystyle \pdfdx{f_\mu(x)}{\mu}\bigg|_{\mu=\mu_0,x=0} \neq 0$ \ .
\end{enumerate}
Then, there exist an interval $I$ about $0$ and a mapping
$q:I\rightarrow \RR$ such that $q(0) = \mu_0$ and $f_{q(x)}(x) = x$.  Moreover,
$q'(0) = 0$ and $q''(0) \neq 0$.
\end{theorem}

Figure~\ref{figSNBDM} illustrates a typical fold bifurcation.  The fixed
points represented by a dashed curve are sources while those
represented by a continuous curve are sinks.  The conditions in the
statement of the theorem above do not determine which branch of the
curve is associated to sources and which branch is associated to
sinks.  Moreover,
\[
q''(0) = \frac{\displaystyle -\pdfdxn{f_{\mu}(x)}{x}{2}\bigg|_{\mu=\mu_0,x=0}}
{\displaystyle \pdfdx{f_{\mu}(x)}{\mu}\bigg|_{\mu=\mu_0,x=0}}
\]
can be used to determine if the curve $\mu=p(x)$ is
{\bfseries supercritical}\label{Supercritical} (namely, $q''(x)>0$ as in
Figure~\ref{figSNBDM}) or {\bfseries subcritical}\index{Subcritical}
(namely, $q''(x)<0$).

\pdfF{discrete_dyn/saddle_node}{Fold bifurcation of discrete maps}
{A typical fold bifurcation diagram for a discrete map}{figSNBDM}

\begin{theorem}[Period doubling or flip bifurcation]
Let $f:\RR^2 \rightarrow \RR$ be a smooth function and let
$f_\mu(x)= f(x,\mu)$.  Suppose that
\begin{enumerate}
\item $f_{\mu}(0)=0$ for all $\mu$ near $\mu_0$,
\item $\displaystyle \pdfdx{f_{\mu}(x)}{x}\bigg|_{\mu=\mu_0,x=0} = -1$,
\item
$\displaystyle \frac{1}{2}
\left(\pdfdxn{f_{\mu}(x)}{x}{2}\bigg|_{\mu=\mu_0,x=0}\right)^2
+\frac{1}{3} \pdfdxn{f_{\mu}(x)}{x}{3}\bigg|_{\mu=\mu_0,x=0} \neq 0$
and
\item $\displaystyle \pdfdxnm{f_{\mu}^2(x)}{\mu}{x}{2}{}{}
\bigg|_{\mu=\mu_0,x=0} \not=0$,
\end{enumerate}
where $f^2_\mu(x) \equiv f(f(x,\mu),\mu)$.  Then, there exist
an interval $I$ about $0$ and a mapping $q:I\rightarrow \RR$ such that
$q(0) = \mu_0$ and $f_{q(x)}(x) \neq x$ but $f^2_{q(x)}(x) = x$.
\label{ThePDTforM}
\end{theorem}

Figure~\ref{figPDB} illustrates a typical period doubling bifurcation.
The fixed points are represented by the straight line $x=0$ and the periodic
points of period two are represented by the curve.   For $\mu$ fixed,
a periodic orbit of period two alternates between the lower and the
upper curve. The fixed points represented by a dashed curve are
sources while those represented by a continuous curve are sinks.   The
periodic points of period two represented by a dashed curve are
unstable while those represented by a continuous curve are
asymptotically stable.    Moreover,
\[
  q''(0) = \frac{\displaystyle
\left(\pdfdxn{f_{\mu}(x)}{x}{2}\bigg|_{\mu=\mu_0,x=0}\right)^2
+\frac{2}{3} \pdfdxn{f_{\mu}(x)}{x}{3}\bigg|_{\mu=\mu_0,x=0}}
{\displaystyle \pdfdxnm{f^2_{\mu}(x)}{x}{\mu}{2}{}{}\bigg|_{\mu=\mu_0,x=0}}
\]
can be used to determine if the curve $\mu=q(x)$ is
{\bfseries supercritical}\index{Supercritical} (namely, $q''(x)>0$ as in
Figure~\ref{figPDB}) or {\bfseries subcritical}\index{Subcritical}
(namely, $q''(x)<0$).

\pdfF{discrete_dyn/period_doubling}{Period doubling bifurcation for
discrete maps}{A typical period doubling bifurcation diagram for a
discrete map}{figPDB}

\begin{rmk}
In the previous theorem, the condition $f_{\mu}(0)=0$ for all $\mu$
near $\mu_0$ is not necessary.  Suppose that
\begin{enumerate}
\item $\displaystyle f_{\mu_0}(x_0) = x_0$\ ,
\item $\displaystyle \pdfdx{f_{\mu}(x)}{x}\bigg|_{\mu=\mu_0,x=x_0} = -1$\ ,
\item $\displaystyle \frac{1}{2}
\left(\pdfdxn{f_{\mu}(x)}{x}{2}\bigg|_{\mu=\mu_0,x=x_0}\right)^2
+\frac{1}{3} \pdfdxn{f_{\mu}(x)}{x}{3}\bigg|_{\mu=\mu_0,x=x_0} \neq 0$ \ 
and
\item $\displaystyle \pdfdxnm{f_{\mu}^2(x)}{\mu}{x}{2}{}{}
\bigg|_{\mu=\mu_0,x=x_0} \neq 0$ \ .
\end{enumerate}
From Theorem~\ref{IFTforDS}, there exists a function
$p$ defined in an open interval $J$ of $\nu_0$ such that
$p(\mu_0)=x_0$ and $p(\mu)$ is a fixed point of $f_\mu$ for all $\mu \in J$.
Let $\hat{f}(x,\mu) \equiv f(x+p(\mu),\mu)-p(\mu)$.  We show that
$\hat{f}$ satisfies the hypotheses of Theorem~\ref{ThePDTforM}.

We have $\hat{f}_\mu(0)=0$ for all $\mu$ near $\mu_0$.  Since
\[
  \pdydxn{\hat{f}}{x}{n}(x,\mu) = \pdydxn{f}{x}{n}(x+p(\mu),\mu)
\]
for $n=1$, $2$, \ldots, we have
\[
\pdfdx{\hat{f}_{\mu}(x)}{x}\bigg|_{\mu=\mu_0,x=0}
= \pdydx{\hat{f}}{x}(0,\mu_0)
= \pdydx{f}{x}(0+p(\mu_0),\mu_0)
= \pdydx{f}{x}(x_0,\mu_0)
= \pdfdx{f_{\mu}(x)}{x} \bigg|_{\mu=\mu_0,x=x_0} = -1 \ .
\]
Moreover,
\begin{align*}
&\frac{1}{2}
\left(\pdfdxn{\hat{f}_{\mu}(x)}{x}{2}\bigg|_{\mu=\mu_0,x=0}\right)^2
+\frac{1}{3} \pdfdxn{\hat{f}_{\mu}(x)}{x}{3}\bigg|_{\mu=\mu_0,x=0}
= 
\frac{1}{2} \left(\pdydxn{\hat{f}}{x}{2}(0,\mu_0) \right)^2
+\frac{1}{3} \pdydxn{\hat{f}}{x}{3}(0,\mu_0) \\
&\qquad = \frac{1}{2} \left(\pdydxn{f}{x}{2}(0+p(\mu_0),\mu_0) \right)^2
+\frac{1}{3} \pdydxn{f}{x}{3}(0+p(\mu_0),\mu_0)
= \frac{1}{2} \left(\pdydxn{f}{x}{2}(x_0,\mu_0) \right)^2
+\frac{1}{3} \pdydxn{f}{x}{3}(x_0,\mu_0) \\
&\qquad =\frac{1}{2}
\left(\pdfdxn{f_{\mu}(x)}{x}{2}\bigg|_{\mu=\mu_0,x=x_0}\right)^2
+\frac{1}{3} \pdfdxn{f_{\mu}(x)}{x}{3}\bigg|_{\mu=\mu_0,x=x_0} \neq 0 \ .
\end{align*}
Finally,
\begin{equation}\label{per_db_sc}
\pdfdxnm{\hat{f}_{\mu}^2(x)}{\mu}{x}{2}{}{}\bigg|_{\mu=\mu_0,x=0}
= \pdfdxnm{f_{\mu}^2(x)}{\mu}{x}{2}{}{}\bigg|_{\mu=\mu_0,x=x_0} \neq 1 \ .
\end{equation}
To prove the first equality requires a little bit of work.  From
\[
\hat{f}_{\mu}^2(x) = f(f(x+p(\mu),\mu),\mu)-p(\mu) \  ,
\]
we get
\[
\pdfdx{\hat{f}_{\mu}^2(x)}{x} = \pdfdx{f(f(x+p(\mu),\mu),\mu)}{x} 
= \pdydx{f}{x}(f(x+p(\mu),\mu),\mu) \pdydx{f}{x}(x+p(\mu),\mu)
\]
and
\begin{align*}
&\pdfdxnm{\hat{f}_{\mu}^2(x)}{\mu}{x}{2}{}{} =
\pdfdxnm{f(f(x+p(\mu),\mu),\mu)-p(\mu)}{\mu}{x}{2}{}{} \\
&\quad = \pdfdx{\left( \pdydx{f}{x}(f(x+p(\mu),\mu),\mu)
\pdydx{f}{x}(x+p(\mu),\mu) \right)}{\mu}  \\
&\quad = \pdydxn{f}{x}{2}(f(x+p(\mu),\mu),\mu)
\left( \pdydx{f}{x}(x+p(\mu),\mu) \dydx{p}{\mu} (\mu) \right. \\
&\qquad \left. + \pdydx{f}{\mu}(x+p(\mu),\mu) \right)
\pdydx{f}{x}(x+p(\mu),\mu)
+ \pdydxnm{f}{x}{\mu}{2}{}{}(f(x+p(\mu),\mu),\mu)
\pdydx{f}{x}(x+p(\mu),\mu)\\
&\qquad +\pdydx{f}{x}(f(x+p(\mu),\mu),\mu)
\left( \pdydxn{f}{x}{2}(x+p(\mu),\mu) \dydx{p}{\mu}(\mu)
+ \pdydxnm{f}{x}{\mu}{2}{}{}(x+p(\mu),\mu) \right) \ .
\end{align*}
Hence, using $p(\mu_0) = x_0$, $f(x_0,\mu_0)=x_0$ and
$\displaystyle \pdydx{f}{x}(x_0,\mu_0) =
\pdfdx{f_{\mu}(x)}{x}\bigg|_{\mu=\mu_0,x=x_0} = -1$, we get
\begin{align*}
&\pdfdxnm{\hat{f}_{\mu}^2(x)}{\mu}{x}{2}{}{} \bigg|_{\mu=\mu_0,x=0}
= \pdydxn{f}{x}{2}(f(x_0,\mu_0),\mu_0)
\left( \pdydx{f}{x}(x_0,\mu_0) \dydx{p}{\mu}(\mu_0)
+ \pdydx{f}{\mu}(x_0,\mu_0) \right)
\pdydx{f}{x}(x_0,\mu_0) \\
&\quad + \pdydxnm{f}{x}{\mu}{2}{}{}(f(x_0,\mu_0),\mu_0) \pdydx{f}{x}(x_0,\mu_0)
 \\
&\quad +\pdydx{f}{x}(f(x_0,\mu_0),\mu_0)
\left( \pdydxn{f}{x}{2}(x_0,\mu_0) \dydx{p}{\mu}(\mu_0)
+ \pdydxnm{f}{x}{\mu}{2}{}{}(x_0,\mu_0) \right) \\
&= -\pdydxn{f}{x}{2}(x_0,\mu_0) \left( - \dydx{p}{\mu}(\mu_0)
+ \pdydx{f}{\mu}(x_0,\mu_0) \right) - \pdydxnm{f}{x}{\mu}{2}{}{}(x_0,\mu_0) \\
&\quad - \left( \pdydxn{f}{x}{2}(x_0,\mu_0) \dydx{p}{\mu}(\mu_0) 
+ \pdydxnm{f}{x}{\mu}{2}{}{}(x_0,\mu_0) \right)
= - 2 \pdydxnm{f}{x}{\mu}{2}{}{}(x_0,\mu_0)
 - \pdydxn{f}{x}{2}(x_0,\mu_0)\pdydx{f}{\mu}(x_0,\mu_0) \ .
\end{align*}

Moreover,
\begin{align*}
\pdfdxnm{f_{\mu}^2(x)}{\mu}{x}{2}{}{}
&= \pdfdx{\left(\pdydx{f}{x}(f(x,\mu),\mu)
\pdydx{f}{x}(x,\mu) \right)}{\mu} \\
&= \pdydxn{f}{x}{2}(f(x,\mu),\mu) \pdydx{f}{\mu}(x,\mu)
\pdydx{f}{x}(x,\mu) + \pdydxnm{f}{x}{\mu}{2}{}{}(f(x,\mu),\mu)
\pdydx{f}{x}(x,\mu) \\
&\quad + \pdydx{f}{x}(f(x,\mu),\mu) \pdydxnm{f}{x}{\mu}{2}{}{}(x,\mu) \ .
\end{align*}
Hence, using $f(x_0,\mu_0)=x_0$ and
$\displaystyle \pdydx{f}{x}(x_0,\mu_0) =
\pdfdx{f_{\mu}(x)}{x} \bigg|_{\mu=\mu_0,x=x_0} = -1$, we get
\begin{align*}
\pdfdxnm{f_{\mu}^2(x) }{\mu}{x}{2}{}{} \bigg|_{\mu=\mu_0,x=x_0} &= 
\pdydxn{f}{x}{2}(f(x_0,\mu_0),\mu_0) \pdydx{f}{\mu}(x_0,\mu_0)
\pdydx{f}{x}(x_0,\mu_0) \\
& + \pdydxnm{f}{x}{\mu}{2}{}{}(f(x_0,\mu_0),\mu_0) \pdydx{f}{x}(x_0,\mu_0)
+\pdydx{f}{x}(f(x_0,\mu_0),\mu_0)
\pdydxnm{f}{x}{\mu}{2}{}{}(x_0,\mu_0) \\
&= -\pdydxn{f}{x}{2}(x_0,\mu_0) \pdydx{f}{\mu}(x_0,\mu_0)
- 2 \pdydxnm{f}{x}{\mu}{2}{}{}(x_0,\mu_0)
\end{align*}
and this proves the first equality of (\ref{per_db_sc}).
\end{rmk}

\subsection{Logistic Map}

This section is mainly about the logistic equation
\[
x_{i+1} = f_\mu(x_i) = \mu x_i (1-x_i) \; .
\]
We have already found the fixed points of $f_\mu$ with their stability
in Examples~\ref{EggLogMap1} and \ref{EggLogMap2}.  We have also
looked at period points of period $2$ for $f_\mu$ in
Example~\ref{EggLogMap3}.  To illustrate the complex behaviour that
discrete dynamical systems may have, we now present some results
about the logistic map $f_\mu$ without giving the proofs.  A good
reference is \cite{D}.  Most of the results about the logistic map
$f_\mu$ that we present in this section are true for mappings having a
graph ``similar'' to the graph of $f_\mu$ on $I=[0,1]$.

Figure~\ref{figBDLM} is the
{\bfseries bifurcation diagram}\index{Bifurcation Diagram} of
$f_\mu$.  Namely, for ``each'' value of $\mu$ we plot the fixed
points, periodic points, \ldots of $f_\mu$.

\mathF{discrete_dyn/route_to_chaos}{9cm}
{Bifurcation diagram for the logistic map}
{Bifurcation diagram for the logistic map}{figBDLM}

To produce this figure, we have chosen a large number
of equally spaced values of $\mu$.  For each of these values of $\mu$,
we have computed the first $200$ iterations of the orbit of $0.5$
under $f_\mu$ and plotted only the last $80$ or so iterations.  By
increasing the number of iterations to compute and plot, we could have
generated a more precise bifurcation diagram.

For $\mu<3$, all iterations converge to $p_\mu$.   When
$\mu$ crosses above $3$, the fixed point $p_\mu$ becomes a
source and there appears an attracting  periodic orbit of period $2$;
all orbits eventually ``bounce back and for'' between the two
points of the orbit.  We have period doubling at $\mu=3$.  This claim
can be rigorously prove by showing that all hypothesis of
Theorem~\ref{ThePDTforM} are satisfied.  Effectively, we have
\begin{enumerate}
\item $f_\mu(p_\mu) = p_\mu$ for all $\mu$ near $\mu=3$.
\item $\displaystyle \pdfdx{f_\mu(x)}{x} \bigg|_{\mu=3,x=p_3} = -1$.
\item
$\displaystyle \frac{1}{2}
\left(\pdfdxn{f_{\mu}(x)}{x}{2}\bigg|_{\mu=3,x=p_3}\right)^2
+\frac{1}{3} \pdfdxn{f_{\mu}(x)}{x}{3}\bigg|_{\mu=3,x=p_3}
= 2\mu^2 \bigg|_{\mu=3} = 18 \neq 0$
\item $\displaystyle \pdfdxnm{f_\mu^2(x) }{\mu}{x}{2}{}{}\big|_{\mu=3,x=p_3} =
\left( 2\mu - 6\mu^2 x+18\mu^2 x^2 - 4\mu x -
12\mu^2 x^3\right)\big|_{\mu=3,x=p_3}
= 2 \not=0$.
\end{enumerate}

When $\mu$ crosses above $3.236\ldots$, the periodic orbit of period $2$
becomes unstable and there appears an attracting periodic orbit of
period $4$.  For $\mu$ slightly larger, the periodic orbit of period
$4$ becomes unstable and there is a bifurcation from the attracting
periodic orbit of period $4$ to an attracting periodic orbit of period
$8$, And so on.  This is the best known example of a
{\bfseries period doubling cascade}\index{Period Doubling Cascade}.

For a constant value of $\mu$, if the attracting periodic orbit
$\OO \subset [0,1]$ of $f_\mu$ is of period $n$ with $n$ very large,
it is reasonable to expect that $\OO$ will ``almost cover'' some
segments of the interval $[0,1]$.  This explains the shaded area.
Figure~\ref{figLPO} illustrates the attracting periodic orbit for
$\mu=3.6$.  The corresponding histogram with 300 subintervals and 10
millions iterations is given in Figure~\ref{figHLPO}.

\mathF{discrete_dyn/logistic8}{7cm}{Cobweb of a periodic orbit with a large
period}{Cobweb of a periodic orbit of the logistic map for $\mu = 3.6$.
The period of this stable periodic orbit is very large.}{figLPO}

\mathF{discrete_dyn/histogram2}{7cm}{Histogram of a periodic orbit with a
large period}{Histogram of a periodic orbit of the logistic map for
$\mu = 3.6$.  The period of this stable periodic orbit is very large.}{figHLPO}

Period doubling accumulates to $\mu = 3.5699456\ldots$.  This number
is known as the {\bfseries Feigenbaum point}\index{Feigenbaum Point}.  Moreover, let $\mu_0=3$,
$\mu_1=3.236\ldots$, $\mu_2$, $\mu_3$, \ldots be the values of $\mu$
for which the logistic mapping undergoes period doubling and let
$d_j = \mu_{j+1}-\mu_j$.  It has been showed that
$\displaystyle \lim_{j\rightarrow \infty} \frac{d_j}{d_{j+1}} =
4.6692016091029\ldots$.  This number is called the
{\bfseries Feigenbaum constant}\index{Feigenbaum Constant}.
Feigenbaum discovered this number in
1975.  This constant is universal in the sense that it is the same
for a whole class of dynamical systems of the form
$x_{n+1} = g_\mu(x_n)$ where the graph of $g_\mu(x)$ looks like the
graph of $f_\mu(x)$.

Period doubling is far from being the must complex type of
bifurcation. To understand the complex behaviour of the orbits of
$f_\mu$, we need the following theorem.

\begin{theorem}[Sarkovskii]
Consider the order on the positive integers defined by
\begin{multline*}
3 \gg 5 \gg 7 \gg \ldots
\gg 3\times 2 \gg 5\times 2 \gg \ldots \\
\gg 3\times 2^2 \gg 5\times 2^2 \gg \ldots
\gg 3\times 2^3 \gg 5\times 2^3 \gg \ldots
2^4 \gg 2^3 \gg 2^2 \gg 2 \gg 1 \; .
\end{multline*}
Let $f:\RR \rightarrow \RR$ be a continuous function and $k$ be a
prime number.  If $f$ has a periodic point of period $k$, then $f$ has
a periodic point of period $m$ for all $m \ll k$.
\end{theorem}

\begin{egg}
For $\mu = 3.839\ldots$,
$\left\{0.14988539433432, 0.48917380192271, 0.95930024021836 \right\}$
is an attracting periodic orbit of period $3$ of $f_\mu$, where all
value have been chopped to $14$ digits after the decimal point
(Figure~\ref{figPTLC}).

\mathF{discrete_dyn/logistic5}{7cm}{Period Three}{Period Three}{figPTLC}

Hence, $f_{3.839\ldots}$ has periodic orbits of all possible periods.
All the periodic orbits, except the one of period $3$, are unstable
(in fact repelling).  So, unless the iteration starts with a point on
a repelling periodic orbit, the iterations will converge toward the
periodic orbit of period $3$.
\end{egg}

This is not the end of the story.  We now consider $f_\mu$ for
$\mu>4$.

Let
\[
A_n = \left\{ x \in [0,1] : f_\mu^j(x) \in [0,1] \text{ for }
0 \leq j \leq n \text{ and } f_\mu^j(x) \not\in [0,1] \text{ for }
j > n \right\} \; .
\]
It can be shown that $A_n$ consists of $2^n$ distinct open subintervals
of $[0,1]$.  We have that any iteration that starts with $x_0 \in A_0$
(i.e.\ such that $f_\mu(x_0)>1$) eventually converges to $-\infty$
(Figure~\ref{figA0}).  We also have that any iteration that starts
with $x_0 \in A_1$ (i.e. such that $f_\mu(x_0) \in A_0$) eventually
converges to $-\infty$ (Figure~\ref{figA1}).  And so on.

\mathF{discrete_dyn/logistic6}{7cm}{$A_0$ for $f_{4.3}$}{$0.5 \in A_0$ for
$f_{4.3}$}{figA0}

\mathF{discrete_dyn/logistic7}{7cm}{$A_0$ for $f_{4.3}$}{$0.1 \in A_1$ for
$f_{4.3}$}{figA1}

Consider
\[
\Delta_\mu = [0,1] \setminus \bigcup_{n=0}^\infty \,A_n \; .
\]
$\Delta_\mu$ contains all the points $x$ such that $f_\mu^j(x)$ stay
in $[0,1]$ for all $j\geq 0$.  In particular, $\Delta_\mu$ contains all
the periodic points.

Recall that a {\bfseries Cantor set}\index{Cantor Set} is a set that
is closed (contains
all its limit points), totally disconnected (does not contain any open
interval), and perfect (every point of the set is the limit of other
points of the set).

\begin{egg}
The best known example of a Cantor set is the Cantor Middle-Thirds
set.  It is also an example of a
{\bfseries Fractal}\index{Fractal} set because of its
self-similarity under zooming.
\end{egg}

\begin{theorem}
For $\mu > 4$, $\Delta_\mu$ is a cantor set.
\end{theorem}

\subsection{Chaos}

The next two definitions are the bases for the definition of chaos.

\begin{defn}
Let $I$ be an interval of $\RR$ and $f:I\rightarrow I$ be a
continuous function.  $f$ is
{\bfseries topologically transitive}\index{Topologically Transitive} if
for any open sets $V$ and $W$ in $I$ there exist $k>0$ such that
$f^k(V) \cap W \neq \emptyset$.
\end{defn}

\begin{defn}
Let $I$ be an interval of $\RR$ and $f:I\rightarrow I$ be a
continuous function.  $f$ has
{\bfseries sensitive dependence on initial conditions}\index{Sensitive
Dependence on Initial Conditions} if there exists
$\delta>0$ such that, for any $x \in I$ and neighbourhood $N \subset I$
of $x$, there exist $y\in N$ and $k>0$ satisfying
$\left| f^k(x) - f^k(y) \right| > \delta$.
\end{defn}

\begin{defn}[Chaos]
Let $I$ be an interval of $\RR$ and $f:I\rightarrow I$ be a
continuous function.  $f$ is said to be
{\bfseries chaotic}\index{Chaotic} on $I$ if
\begin{enumerate}
\item $f$ has sensitive dependence on initial conditions.
\item $f$ is topologically transitive.
\item The set of all periodic points of $f$ is dense in $I$ (every
  non-periodic point of $I$ is the limit of some periodic points).
\end{enumerate}
\label{chaos}
\end{defn}

\begin{rmk}
It has been proved in \cite{BBCDS} that 2 and 3 implies 1.
Nevertheless, we keep the tradition of using Definition~\ref{chaos} as
the definition of chaos because it lists three of the must important
properties of a chaotic function.  Moreover, it has been proved in
\cite{AG} that 1 and 3 do not imply 2, and 1 and 2 do not imply 3.
\end{rmk}

\begin{egg}
The logistic map $f_4:I \to I$, where $I=[0,1]$, is chaotic. 
\end{egg}

We may expand our definition of attracting and repelling periodic
orbits to more general sets.

\begin{defn}
Let $V$ be a subset of $\RR$ and $f:\RR\rightarrow \RR$ be a
continuous function.  $V$ is an
{\bfseries attracting}\index{Hyperbolic Set!Attracting}
(respectively {\bfseries repelling}\index{Hyperbolic Set!Repelling})
{\bfseries hyperbolic set} if
\begin{enumerate}
\item $V$ is closed and bounded.
\item $V$ is {\bfseries invariant}\index{Invariant Set} under $f$
(i.e.\ $f(V) \subset V$).
\item There exists $N>0$ such that
$\displaystyle \left| \dydx{f^n}{x}(x)  \right| < 1$
(respectively $>1$) for all $n \geq N$ and $x \in V$.
\end{enumerate}
\end{defn}

\begin{egg}
It can be proved that for $\mu>2+\sqrt{5}$, $\Delta_\mu$ is a repelling
hyperbolic set for the logistic map $f_\mu$.  The behaviour of
$f_\mu:\Delta_\mu \to \Delta_\mu$ is a lot more complex than we may
imagine.  $f_\mu$ has a dense orbit in $\Delta_\mu$.  Moreover,
$f_\mu:\Delta_\mu \to \Delta_\mu$ is choatic\footnote{The definition
of choatic map can be extended to any topological space $I$, not just
intervals.}.
\end{egg}

%%% Local Variables: 
%%% mode: latex
%%% TeX-master: "notes"
%%% End: 
