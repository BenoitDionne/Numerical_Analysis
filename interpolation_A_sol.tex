\nonumsection{Chapter~\ref{chaptInterA} : Polynomial Interpolation}

\solution{\SOL}{\ref{intAQ1}}{
Let
\[
q(x) = \sum_{j=0}^n \ell_j(x)
\]
for all $x$.  We first show that $q(x) = 1$ for all $x$.  The
polynomial $q(x)-1$ is of degree at most $n$ and has
$n+1$ distinct roots at $x_0$, $x_1$, \ldots, $x_n$ because
\[
\ell_j(x_i) = \begin{cases} 1 & \text{ if } i=j \\
0 & \text{ if } i\neq j
\end{cases}
\]
Since polynomial of degree $n$ cannot have more than $n$ roots,
$q(x)-1 =0$ for all $x$.

Hence
\[
\sum_{j=0}^n \left(f(x)-p(x_j)\right) \ell_j(x)
= f(x)\,\sum_{j=0}^n \ell_j(x) - \sum_{j=0}^n p(x_j)\, \ell_j(x)
= f(x) - p(x)
\]
because $\displaystyle p(x) = \sum_{j=0}^n p(x_j)\, \ell_j(x)$.
}

\solution{\SOL}{\ref{intAQ2}}{
Our theory of interpolation does not apply here because $p(\xi)$ is not
fixed.  We have to find the coefficients of the polynomial
$p(x)=a + b\,x+c\,x^2$ such that
$p(0) = a = \alpha$, $p(1) = a+b+c = \beta$ and
$p'(\xi) = b + 2c\,\xi = \gamma$.
The first equation gives the value for $a$ which is substituted into
the other equations to give
$b+c = \beta - \alpha$ and $b + 2c\,\xi = \gamma$.
If we subtract the first equation from the second equation, we get the system
$b+c = \beta - \alpha$ and $(2\xi-1)\,c = \gamma - \beta + \alpha$.

If $\xi \neq 1/2$, then this system has a unique solution
\[
c = \frac{\gamma-\beta+\alpha}{2\xi-1} \ , \quad
b = (\beta-\alpha) - \frac{\gamma-\beta+\alpha}{2\xi-1}
= \frac{-\gamma +2\xi\,\beta -2\xi\,\alpha}{2\xi -1} \quad \text{and} \quad
a = \alpha \ .
\]

If $\xi= 1/2$ and $\gamma = \beta - \alpha$, then $c$ is free,
$b= \beta -\alpha - c$ and $a = \alpha$.  There is an
infinite number of solutions, and thus an infinite number of
interpolating polynomial of degree at most $2$.  Note that
$p'(1/2) = b+c = \beta - \alpha = \gamma$.  Several interpolating
polynomial are drawn in the following figure.
\pdfbox{interpolation_A/interpol_quest3}
If $\xi = 1/2 $ and $\gamma \neq \beta - \alpha$, there is no
solution and thus no interpolating polynomial of degree at most
$2$.
}

\solution{\SOL}{\ref{intAQ3}}{
The polynomial $r$ is of degree at most $n$ because $p$ and $q$ are
polynomial of degree at most $n-1$.  Moreover,
$r(x_0) = p(x_0) = f(x_0)$, $r(x_n) = q(x_n) = f(x_n)$ and
\begin{align*}
  r(x_j) &= \frac{x_j-x_n}{x_0-x_n} \, p(x_j) +
\frac{x_j-x_0}{x_n-x_0}\, q(x_j)
= \frac{x_j-x_n}{x_0-x_n} \, f(x_j) +
\frac{x_j-x_0}{x_n-x_0}\, f(x_j) \\
&= \left( \frac{x_j-x_n}{x_0-x_n} + \frac{x_j-x_0}{x_n-x_0}\right)
f(x_j) = f(x_j)
\end{align*}
for $0<j<n$.  Hence, $r$ is the interpolating polynomial of degree at
most $n$ at $x_0$, $x_1$, \ldots, $x_n$ since this polynomial is
unique.
}

\solution{\SOL}{\ref{intAQ4}}{
The Lagrange's form of the polynomial $p$ is
\[
p(x) = \sum_{i=0}^n f(x_i) \prod_{\substack{j=0\\i\neq j}}^n \left(
\frac{x-x_j}{x_i-x_j} \right) \ .
\]
The coefficient of $x^n$ in
$\displaystyle f(x_i) \prod_{\substack{j=0\\i\neq j}}^n \left(
\frac{x-x_j}{x_i-x_j} \right)$ is
$\displaystyle f(x_i) \prod_{\substack{j=0\\i\neq j}}^n \left(
\frac{1}{x_i-x_j} \right) = f(x_i) \ell_i$.  The sum of these
coefficients gives the coefficient of $x^n$ in $p$.

If $f$ is a polynomial of degree less than $n$, the interpolating
polynomial $p$ of $f$ of degree at most $n$ at $x_0$, $x_1$, \ldots,
$x_n$ is $f$ itself by uniqueness.  The coefficient of $x^n$ in
$p = f$ is therefore $0$.  Since the coefficient of $x^n$ is
$\displaystyle \sum_{i=0}^n f(x_i) \ell_i$, we have that
$\displaystyle \sum_{i=0}^n f(x_i) \ell_i=0$.
}

\solution{\SOL}{\ref{intAQ5}}{
\subQ{a} For $k\in\{0,1,2,\ldots,n\}$,
\[
\prod_{\substack{j=0\\i\neq j}}^n (x_k-x_j) =
\underbrace{(x_k-x_0)(x_k-x_1)\ldots(x_k-x_{k-1})}_{k \text{ positive factors}}
\ \underbrace{(x_k-x_{k+1}) \ldots (x_k-x_n)}_{n-k \text{ negative factors}}
\ .
\]
Hence $\sgn(\ell_k) = (-1)^{n-k}$.

\subQ{b} To prove (\ref{sgnEQ1}), we consider $f(x) = x^n$.  Since the
interpolating polynomial $p$ of $f(x) = x^n$ of degree at most $n$ at
the points $x_0$, $x_1$, \ldots, $x_n$ is $f$ itself by uniqueness,
the coefficient of $x^n$ in $f(x) = p(x) = x^n$ is
$\displaystyle f[x_0,x_1,\ldots, x_n] = \sum_{j=0}^n x_i^n \ell_j = 1$
according to (\ref{sgnEQ0}).

To prove (\ref{sgnEQ2}), we consider $f(x) = 1$ for all $x$.  The
interpolating polynomial $p$ of $f$ of degree at most $n$ at the
points $x_0$, $x_1$, \ldots, $x_n$ is $p(x)=f(x)=1$ for all $x$ by
uniqueness. Hence, the coefficient of $x^n$ in $f(x) = p(x) = 1$ is
\[
f[x_0,x_1,\ldots, x_n] = \sum_{j=0}^n \ell_j
= \begin{cases} 1 & \quad \text{if} \quad n=0 \\
0 & \quad \text{if}\quad n>0 \end{cases}
\]
according to (\ref{sgnEQ0}).
}

\solution{\SOL}{\ref{intAQ6}}{
The proof is by induction.  (\ref{binewt}) is true when $m=0$ because
$\displaystyle
\frac{1}{0!} \sum_{j=0}^0 (-1)^{0-j} \binom{0}{j} f(j) = f(0)$ and
$f[0] = f(0)$.

We assume that (\ref{binewt}) is true for $m=k$ and show that it is
then true for $m=k+1$.
 
Let $g(n) = f(n+1)$ for all $n$.  We claim that
$f[n,n+1,\ldots,n+r]=g[n-1,n,\ldots,n+r-1]$ for all $n$ and $r \geq 0$.
The proof of this claim is by induction on $r$.  For $r=0$, we have
$f[n] = f(n)$ and $g[n-1]= g(n-1) = f(n)$ for all $n$.  Thus
$f[n]=g[n-1]$ for all $n$.
Suppose that $f[n,n+1,\ldots,n+r]=g[n-1,n,\ldots,n+r-1]$ is true for
all $n$ and some $r \geq 0$.  Then
\begin{align*}
f[n,n+1, \ldots, n+r+1] &=
\frac{f[n+1,n+2,\ldots,n+r+1] - f[n,n+1,\ldots,n+r]}{(n+r+1)-n} \\
&= \frac{g[n,n+1,\ldots,n+r] - g[n-1,n,\ldots,n+r-1]}{(n+r)-(n-1)} \\
&= g[n-1,n,\ldots,n+r]
\end{align*}
for all $n$, where the hypothesis of induction has been used for the second
equality.  Hence, $f[n,n+1,\ldots,n+r]=g[n-1,n,\ldots,n+r-1]$ is true
for all $n$ and $r$ replaced by $r+1$.

We now go back to our main proof by induction.   We have
\begin{align*}
f[0,1,2,\ldots,k+1] &= \frac{f[1,2,\ldots,k+1]-f[0,1,\ldots,k]}{k+1}
= \frac{g[0,1,\ldots,k]-f[0,1,\ldots,k]}{k+1} \\
&= \frac{1}{k+1} \left(
\frac{1}{k!}\sum_{j=0}^k (-1)^{k-j} \binom{k}{j} g(j)
- \frac{1}{k!}\sum_{j=0}^k (-1)^{k-j} \binom{k}{j} f(j) \right) \\
&= \frac{1}{k+1} \left(
\frac{1}{k!}\sum_{j=0}^k (-1)^{k-j} \binom{k}{j} f(j+1)
- \frac{1}{k!}\sum_{j=0}^k (-1)^{k-j} \binom{k}{j} f(j) \right)
\end{align*}
The third equality is a consequence of the hypothesis of induction;
namely, (\ref{binewt}) with $m=k$ for both $f$ and $g$.

If we replace $j$ by $j-1$ in the first sum, we get
\begin{align*}
f[0,1,2,\ldots,k+1] &= \frac{1}{k+1} \left(
\frac{1}{k!}\sum_{j=1}^{k+1} (-1)^{k-j+1} \binom{k}{j-1} f(j)
- \frac{1}{k!}\sum_{j=0}^k (-1)^{k-j} \binom{k}{j} f(j) \right) \\
&= \frac{1}{(k+1)!} \sum_{j=0}^{k+1} (-1)^{k-j+1} \left(
  \binom{k}{j-1} + \binom{k}{j}\right) f(j)
\end{align*}
where we have made used of the fact that
$\displaystyle \binom{k}{-1}=0$ and $\displaystyle \binom{k}{k+1} =0$.
Using the hint, we get
\[
f[0,1,2,\ldots,k+1] =
\frac{1}{(k+1)!} \sum_{j=0}^{k+1} (-1)^{k-j+1} \binom{k+1}{j} f(j)
\]
which is (\ref{binewt}) with $m=k+1$.  This prove that (\ref{binewt})
is true for all $m$ by induction.
}

\solution{\SOL}{\ref{intAQ7}}{
The table of divided differences is
{\small
\[
\begin{array}{cccc}
x_i & f[\cdot] & f[\cdot,\cdot] & f[\cdot,\cdot,\cdot] \\
\hline
1 & \multicolumn{1}{|c}{1} & \multicolumn{1}{|c}{-0.951625820} &
\multicolumn{1}{|c|}{0.438432783} \\ 
\cline{2-4}
1.1 & 0.904837418 & -0.820095985 & 0.383714117 \\
1.3 & 0.740818221 & -0.704981750 & 0.324799000 \\
1.4 & 0.670320046 & -0.607542050 & 0.275321725 \\
1.6 & 0.548811636 & -0.497413360 &  \\
1.8 & 0.449328964 &  &  \\
& & & \\
f[\cdot,\cdot,\cdot,\cdot] & f[\cdot,\cdot,\cdot,\cdot,\cdot] &
f[\cdot,\cdot,\cdot,\cdot,\cdot,\cdot,\cdot] \\
\cline{1-3}
\multicolumn{1}{|c}{-0.136796667} &
\multicolumn{1}{|c}{0.0316107222} &
\multicolumn{1}{|c|}{-0.00580682540} & \\
\cline{1-3}
-0.1178302333 & 0.0269652619 & & \\
-0.0989545500 & & &
\end{array}
\]
}
The interpolating polynomial is
\begin{align*}
p(x) &\approx 1 -0.951625820\,(x-1) + 0.438432783\,(x-1)(x-1.1) \\
&\quad -0.136796667\,(x-1)(x-1.1)(x-1.3) \\
&\quad +0.0316107222\,(x-1)(x-1.1)(x-1.3)(x-1.4) \\
&\quad -0.00580682540\,(x-1)(x-1.1)(x-1.3)(x-1.4)(x-1.6) \ ,
\end{align*}
where we have rounded the coefficients to $9$ digits.
The nested form of this polynomial is
\begin{align*}
p(x) &\approx 1 + (x-1)\big( -0.951625820 + (x-1.1) \big( 0.438432783
+(x-1.3) \big(-0.136796667 \\
& \quad +(x-1.4) \big(0.0316107222 -0.00580682540\,(x-1.6) \big)\big)\big)\big)
\ .
\end{align*}
Hence,
\begin{align*}
f(1.35) & \approx p(1.35)
\approx
1 + 0.35 \big( -0.951625820 + 0.25 \big( 0.438432783
+ 0.05 \big(-0.136796667 \\
& \quad -0.05 \big(0.0316107222 - 0.00580682540\times (-0.25)
\big)\big)\big)\big) \approx 0.704688114 \ .
\end{align*}
}

\solution{\SOL}{\ref{intAQ8}}{
\subQ{a} We have $f(x) = e^{x/2}$, $f'(x) = e^{x/2}/2$ and
$f''(x)=e^{x/2}/4$.  The table of divided differences is
{
\small
\[
\begin{array}{ccccc}
x & \quad f[\cdot] \quad & f[\cdot, \cdot] & f[\cdot,\cdot,\cdot] &
f[\cdot,\cdot,\cdot,\cdot] \\
\hline
0 & \multicolumn{1}{|c}{1} & \multicolumn{1}{|c}{(e-1)/2} &
\multicolumn{1}{|c}{1/4} & \multicolumn{1}{|c|}{(e-2)/16} \\
\cline{2-5}
2 & e & e/2 & e/8 & \\
2 & e & e/2 & & \\
2 & e & & & \\
\end{array}
\]
}
The interpolating polynomial is
\begin{align*}
p(x) &= 1 + \left(\frac{e-1}{2}\right)x + \frac{1}{4}\,x(x-2) +
\left(\frac{e-2}{16}\right)x(x-2)^2 \\
&= 1 + x\left( \frac{e-1}{2} + (x-2) \left(\frac{1}{4}
+ \left(\frac{e-2}{16}\right)(x-2) \right)\right) \\
&\approx 1 + x\left(0.8591409142 + (x-2) \left(0.25
+ 0.04489261428\,(x-2) \right)\right) \ .
\end{align*}

\subQ{b}
\[
f(1) \approx p(1) = 1 + \left(0.8591409142 - \left(0.25
- 0.04489261428 \right)\right) \approx 1.65403352848 \ .
\]

\subQ{c}
It follows from Theorems~\ref{InterpTh} and \ref{InterpProp} that, for
each $x\in[0,2]$, there exists $\xi = \xi(x) \in [0,2]$ such that
\[
| f(x) - p(x) | = \left| \frac{1}{4!} f^{(4)}(\xi)\, x (x-2)^3\right| \ .
\]
Moreover
\[
|f^{(4)}(x)| = \left| \frac{e^{x/2}}{2^4} \right|
\leq \frac{e}{2^4}
\]
for all $x\in[0,2]$.  Hence,
\[
| f(x) - p(x) | \leq \frac{e}{2^4\,4!} \left| x (x-2)^3 \right|
\leq \frac{e}{4!}
\approx 0.113261743 \ .
\]

We have use the upper bound $2^4$ for $|x(x-2)^3|$.  A better
(i.e.\ smaller) bound could be found by maximizing
$g(x) = |x(x-2)^3| = x(2-x)^3$ on the interval $[0,2]$.
Since $g'(x) = (2-x)^2(2-4x)=0$, the critical points of $g$ on $[0,2]$
are $x=1/2$ and $x=2$.  It follows from the Extremum Theorem,
Theorem~\ref{Th2}, that the maximum of $g$ on $[0,2]$ is the maximum
of $g(0) = 0$, $g(2)=0$ and $g(1/2) = 3^3/2^4$. 
Hence,
\[
| f(x) - p(x) | \leq \frac{e}{2^4\,4!} \left| x (x-2)^3 \right|
\leq \frac{3^3\,e}{2^8\,4!}
\approx 0.01194557444 \ .
\]
It is a better upper bound than the previous one.  However, we do not
usually maximize the polynomial part of the truncation error as we
have just done because the roots of its derivative may be too
difficult to find if the degree of this polynomial is high.
}

\solution{\SOL}{\ref{intAQ9}}{
\subQ{a}
The table of divided differences is
{
\small
\[
\begin{array}{cccccc}
x_i & f[\cdot] & f[\cdot,\cdot] & f[\cdot,\cdot,\cdot] &
f[\cdot,\cdot,\cdot,\cdot] & f[\cdot,\cdot,\cdot,\cdot,\cdot] \\
\hline
\rule{0em}{1.1em} 0 & \multicolumn{1}{|c}{e} & \multicolumn{1}{|c}{1-e} &
\multicolumn{1}{|c}{e-2} & \multicolumn{1}{|c}{5/2 - e} &
\multicolumn{1}{|c|}{(e+e^{-1}-3)/2} \\
\cline{2-6}
\rule{0em}{1.1em} 1 & 1 & -1 & 1/2 & e^{-1} -1/2  & \\
1 & 1 & -1 & e^{-1} & & \\
1 & 1 & e^{-1}-1 & & & \\
2 & e^{-1} & & & &
\end{array}
\]
}
The interpolating polynomial is
\begin{align*}
&p(x) = e + (1-e) x + (e-2) x(x-1) + \left(\frac{5}{2}-e\right) x(x-1)^2
+ \left( \frac{e+e^{-1}-3}{2} \right) x(x-1)^3 \\
&\quad = e + x\bigg( (1-e) + (x-1) \bigg( (e-2) + (x-1)\bigg(
\left(\frac{5}{2}-e\right)  + \left( \frac{e+1/e -3}{2} \right) (x-1)
\bigg)\bigg)\bigg) \ .
\end{align*}

\subQ{b}
\[
f(1.1) \approx
e + 1.1 \bigg( (1-e) + 0.1 \bigg( (e-2) + 0.1 \bigg(
\left(\frac{5}{2}-e\right)  + 0.1 \left( \frac{e+1/e -3}{2} \right)
\bigg)\bigg)\bigg)
\approx 0.9048291 \ .
\]

\subQ{c} It follows from Theorems~\ref{InterpTh} and \ref{InterpProp}
that, for each $x \in [0,2]$, there exists $\xi = \xi(x) \in [0,2]$
such that
\[
| f(x) - p(x) | = \left| \frac{f^{(5)}(\xi)}{5!}\,  x(x-1)^3 (x-2) \right| \ .
\]
Hence
\[
| f(x) - p(x) | \leq \frac{e}{5!}\, |x(x-1)^3 (x-2)|
\leq \frac{e}{30}
\]
for $0\leq x \leq 2$.  We have use the conservative upper bound
$|x(x-1)^3 (x-2)| \leq 4$ for $0\leq x \leq 2$ provided by
$|x|\leq 2$, $|(x-1)^3|\leq 1$ and $|x-2|\leq 2$ for
$0 \leq x \leq 2$.
}

\solution{\SOL}{\ref{intAQ10}}{
The table of divided differences is
{
\small
\[
\begin{array}{cccc}
x_i & f[\cdot] & f[\cdot,\cdot] & f[\cdot,\cdot,\cdot] \\
\hline
1 & \multicolumn{1}{|c}{1.7165256995} &
\multicolumn{1}{|c}{-1.4444065708} &
\multicolumn{1}{|c|}{0.14399171305} \\
\cline{2-4}
1 & 1.7165256995 & -1.4444065708 & 0.36837390975 \\
1 & 1.7165256995 & -1.1497074430 & 0.44217113417 \\
1.8 & 0.79675974510 & -0.53066785517 & 0.34592323664 \\
2.4 & 0.47835903200 & -0.32311391318 &  \\
2.4 & 0.47835903200 &  &  \\
& & & \\
f[\cdot,\cdot,\cdot,\cdot] &
f[\cdot,\cdot,\cdot,\cdot,\cdot] &
f[\cdot,\cdot,\cdot,\cdot,\cdot,\cdot,\cdot] \\
\cline{1-3}
\multicolumn{1}{|c}{0.28047774588} &
\multicolumn{1}{|c}{-0.16268960195} &
\multicolumn{1}{|c|}{0.054237061909} & \\
\cline{1-3}
0.052712303155 & -0.086757715275 & & \\
-0.06874849823 & & & \\
\end{array}
\]
}
The interpolating polynomial is
\begin{align*}
p(x) &\approx 1.7165256995 -1.4444065708\,(x-1) +
0.14399171305\,(x-1)^2 \\
&\quad +0.28047774588\,(x-1)^3  - 0.16268960195\,(x-1)^3(x-1.8) \\
&\quad +0.054237061909\,(x-1)^3(x-1.8)(x-2.4) \ ,
\end{align*}
where we have rounded the coefficients to $11$ digits.  The nested
form of this polynomial is
\begin{align*}
p(x) &\approx 1.7165256995 +(x-1)\big( -1.4444065708 + (x-1)
\big( 14399171305 \\
& + (x-1)\big(0.28047774588 +
(x-1.8)\big(-0.16268960195 + 0.054237061909\,(x-2.4) \big)\big)\big)\big) \ .
\end{align*}
Hence,
\begin{align*}
f(1.75) & \approx p(1.75)
\approx
1.7165256995 + 0.75 \big( -1.4444065708 + 0.75
\big( 0.14399171305 \\
&\quad + 0.75 \big(0.28047774588 -0.05\big(-0.16268960195
+0.054237061909\,(-0.65) \big)\big)\big)\big) \\
&\approx 0.83671803379 \ .
\end{align*}

The absolute error is
\[
|f(1.75)-p(1.75)| \approx |0.83673651441075 - 0.83671803379|
\approx 0.0000184806 \ .
\]
The relative error is
\[
\frac{|f(1.75)-p(1.75)|}{|f(1.75)|}
\approx \frac{|0.83673651441075 - 0.8367180338|}{0.83673651441075}
\approx 0.0000220865 \ .
\]
Since $k=5$ is the largest positive integer such that
$0.0000220865 < 5 \times 10^{-k}$, there are $5$ significant digits.
}

\solution{\SOL}{\ref{intAQ12}}{
If the divided differences of order three are always egal to $1$, then
the divided differences of order fourth and higher are $0$.  Thus $p$
is a polynomial of degree $3$.

A table of divided differences of $p$ at $0$, $1$, $2$ and $3$ (any
other value greater than $2$ would have been fine) will have the
following table.
{
\small
\[
\begin{array}{ccccc}
x_i & p[x_i] & p[x_i,x_j] & p[x_i,x_j,x_k]& p[x_i,x_j,x_k,x_l] \\
\hline
0 & \multicolumn{1}{|c}{2} & \multicolumn{1}{|c}{-1} &
\multicolumn{1}{|c}{2} & \multicolumn{1}{|c|}{1} \\ 
\cline{2-5}
1 & 1 & 3 & c_1 & \\
2 & 4 & c_2 & & \\
3 & c_3 & & &
\end{array}
\]
}
where $c_1$, $c_2$, and $c_3$ are constants.  Since $p$ is a
polynomial of degree $3$, we have
\[
p(x) = 2 - x + 2x(x-1) + x(x-1)(x-2) = 2 -x -x^2 + x^3 \ .
\]
}

\solution{\SOL}{\ref{intAQ13}}{
\subQ{a} 
We have
$\displaystyle f(x) = \cos\left(\frac{\pi}{2}-x\right)$,
$\displaystyle f'(x) = \sin\left(\frac{\pi}{2}-x\right)$ and
$\displaystyle f''(x) = -\cos\left(\frac{\pi}{2}-x\right)$.

The table of divided differences is
{
\small
\[
\begin{array}{cccccc}
x_i & f[\cdot] & f[\cdot,\cdot] & f[\cdot,\cdot,\cdot] &
f[\cdot,\cdot,\cdot,\cdot] & f[\cdot,\cdot,\cdot,\cdot,\cdot] \\ 
\hline
0 & \multicolumn{1}{|r}{0} & \multicolumn{1}{|r}{0.90031632} & \multicolumn{1}{|r}{-0.24600202} & \multicolumn{1}{|r}{-0.13693866} & \multicolumn{1}{|r|}{0.02886361} \\
\cline{2-6}
0.78539816 & 0.70710678 & 0.70710678 & -0.35355339 & -0.09159981 &  \\ 
0.78539816 & 0.70710678 & 0.70710678 & -0.42549571 &  &  \\ 
0.78539816 & 0.70710678 & 0.37292323 &  &  &  \\ 
1.57079633 & 1.00000000 &  &  &  &
\end{array}
\]
}
To save some space, we have only printed the numbers to $8$ decimal
places in the table above.  However, computations were done with full
Matlab accuracy.

The interpolating polynomial is
\begin{align*}
p(x) &\approx 0.900316316157 \,x  -0.2460020203444\,x(x-\pi/4) \\
&\qquad  -0.136938657691 \,x(x-\pi/4)^2 + 0.0288636058864\,x(x-\pi/4)^3 \ ,
\end{align*}
where we have rounded the coefficients to $12$ digits.

\subQ{b} The nested form of this polynomial is
\begin{align*}
p(x) &\approx \bigg( 0.900316316157 + \bigg(-0.2460020203444 + \\
& \qquad \bigg(-0.136938657691 + 0.0288636058864\,
\left(x-\frac{\pi}{4}\right)\bigg)
\left(x-\frac{\pi}{4}\right)\bigg)\left(x-\frac{\pi}{4}\right)\bigg) x \ .
\end{align*}
Hence,
\begin{align*}
f(\pi/8) & \approx p(1.75) \approx
\bigg( 0.900316316157 + \bigg(-0.2460020203444 + \\
& \qquad \bigg(-0.136938657691 + 0.0288636058864
\,\left(\frac{-\pi}{8}\right)\bigg)
\left(\frac{-\pi}{8}\right)\bigg)\left(\frac{-\pi}{8}\right)\bigg)
\left(\frac{\pi}{8}\right) \\
&\approx 0.382510687216 \ .
\end{align*}

\subQ{c}
It follows from Theorems~\ref{InterpTh} and \ref{InterpProp}
that, for each $x \in [0,\pi/2]$, there exists $\xi = \xi(x) \in [0,\pi/2]$
such that
\[
| f(x) - p(x) | = \left| \frac{1}{5!} f^{(5)}(\xi)
\, x \left(x-\frac{\pi}{4}\right)^3\left(x-\frac{\pi}{2}\right)\right| \ .
\]
However
\[
|f^{(5)}(x)| = \left| \sin\left( \frac{\pi}{2} - x\right)\right| \leq 1
\]
for all $x$.  Hence,
\[
| f(x) - p(x) | \leq \frac{1}{5!}\, \left|x \left(x-\frac{\pi}{4}\right)^3
\left(x-\frac{\pi}{2}\right) \right|
\leq \frac{1}{5!} \left(\frac{\pi}{4}\right)^5
\approx 0.0025
\]
because $\displaystyle \left|x (x-\pi/2) \right| \leq (\pi/4)^2$
(the maximum is reached at $x = \pi/4$) and
$\left|x-\pi/4\right|\leq \pi/4$ for $x \in [0,\pi/2]$.

\subQ{d} The following figure contains the graph of $p$ in blue and
the graph of $f$ in red.  The graph of $p$ was drawn first and is
almost completely covered by the graph of $f$.  The two graphs are
basically indistinguishable at the level of the graph accuracy.
\figbox{interpolation_A/interpol_quest4}{7cm}
}

%%% Local Variables: 
%%% mode: latex
%%% TeX-master: "notes"
%%% End:
