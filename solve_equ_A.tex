\chapter{Iterative Methods to Solve Nonlinear Equations of One
Variable}\label{chaptSeqA}

The classical problem is to find the solutions of the equation
\begin{equation} \label{equ1}
f(x)  = 0 \ ,
\end{equation}
where $f:\RR \rightarrow \RR$ is a given function.  Namely, the goal
is to find the numbers $p$ such that $f(p) = 0$.  The numbers $p$ are called
the {\bfseries roots}\index{Functions!Root} or
{\bfseries zeros}\index{Functions!Zero} of $f$.

\section{Real Analysis Background}

We present some of the well know results in real analysis that will be
used to justify the numerical methods presented in this book.

\begin{theorem}
If $\{x_n\}_{n=0}^\infty$ is a bounded and increasing
sequence of $\RR$, then it converges to
$M = \sup \{x_n : n \geq 0 \} \in \RR$.
\label{Th0}
\end{theorem}

\begin{theorem}[Intermediate Value Theorem]
Let $a<b$ be two real numbers and $f:[a,b] \rightarrow \RR$ be a
continuous function.  If $\alpha$ is between $f(a)$ and
$f(b)$ ($\alpha$ may be $f(a)$ or $f(b)$\,), then there exists $c$
between $a$ and $b$ ($c$ may be $a$ or $b$) such that $f(c)=\alpha$.
\label{Th1}
\end{theorem}

\begin{cor}
Let $a<b$ be two real numbers and $f:[a,b] \rightarrow \RR$ be a
continuous function.  If $f(a)\,f(b)<0$, then there exists a zero of
$f$ in the interval $]a,b[$.
\label{Cor1}
\end{cor}

\begin{proof}
Since $f(a)$ and $f(b)$ are of opposite sign, $0$ is between $f(a)$
and $f(b)$.  By the previous theorem with $\alpha = 0$, there exists
$c$ between $a$ and $b$ such that $f(c) = 0$.  We have $c \neq a$ and
$c \neq b$ because $f(a) \neq 0$ and $f(b) \neq 0$.
\end{proof}

\begin{theorem}[Extremum Theorem]
Let $a<b$ be two real numbers and $f:[a,b] \rightarrow \RR$ be a
continuous function.  Then there exist $x_s$ and $x_i$ in $[a,b]$ such
that
\[
f(x_i) \leq f(x) \leq f(x_s)
\]
for all $x \in [a,b]$.
\label{Th2}
\end{theorem}

\begin{theorem}[Mean Value Theorem] \label{Th3}
Let $a<b$ be two real numbers and $f:[a,b] \rightarrow \RR$ be a
continuous function.  Suppose that $f$ is differentiable on $]a,b[$.
Then there exists $c$ between $a$ and $b$ such that
\[
f'(c)=\frac{f(b)-f(a)}{b-a} \ .
\]
\end{theorem}

\begin{theorem}[Taylor's Theorem]
Let $a<b$ be two real numbers.  Suppose that $f:[a,b] \rightarrow \RR$
is a $n$-time continuously differentiable function on $[a,b]$, that
$f^{(n+1)}(x)$ exists for all $x \in ]a,b[$, and that
$c \in ]a,b[$.  Then, for every $x \in [a,b]$, there exists $\xi(x,c)$
between $x$ and $c$ such that
\[
f(x) = p_n(x) + r_n(x) \ ,
\]
where
\begin{align*}
p_n(x) &= f(c) + f'(c)(x-c) + \frac{f'(c)}{2!}(x-c)^2 + \ldots
+ \frac{f^{(n)}(c)}{n!} \, (x-c)^n\\
\intertext{and}
r_n(x) &= \frac{f^{(n+1)}(\xi(x,c))}{(n+1)!}(x-c)^{n+1} \ .
\end{align*}
\label{TaylorTheo}
\end{theorem}

\section{Bisection Method}

The idea is to construct a sequence of nested intervals
$\displaystyle \left\{ [a_n,b_n] \right\}_{n=0}^\infty$ of decreasing
length such that the sign of a function $f$ at $a_n$ is different than
its sign at $b_n$.  Thus, $f$ must have a root at some point in the
interval $[a_n,b_n]$ according to Corollary~\ref{Cor1}.

\begin{algo}[Bisection]
Suppose that $f$ is continue on $[a,b]$ and $f(a)\,f(b)<0$.
\begin{enumerate}
\item Choose $a_0=a$ and $b_0=b$.
\item Stop if $f(a_0)f(b_0) = 0$ because one of $a_0$ or $b_0$ is a
root of $f$. \label{BisectAlgo2}
\item Given $a_n$ and $b_n$ such that $f(a_n)\,f(b_n)<0$,
let $\displaystyle x_{n+1}=\frac{a_n+b_n}{2}$.
\item Stop if $f(x_{n+1})=0$ since $p=x_{n+1}$ is a root of $f$.
\label{BisectAlgo4}
\item If $f(x_{n+1})\,f(a_n) < 0$, set $a_{n+1}=a_n$ and
$b_{n+1}=x_{n+1}$.  If $f(x_{n+1})\,f(a_n) > 0$, set $a_{n+1}=x_{n+1}$
and $b_{n+1}=b_n$.
\item Repeat (3), (4) and (5) until the interruption criteria are satisfied
(more on the interruption criteria later).
\end{enumerate}
\label{BisectAlgo}
\end{algo}

\begin{prop}
In the algorithm for the bisection method, $b_n - a_n = (b-a)/2^n$.
\label{bisecProp1}
\end{prop}

\begin{proof}
We prove by induction that the interval $[a_n,b_n]$ is of length
$(b-a)/2^n$.

We have $b_0 - a_0 = b - a = (b-a)/2^0$.  Hence, the interval
$[a_0,b_0]$ is of length $(b-a)/2^0$.

Suppose that the interval $[a_n,b_n]$ is of length $(b-a)/2^n$;
namely, $b_n-a_n = (b-a)/2^n$.  Since $[a_{n+1},b_{n+1}]$ is
half the length of $[a_n,b_n]$, we have
\[
b_{n+1} - a_{n+1} = (b_n-a_n)/2 = (b-a)/2^{n+1} \ ,
\]
where we have used the hypothesis of induction for the second
equality.  Hence, the interval $[a_{n+1},b_{n+1}]$ is of length
$(b-a)/2^{n+1}$.

By induction, we then have that $[a_n,b_n]$ is of length $(b-a)/2^n$
for all $n\geq 0$.
\end{proof}

\begin{cor}
In the algorithm for the bisection method,, the approximation $x_{n}$
is within $(b-a)/2^n$ of a root $r$ of $f$ in the interval $[a,b]$.
\label{bisectCor}
\end{cor}

\begin{proof}
Since $f$ change sign in the interval $[a_{n-1},b_{n-1}]$, there is a
root $r$ of $f$ in the interval $[a_{n-1},b_{n-1}]$.
Since the approximation $x_n$ of $r$ is the middle point of the
interval $[a_{n-1},b_{n-1}]$, the absolute error $|x_n - r |$ 
satisfies $|x_n - r | < (b_{n-1} - a_{n-1})/2 = (b-a)/2^n$ according
to Proposition~\ref{bisecProp1}.
\end{proof}

\begin{prop}
In the algorithm for the bisection method,
\[
\lim_{n\rightarrow \infty} a_n = \lim_{n\rightarrow \infty} b_n
= \lim_{n\rightarrow \infty} x_n
\]
and this limit is a root of $f$.
\end{prop}

\begin{proof}
Since $a_0 \leq a_1 \leq a_2 \leq \ldots \leq b$, the sequence
$\{a_n\}_{n=0}^\infty$ is an increasing and bounded sequence.  It
follows from Theorem~\ref{Th0} that $\{a_n\}_{n=0}^\infty$ converges.
Let $\alpha$ be this limit.

Similarly, since $b_0 \geq b_1 \geq b_2 \geq \ldots \geq a$, the sequence
$\{b_n\}_{n=0}^\infty$ is a decreasing and bounded sequence.  Thus
$\{b_n\}_{n=0}^\infty$ converges.  Let $\beta$ be this limit.

Moreover,
\[
\alpha - \beta = \lim_{n\rightarrow \infty} a_n
-\lim_{n\rightarrow \infty} b_n
= \lim_{n\rightarrow \infty} (a_n-b_n) = 0
\]
by Proposition~\ref{bisecProp1}.

Since $a_n \leq x_{n+1} \leq b_n$ for all $n$, we have by the sandwich
theorem that $\{x_n\}_{n=0}^\infty$ also converge to
$\alpha = \beta$.

Finally, since $f(a_n)f(b_n)\leq 0$ for all $n$, we have
\[
\left( f(\alpha) \right)^2 = f(\alpha)f(\alpha)
= \lim_{n\rightarrow \infty} f(a_n)f(b_n) \leq 0 \ .
\]
Hence $f(\alpha) = 0$ and $\alpha$ is a root of $f$.
\end{proof}

\begin{egg}
Find an approximation of $\sqrt{2}$ using the bisection method.  Stop
when the length of the interval is less than $10^{-2}$.  Find a bound
on the absolute error.

The question is to find the positive root of $f(x) = x^2 -2 = 0$.  Let
$a_0=1$ and $b_0=2$.  Since $f(1) = -1 < 0 < 2 = f(2)$, there is a
root of $f$ in the interval $[1,2]$.  If
\[
x_{n+1}=\frac{a_n+b_n}{2} \ ,
\]
we get
\[
\begin{array}{c|c|c|c|c|c|c}
\hline
 n & x_n & a_n & b_n & |b_n-a_n|  & f(x_n)  &  f(a_{n-1})  \\
\hline
0&  &1 &2 & 1.0 & & \\
1& 1.500000 & 1 & 1.500000 & .500000 & + & - \\
2& 1.250000 & 1.250000 & 1.500000 & .250000 & - & - \\
3& 1.375000 & 1.375000 & 1.500000 & .125000 & - & - \\
4& 1.437500 & 1.375000 & 1.437500 & .062500 & + & - \\
5& 1.406250 & 1.406250 & 1.437500 & .031250 & - & - \\
6& 1.421875 & 1.406250 & 1.421875 & .015625 & + & - \\
7& 1.4140625 & 1.4140625 & 1.421875 & .0078125 & - & - \\
8& 1.4179688 & & & & & \\
\hline
\end{array}
\]
The answer is $\sqrt{2} \approx 1.4179688$.  There is a root in the
interval $[1.4140625,1.421875]$.  So
$(1.421875 - 1.4140625)/2 = 1/2^8 = 0.00390625$ is an upper-bound on the
absolute error.
\label{ExSqrt2}
\end{egg}

\begin{egg}
Using the formula provided by the bisection method, determine the
smallest number of iterations in the previous example to get an
absolute error lest than $10^{-4}$\ ?

We choose $n$ such that $|b_n- a_n| = (b-a)/2^n < 10^{-4}$.  This is
$2^{-n}< 10^{-4}$.  Thus,
\[
  \ln(2^{-n})< \ln(10^{-4}) \Rightarrow 
-n\,\ln(2) < -4\,\ln(10) \Rightarrow n>4\,\ln(10)/\ln(2) \approx
13.2877
\]
and $14$ iterations will be sufficient.
\end{egg}

\section{Interruption criteria}

There are three interruption criteria that are usually used in the
implementation of iteration methods:
\begin{enumerate}
\item  Stop after $N$ iterations ($N$ is given).
\item  Stop when $|x_{n+1}-x_n|<\epsilon$ ($\epsilon$ is given).
\item  Stop when $|f(x_n)|<\eta$ ($\eta$ is given).
\end{enumerate}

We give below an implementation of the bisection method in Matlab,
where we make use of the criteria 1 and 3.

\begin{code}[Bisection]
To approximate the zeros of a function $f$.\\
\subI{Input} The function $f$ (funct in the code below).\\
The endpoints $a$ and $b$ of the interval on which $f$ changes sign.\\
The error tolerance (tol in the code below).\\
\subI{Output} The approximation x to a root of $f$.
\small
\begin{verbatim}
% x = bisection(funct,a,b,tol)

function x = bisection(funct,a,b,tol)
  fa = feval(funct,a);
  fb = feval(funct,b);
  x = NaN;

  if ( a >= b )
    disp(['a must be smaller than b.'])
    return;
  end

  % We compute the theoritical number of iterations needed to reach
  % the accuracy requested.  This also prevent any infinite loops.
  %
  % From  (b-a)/2^n < tol  we get
  N = ceil(log2((b-a)/tol));

  % We replace fa*fb > 0 by a simple comparison of the signs of these
  % values.  We avoid a multiplication.
  if ( sign(fa) == sign(fb) )
    disp(sprintf('The bisection algorithm cannot be used because f(%f)
= %f and f(%f) = %f have the same sign.',a,b,fa,fb));
    return;
  end

  p = b - a;
  % We stop at i = N-1 because x_N is computed at i = N-1.
  for i=1:N-1
    p = p/2;

    % Instead of using the formula  (a+b)/2  to compute the middle
    % point, we simply add p to a.
    x = a + p;
    fx = feval(funct,x);

    % The test fx == 0 is not reliable because it is extremely rare
    % that the numerical evaluation of a function will give exactly 0.
    % We replace this test by  abs(fx) < 2*realmin , where realmin is the
    % smallest number that the computer may handle.
    if ( abs(fx) <= 2*realmin )
      return;
    end
       
    % We replace fa*fx < 0 by a simple comparison of the signs of these
    % values.  We avoid a multiplication.
    % We also store the value fx of f at the midpoint x into fa if
    % a takes the value x or into fb if b takes the value x.
    % This eliminates the need to compute f again at x.
    if ( sign(fx) ~= sign(fa) )
      b = x;
      fb = fx;
    else
      a = x;
      fa = fx;
    end
  end
end
\end{verbatim}
\end{code}

\section{Fixed Point Method}

To find a root of $f$, we rewrite (\ref{equ1}) as
\begin{equation} \label{equ2}
x  =  g(x) \ ,
\end{equation}
where $g:\RR \rightarrow \RR$.

Given $x_0$, we hope that the sequence $x_0$, $x_1$, \dots defined by
\begin{equation} \label{FPMformula1}
x_{n+1} = g(x_n) \quad \text{for} \quad n=0,1,2,\dots
\end{equation}
will converge to a
{\bfseries fixed point}\index{Functions!Fixed Point} $p$ of $g$;
namely, a point $p$ such that $g(p) = p$.

We say that (\ref{equ1}) and (\ref{equ2}) are
{\bfseries equivalent}\index{Equations!Equivalent}
(on a given interval) if a root of $f$ is a fixed point of $g$ and
vice-versa.  The problem is to choose $g$ and $x_0$ adequately.

\begin{egg}\label{fixit}
\[
f(x) = x^3+9\,x-9 = 0
\]
is equivalent to
\[
g(x) = (9-x^3)/9 = x \ .
\]
\end{egg}

\begin{theorem}[Fixed Point Theorem]
Let $g$ be a real valued function satisfying the following conditions.
\begin{enumerate}
\item $g(x)\in [a,b]$ for all $x\in[a,b]$.
\item There exists a number $K$ such that $0<K<1$
and $|g(x)-g(y)| \leq K |x-y|$ for all $x,y\in [a,b]$.
\end{enumerate}
Then $g$ has a unique fixed point $p \in [a,b]$ and, given
$x_0 \in [a,b]$, the sequence defined by (\ref{FPMformula1}) converges to
$p$ as $n$ goes to $\infty$.  Moreover,
\begin{align}
|x_n - p | & \leq K^n \, \max\{x_0 - a,b- x_0\} \label{ineq1} \\
\intertext{and}
|x_n - p | & \leq \frac{K^n}{1-K} \, |x_1 - x_0 | \ . \label{ineq2}
\end{align}
\label{FxPtTh}
\end{theorem}

\begin{proof}
We begin by proving the existence and uniqueness of the fixed point.
Note that the second hypothesis of the theorem implies that $g$ is a
continuous function on $[a,b]$.

Since $g(a) \geq a$ and $g(b) \leq b$, the function $h(x) = g(x) -x$
is a continuous function on $[a,b]$ such that $h(b) \leq 0 \leq h(a)$.
By the Intermediate Value Theorem, there exists $p \in [a,b]$ such
that $h(p) = 0$; namely, $g(p) = p$.

Suppose that $p_1$ and $p_2$ are two distinct fixed points of $g$ in
$[a,b]$.  We have
\[
|p_1 -p_2| = |g(p_1) - g(p_2)| \leq K|p_1 - p_2| < |p_1 - p_2| \ .
\]
This is a contradiction.

\noindent We now prove (\ref{ineq1}) and (\ref{ineq2}).

Let $p$ be the unique fixed point of $g$ in $[a,b]$ and let $x_0$ be a
point in $[a,b]$.  Since $g:[a,b] \rightarrow [a,b]$, the sequence
$\{x_n\}_{n=0}^\infty$ defined by $x_{n+1} = g(x_n)$ for $n \geq 0$ is a
well defined sequence in $[a,b]$.  Hence,
\[
\begin{split}
|x_n -p| &= |g(x_{n-1}) - g(p)| \leq K |x_{n-1} - p|
= K |g(x_{n-2}) - g(p)| \leq K^2 |x_{n-2} -p| \\
&= \ldots \leq K^n |x_0 - p| \rightarrow 0
\end{split}
\]
as $n \rightarrow \infty$ because $0<K<1$.
Moreover, since $|x_0 -p| \leq \max\{x_0 -a, b-x_0\}$, we get
$|x_n -p| \leq K^n \max\{x_0 -a,b - x_0\}$.  This prove (\ref{ineq1}).

To prove (\ref{ineq2}), we write
\[
\begin{split}
|x_{n+1} -x_n| &= |g(x_n) - g(x_{n-1})| \leq K |x_n - x_{n-1}|
= K |g(x_{n-1}) - g(x_{n-2})| \leq K^2 |x_{n-1} -x_{n-2}|\\
&= \ldots \leq K^n |x_1 - x_0| \ .
\end{split}
\]
Hence, for $m > n$,
\[
\begin{split}
|x_m - x_n| &= |x_m - x_{m-1} + x_{m-1} - x_{m-2} + \ldots - x_{n+1} +
x_{n+1} - x_n | \\
&\leq |x_m - x_{m-1}| + |x_{m-1} - x_{m-2}| + \ldots + |x_{n+1} - x_n| \\
&\leq (K^{m-1} + K^{m-2} + \ldots + K^n)|x_1 - x_0|\\
&= K^n(K^{m-n-1} + K^{m-n-2} + \ldots + K + 1)|x_1 - x_0| \ .
\end{split}
\]
If we let $m$ goes to infinity, we get
\[
|p - x_n| \leq K^n\left( \sum_{i=0}^\infty \, K^i\right)|x_1 - x_0| =
\frac{K^n}{1-K} |x_1 - x_0| \ .
\]
The series in the previous expression is the geometric series which
converges because $|K|<1$.
\end{proof}

\begin{defn}
A continuous function $g:[a,b]\rightarrow \RR$ for which there exists
$0<K<1$ satisfying $|g(x)-g(y)|\leq K |x-y|$ for all $x, y, \in [a,b]$
is called a {\bfseries contraction}\index{Functions!Contraction} on
$[a,b]$.
\end{defn}

\begin{rmk}
In Theorem~\ref{FxPtTh}, the second hypothesis is that
$g:[a,b]\rightarrow [a,b]$ is a contraction.

If $g$ in Theorem~\ref{FxPtTh} is differentiable and there exists
$0<K<1$ such that $|g'(x)|\leq K$ for all $x\in [a,b]$, then the
second hypothesis is satisfied.  This is a consequence of the Mean
Value Theorem.  For every $x, y \in [a,b]$, there exists $\eta$ between
$x$ and $y$ such that
\[
|g(x)-g(y)| = |g'(\eta)|\, |x-y| \leq K |x-y|
\]
because $\eta \in [a,b]$.
\label{FxPtThDer}
\end{rmk}

\begin{egg}
Find an approximation to a root of $f(x)=x^3+9\,x-9$.

Because $f(0)\,f(1)=-9<0$, the function $f$ has a root between $0$ and
$1$.  In Example~\ref{fixit}. we saw that $f(x)=x^3+9\,x-9=0$ is
equivalent to $g(x)=(9-x^3)/9 =x$.  Thus, the problem is to
approximate a fixed point of $g$ in $[0,1]$.

We show that $g$ on the interval $[0,1]$ satisfies the hypotheses of
the Fixed Point Theorem,  Because $g'(x) = -x^2/3 < 0$ for all $x>0$,
the function $g$ is decreasing on $[0,1]$.  Hence,
$8/9 =g(1) \leq g(x) \leq g(0) = 1$ for all $x \in [0,1]$.  We have
shown that $g:[0,1]\rightarrow [0,1]$ and thus the first hypothesis of
the Fixed Point Theorem is satisfied with $[a,b]=[0,1]$.  As mentioned
in Remark~\ref{FxPtThDer}, the second hypothesis of the Fixed Point
Theorem is satisfied with $K=1/3$ because $|g'(x)|=|-x^2/3|\leq1/3$ for
all $x$ in $[0,1]$.

All conditions of the Fixed Point Theorem are satisfied.  So, we
may use it to approximate a fixed point $p$ of $g$.  The following table
gives the first five iterations of $x_{n+1} = g(x_n)$ with
$x_0=0.5$.  The absolute and relative errors have been computed using
the exact value of the fixed point $p$; namely,
$p=0.91490784153366\ldots$.

\[
\begin{array}{c|c|c|c|c}
\hline
n & x_n & |x_n-p| & |x_n-p|/|p| & \text{number of} \\
 & & & & \text{significant digits} \\
\hline
0 & 0.5000000000 & 0.4149078415 & 0.4534968690 & 1 \\
1 & 0.9861111111 & 0.0712032696 & 0.0778256195 & 1 \\
2 & 0.8934545158 & 0.0214533257 & 0.0234486194 & 2 \\
3 & 0.9207544589 & 0.0058466174 & 0.0063903894 & 2 \\
4 & 0.9132660785 & 0.0016417630 & 0.0017944573 & 3 \\
5 & 0.9153651027 & 0.0004572612 & 0.0004997894 & 4 \\
\hline
\end{array}
\]
\end{egg}

\begin{egg}
Suppose that we want to approximate a root of $f(x) = x^3+4\,x^2-10$.
The function $f$ has a root in $[1,2]$ (show it).  The four functions
\begin{align*}
g_1(x) &= 10+x-4 x^2-{x^3} \ , \quad
g_2(x) = \sqrt{\frac{10}{x}-4 x} \ , \quad
g_3(x) = \frac{1}{2} {\sqrt{10-x^3}}
\intertext{and}
g_4(x) &= x-\frac{-10+4 x^2+x^3}{8 x+3 x^2}
\end{align*}
are equivalent to $f(x) = 0$ on the interval $[1,2]$.  We apply the
fixed point method (without checking if the conditions of
the Fixed Point Theorem are satisfied) to each function $g_i$ with
$x_0 = 1.5$.
\[
\begin{array}{c|c|c|c|c}
\hline
n & x_n=g_1(x_{n-1}) & x_n=g_2(x_{n-1}) & x_n=g_3(x_{n-1}) & x_n=g_4(x_{n-1}) \\
\hline
0 & 1.5 & 1.5 & 1.5 & 1.5 \\
1 & -0.875 & 0.81649658 & 1.2869538 & 1.373333333 \\
2 & 6.7324219 & 2.9969088 & 1.4025408 & 1.365262015 \\
3 & -469.72001 & & 1.3454584 & 1.365230014 \\
4 & 1.0275456\times 10^8 & & 1.3751703 & 1.365230013 \\
5 & -1.0849339\times 10^{24} & & 1.3600942 & 1.365230013 \\
\hline
\end{array}
\]
$g_1$ and $g_2$ generate sequences that do not converge.  $g_2$ even
ends up generating complex numbers.  This shows that not all functions
equivalent to $f$ give converging fixed point iterations.  We note the
fast convergence of the fixed point iteration for the function $g_4$.  We
will show in the next section why it is so.
\label{Fix}
\end{egg}

\section{Newton's Method}

The idea is to construct a sequence $\{x_i\}_{i=0}^\infty$ which
converges to a root $p$ of a function $f$.

\begin{algo}[Newton]
\begin{enumerate}
\item Choose $x_0$ closed to a root $p$ of $f$ (if possible).
\item Given $x_n$ , compute
\begin{equation} \label{formula2}
x_{n+1}=x_n-\frac{f(x_n)}{f'(x_n)}
\end{equation}
if $f'(x_n)\neq 0$.  If $f'(x_n)=0$, start over with a better choice
of $x_0$.
\item Repeat (2) until the interruption criteria are satisfied.
\end{enumerate}
\end{algo}

This method is also known as
{\bfseries Newton-Raphson's Algorithm}\index{Newton-Raphson's Algorithm}.

There is a nice graphical representation of the Newton's method
that can be found in Figure~\ref{NRM}.  Let $x_n$ be an
approximation of a root $p$ of $f$ obtained from Newton's method.
$x_{n+1}$ is the $x$ coordinate of the intersection of the
tangent line to the curve $y=f(x)$ at $(x_n,f(x_n))$ with the
$x$-axis.  The equation of the tangent line to the curve $y=f(x)$ at
$(x_n,f(x_n))$ is $y=f(x_n)+f'(x_n)\,(x-x_n)$.  Hence $x_{n+1}$ is the
solution of $0 = f(x_n)+f'(x_n)\,(x-x_n)$.  If $f'(x_n)\not=0$, this
is
\[
x_{n+1} = x_n-\frac{f(x_n)}{f'(x_n)} \ .
\]

\pdfF{solve_equ_A/newton}{Newton's Method}{Newton's Method}{NRM}

\begin{theorem}
Let $f$ be a twice continuously differentiable function on $[a,b]$.
Suppose that $p \in [a,b]$ is a root of $f$ such that $f'(p) \neq 0$.
Then there exists $\delta > 0$ such that, for any
$x_0 \in [p-\delta, p + \delta]$, the sequence defined by
(\ref{formula2}) converges to $p$ as $n$ goes to $\infty$.
\end{theorem}

This theorem will be proved as part of the more informative
Theorem~\ref{newton_order}.

\begin{rmk} The Newton's method is the fixed point method
defined by $x_{n+1}=g(x_n)$ with
$\displaystyle g(x)=x - \frac{f(x)}{f'(x)}$.  If $p$ is a fixed point
of $g$, then $\displaystyle p = p - \frac{f(p)}{f'(p)}$ and we get
$f(p) = 0$.
\end{rmk}

\begin{egg}
Find an approximation of $\sqrt{2}$ using the Newton's method.
Stop when the difference between two consecutive iterations is smaller
than $10^{-4}$.

As in Example~\ref{ExSqrt2}, we find an approximation of the
positive root of $f(x) = x^2 - 2$.  We have
\[
x_{n+1} = x_n - \frac{f(x_n)}{f'(x_n)}
= x_n - \frac{x_n^2-2}{2 x_n}
= \frac{x_n^2+2}{2 x_n}.
\]
and start with $x_0=2$.

\[
\begin{array}{c|c|c|c}
\hline
n & x_n & |x_{n-1}-x_n| & < 10^{-4} \\
& \text{(rounded to 6 decimals)} & & \\  
\hline
0 & 2 & & \\
1 & 1.5 & 0.5 & \text{no} \\
2 & 1.416667 & 0.083333 & \text{no} \\
3 & 1.414216 & 0.002451 & \text{no} \\
4 & 1.414214 & 0.000002 & \text{yes} \\
\hline
\end{array}
\]

The answer we are looking for is $x_4 \approx 1.414214$.
\end{egg}

\begin{egg}
Use Newton's method to find an approximation of
a root of $f$ given in Example~\ref{Fix}.  Stop when the difference
between two consecutive iterations is smaller than $10^{-10}$.

We have
\[
x_{n+1} = x_n - \frac{f(x_n)}{f'(x_n)}
= x_n - \frac{x_n^3 + 4 x_n^2 - 10}{3 x_n^2 + 8 x_n}
= \frac{2 \left( x_n^3 + 2 x_n^2 + 5\right)}{3 x_n^2 + 8 x_n} \ ,
\]
and we take $x_0 = 1.5$.
\[
\begin{array}{c|c|c|c}
\hline
n & x_n & |x_{n-1}-x_n| & < 10^{-10}\\
& \text{(rounded to 13 decimals)} & & \\
\hline
0 & 1.5 & & \\
1 & 1.3733333333333 & 0.126667 & \text{no} \\
2 & 1.3652620148746 & 0.00807132 & \text{no} \\
3 & 1.3652300139162 & 0.000032001 & \text{no} \\
4 & 1.3652300134141 & 5.0205\times 10^{-10} & \text{no} \\
5 & 1.3652300134141 & 2.22045\times 10^{-16} & \text{yes} \\
\hline
\end{array}
\]
The required approximation for the root of $f$ is
$x_5 \approx 1.3652300134141$.
\end{egg}

\section{Secant Method}

As for Newton's method, the idea is to construct a
sequence $\{x_i\}_{i=0}^\infty$ that converges to a root $p$ of
$f$.  The convergence of the secant method is generally slower than
the convergence of the Newton's method but this secant method does not
use the derivative of $f$.  Moreover, only one evaluation of $f$ is
needed at each step of the secant method while one evaluation of $f$
and one evaluation of $f'$ are needed at each step of the Newton's method.

\begin{algo}[Secant]
\begin{enumerate}
\item Choose two distinct values $x_0$ and $x_1$ near a root $p$ of
  $f$ (if possible)
\item Given two distinct values $x_{n-1}$ and $x_n$, compute
\begin{equation} \label{formula3}
x_{n+1} = x_n - f(x_n) \left(\frac{f(x_n)-f(x_{n-1})}{x_n-x_{n-1}}\right)^{-1}
= x_n - \frac{f(x_n)\,(x_n-x_{n-1})}{f(x_n)-f(x_{n-1})}
\end{equation}
if $f(x_n)-f(x_{n-1})\not=0$.  If $f(x_n)-f(x_{n+1})=0$, start over
with a better choice of $x_0$ and $x_1$.
\item Repeat (2) until the interruption criteria are satisfied.
\end{enumerate}
\end{algo}

There is a graphical interpretation of the secant method which is
given in Figure~\ref{SM}.  Let $x_{n-1}$ and $x_n$ be two
approximations of a root $p$ of $f$. the next approximation $x_{n+1}$
of $p$ is the $x$-coordinate of the intersection of the $x$-axis with
the secant line for the curve $y=f(x)$ through $(x_n,f(x_n))$ and
$(x_{n-1},f(x_{n-1}))$.  The equation of the secant line is
\[
y= f(x_n) + \frac{f(x_n)-f(x_{n-1})}{x_n-x_{n-1}}\,(x-x_n) \ .
\]
Thus $x_{n+1}$ is the solution of
\[
0= \frac{f(x_n)-f(x_{n-1})}{x_n-x_{n-1}}  (x-x_n) +f(x_n) \ .
\]
If $f(x_n) - f(x_{n-1}) \neq 0$, this is
\[
x_{n+1} = x_n - f(x_n) \left(\frac{f(x_n)-f(x_{n-1})}{x_n-x_{n-1}}\right)^{-1}
\ .
\]

\pdfF{solve_equ_A/secant}{Secant Method}{Secant Method}{SM}

\begin{rmk}
It is preferable to use the formula
$\displaystyle x_{n+1} = x_n -
\frac{f(x_n)\,(x_n-x_{n-1})}{f(x_n)-f(x_{n-1})}$ instead of
$\displaystyle x_{n+1} =
x_n - f(x_n) \left(\frac{f(x_n)-f(x_{n-1})}{x_n-x_{n-1}}\right)^{-1}$
to reduce the risk of divisions by numbers (i.e.\ $x_n-x_{n-1}$) almost
equal to $0$.
\end{rmk}

\begin{rmk}
The secant method is the fixed point method defined by
$\displaystyle
\begin{pmatrix} x_n \\ x_{n+1} \end{pmatrix}
= g\begin{pmatrix} x_{n-1}\\ x_n \end{pmatrix}$
with
\begin{align*}
g:\RR^2 &\rightarrow \RR^2 \\
\
\begin{pmatrix} x \\ y \end{pmatrix}
&\mapsto
\begin{pmatrix}
y \\ 
\displaystyle y - \frac{f(y)}{F(x,y)}
\end{pmatrix} \ ,
\end{align*}
where $F$ is defined by
\[
F(x,y) = \begin{cases}
\displaystyle \frac{f(x)-f(y)}{x-y} & \quad \text{if} \quad x\neq y \\
f'(x) & \quad \text{if} \quad x = y
\end{cases}
\]
The point $p$ is a root of $f$ if and only if
$\displaystyle \begin{pmatrix} p \\ p \end{pmatrix}$ is a fixed point
of $g$,  The sequence $\{x_n\}_{n=0}^\infty$ converges to a root $p$ of $f$
if and only if the sequence
$\displaystyle \left\{ \begin{pmatrix} x_n \\
    x_{n+1} \end{pmatrix}\right\}_{n=0}^\infty$ 
converges to a fixed point
$\displaystyle \begin{pmatrix} p \\ p \end{pmatrix}$  of $g$.

We will study the fixed point method in $\RR^n$ in Chapter~\ref{chaptSeqD}.
\label{secantFP}
\end{rmk}

\section{Order of Convergence} \label{OrderConvNBmeth}

The following definition is used to determine the ``quality'' of an
iterative method.

\begin{defn}
Suppose that the sequence $\{x_n\}_{n=0}^\infty$ converges to $p$.
Let $e_n = x_n - p$.  We say that $\{x_n\}_{n=0}^\infty$
{\bfseries converges to $p$ of order $\alpha$}\index{Sequences!Order
  of Convergence} if there exists a non-zero real number $\lambda$
such that
\[
\lim_{n\rightarrow \infty}
\frac{|e_{n+1}|}{|e_n|^\alpha} = \lambda
\]
\end{defn}

If $\alpha = 1$, we talk of {\bfseries linear
convergence}\index{Sequences!Linear Convergence}.  If $\alpha = 2$,
we talk of {\bfseries quadratic convergence}\index{Sequences!Quadratic
Convergence}.

\begin{theorem}
Let $g:[a,b] \rightarrow \RR$ be a sufficiently continuously
differentiable function.  Suppose that $p$ is a fixed point of $g$ in
$[a,b]$ such that one of the following conditions is satisfied. 
\begin{description}
\item[$k=1$]:
\[
0 < |g'(p)| < 1 \ .
\]
\item[$k=2,3,\ldots$]:
\[
g'(p) = g''(p) = \ldots = g^{(k-1)}(p) = 0 \text{ and } g^{(k)}(p) \not=0 \ .
\]
\end{description}
Then there exists $\delta > 0$ such that, for
$x_0 \in [p-\delta, p + \delta]$, the sequence defined by
(\ref{FPMformula1}) converges to $p$ of order $k$ as $n$ goes to
$\infty$.
\label{FPorder}
\end{theorem}

\begin{proof}
Choose $K$ such that $|g'(p)| < K <1$.  By continuity of $g'$,
there exists $\delta$ such that $|g'(x)| \leq K$ for all $x$ in
$[p-\delta, p+\delta]$.  Using the Mean Value Theorem, it is easy to
see that $g:[p-\delta, p+\delta] \rightarrow [p-\delta, p+\delta]$
(show it).  Hence, when restricted to $[p-\delta, p + \delta]$, the
function $g$ satisfies all the hypothesis of the Fixed Point Theorem.

From the Fixed Point Theorem, $p$ is the unique fixed point of $g$ in
$[p-\delta, p + \delta]$.  Moreover, if
$x_0 \in [p-\delta, p + \delta]$, the sequence $\{x_n\}_{n=0}^\infty$
defined by $x_{n+1} = g(x_n)$ for $n \geq 0$ converges to $p$.

The Taylor series expansion of $g$ at $p$ yields
\[
x_{n+1} - p = g(x_n) - g(p) = \frac{1}{k!} g^{(k)}(\xi_n)\,(x_n - p)^k
\]
for some $\xi_n$ between $x_n$ and $p$.  If $x_n \rightarrow p$ as
$n \rightarrow \infty$, then $\xi_n \rightarrow p$ as
$n \rightarrow \infty$ because $\xi_n$ is between $x_n$ and $p$.
Hence,
\[
\lim_{n\rightarrow \infty}\frac{|e_{n+1}|}{|e_n|^k}
= \lim_{n\rightarrow \infty}\frac{|x_{n+1}-p|}{|x_n-p|^k} \\
= \lim_{n\rightarrow \infty}\frac{|g^{(k)}(\xi_n)|}{k!}
= \frac{|g^{(k)}(p)|}{k!} \neq 0 \ .  \qedhere
\]
\end{proof}

From the proof above, we have that the order of a fixed point method
to find a root $p$ of a function $g$ is the order of the first
non-null derivative of $g$ at $p$.

\begin{theorem}
Let $f:[a,b] \rightarrow \RR$ be a twice continuously differentiable
function.  Suppose that $p$ is a root of $f$ in $[a,b]$ and
$f'(p) \neq 0$.  Then there exists $\delta > 0$ such that, for
$x_0 \in [p-\delta, p + \delta]$, the sequence
$\displaystyle \left\{x_n\right\}_{n=0}^\infty$ produced by the
Newton's method defined by (\ref{formula2}) converges to $p$ at least
quadratically as $n$ goes to $\infty$.
\label{newton_order}
\end{theorem}

\begin{proof}
By continuity of $f'$, there exists $\delta'$ such that $f'(x) \neq 0$
for all $x$ in $[p-\delta',p+\delta']$.  Consider
$g:[p-\delta',p+\delta'] \rightarrow \RR$ defined by
\[
g(x) = x - \frac{f(x)}{f'(x)} \ .
\]
This function satisfies the hypotheses of Theorem~\ref{FPorder} with
$k > 1$ because
\[
g'(x) = \frac{f(x)f''(x)}{(f'(x))^2}
\]
on $[p-\delta',p+\delta']$ and so $g'(p) = 0$ because $f(p) = 0$.
\end{proof}

\begin{rmk}
If $f'(p) = 0$ in the previous theorem, the convergence (if there is
convergence) of the sequence produced by the Newton's method may
not be quadratic.  If the Newton's method does not produce a
sequence converging to a root of the function $f$, or if it
produces a sequence converging very slowly to a root of $f$, it is
possible to slightly modify the Newton's method to obtain a method
that will produce a sequence converging quadratically to a root of
$f$.

Suppose that $p$ is a {\bfseries zero of multiplicity $k> 1$ of
$f$}\index{Functions!Multiplicity of Zeros};
that is, $f(p) = f'(p) = \ldots = f^{(k-1)}(p) = 0$ and $f^k(p) \neq 0$. 
Instead of (\ref{formula2}), one uses the fixed point method
with the function
\[
g(x) = \begin{cases}
\displaystyle x - \frac{k\,f(x)}{f'(x)} & \quad \text{if} \quad x \neq p \\
  p & \quad \text{if} \quad x = p
\end{cases}
\]
Because $p$ is a zero of multiplicity $k$ of $f$, it is shown in
Question~\ref{solvAQ29} that we can write $f$ as
$f(x) = (x-p)^k\,q(x)$ for some function $q$ such that
$q(p) \neq 0$. Hence, for $x \neq p$,
\[
g(x) = x - \frac{k\,(x-p)^k\,q(x)}
{k\,(x-p)^{k-1}\,q(x)+(x-p)^k\,q'(x)}
= x - \frac{k\,(x-p)\,q(x)}{k\,q(x) + (x-p)\,q'(x)} \ .
\]
This expression of $g$ is well defined at $p$ because the denominator
of the fraction is different of $0$ at $p$.  In fact, the right-hand
side evaluated at $p$ gives $p$.

Moreover,
\[
g'(x) = 1 -
\frac{k\,q(x) + k\,(x-p)\,q'(x)}{k\,q(x) + (x-p)\,q'(x)}
+ \frac{\left(k\,(x-p)\,q(x)\right)
\left(k\,q'(x) + q'(x) + (x-p)\,q''(x)\right)}
{\left(kq(x) + (x-p)q'(x) \right)^2} \ .
\]
Hence
\[
g'(p) = 1 - \frac{kq(p)}{kq(p)} = 0
\]
and the convergence is at least quadratic.
\label{multiplicity}
\end{rmk}

\begin{theorem}
Let $f:[a,b] \rightarrow \RR$ be a twice continuously differentiable
function.  Suppose that $p$ is a root of $f$ in $[a,b]$,
$f'(p) \neq 0$ and $f''(p) \neq 0$.  Then there exists $\delta > 0$
such that, for $x_0$ and $x_1$ in $[p-\delta, p + \delta]$, the
sequence $\displaystyle \left\{x_n\right\}_{n=0}^\infty$ produced by
the secant method defined by (\ref{formula3}) converges to $p$ of order
$(1+\sqrt{5})/2 \approx 1.618\ldots$ as $n$ goes to $\infty$.
\end{theorem}

The proof of the convergence of the secant method is based on
proving that the function $g$ defined in Remark~\ref{secantFP}
satisfies the hypothesis of the Fixed Point Theorem.  This proof will
not be given here.

We will also not prove that there exist $\alpha>0$ and $\lambda \neq 0$
such that
\begin{equation} \label{bisOrderGoldF}
\lim_{n\rightarrow \infty} \frac{|e_{n+1}|}{|e_n|^\alpha} = \lambda \ .
\end{equation}
This proof is tricky.  We prove in Remark~\ref{bisOrderGold}
of Section~\ref{GenNewtonInter} that if there exist $\alpha>0$ and
$\lambda \neq 0$ such that (\ref{bisOrderGoldF}) is satisfied, then
$\alpha$ must be the {\bfseries golden ratio}\index{Golden Ratio}
$(1+\sqrt{5})/2$.   To prove this, we use divide difference
formulae that are presented in Chapter~\ref{chaptInterA}.

\section{Aitken's $\Delta^2$ Process and Steffensen's Algorithm}

Suppose that $p_0$ is an initial approximation for a fixed point $p$
of a function $g$.  Moreover, suppose that the sequence
$\{p_n\}_{n=0}^\infty$ defined by $p_{n+1} = g(p_n)$ for $n \geq 0$
converges linearly to $p$.  We give a procedure to build a new
sequence $\displaystyle \{\hat{p}_n\}_{n=0}^\infty$ that converges
``faster'' to $p$ than $\{p_n\}_{n=0}^\infty$.

Let
\begin{align*}
\dtx{p_n} &= p_{n+1} - p_n\\
\dtxn{2}{p_n} &= \dtx{(\dtx{p_n})} = \dtx{p_{n+1}} - \dtx{p_n}
= p_{n+2} - 2p_{n+1} + p_n \\
\ldots &\quad \ldots \\
\dtxn{k}{p_n} &= \dtx{(\dtxn{k-1}{p_n})}
\end{align*}
for $n \geq 0$.

The sequence $\{\hat{p}_n\}_{n=0}^\infty$ defined by
\begin{equation} \label{Aitken}
\begin{split}
\hat{p}_n &= p_n - \frac{(\dtx{p_n})^2}{\dtxn{2}{p_n}}
= p_n - \frac{(p_{n+1}- p_n)^2}{p_{n+2} -2p_{n+1} + p_n}
\end{split}
\end{equation}
converges to $p$ and the order of convergence of
$\{\hat{p}_n\}_{n=0}^\infty$ to $p$ is greater than 1.
The procedure used to construct $\{\hat{p}_n\}_{n=0}^\infty$ is
called the
{\bfseries Aitken's $\Delta^2$ process}\index{Aitken's $\Delta^2$ process}.

\[
\begin{array}{l*{2}{@{\ ,\ }l}@{\quad\text{give}\quad}l}
p_0 & p_1 & p_2 & \hat{p}_0 \\
p_1 & p_2 & p_3 & \hat{p}_1 \\
p_2 & p_3 & p_4 & \hat{p}_2 \\
\multicolumn{4}{c}{\ldots}
\end{array}
\]

Since $\hat{p}_n$ is generally a better approximation of $p$ than
$p_{n+1}$, it is better to replace $p_n$, $p_{n+1}$ and $p_{n+2}$ in
(\ref{Aitken}) by $\hat{p}_{n-1}$, $g(\hat{p}_{n-1})$ and
$g(g(\hat{p}_{n-1}))$.  Using this idea, we get the following
algorithm.

\begin{algo}[Steffensen's]
\begin{enumerate}
\item Choose $p_0$ closed to a fixed point $p$ of $g$ (if possible).
\item Let $\hat{p}_{-1} = p_0$.
\item For $n \geq -1$, compute
\begin{equation} \label{formula4}
\hat{p}_{n+1} = \hat{p}_n - \frac{(g(\hat{p}_n) -
\hat{p}_n)^2}{g(g(\hat{p}_n)) -2 g(\hat{p}_n) + \hat{p}_n} \ .
\end{equation}
\item Repeat (3) until the interruption criteria are satisfied.
\end{enumerate}
\label{SteffAlgo}
\end{algo}

\[
\begin{array}{l*{2}{@{\ ,\ }l}@{\quad\text{give}\quad}l}
p_0 & g(p_0) & g(g(p_0)) & \hat{p}_0 \ . \\
\hat{p}_0 & g(\hat{p}_0) & g(g(\hat{p}_0)) & \hat{p}_1 \ . \\
\hat{p}_1 & g(\hat{p}_1) & g(g(\hat{p}_1)) & \hat{p}_2 \ . \\
\multicolumn{4}{c}{\ldots}
\end{array}
\]

\begin{theorem}
Let $g:[a,b] \rightarrow \RR$ be a $3$-time continuously
differentiable function.  Suppose that $p$ is a fixed point of $g$ in
$[a,b]$ and $g'(p) \not= 0$.  Then there exists $\delta > 0$ such
that, for $p_0 \in [p-\delta, p + \delta]$, the sequence
$\{\hat{p}_n\}_{n=0}^\infty$ defined by (\ref{formula4}) converges to
$p$ of order two as $n$ goes to $\infty$.
\end{theorem}

\begin{proof}[Proof (Idea)]
The idea of the proof is to apply Theorem~\ref{FPorder} with $k=2$ to the
function
\[
G(x) =
\begin{cases}
\displaystyle x - \frac{(g(x)-x)^2}{g(g(x)) -2g(x) + x} & \quad
\text{if} \quad x \neq p \\
p & \quad \text{if} \quad x = p
\end{cases}  \qedhere
\]
\end{proof}

\begin{rmk}
Though the order of the Steffensen's Algorithm is greater than the
order of the secant method, the Steffensen's Algorithm is not always
faster on computer than the secant method because there are two
function evaluations, four additions/subtractions and three
multiplications/divisions at each step for the Steffensen's Algorithm
while there are one function evaluation, three additions/subtractions
and two multiplications/divisions at each step for the secant
method.  A function evaluation may be time consuming.
\end{rmk}

\section{Real Roots of Polynomials}

In this section, we do not introduce any new iterative algorithms but
show how to efficiently use Newton's method to approximate
the real roots of a polynomial.

Let $p$ be a polynomial of degree $m$.  If we apply Newton's method
to this polynomial, we get the formula
\[
x_{n+1} = x_n - \frac{p(x_n)}{p'(x_n)}
\]
for $n \geq 0$.  The following theorem gives an algorithm to compute
$p(x_n)$ and $p'(x_n)$ with a lot less arithmetic operations than the
direct computation of $p(x_n)$ and $p'(x_n)$.

\begin{theorem}[Horner]
Let $p(x) = a_m\,x^m + a_{m-1}\,x^{m-1} + \ldots + a_1\,x + a_0$ and
\[
\begin{split}
b_m &= a_m \\
b_{m-1} &= a_{m-1} + b_m \, \alpha \\
b_{m-2} &= a_{m-2} + b_{m-1} \, \alpha \\
\ldots & \qquad \ldots \\
b_k &= a_k + b_{k+1} \, \alpha \\
\ldots & \qquad \ldots \\
b_0 &= a_0 + b_1 \, \alpha
\end{split}
\]
Then  $b_0 = f(\alpha)$ and $p(x) = (x - \alpha)\,q(x) + b_0$ where
$q(x) = b_m\,x^{m-1} + b_{m-1}\,x^{m-2} + \ldots + b_2\,x + b_1$.
Moreover $p'(\alpha) = q(\alpha)$.
\label{HornerTh}
\end{theorem}

\begin{proof}
We have
\begin{align*}
(x-\alpha)\,q(x) + b_0 &= (x-\alpha)(b_m\,x^{m-1} + b_{m-1}\,x^{m-2} +
\ldots + b_2\,x + b_1) + b_0\\
&= b_m\,x^m + b_{m-1}\,x^{m-1} + b_{m-2}\,x^{m-2} + \ldots + b_2\,x^2
+ b_1\,x \\
& \quad - b_m\,\alpha\,x^{m-1} - b_{m-1}\,\alpha\,x^{m-2} - \ldots -
b_3\,\alpha\,x^2 - b_2\,\alpha\,x - b_1\,\alpha + b_0\\
&= b_m\,x^m + (b_{m-1} -b_m\,\alpha)x^{m-1} + (b_{m-2}
-b_{m-1}\,\alpha)x^{m-2} + \ldots \\
& \quad + (b_2 - b_3\,\alpha)x^2 + (b_1 - b_2\,\alpha)x +
(b_0 - b_1\,\alpha)\\
&= a_m\,x^m + a_{m-1}\,x^{m-1} + a_{m-2}\,x^{m-2} + \ldots + a_2\,x^2
+ a_1\,x + a_0
= f(x)
\end{align*}
because $b_m = a_m$ and $b_k - b_{k+1}\,\alpha = a_k$ for
$k = 0,1,2,\ldots,m-1$.  Moreover,
$f(\alpha) = (\alpha -\alpha)q(\alpha) + b_0 = b_0$.

Since $p'(x) = (x-\alpha)q'(x) + q(x)$, we get $p'(\alpha) = q(\alpha)$.
\end{proof}

At the same time that $p(\alpha)$ is computed with Horner's Algorithm, a
second used of Horner's Algorithm with $p$ replaced by $q$ may compute
$p'(\alpha) = q(\alpha)$,  More precisely, if
\[
\begin{split}
d_m &= b_m \\
d_{m-1} &= b_{m-1} + d_m \, \alpha \\
d_{m-2} &= b_{m-2} + d_{m-1} \, \alpha \\
\ldots & \quad \ldots \\
d_k &= b_k + d_{k+1} \, \alpha \\
\ldots & \quad \ldots \\
d_1 &= b_1 + d_2 \, \alpha
\end{split}
\]
then $p'(\alpha) = q(\alpha) = d_1$.  This may be expanded to higher
order derivatives.

Hence, Horner's theorem gives an efficient way to compute $p(x_n)$ and
$p'(x_n)$ in the Newton's method.  If $\alpha = x_n$ in Horner's
theorem, then $p(x_n) = b_0$ and $p'(x_n) = d_1$.

The computation of $p(x_n)$ and $p'(x_n)$ are combined in the
following algorithm.

\begin{code}[Horner's Algorithm]
To evaluate a polynomial $\displaystyle p(x) = \sum_{i=0}^n a_i x^i$
and its derivative at a point $\alpha$.\\
\subI{Input} The coefficients $a_i$ (the vector a in the code below).
The coefficient $a_n$ must be given even if it is zero.\\
The value of $\alpha$ (x in the code below.)\\
\subI{Output} $y = p(\alpha)$ and $z = p'(\alpha)$.
\small
\begin{verbatim}
% [y,z] = horner(a,x)

function [y,z] = horner(a,x)
  m = length(a);
  y = a(m);
  z = a(m);
  for i = m-1:-1:2
    y = a(i) + x*y;
    z = y + x*z;
  end
  y = a(1) + x*y;
end
\end{verbatim}
\end{code}

If we combine Newton's method and Horner's Algorithm, we get

\begin{code}[Newton's Method with Horner's Algorithm]
To approximate a real root of a polynomial
$\displaystyle p(x) = \sum_{i=0}^n a_i x^i$\\
\subI{Input} The coefficients $a_i$ (The vector a in the code below)
The coefficient $a_n$ must be given even if it is zero.\\
The initial approximation $x_0$ (x in the code below) of a root
$c$ of $p$.\\
The maximal tolerance $T$.\\
The maximal number $N$ of iterations.\\
\subI{Output} An approximation (xf in the code below) of the
real root $c$\\ or\\
an error message if the real root cannot be approximate with the
desired tolerance in less than $N$ iterations.
\small
\begin{verbatim}
% xf = realroot(a,x,N,tol)

function xf = realroot(a,x,N,tol)
  xf = NaN;
  m = length(a);

  for k=1:N
    y = a(m);
    z = a(m);
    for i=m-1:-1:2
      y = a(i) + x*y;
      z = y + x*z;
    end
    y = a(1) + x*y;

    if ( abs(z) < tol )
      disp 'The derivative is almost null.  Cannot proceed.'
      break;
    end

    % y = p(x)  and  z = p'(x) .
    ratio = y/z;
    x = x - ratio;
    if (abs(ratio) < tol)
      xf = x;
      return;
    end
  end
  
  disp 'The program fails to give an approximation to a root of'
  disp 'the polynomial in less than the N iterations.'
  xf = NaN;
end
\end{verbatim}
\label{NRHorner}
\end{code}

\begin{rmkList}
\begin{enumerate}
\item Newton's method may not be so good if we try to
approximate a root of multiplicity greater than one of a polynomial.
See Remark~\ref{multiplicity}.
\item A good initial approximation $x_0$ of a root $c$ of a polynomial
$p$ must be given if we want the Newton's method to generate a
sequence $\{x_n\}_{n=0}^\infty$ that converges to $c$.  A bad choice for
$x_0$ and the sequence may converge toward another root of $p$ or may
not converge at all.
\item Small changes in the coefficients of a polynomial of high degree
may produce very large changes in the roots of this polynomial.

For instance, the polynomial
\[
p(x) = x^7 - 28x^6 +322x^5 - 1960x^4 + 6769x^3
- 13132x^2 +13068x - 5040
\]
has the roots $1$, $2$, $3$, $4$, $5$, $6$ and $7$.  However, the
polynomial
\[
\tilde{p}(x) = x^7 - 28x^6 +322x^5 -1960x^4 + 6769x^3
-13133x^2 +13068x -5040 \ ,
\]
where only the coefficient of $x^2$ has be changed from $13132$ to
$13133$, has the roots (rounded to seven decimals) $1.0013976$,
$1.9689208$, $3.3183233$, $3.5050604$, $5.5731849 \pm 0.2641298\, i$
and $7.0599281$ which are quite different from those of the initial
polynomial.
\item In theory, if we have a real root $c$ of $p$, we can use Horner's
theorem with $\alpha = c$ to express $p$ as $p(x) = (x-c)q(x)$
because $b_0 = p(c) = 0$.  To find a second root of $p$, we only have to
find a root of $q$.  the polynomial $q$ is called the
{\bfseries reduced}\index{Reduced Polynomial} or
{\bfseries deflated polynomial}\index{Deflated Polynomial} associate to
$p$.  In reality, we only have an approximation of the root $c$ and
Horner's theorem gives only approximations of the coefficients $b_j$
of $q$.  In light of item 3 above, the approximation of a real root of
$q$ may have little relation with a real root of $p$.  However, we may
use this approximation as $x_0$ in the Newton's method applied to
$p$ to get an approximation of a new root (we hope) of $p$.
\end{enumerate}
\end{rmkList}

\begin{egg}
Let $p(x) = x^7 - 28x^6 +322x^5 - 1960x^4 + 6769x^3
- 13132x^2 +13068x - 5040$.  Approximate all the roots of $p$ within
$10^{-10}$.

In the following table
\begin{enumerate}
\item $q_0 = p$
\item $c_0$ is an approximation of a root of $q_0$ obtained with the
  Newton's method and the initial value $x_0 = 2.5$ (any
  other initial value could have been used).
\item $r_0 = c_0$
\item For $i=1$, $2$, \ldots, $6$.
{
\renewcommand{\labelenumii}{(\roman{enumii})}
\begin{enumerate}
\item The polynomial $q_i$ is the deflated polynomial obtained from the
  previous deflated polynomial $q_{i-1}$ with the help of Horner's
  Algorithm.  In theory, we have $q_{i-1} = (x-r_{i-1})q_i$.
\item The number $r_i$ is an approximation of a root of the deflated
  polynomial $q_i$ obtained with the Newton's method the initial value
  $x_0 = 2.5$.
\item The number $c_i$ is an approximation of a root of the polynomial
  $p$ obtained with the Newton's method and the initial value $x_0 = r_i$.
\end{enumerate}
}
\end{enumerate}

\[
\begin{array}{c|r|c|l}
\hline
i & \multicolumn{1}{c|}{q_i} & r_i & \multicolumn{1}{c}{c_i} \\
\hline
0 & -5040 + 13068x -13132x^2 + 6769x^3 -1960x^4 + 322x^5 -28x^6 + x^7
&  1 &  1 \\
1 & 5040 -8028x + 5104x^2 -1665x^3 +295x^4 -27x^5 + x^6 & 2 & 2 \\
2 & -2520 +2754x -1175x^2 + 245x^3 -25x^4 + x^5 & 3 & 3 \\
3 & 840 -638x + 179x^2 -22x^3 + x^4 & 4 & 4 \\
4 & -210 + 107x -18x^2 +x^3 & 5 & 5 \\
5 & 42 -13x + x^2 & 6 & 6 \\
6 & -7 + x & 7 & 7 \\
\hline
\end{array}
\]

We get the exact roots after rounding.
\end{egg}

\begin{egg}
Let $p(x) = 5 -3x -4x^2+x^4$.  Approximate all the roots of $p$ within
$10^{-10}$.

In the following table
\begin{enumerate}
\item $q_0 = p$
\item $c_0$ is an approximation of a root of $p$ obtained with the
  Newton's method and the initial value $x_0 = 2$ (any other initial
  value could have been used).
\item $r_0 = c_0$
\item For $i=1$ and $2$.
{
\renewcommand{\labelenumii}{(\roman{enumii})}
\begin{enumerate}
\item The polynomial $q_i$ is the deflated polynomial obtained from
  the previous deflated polynomial $q_{i-1}$ with the help of Horner's
  Algorithm.  In theory, we have $q_{i-1} = (x-r_{i-1})q_i$.
\item If $i=1$, the number $r_i$ is an approximation of a root of the
  deflated polynomial $q_i$ obtained with the Newton's method
  and the initial value $x_0 = 2$.
\item If $i=1$, the number $c_i$ is an approximation of a root of the
  polynomial $p$ obtained with the Newton's method and the
  initial value $x_0 = r_1$.
\end{enumerate}
}
\end{enumerate}

\[
\begin{array}{c|r|c|c}
\hline
i & \multicolumn{1}{c|}{q_i} & r_i & c_i \\
& \multicolumn{1}{c|}{\text{coefficients rounded to 10 decimals}} & & \\
\hline
0 & 5 -3 x -4 x^2 +x^4 & 2.0693229488 & 2.0693229488 \\
1 & -2.4162492389 + 0.2820974665 x + 2.0693229488 x^2 +x^3 &
0.8611735320 & 0.8611735320 \\
2 & 2.8057634715 + 2.9304964809 x + x^2 & \text{NaN} & \text{NaN} \\
\hline
\end{array}
\]

The method using the deflated polynomials combined with Newton's method
fails to give all the roots of the polynomial $p$.  The
deflated polynomial, where the coefficients have been rounded to $14$
decimals,
\[
q_2(x) = 2.80576347152215 + 2.93049648085253 \, x + x^2
\]
does not have real roots.  Since $q_2$ is a polynomial of degree two,
we can use the quadratic formula to find the roots of $q_2$.  We find
$-1.46524824042627 \pm 0.81167177199277i$, where the real and
imaginary parts have been rounded to 14 decimals.

The Newton's method works for the complex roots of
polynomials with complex coefficients.  So, we may use the
Newton's method with $p$ and the initial value $x_0$ given by one of the
complex roots of $q_2$.  Since $p$ has real coefficients, we know that
complex roots come in pair.
\end{egg}

\section{Appendix}

This section is to illustrate how complex a simple
{\bfseries discrete dynamical system}\label{Discrete Dynamical System}
of the form
\[
x_{i+1} = f(x_i)  \; ,
\]
where $f:\RR\rightarrow \RR$ is a continuous function, can be.  The
complexity is even greater if $f:\RR^k \rightarrow \RR^k$ with $k>1$.
Discrete dynamical systems show up in many numerical algorithms.
For instance, the Newton's Method to find zeros of functions yields
discrete dynamical systems, some numerical methods to solve ordinary
differential equations or partial differential equations are
discrete dynamical systems, etc.  It is therefore important to
understand the behaviour of discrete dynamical systems, or at least to
be aware of the complex behaviour of these systems. 

A good introduction to the subject of this appendix is \cite{D}.  It
is also a good reference for the proofs of most of results stated in this
appendix.

\subsection{Elementary Concepts of Discrete Dynamical Systems}

\begin{defn}
Consider a continuous function $f:\RR \rightarrow \RR$ and $x \in \RR$.
The {\bfseries forward orbit}\index{Orbit!Forward} of $x$ is the set
\[
\OO_x^+ = \left\{ x, f(x), f^2(x) = f(f(x)), f^3(x)= f(f(f(x))), \ldots
\right\} \  .
\]
If $f$ has an inverse $f^{-1}:\RR\rightarrow \RR$, the
{\bfseries backward orbit}\index{Orbit!Backward} of $x$ is the set
\[
\OO_x^- = \left\{ x, f^{-1}(x), f^{-2}(x) = f^{-1}(f^{-1}(x)),
f^{-3}(x)= f^{-1}(f^{-1}(f^{-1}(x))), \ldots \right\}
\]
and the {\bfseries orbit}\index{Orbit} of $x$ is
$\OO_x = \OO_x^+ \cup \OO_x^-$.
\end{defn}

\begin{defn}
Consider a continuous function $f:\RR \rightarrow \RR$.  The point
$p \in \RR$ is a {\bfseries periodic point}\index{Periodic Point} of
$f$ if there exists a positive integer $n$ such that $f^n(p) = p$.  If
$n$ is the smallest positive integer such that $f^n(p)=p$, we say that
$p$ is of {\bfseries period}\index{Period} $n$.  When $n=1$, $p$ is a
{\bfseries fixed point}\index{Fixed Point}.
We denote by $\Per_n(f)$ the set of all periodic point of $f$ of
period $n$.  In particular, $\Fix(f) = \Per_1(f)$ is the set of fixed
points of $f$.  If $p$ is a periodic point of $f$, $\OO_p$ is a
{\bfseries periodic orbit}\index{Periodic Orbit}.
\end{defn}

\begin{egg}
Consider the {\bfseries logistic map}\index{Logistic Map}
\[
f_\mu(x) = \mu x(1-x)
\]
for $0\leq x \leq 1$.  For $0 \leq \mu \leq 4$, we have
$f_\mu:[0,1]\rightarrow [0,1]$.  Moreover, $f_\mu$ has two fixed
points: $p_0 = 0$ and
$\displaystyle p_{\mu} = \frac{\mu-1}{\mu}$ for $\mu > 0$.

For $\mu=3.4$, $p_{3.4} = 0.45195878844045\ldots$ is a periodic point of
$f_\mu$ of period $2$.  The orbit of period two is
\[
\left\{ 0.45195878844045, 0.84215476876273, 0.45195878844045,
0.84215476876273, \ldots \right\} \; ,
\]
where the values have been chopped
to $14$ digits after the decimal point.  This can be easily seen from the
{\bfseries staircase diagram}\index{Staircase Diagram} or
{\bfseries cobweb}\index{Cobweb} of $f_\mu$ shown in Figure~\ref{figCW}.

\mathF{discrete_dyn/logistic1}{7cm}{Cobweb}{Cobweb}{figCW}

Another way to illustrate the behaviour of $f_\mu$ is with the
{\bfseries phase portrait}\index{Phase Portrait} of $f_\mu$ shown in
Figure~\ref{figPP}.

\pdfF{discrete_dyn/phase_portrait}{Phase Portrait}{Phase Portrait}{figPP}

Finally, one can plot the histogram of $f_\mu$.  Namely, we divide
the interval $[0,1]$ into a large number of subintervals of equal
lengths and we compute the percentage of iterations that enter each
subinterval.  For $\mu=3.4$, the histogram with $200$ subintervals of
$[0,1]$ and $10,000$ iterations is given in Figure~\ref{figHG}.

\mathF{discrete_dyn/histogram1}{7cm}{Histogram for $f_{3.4}$}
{Histogram for $f_{3.4}$}{figHG}

For $\mu=3.5$, $p_{3.5} = 0.87499726360246\ldots$ is a periodic point of
$f_\mu$ of period $4$.  The orbit of period four is
\[
\left\{0.87499726360246, 0.38281968301732, 0.82694070659144,
0.50088421030722, \ldots \right\} \; ,
\]
where the values have been chopped
to $14$ digits after the decimal point (see Figure~\ref{figPF}).

\mathF{discrete_dyn/logistic2}{7cm}{Period Four}{Period Four}{figPF}
\label{EggLogMap1}
\end{egg}

\begin{defn}
Consider a continuous function $f:\RR \rightarrow \RR$.  The points
$p$ and $q$ in $\RR$ are
{\bfseries forward asymptotic}\index{Forward Asymptotic} if
\[
\lim_{j\rightarrow \infty} \left| f^j(p) - f^j(q) \right| = 0 \ .
\]
In particular, if $p \in \RR$ is a periodic point of period $n$, then a
point $q$ is
{\bfseries forward asymptotic to $p$}\index{Forward Asymptotic to a Point} if
\[
p = \lim_{j\rightarrow \infty} f^{jn}(q) \  .
\]
The set of all points forward asymptotic to $p$ is denoted by $W^s(p)$.
There are similar definitions for
{\bfseries backward asymptotic}\index{Backward Asymptotic}.
\end{defn}

\begin{defn}
Consider a continuous function $f:\RR \rightarrow \RR$.
A fixed point $p$ of $f$ is {\bfseries stable}\index{Stable} if, for any open
neighbourhood $U$ of $p$, there exists an open neighbourhood $V$ of $p$
such that $f^i(V) \subset U$ for all $i > 0$.  Fixed points that are
not stable are called {\bfseries unstable}\index{Unstable}.   A fixed
point $p$ of
$f$ is {\bfseries asymptotically stable}\index{Asymptotically Stable} if it is
stable and there exists an open neighbourhood $W$ of $p$ such
$\displaystyle \lim_{i\rightarrow \infty} f^{i}(x) = p$ for all
$x\in W$.

If $p$ is a period point of period $n$ for $f$, we say that the
periodic orbit $\OO_p$ is {\bfseries stable}\index{Stable} if $p$ is a stable
fixed point of $f^n$.  We say that the periodic orbit is
{\bfseries asymptotically stable}\index{Asymptotically Stable} if $p$
is an asymptotically stable fixed point of $f^n$.
\end{defn}

\begin{rmk}
The previous definition of stability and asymptotic stability for a
period orbit is independent of the point $p$ of the orbit used to
determine the stability or the asymptotic stability.

Suppose that $p$ is a periodic point of period $n>1$ for a continuously
invertible function $f:\RR\to \RR$.  Suppose that $p$ is stable
for $f^n$ and let $q=f^k(p)$ be another point on the orbit $\OO_p$.
If $U_q$ is an open neighbourhood of $q$, then $f^{-k}(U_q)$ is an
open neighbourhood of $p$.  Since $p$ is a stable fixed point for
$f^n$, there exists an open neighbourhood $V_p$ of $p$ such that
$f^{ni}(V_q) \subset f^{-k}(U_q)$ for all $i>0$. 
Hence $f^{ni}(f^k(V_p)) = f^{ni+k}(V_q) \subset U_q$ for all $i>0$,
where $f^k(V_p)$ is an open neighbourhood of $q$.  This proves that
$q$ is a stable fixed point of $f^n$.

Suppose furthermore that $p$ is an asymptotically stable fixed point
of $f^n$.  Then there exists an open neighbourhood $W_p$ of $p$ such
that $\displaystyle \lim_{i\to \infty}f^{ni}(x) = p$ for all $x \in W_p$.
Thus $\displaystyle \lim_{i\to \infty}f^{ni}(f^k(x)) = f^{k}(p) = q$
for all $x \in W_p$; namely,
$\displaystyle \lim_{i\to \infty}f^{ni}(y) = q$ for all
$y$ in the open neighbourhood $f^k(W_p)$ of $q$.  This proves that
$q$ is an asymptotically stable fixed point of $f^n$.
\end{rmk}

\subsection{Qualitative Study}

Consider the discrete dynamical system
\begin{equation} \label{DDS}
x_{i+1} = f(x_i) \; .
\end{equation}
For a qualitative study of this system, we would like to find all the
fixed points, periodic orbits, \ldots.  We would also like to find the
sets of points forward asymptotic to these objects.

We first study the fixed points of (\ref{DDS}).

\begin{defn}
A fixed point $p$ of $f$ is
{\bfseries hyperbolic}\index{Hyperbolic Fixed Point} if
$\left|f'(p)\right| \neq 1$.
\end{defn}

A proof similar to the proof of the Fixed Point Theorem yields the next
theorem.

\begin{prop}
Let $p$ be an hyperbolic fixed point of $f$.  If
$\left|f'(p)\right|<1$, then $p$ is asymptotically stable
However, if $\left|f'(p)\right|>1$, then there exists an open
interval $I$ containing $p$ such that for each $x$ in $I$,
$x \neq p$, one can find $j \in \NN$ such that $f^j(x) \not\in I$.
This implies that $f^j(x) \not\in I$ for infinitely many values of $j$.
\label{fp_stability}
\end{prop}

\begin{defn}
If $p$ is an hyperbolic fixed point of $f$ such that
$\left|f'(p)\right|<1$, then $p$ is called an
{\bfseries attracting}\index{Attracting} fixed point or a
{\bfseries sink}\index{Sink}.
If $p$ is an hyperbolic fixed point of $f$ such that
$\left|f'(p)\right|>1$, then $p$ is called a
{\bfseries repelling}\index{Repelling} fixed point or a
{\bfseries source}\index{source}.
\end{defn}

\begin{egg}
For the logistic map $f_\mu(x) = \mu x(1-x)$, the origin is a source
if $\mu>1$, and $\displaystyle p_\mu  = \frac{\mu-1}{\mu}$ is a sink
if $1<\mu<3$.  This follows from
$\displaystyle \pdfdx{f_\mu(x)}{x}\bigg|_{x=0} = \mu$ and
$\displaystyle \pdfdx{f_\mu(x)}{x}\bigg|_{x=p_\mu} = 2-\mu$.
\label{EggLogMap2}
\end{egg}

We can expand the notion of hyperbolicity to periodic points.

\begin{defn}
A periodic point $p$ of period $n$ for $f$ is
{\bfseries hyperbolic}\index{Hyperbolic Period Point}
if $\displaystyle \left| \dydx{f^n}{x}(p) \right| \neq 1$.
\end{defn}

A periodic point $p$ of $f$ of period $n$ is a fixed point of $f^n$.
The stability of the periodic point $p$ of $f$ is determined by the
stability of the fixed point $p$ of $f^n$.  Proposition
\ref{fp_stability} holds for a periodic point $p$ of period $n$ if
$f$ is replaced by $f^n$.

\begin{egg}
To study the stability of the periodic points of period $2$ for the
logistic map $f_\mu$ with $\mu = 3.4$, we consider the iterative system
$x_{i+1} = f_\mu^2(x_i) = f_\mu(f_\mu(x_i))$ for $i=0$, $1$, $2$, \ldots

\mathF{discrete_dyn/logistic3}{7cm}{Period Two}{Period Two}{figPT}

From the graph of $f_\mu^2$ given in Figure~\ref{figPT}, we see that
the periodic point\\ $0.45195878844045\ldots$ of period $2$ is a sink.
\label{EggLogMap3}
\end{egg}

\subsection{Bifurcation}

Consider a nice function $f:\RR^2 \rightarrow \RR$.  It defines a
one-parameter family of functions $f_\mu(x) = f(x,\mu)$.

We say that $\mu = \mu_0$ is a
{\bfseries bifurcation point}\index{Bifurcation Point} of the
discrete dynamical system
\[
x_{i+1} = f_\mu(x_i)
\]
if the qualitative behaviour of the phase portrait changes as $\mu$
goes through $\mu_0$.  For instance, the number of fixed points
change, new periodic solutions appear, etc.

There are three major results related to bifurcation.  The first
result is a simple consequence of the Implicit Function Theorem
applied to the function $g(x,\mu) = f(x,\mu) - x$.

\begin{theorem}
Let $f:\RR^2 \rightarrow \RR$ be a smooth function and let
$f_\mu(x)= f(x,\mu)$.  Suppose that $f_{\mu_0}(x_0) = x_0$ and
\[
\pdfdx{f_{\mu}(x)}{x} \bigg|_{\mu=\mu_0,x=x_0}
= \pdydx{f}{x}(x,\mu) \bigg|_{\mu=\mu_0,x=x_0} \neq 1  \ ,
\]
then there exist an open interval $I$ about $x_0$, an open interval
$J$ about $\mu_0$, and a mapping $p:J\rightarrow I$ such that
$p(\mu_0)=x_0$ and $f_\mu(p(\mu))=p(\mu)$ for all $\mu \in U$.
Moreover, $f_\mu$ has no other fixed point in $I$ (Figure~\ref{figIFT}).
\label{IFTforDS}
\end{theorem}

\pdfF{discrete_dyn/impl_funct_th}{The Implicit Function Theorem}{The Implicit
Function Theorem}{figIFT}

The next theorems describe two ``generic'' types of bifurcation.  We
have bifurcation only when
$\displaystyle \left| \pdfdx{f_{\mu}(x)}{x}\right| = 1$.  We use the
word generic because the other conditions to classify these types of
bifurcation require only that some derivatives be non-null.

\begin{theorem}[Saddle-node, tangent or fold bifurcation]
Let $f:\RR^2 \rightarrow \RR$ be a smooth function and let
$f_\mu(x)= f(x,\mu)$.  Suppose that
\begin{enumerate}
\item $\displaystyle f_{\mu_0}(0)=0$\ ,
\item $\displaystyle \pdfdx{f_{\mu}(x)}{x}\bigg|_{\mu=\mu_0,x=0} = 1$ \ ,
\item $\displaystyle \pdfdxn{f_{\mu}(x)}{x}{2}\bigg|_{\mu=\mu_0,x=0} \neq 0$
\quad and
\item $\displaystyle \pdfdx{f_\mu(x)}{\mu}\bigg|_{\mu=\mu_0,x=0} \neq 0$ \ .
\end{enumerate}
Then, there exist an interval $I$ about $0$ and a mapping
$q:I\rightarrow \RR$ such that $q(0) = \mu_0$ and $f_{q(x)}(x) = x$.  Moreover,
$q'(0) = 0$ and $q''(0) \neq 0$.
\end{theorem}

Figure~\ref{figSNBDM} illustrates a typical fold bifurcation.  The fixed
points represented by a dashed curve are sources while those
represented by a continuous curve are sinks.  The conditions in the
statement of the theorem above do not determine which branch of the
curve is associated to sources and which branch is associated to
sinks.  Moreover,
\[
q''(0) = \frac{\displaystyle -\pdfdxn{f_{\mu}(x)}{x}{2}\bigg|_{\mu=\mu_0,x=0}}
{\displaystyle \pdfdx{f_{\mu}(x)}{\mu}\bigg|_{\mu=\mu_0,x=0}}
\]
can be used to determine if the curve $\mu=p(x)$ is
{\bfseries supercritical}\label{Supercritical} (namely, $q''(x)>0$ as in
Figure~\ref{figSNBDM}) or {\bfseries subcritical}\index{Subcritical}
(namely, $q''(x)<0$).

\pdfF{discrete_dyn/saddle_node}{Fold bifurcation of discrete maps}
{A typical fold bifurcation diagram for a discrete map}{figSNBDM}

\begin{theorem}[Period doubling or flip bifurcation]
Let $f:\RR^2 \rightarrow \RR$ be a smooth function and let
$f_\mu(x)= f(x,\mu)$.  Suppose that
\begin{enumerate}
\item $f_{\mu}(0)=0$ for all $\mu$ near $\mu_0$,
\item $\displaystyle \pdfdx{f_{\mu}(x)}{x}\bigg|_{\mu=\mu_0,x=0} = -1$,
\item
$\displaystyle \frac{1}{2}
\left(\pdfdxn{f_{\mu}(x)}{x}{2}\bigg|_{\mu=\mu_0,x=0}\right)^2
+\frac{1}{3} \pdfdxn{f_{\mu}(x)}{x}{3}\bigg|_{\mu=\mu_0,x=0} \neq 0$
and
\item $\displaystyle \pdfdxnm{f_{\mu}^2(x)}{\mu}{x}{2}{}{}
\bigg|_{\mu=\mu_0,x=0} \not=0$,
\end{enumerate}
where $f^2_\mu(x) \equiv f(f(x,\mu),\mu)$.  Then, there exist
an interval $I$ about $0$ and a mapping $q:I\rightarrow \RR$ such that
$q(0) = \mu_0$ and $f_{q(x)}(x) \neq x$ but $f^2_{q(x)}(x) = x$.
\label{ThePDTforM}
\end{theorem}

Figure~\ref{figPDB} illustrates a typical period doubling bifurcation.
The fixed points are represented by the straight line $x=0$ and the periodic
points of period two are represented by the curve.   For $\mu$ fixed,
a periodic orbit of period two alternates between the lower and the
upper curve. The fixed points represented by a dashed curve are
sources while those represented by a continuous curve are sinks.   The
periodic points of period two represented by a dashed curve are
unstable while those represented by a continuous curve are
asymptotically stable.    Moreover,
\[
  q''(0) = \frac{\displaystyle
\left(\pdfdxn{f_{\mu}(x)}{x}{2}\bigg|_{\mu=\mu_0,x=0}\right)^2
+\frac{2}{3} \pdfdxn{f_{\mu}(x)}{x}{3}\bigg|_{\mu=\mu_0,x=0}}
{\displaystyle \pdfdxnm{f^2_{\mu}(x)}{x}{\mu}{2}{}{}\bigg|_{\mu=\mu_0,x=0}}
\]
can be used to determine if the curve $\mu=q(x)$ is
{\bfseries supercritical}\index{Supercritical} (namely, $q''(x)>0$ as in
Figure~\ref{figPDB}) or {\bfseries subcritical}\index{Subcritical}
(namely, $q''(x)<0$).

\pdfF{discrete_dyn/period_doubling}{Period doubling bifurcation for
discrete maps}{A typical period doubling bifurcation diagram for a
discrete map}{figPDB}

\begin{rmk}
In the previous theorem, the condition $f_{\mu}(0)=0$ for all $\mu$
near $\mu_0$ is not necessary.  Suppose that
\begin{enumerate}
\item $\displaystyle f_{\mu_0}(x_0) = x_0$\ ,
\item $\displaystyle \pdfdx{f_{\mu}(x)}{x}\bigg|_{\mu=\mu_0,x=x_0} = -1$\ ,
\item $\displaystyle \frac{1}{2}
\left(\pdfdxn{f_{\mu}(x)}{x}{2}\bigg|_{\mu=\mu_0,x=x_0}\right)^2
+\frac{1}{3} \pdfdxn{f_{\mu}(x)}{x}{3}\bigg|_{\mu=\mu_0,x=x_0} \neq 0$ \ 
and
\item $\displaystyle \pdfdxnm{f_{\mu}^2(x)}{\mu}{x}{2}{}{}
\bigg|_{\mu=\mu_0,x=x_0} \neq 0$ \ .
\end{enumerate}
From Theorem~\ref{IFTforDS}, there exists a function
$p$ defined in an open interval $J$ of $\nu_0$ such that
$p(\mu_0)=x_0$ and $p(\mu)$ is a fixed point of $f_\mu$ for all $\mu \in J$.
Let $\hat{f}(x,\mu) \equiv f(x+p(\mu),\mu)-p(\mu)$.  We show that
$\hat{f}$ satisfies the hypotheses of Theorem~\ref{ThePDTforM}.

We have $\hat{f}_\mu(0)=0$ for all $\mu$ near $\mu_0$.  Since
\[
  \pdydxn{\hat{f}}{x}{n}(x,\mu) = \pdydxn{f}{x}{n}(x+p(\mu),\mu)
\]
for $n=1$, $2$, \ldots, we have
\[
\pdfdx{\hat{f}_{\mu}(x)}{x}\bigg|_{\mu=\mu_0,x=0}
= \pdydx{\hat{f}}{x}(0,\mu_0)
= \pdydx{f}{x}(0+p(\mu_0),\mu_0)
= \pdydx{f}{x}(x_0,\mu_0)
= \pdfdx{f_{\mu}(x)}{x} \bigg|_{\mu=\mu_0,x=x_0} = -1 \ .
\]
Moreover,
\begin{align*}
&\frac{1}{2}
\left(\pdfdxn{\hat{f}_{\mu}(x)}{x}{2}\bigg|_{\mu=\mu_0,x=0}\right)^2
+\frac{1}{3} \pdfdxn{\hat{f}_{\mu}(x)}{x}{3}\bigg|_{\mu=\mu_0,x=0}
= 
\frac{1}{2} \left(\pdydxn{\hat{f}}{x}{2}(0,\mu_0) \right)^2
+\frac{1}{3} \pdydxn{\hat{f}}{x}{3}(0,\mu_0) \\
&\qquad = \frac{1}{2} \left(\pdydxn{f}{x}{2}(0+p(\mu_0),\mu_0) \right)^2
+\frac{1}{3} \pdydxn{f}{x}{3}(0+p(\mu_0),\mu_0)
= \frac{1}{2} \left(\pdydxn{f}{x}{2}(x_0,\mu_0) \right)^2
+\frac{1}{3} \pdydxn{f}{x}{3}(x_0,\mu_0) \\
&\qquad =\frac{1}{2}
\left(\pdfdxn{f_{\mu}(x)}{x}{2}\bigg|_{\mu=\mu_0,x=x_0}\right)^2
+\frac{1}{3} \pdfdxn{f_{\mu}(x)}{x}{3}\bigg|_{\mu=\mu_0,x=x_0} \neq 0 \ .
\end{align*}
Finally,
\begin{equation}\label{per_db_sc}
\pdfdxnm{\hat{f}_{\mu}^2(x)}{\mu}{x}{2}{}{}\bigg|_{\mu=\mu_0,x=0}
= \pdfdxnm{f_{\mu}^2(x)}{\mu}{x}{2}{}{}\bigg|_{\mu=\mu_0,x=x_0} \neq 1 \ .
\end{equation}
To prove the first equality requires a little bit of work.  From
\[
\hat{f}_{\mu}^2(x) = f(f(x+p(\mu),\mu),\mu)-p(\mu) \  ,
\]
we get
\[
\pdfdx{\hat{f}_{\mu}^2(x)}{x} = \pdfdx{f(f(x+p(\mu),\mu),\mu)}{x} 
= \pdydx{f}{x}(f(x+p(\mu),\mu),\mu) \pdydx{f}{x}(x+p(\mu),\mu)
\]
and
\begin{align*}
&\pdfdxnm{\hat{f}_{\mu}^2(x)}{\mu}{x}{2}{}{} =
\pdfdxnm{f(f(x+p(\mu),\mu),\mu)-p(\mu)}{\mu}{x}{2}{}{} \\
&\quad = \pdfdx{\left( \pdydx{f}{x}(f(x+p(\mu),\mu),\mu)
\pdydx{f}{x}(x+p(\mu),\mu) \right)}{\mu}  \\
&\quad = \pdydxn{f}{x}{2}(f(x+p(\mu),\mu),\mu)
\left( \pdydx{f}{x}(x+p(\mu),\mu) \dydx{p}{\mu} (\mu) \right. \\
&\qquad \left. + \pdydx{f}{\mu}(x+p(\mu),\mu) \right)
\pdydx{f}{x}(x+p(\mu),\mu)
+ \pdydxnm{f}{x}{\mu}{2}{}{}(f(x+p(\mu),\mu),\mu)
\pdydx{f}{x}(x+p(\mu),\mu)\\
&\qquad +\pdydx{f}{x}(f(x+p(\mu),\mu),\mu)
\left( \pdydxn{f}{x}{2}(x+p(\mu),\mu) \dydx{p}{\mu}(\mu)
+ \pdydxnm{f}{x}{\mu}{2}{}{}(x+p(\mu),\mu) \right) \ .
\end{align*}
Hence, using $p(\mu_0) = x_0$, $f(x_0,\mu_0)=x_0$ and
$\displaystyle \pdydx{f}{x}(x_0,\mu_0) =
\pdfdx{f_{\mu}(x)}{x}\bigg|_{\mu=\mu_0,x=x_0} = -1$, we get
\begin{align*}
&\pdfdxnm{\hat{f}_{\mu}^2(x)}{\mu}{x}{2}{}{} \bigg|_{\mu=\mu_0,x=0}
= \pdydxn{f}{x}{2}(f(x_0,\mu_0),\mu_0)
\left( \pdydx{f}{x}(x_0,\mu_0) \dydx{p}{\mu}(\mu_0)
+ \pdydx{f}{\mu}(x_0,\mu_0) \right)
\pdydx{f}{x}(x_0,\mu_0) \\
&\quad + \pdydxnm{f}{x}{\mu}{2}{}{}(f(x_0,\mu_0),\mu_0) \pdydx{f}{x}(x_0,\mu_0)
 \\
&\quad +\pdydx{f}{x}(f(x_0,\mu_0),\mu_0)
\left( \pdydxn{f}{x}{2}(x_0,\mu_0) \dydx{p}{\mu}(\mu_0)
+ \pdydxnm{f}{x}{\mu}{2}{}{}(x_0,\mu_0) \right) \\
&= -\pdydxn{f}{x}{2}(x_0,\mu_0) \left( - \dydx{p}{\mu}(\mu_0)
+ \pdydx{f}{\mu}(x_0,\mu_0) \right) - \pdydxnm{f}{x}{\mu}{2}{}{}(x_0,\mu_0) \\
&\quad - \left( \pdydxn{f}{x}{2}(x_0,\mu_0) \dydx{p}{\mu}(\mu_0) 
+ \pdydxnm{f}{x}{\mu}{2}{}{}(x_0,\mu_0) \right)
= - 2 \pdydxnm{f}{x}{\mu}{2}{}{}(x_0,\mu_0)
 - \pdydxn{f}{x}{2}(x_0,\mu_0)\pdydx{f}{\mu}(x_0,\mu_0) \ .
\end{align*}

Moreover,
\begin{align*}
\pdfdxnm{f_{\mu}^2(x)}{\mu}{x}{2}{}{}
&= \pdfdx{\left(\pdydx{f}{x}(f(x,\mu),\mu)
\pdydx{f}{x}(x,\mu) \right)}{\mu} \\
&= \pdydxn{f}{x}{2}(f(x,\mu),\mu) \pdydx{f}{\mu}(x,\mu)
\pdydx{f}{x}(x,\mu) + \pdydxnm{f}{x}{\mu}{2}{}{}(f(x,\mu),\mu)
\pdydx{f}{x}(x,\mu) \\
&\quad + \pdydx{f}{x}(f(x,\mu),\mu) \pdydxnm{f}{x}{\mu}{2}{}{}(x,\mu) \ .
\end{align*}
Hence, using $f(x_0,\mu_0)=x_0$ and
$\displaystyle \pdydx{f}{x}(x_0,\mu_0) =
\pdfdx{f_{\mu}(x)}{x} \bigg|_{\mu=\mu_0,x=x_0} = -1$, we get
\begin{align*}
\pdfdxnm{f_{\mu}^2(x) }{\mu}{x}{2}{}{} \bigg|_{\mu=\mu_0,x=x_0} &= 
\pdydxn{f}{x}{2}(f(x_0,\mu_0),\mu_0) \pdydx{f}{\mu}(x_0,\mu_0)
\pdydx{f}{x}(x_0,\mu_0) \\
& + \pdydxnm{f}{x}{\mu}{2}{}{}(f(x_0,\mu_0),\mu_0) \pdydx{f}{x}(x_0,\mu_0)
+\pdydx{f}{x}(f(x_0,\mu_0),\mu_0)
\pdydxnm{f}{x}{\mu}{2}{}{}(x_0,\mu_0) \\
&= -\pdydxn{f}{x}{2}(x_0,\mu_0) \pdydx{f}{\mu}(x_0,\mu_0)
- 2 \pdydxnm{f}{x}{\mu}{2}{}{}(x_0,\mu_0)
\end{align*}
and this proves the first equality of (\ref{per_db_sc}).
\end{rmk}

\subsection{Logistic Map}

This section is mainly about the logistic equation
\[
x_{i+1} = f_\mu(x_i) = \mu x_i (1-x_i) \; .
\]
We have already found the fixed points of $f_\mu$ with their stability
in Examples~\ref{EggLogMap1} and \ref{EggLogMap2}.  We have also
looked at period points of period $2$ for $f_\mu$ in
Example~\ref{EggLogMap3}.  To illustrate the complex behaviour that
discrete dynamical systems may have, we now present some results
about the logistic map $f_\mu$ without giving the proofs.  A good
reference is \cite{D}.  Most of the results about the logistic map
$f_\mu$ that we present in this section are true for mappings having a
graph ``similar'' to the graph of $f_\mu$ on $I=[0,1]$.

Figure~\ref{figBDLM} is the
{\bfseries bifurcation diagram}\index{Bifurcation Diagram} of
$f_\mu$.  Namely, for ``each'' value of $\mu$ we plot the fixed
points, periodic points, \ldots of $f_\mu$.

\mathF{discrete_dyn/route_to_chaos}{9cm}
{Bifurcation diagram for the logistic map}
{Bifurcation diagram for the logistic map}{figBDLM}

To produce this figure, we have chosen a large number
of equally spaced values of $\mu$.  For each of these values of $\mu$,
we have computed the first $200$ iterations of the orbit of $0.5$
under $f_\mu$ and plotted only the last $80$ or so iterations.  By
increasing the number of iterations to compute and plot, we could have
generated a more precise bifurcation diagram.

For $\mu<3$, all iterations converge to $p_\mu$.   When
$\mu$ crosses above $3$, the fixed point $p_\mu$ becomes a
source and there appears an attracting  periodic orbit of period $2$;
all orbits eventually ``bounce back and for'' between the two
points of the orbit.  We have period doubling at $\mu=3$.  This claim
can be rigorously prove by showing that all hypothesis of
Theorem~\ref{ThePDTforM} are satisfied.  Effectively, we have
\begin{enumerate}
\item $f_\mu(p_\mu) = p_\mu$ for all $\mu$ near $\mu=3$.
\item $\displaystyle \pdfdx{f_\mu(x)}{x} \bigg|_{\mu=3,x=p_3} = -1$.
\item
$\displaystyle \frac{1}{2}
\left(\pdfdxn{f_{\mu}(x)}{x}{2}\bigg|_{\mu=3,x=p_3}\right)^2
+\frac{1}{3} \pdfdxn{f_{\mu}(x)}{x}{3}\bigg|_{\mu=3,x=p_3}
= 2\mu^2 \bigg|_{\mu=3} = 18 \neq 0$
\item $\displaystyle \pdfdxnm{f_\mu^2(x) }{\mu}{x}{2}{}{}\big|_{\mu=3,x=p_3} =
\left( 2\mu - 6\mu^2 x+18\mu^2 x^2 - 4\mu x -
12\mu^2 x^3\right)\big|_{\mu=3,x=p_3}
= 2 \not=0$.
\end{enumerate}

When $\mu$ crosses above $3.236\ldots$, the periodic orbit of period $2$
becomes unstable and there appears an attracting periodic orbit of
period $4$.  For $\mu$ slightly larger, the periodic orbit of period
$4$ becomes unstable and there is a bifurcation from the attracting
periodic orbit of period $4$ to an attracting periodic orbit of period
$8$, And so on.  This is the best known example of a
{\bfseries period doubling cascade}\index{Period Doubling Cascade}.

For a constant value of $\mu$, if the attracting periodic orbit
$\OO \subset [0,1]$ of $f_\mu$ is of period $n$ with $n$ very large,
it is reasonable to expect that $\OO$ will ``almost cover'' some
segments of the interval $[0,1]$.  This explains the shaded area.
Figure~\ref{figLPO} illustrates the attracting periodic orbit for
$\mu=3.6$.  The corresponding histogram with 300 subintervals and 10
millions iterations is given in Figure~\ref{figHLPO}.

\mathF{discrete_dyn/logistic8}{7cm}{Cobweb of a periodic orbit with a large
period}{Cobweb of a periodic orbit of the logistic map for $\mu = 3.6$.
The period of this stable periodic orbit is very large.}{figLPO}

\mathF{discrete_dyn/histogram2}{7cm}{Histogram of a periodic orbit with a
large period}{Histogram of a periodic orbit of the logistic map for
$\mu = 3.6$.  The period of this stable periodic orbit is very large.}{figHLPO}

Period doubling accumulates to $\mu = 3.5699456\ldots$.  This number
is known as the {\bfseries Feigenbaum point}\index{Feigenbaum Point}.  Moreover, let $\mu_0=3$,
$\mu_1=3.236\ldots$, $\mu_2$, $\mu_3$, \ldots be the values of $\mu$
for which the logistic mapping undergoes period doubling and let
$d_j = \mu_{j+1}-\mu_j$.  It has been showed that
$\displaystyle \lim_{j\rightarrow \infty} \frac{d_j}{d_{j+1}} =
4.6692016091029\ldots$.  This number is called the
{\bfseries Feigenbaum constant}\index{Feigenbaum Constant}.
Feigenbaum discovered this number in
1975.  This constant is universal in the sense that it is the same
for a whole class of dynamical systems of the form
$x_{n+1} = g_\mu(x_n)$ where the graph of $g_\mu(x)$ looks like the
graph of $f_\mu(x)$.

Period doubling is far from being the must complex type of
bifurcation. To understand the complex behaviour of the orbits of
$f_\mu$, we need the following theorem.

\begin{theorem}[Sarkovskii]
Consider the order on the positive integers defined by
\begin{multline*}
3 \gg 5 \gg 7 \gg \ldots
\gg 3\times 2 \gg 5\times 2 \gg \ldots \\
\gg 3\times 2^2 \gg 5\times 2^2 \gg \ldots
\gg 3\times 2^3 \gg 5\times 2^3 \gg \ldots
2^4 \gg 2^3 \gg 2^2 \gg 2 \gg 1 \; .
\end{multline*}
Let $f:\RR \rightarrow \RR$ be a continuous function and $k$ be a
prime number.  If $f$ has a periodic point of period $k$, then $f$ has
a periodic point of period $m$ for all $m \ll k$.
\end{theorem}

\begin{egg}
For $\mu = 3.839\ldots$,
$\left\{0.14988539433432, 0.48917380192271, 0.95930024021836 \right\}$
is an attracting periodic orbit of period $3$ of $f_\mu$, where all
value have been chopped to $14$ digits after the decimal point
(Figure~\ref{figPTLC}).

\mathF{discrete_dyn/logistic5}{7cm}{Period Three}{Period Three}{figPTLC}

Hence, $f_{3.839\ldots}$ has periodic orbits of all possible periods.
All the periodic orbits, except the one of period $3$, are unstable
(in fact repelling).  So, unless the iteration starts with a point on
a repelling periodic orbit, the iterations will converge toward the
periodic orbit of period $3$.
\end{egg}

This is not the end of the story.  We now consider $f_\mu$ for
$\mu>4$.

Let
\[
A_n = \left\{ x \in [0,1] : f_\mu^j(x) \in [0,1] \text{ for }
0 \leq j \leq n \text{ and } f_\mu^j(x) \not\in [0,1] \text{ for }
j > n \right\} \; .
\]
It can be shown that $A_n$ consists of $2^n$ distinct open subintervals
of $[0,1]$.  We have that any iteration that starts with $x_0 \in A_0$
(i.e.\ such that $f_\mu(x_0)>1$) eventually converges to $-\infty$
(Figure~\ref{figA0}).  We also have that any iteration that starts
with $x_0 \in A_1$ (i.e. such that $f_\mu(x_0) \in A_0$) eventually
converges to $-\infty$ (Figure~\ref{figA1}).  And so on.

\mathF{discrete_dyn/logistic6}{7cm}{$A_0$ for $f_{4.3}$}{$0.5 \in A_0$ for
$f_{4.3}$}{figA0}

\mathF{discrete_dyn/logistic7}{7cm}{$A_0$ for $f_{4.3}$}{$0.1 \in A_1$ for
$f_{4.3}$}{figA1}

Consider
\[
\Delta_\mu = [0,1] \setminus \bigcup_{n=0}^\infty \,A_n \; .
\]
$\Delta_\mu$ contains all the points $x$ such that $f_\mu^j(x)$ stay
in $[0,1]$ for all $j\geq 0$.  In particular, $\Delta_\mu$ contains all
the periodic points.

Recall that a {\bfseries Cantor set}\index{Cantor Set} is a set that
is closed (contains
all its limit points), totally disconnected (does not contain any open
interval), and perfect (every point of the set is the limit of other
points of the set).

\begin{egg}
The best known example of a Cantor set is the Cantor Middle-Thirds
set.  It is also an example of a
{\bfseries Fractal}\index{Fractal} set because of its
self-similarity under zooming.
\end{egg}

\begin{theorem}
For $\mu > 4$, $\Delta_\mu$ is a cantor set.
\end{theorem}

\subsection{Chaos}

The next two definitions are the bases for the definition of chaos.

\begin{defn}
Let $I$ be an interval of $\RR$ and $f:I\rightarrow I$ be a
continuous function.  $f$ is
{\bfseries topologically transitive}\index{Topologically Transitive} if
for any open sets $V$ and $W$ in $I$ there exist $k>0$ such that
$f^k(V) \cap W \neq \emptyset$.
\end{defn}

\begin{defn}
Let $I$ be an interval of $\RR$ and $f:I\rightarrow I$ be a
continuous function.  $f$ has
{\bfseries sensitive dependence on initial conditions}\index{Sensitive
Dependence on Initial Conditions} if there exists
$\delta>0$ such that, for any $x \in I$ and neighbourhood $N \subset I$
of $x$, there exist $y\in N$ and $k>0$ satisfying
$\left| f^k(x) - f^k(y) \right| > \delta$.
\end{defn}

\begin{defn}[Chaos]
Let $I$ be an interval of $\RR$ and $f:I\rightarrow I$ be a
continuous function.  $f$ is said to be
{\bfseries chaotic}\index{Chaotic} on $I$ if
\begin{enumerate}
\item $f$ has sensitive dependence on initial conditions.
\item $f$ is topologically transitive.
\item The set of all periodic points of $f$ is dense in $I$ (every
  non-periodic point of $I$ is the limit of some periodic points).
\end{enumerate}
\label{chaos}
\end{defn}

\begin{rmk}
It has been proved in \cite{BBCDS} that 2 and 3 implies 1.
Nevertheless, we keep the tradition of using Definition~\ref{chaos} as
the definition of chaos because it lists three of the must important
properties of a chaotic function.  Moreover, it has been proved in
\cite{AG} that 1 and 3 do not imply 2, and 1 and 2 do not imply 3.
\end{rmk}

\begin{egg}
The logistic map $f_4:I \to I$, where $I=[0,1]$, is chaotic. 
\end{egg}

We may expand our definition of attracting and repelling periodic
orbits to more general sets.

\begin{defn}
Let $V$ be a subset of $\RR$ and $f:\RR\rightarrow \RR$ be a
continuous function.  $V$ is an
{\bfseries attracting}\index{Hyperbolic Set!Attracting}
(respectively {\bfseries repelling}\index{Hyperbolic Set!Repelling})
{\bfseries hyperbolic set} if
\begin{enumerate}
\item $V$ is closed and bounded.
\item $V$ is {\bfseries invariant}\index{Invariant Set} under $f$
(i.e.\ $f(V) \subset V$).
\item There exists $N>0$ such that
$\displaystyle \left| \dydx{f^n}{x}(x)  \right| < 1$
(respectively $>1$) for all $n \geq N$ and $x \in V$.
\end{enumerate}
\end{defn}

\begin{egg}
It can be proved that for $\mu>2+\sqrt{5}$, $\Delta_\mu$ is a repelling
hyperbolic set for the logistic map $f_\mu$.  The behaviour of
$f_\mu:\Delta_\mu \to \Delta_\mu$ is a lot more complex than we may
imagine.  $f_\mu$ has a dense orbit in $\Delta_\mu$.  Moreover,
$f_\mu:\Delta_\mu \to \Delta_\mu$ is choatic\footnote{The definition
of choatic map can be extended to any topological space $I$, not just
intervals.}.
\end{egg}

%%% Local Variables: 
%%% mode: latex
%%% TeX-master: "notes"
%%% End: 


\section{Exercises}

\begin{question}
Find small intervals containing the solutions (one solution per
interval) of $4x^2 - e^x = 0$.  Do not forget to justify your answer.
\label{solvAQ1}
\end{question}

\begin{question}
Use the bisection method to find an approximation of $\sqrt[3]{25}$
correct to within $10^{-4}$.
\label{solvAQ2}
\end{question}

\begin{question}
In the algorithm for the bisection method, Algorithm~\ref{BisectAlgo},
if $a_0 >0$ and
\begin{equation}\label{first}
n \geq \frac{\ln(b_0-a_0)-\ln(\epsilon)-\ln(a_0)}{\ln(2)}
\end{equation}
Show that the $n^{th}$ iteration $x_n$ of the bisection method 
is an approximation of a root $r$ with a relative error smaller than
$\epsilon$.
\label{solvAQ3}
\end{question}

\begin{question}
In the algorithm for the bisection method, Algorithm~\ref{BisectAlgo},
show that $| x_n - x_{n+1}| = 2^{-n-1}(b_0 - a_0)$.
\label{solvAQ4}
\end{question}

\begin{question}
In the algorithm for the bisection method, Algorithm~\ref{BisectAlgo},
is it possible to have $a_n <a_{n+1}$ (strict inequalities) for all
$n$?  If it is possible, give the conditions under which it is
possible.  If it is not possible, prove it.
\label{solvAQ5}
\end{question}

\begin{question}
Find a solution accurate to within $10^{-4}$ for $x=\tan(x)$ on
$\pi/2 < x < 3 \pi/2$.\\
\subQ{a} Use the bisection method.\\
\subQ{b} Use the fixed point method.\\
\subQ{c} Use Newton's method.
\label{solvAQ6}
\end{question}

\begin{question}
Find the solutions accurate to within $10^{-5}$ for $x^2 + 11\cos(x)=0$.\\
\subQ{a} Use the bisection method.\\
\subQ{b} Use the fixed point method.\\
\subQ{c} Use Newton's method.
\label{solvAQ7}
\end{question}

\begin{question}
Find the smallest value $x_0>0$ such that the Newton's method for
$f(x) = \arctan(x)$ does not converge.
\label{solvAQ8}
\end{question}

\begin{question}
For which functions $f$ is the iterative equation
\[
x_{n+1} = 2x_n - C x_n^2
\]
the result of the formula for the Newton's method?  $C$ is a
constant.
\label{solvAQ9}
\end{question}

\begin{question}
If $x_0 = 0$ and
\[
x_{n+1} = x_n - \frac{\tan(x_n) - 1}{\sec^2(x_n)}
\]
for $n=0$, $1$, $2$, \ldots\  Without doing any computation, find the
limit of this sequence.
\label{solvAQ10}
\end{question}

\begin{question}
Use Newton's method to find an approximation of a root of
$f(x) = \tan(x)$ in the interval $[4.8, 7.7]$ with an accuracy of
$10^{-8}$.
\label{solvAQ11}
\end{question}

\begin{question}
Use the secant method to find an approximation of the first positive
root of $f(x) = e^x - \tan(x)$ with an accuracy of $10^{-8}$.\\
Hint: To choose $x_0$ and $x_1$, draw the graph of $e^x$ and $\tan(x)$.
\label{solvAQ12}
\end{question}

\begin{question}
\subQ{a} Suppose that the Newton's method is used to generate a sequence
$\{x_n\}_{n=0}^\infty$ converging to a root $r$ of a function $f$. 
Let $e_n = x_n-r$.  Show that
\[
e_{n+1} = \frac{f''(\xi_n)}{2f'(x_n)}\, e_n^2
\]
for some $\xi_n$ between $r$ and $x_n$.\\
\subQ{b} Let $f(x) = x-e^{-x}$ and assume that the Newton's method is
used to generate a sequence $\{x_n\}_{n=0}^\infty$ converging to the
root $r$ of $f$ in the interval $[0,1]$.  If $e_n =x_n-r$, show that 
\begin{equation}\label{NMconvQuest}
|e_n| \leq 2\left(\frac{e_0}{2}\right)^{2^n}
\end{equation}
for $n \geq 0$ whenever $x_0\geq 0$.\\
\subQ{c} If $x_0=1$ in (b), how many iterations of the Newton's method
will be sufficient to get an approximation of the root $r$ of $f$ with
an accuracy of $10^{-5}$; namely, such that $|e_n| < 10^{-5}$.
\label{solvAQ13}
\end{question}

\begin{question}
Use Newton's method with Horner's Algorithm to approximate the
three roots of $f(x) = x^3 -x$; namely, to approximate $p_1=-1$,
$p_2=0$ and $p_3=1$.  For each value of $i$, can you find a subinterval
$I_i$ of $[-0.451,-0.446]$ such that the Newton's method with Horner's
Algorithm converges to the root $p_i$ of $f(x)$?  For each $i$,
the subsets of the real line containing the points $x_0$ such that
the Newton's method converges to $p_i$ form a Cantor type of set.
\label{solvAQ14}
\end{question}

\begin{question}
Suppose that $g:[a,b] \rightarrow [a,b]$ satisfies the Fixed Point
Theorem and $g'(x) < 0$ for all $x \in [a,b]$.  Describe the behaviour
of the sequence $\{ x_n\}_{n=0}^\infty$ given by $x_{n+1} = g(x_n)$
for $x_0 \in [a,b]$ as it converges to the fixed point.  You may want
to sketch a typical graph.
\label{solvAQ15}
\end{question}

\begin{question}
Let $\displaystyle g(x) = \frac{1}{x^2+1}$.

\subQ{a} Show that $g$ has a unique fixed point in the interval $[0,1]$.\\
\subQ{b} Show that we can use the Fixed Point Theorem to find the
fixed point of $g$ in the interval $[0,1]$.\\
\subQ{c} Determine the order of convergence of this fixed point method.
\label{solvAQ16}
\end{question}

\begin{question}
Consider the function $g(x)=2^{-x}$.\\
\subQ{a} Show that you can use the Fixed Point Theorem to approximate
the fixed point of $g$ in the interval $[1/3,1]$.\\
\subQ{b} Find a small value of $n$ ensuring to the approximation $x_n$
of the fixed point of $g$ has an accuracy of $10^{-4}$.  You may
assume that $x_0=0.5$\ .\\
\subQ{c} Use the Fixed Point Theorem to find an approximation
$x_{n+1}$ to the fixed point of $g$ in the interval $[1/3,1]$ such
that $|x_{n+1} - x_n| < 10^{-4}$.  As in (b), you may assume that
$x_0=0.5$.
\label{solvAQ17}
\end{question}

\begin{question}
Consider the function
\[
g(x) = 12 - \frac{20}{x} \ .
\]
\subQ{a} Explain why this function has two fixed points.\\
\subQ{b} Using the Fixed Point Theorem, show that $g$ 
has a unique fixed point $p$ in the interval $[9.5,11.5]$, and that
the sequence $\displaystyle \{x_n\}_{n=0}^\infty$ generated by
$x_{n+1} = g(x_n)$ converge to $p$ whatever the choice
$x_0\in [9.5,11.5]$.\\
\subQ{c} How many iterations are needed to get an approximation of the
fixed point of $g$ in the interval $[9.5,11.5]$ with an accuracy of
$10^{-7}$?  You may assume that $x_0 = 9.5$.\\
\subQ{d} What is the order of convergence of the sequence
$\{x_n\}_{n=0}^\infty$ that is generated by $x_{n+1}= g(x_n)$ with $x_0$ 
in $[9.5,11.5]$.\\
\subQ{e} Use Steffensen's Algorithm to find an
approximation of the fixed point of $g$ in the interval $[9.5,11.5]$
with an accuracy of $10^{-7}$.  Use $x_0 = 9.5$.  Why does this
method converge faster than the fixed point method?
\label{solvAQ18}
\end{question}

\begin{question}
Let $f(x) = e^x -2x-1$.

\subQ{a} Show that $f$ has a unique root in the interval $[1,2]$.\\
\subQ{b} Show that a root of $f$ is a fixed point of
$g(x) = \ln(1+2x)$ and vice-versa.\\
\subQ{c} Show that for any $x_0\in [1,2]$, the sequence
$\displaystyle \{x_n\}_{n=0}^\infty$ generated by $x_{n+1} = g(x_n)$
for $n=0$, $1$, $2$, \ldots\ converges to the fixed point of $g$ in
the interval $[1,2]$.\\
\subQ{d} Determine the order of convergence of this fixed point method.
\label{solvAQ19}
\end{question}

\begin{question}
Our goal is to approximate the value of $\sqrt[3]{25}$ using the Fixed
Point Theorem.\\
\subQ{a} Show that $\sqrt[3]{25}$ is the unique root of $f(x) = x^3-25$.\\
\subQ{b} Show that $p>0$ is a root of $f$ if and only if $p$ is a
fixed point of $\displaystyle g(x) = 5/\sqrt{x}$, and conclude that
this fixed point $p$ is $\sqrt[3]{25}$. \\
\subQ{c} Using the graph of $g$, give an interval $[a,b]$ with $a>0$
such that $g$ satisfies the Fixed Point Theorem on $[a,b]$.  Verify
that $g$ satisfies all the hypotheses of the Fixed Point Theorem.\\
\subQ{d} Choose $x_0$ in the interval $[a,b]$ that you have found in (c).
Without doing any iteration, find a small value of $n$ such that $x_n$,
the $(n+1)^{th}$ term in the sequence $\{x_n\}_{n=0}^\infty$ produced by
the Fixed Point Theorem applied to the function $g$, is an approximation of 
$p$ with an accuracy of $10^{-5}$.\\ 
\subQ{e} Use the Fixed Point Theorem to find an approximation $x_n$ to
the fixed point of $g$ in the interval $[a,b]$ such that
$|x_n - x_{n-1}| < 10^{-5}$.  Use the $x_0$ that you have chosen in (d).
\label{solvAQ20}
\end{question}

\begin{question}
Let $f(x) = e^{2-x} -x^2$.

\subQ{a} Show that $f$ has a unique root $p$ in the interval $[1,2]$.\\
\subQ{b} Find a function $g$ satisfying all the hypotheses of the Fixed
Point Theorem such that a root of $f$ in the interval $[1,2]$ is a
fixed point of $g$ in $[1,2]$.  Verify that your function $g$ satisfies
the hypotheses of the Fixed Point Theorem.\\
\subQ{c} Let $x_0= 1$.  Without doing any iteration, find a small value
of $n$ such that $x_n$, the $(n+1)^{th}$ term in the sequence
$\{x_n\}_{n=0}^\infty$ produced by the Fixed Point Theorem applied to
the function $g$, is an approximation of $p$ with an accuracy of
$10^{-5}$.\\
\subQ{d} Determine the order of convergence of this fixed point
method.
\label{solvAQ21}
\end{question}

% g(x) = e^{1-x/2}, g'(x) = -e^{1-x/2} / 2
% g(1) = 1.64872...  , g(2) = 1
% g'(x) <= g'(1) < 0.825  for 1 <=x <= 2
% 0.825^n < 10^{-5} (1-0.825)/(1.649-1)  -->  n > 66.66

\begin{question}
The first positive value $p$ such that $p=\tan(p)$ is between $\pi$
and $3\pi/2$.  Let
\[
g(x) = \pi + \arctan(x) \ .
\]
\subQ{a} Show graphically that $g$ has a unique fixed point in
$[\pi,3\pi/2]$ and that it is the point $p$ above.\\
\subQ{b} Show that $g$ satisfies the hypotheses of the Fixed Point
Theorem on the interval $[\pi,3\pi/2]$.\\
\subQ{c} Without doing any iteration, find a small value
of $n$ such that $x_n$, the $(n+1)^{th}$ term in the sequence
$\{x_n\}_{n=0}^\infty$ produced by the Fixed Point Theorem applied to
the function $g$, is an approximation of $p$ with an accuracy of
$10^{-5}$.
\label{solvAQ22}
\end{question}

\begin{question}
Let $a$ be a positive number and
\begin{equation}\label{classicAnal}
g(x) = \frac{x}{2} + \frac{a}{2x} \ .
\end{equation}
Given $x_0>0$, let $\displaystyle \{x_n\}_{n=0}^{\infty}$ be the
sequence generated by $x_{n+1} = g(x_n)$ for $n=0$, $1$, $2$, \ldots

\subQ{a} Show that the positive fixed point of $g$ is $\sqrt{a}$.\\
\subQ{b} Use the Fixed Point Theorem to prove that for any $x_0 > 0$
the sequence $\displaystyle \{x_n\}_{n=0}^{\infty}$ converges to the
unique positive fixed point of $f$.\\
Hint: Show first that if $0<x_0<\sqrt{a}$, then $x_1 \geq \sqrt{a}$.
Then show that $g$ satisfies the Fixed Point Theorem on any interval
of the form $[\sqrt{a},m]$.
\label{solvAQ23}
\end{question}

\begin{question}
\subQ{a} If $f'$ is continuous and positive on $[a,b]$, and
$f(a)f(b)<0$, prove that $f$ has a unique zero in the open interval
$]a,b[$.\\
\subQ{b} Find $\lambda$ such that the sequence
$\displaystyle \{x_n\}_{n=0}^\infty$ generated by the iteration
$x_{n+1} = x_n +\lambda f(_n)$ for $n=0$, $1$, $2$, \ldots\ converges
to a zero of $f$.
\label{solvAQ24}
\end{question}

\begin{question}
Suppose that $g$ is a continuously differentiable function on an
interval $[a,b]$.  Let $m=(a+b)/2$ be the middle point of the interval
$[a,b]$.  If $|g'(x)| < 1$ for all $x\in [a,b]$ and
$g(m)=m$, prove or disprove that the sequence $\{x_n\}_{n=0}^\infty$
defined by $x_{n+1} = g(x_n)$ converges to the fixed point $m$ of $g$ in
$[a,b]$ whatever the choice of $x_0 \in[a,b]$.
\label{solvAQ25}
\end{question}

\begin{question}
Suppose that $g:\RR\rightarrow \RR$ is a continuously differentiable
function and that $p$ is a fixed point of $g$ such that $|g'(p)| > 1$.
Prove or disprove that for all $x_0\in \RR$ the sequence
$\{x_n\}_{n=0}^\infty$ does not converge to $p$.  If there are
sequences $\{x_n\}_{n=0}^\infty$ that converge to $p$, describe all of
them.
\label{solvAQ26}
\end{question}

\begin{question}
Suppose that $|g'(x)| \leq \lambda < 1$ for all
$x \in [x_0-\rho,x_0+\rho]$, where
$\displaystyle \rho = \frac{|g(x_0)-x_0|}{1-\lambda}$.
Prove that the sequence $\{x_n\}_{n=0}^\infty$ defined by
$x_{n+1} = g(x_n)$ for $n \geq 0$ converges to a fixed point of $g$ in
the interval $[x_0-\rho,x_0+\rho]$.
\label{solvAQ27}
\end{question}

\begin{question}
Prove or disprove that if $f$ is a contraction on $[a,b]$, then $f$
has a unique fixed point and the iterative system $x_{n+1} = f(x_n)$
for $n \geq 0$ yields a sequence $\{x_n\}_{n=0}^\infty$
that converges toward this root whatever the choice of $x_0 \in [a,b]$. 
\label{solvAQ28}
\end{question}

\begin{question}
Suppose that $f$ is $m$ times continuously differentiable.  Show that
$f(x) = (x-p)^m q(x)$ with $q(p) \neq 0$  if and only if
$f(p) = f'(p) = ... = f^{(m-1)}(p) = 0$ and $f^{(m)}(p) \neq 0$; namely,
if and only if $f$ has a zero of multiplicity $m$ at $p$.
\label{solvAQ29}
\end{question}

\begin{question}
Let $F(x) = x - f(x)f'(x)$, where $f$ is a three times continuously
differentiable function satisfying $f(r)=0$ and $f'(r)\neq 0$.
Find the conditions on $f$ to obtain an iterative method
$x_{n+1} = F(x_n)$ for $n \geq 0$ that generates sequences
converging toward $r$ and such that the convergence is of order
exactly three.
\label{solvAQ30}
\end{question}

\begin{question}
Let $F(x) = x + f(x)g(x)$, where $f$ and $g$ are sufficiently
continuously differentiable functions.  Moreover, assume that $f$
satisfies $f(r)=0$ and $f'(r)\neq 0$.  Find the conditions on $g$ to
obtain an iterative method $x_{n+1} = F(x_n)$ for $n\geq 0$ that
generates sequences converging toward $r$ and such that the order of
convergence is exactly three.
\label{solvAQ31}
\end{question}

\begin{question}
Which of the following sequences converge quadratically?
\begin{center}
\begin{tabular}{*{2}{l@{\hspace{1em}}l@{\hspace{2.5em}}}l@{\hspace{1em}}l}
\subQ{a} & $\displaystyle \left\{\frac{1}{n^2}\right\}_{n=1}^\infty$ &
\subQ{b} & $\displaystyle \left\{\frac{1}{2^{2^n}}\right\}_{n=0}^\infty$ &
\subQ{c} & $\displaystyle \left\{\frac{1}{\sqrt{n}}\right\}_{n=1}^\infty$ \\
\subQ{d} & $\displaystyle \left\{\frac{1}{e^n}\right\}_{n=0}^\infty$ &
\subQ{e} & $\displaystyle \left\{\frac{1}{n^n}\right\}_{n=1}^\infty$ & &
\end{tabular}
\end{center}
\label{solvAQ32}
\end{question}

\begin{question}
\subQ{a} Show that the convergence of the sequence $p_n = 10^{-k^n}$
to $0$ is of order $k$.\\
\subQ{b} Show that the sequence $p_n = 10^{-n^k}$ does not converge to
$0$ quadratically regardless of the size of the exponent $k > 1$.
\label{solvAQ33}
\end{question}

\begin{question}
Solve $x - 2^{-x} = 0$ for $x \in [0,1]$ with an accuracy of $10^{-4}$
using Steffensen's Algorithm.
\label{solvAQ34}
\end{question}

\begin{question}
Use the method of deflation to approximate all the roots of
\[
p(x) = x^3 - 5.974925987\,x^2 + 9.734512519\,x -2.617993878
\]
with an accuracy of $10^{-10}$.  Do not used any formula to computer
the roots of a polynomial of degree two.
\label{solvAQ35}
\end{question}

\begin{question}
Use the method of deflation to approximate all the roots of
$p(x) = x^4 -2 x^3 -12 x^2 + 16 x -40$ with an accuracy of $10^{-9}$.
You must use Newton's method with Horner's Algorithm.
\label{solvAQ36}
\end{question}

\begin{question}
Use the method of deflation to approximate all the roots of
$x^3 - 53 x^2 + 151 x - 3$ with an accuracy of $10^{-3}$.
You must use Newton's method with Horner's Algorithm.
\label{solvAQ37}
\end{question}

% p(x) = (x-3) q(x)  where  q(x) = x^2 -50 x + 1
% Two iterations of Newton's method applied to q(x) with x_0 = 0 gives
% 0.02000800640512 .
% Iterations of Newton's method applied to p(x) with x_0 = 0.02000800640641
% Exact value 0.02000800640641...

\begin{question}
Use the method of deflation to approximate all the roots of
\[
x^4-10.07251864 x^3+34.83068793 x^2-44.63745043 x+11.36978427
\]
with an acuracy of $10^{-5}$.
You must use Newton's method with Horner's Algorithm.
\label{solvAQ38}
\end{question}
% (x-3.456)(x-\pi)^2(x-0.3333333333333)

%%% Local Variables: 
%%% mode: latex
%%% TeX-master: "notes"
%%% End: 
