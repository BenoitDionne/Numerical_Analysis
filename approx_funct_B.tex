\chapter{Uniform Approximation} \label{chaptApproxB}

Suppose that $f:\RR \to \RR$ is a sufficiently differential function.
The Taylor expansion of the function $f$ at a point $c \in \RR$ is
given by
\[
  f(x) = p_n(x) + r_n(x) \ ,
\]
where
\begin{align*}
p_n(x) &= f(c) + f'(c)(x-c) + \frac{f'(c)}{2!}(x-c)^2 + \ldots
+ \frac{f^{(n)}(c)}{n!} \, (x-c)^n\\
\intertext{and}
r_n(x) &= \frac{f^{(n+1)}(\xi(x,c))}{(n+1)!}(x-c)^{n+1}
\end{align*}
for some $\xi(x,c)$ between $x$ and $c$.  If $c \in [a,b]$ and
$\displaystyle \left|f^{(n+1)}(x)\right| < M$ for $[a\leq x \leq b$,
then we get that
\[
  \sup_{a\leq x \leq b} |f(x) - p_n(x)| \leq \frac{M}{(n-1)!}(b-a)^{n+1} \ .
\]
This is an uniform approximation of $f$ by a polynomial $p_n$.  If $M$
is small and $b-a \leq 1$, then $p$ can be a good uniform
approximation of $f$.  So, instead of evaluating $f(x)$, it may be
simpler and sufficiently accurate to evaluate $p_n(x)$.
Unfortunately, in practice, it may be very hard to compute the
derivatives of $f$ and to find an upper bound on $|f^{n+1}|$.
Moreover, the interval $[a,b]$ may be of length greater than $1$ and
so $(b-a)^{n+1} \to \infty$ as $n \to \infty$.

Therefore, it is still preferable to use interpolation polynomials
as presented in Chapter~\ref{chaptInterA} to approximate $f$.
However, among all the possible interpolating polynomials, it is
possible to choose the points of interpolation to minimize the degree
of the interpolating polynomial and the error.
This is the major result of this chapter that we present in
Section~\ref{bestInterPol} below.

\section{Stone-Weierstrass Theorem}

The fundamental theorem in this chapter is the following.

\begin{theorem}[Stone-Weierstrass]
Given a continuous function $f:[a,b] \to \CC$, there exists a sequence
of polynomials $\{p_n\}_{n=0}^\infty$ over $\CC$ such that
\[
  \max_{a\leq x \leq b} |f(x)-p_n(x)| \to 0 \quad \text{as}
  \quad n \to \infty \ .
\]
If $f:[a,b]\to \RR$, the polynomials can be assumed to be over $\RR$.
\label{SWtheorem}
\end{theorem}

Basically, the theorem states that for any continuous function
$f:[a,b]\to \CC$, we can find a sequence of polynomials converging to
$f$ uniformly on $[a,b]$.  We will not prove this theorem.  There
exist many proofs of it.  The reader can find one of them in any good
analysis textbook.

\section{Chebyshev Polynomials} \label{appr_Cheb_sect}

We have seen in Example~\ref{appr_cheb_egg} that the Chebyshev polynomials
were defined by
\[
T_{n+1}(x) = 2x T_n(x) - T_{n-1}(x)
\]
for $n = 1$, $2$, $3$, \ldots\ with $T_0(x) = 1$ and $T_1(x) = x$.
The tradition is to denote the Chebyshev polynomials with the letter
$T$ instead of $P$ because the translation from Russian to French of
Chebyshev is Tchébyshev.

There is an equivalent way to define the Chebyshev polynomials from which it
is easier to deduce some of the properties of the Chebyshev polynomials.

The Chebyshev polynomial $T_n$ is also defined by
\begin{equation} \label{appr_Cheb_defn}
T_n(x) = \cos( n \arccos(x)) \quad , \quad -1 \leq x \leq 1 \ .
\end{equation}
To verify that this is true, we note that $T_0(x) = \cos(0) = 1$ and
$T_1(x)=\cos(\arccos(x)) = x$ for $x\in[-1,1]$.  Moreoevr, it follows
from the addition formulae for the cosine function that
\begin{align*}
\cos((n+1)\theta) &= \cos(n\theta+\theta) = \cos(n\theta) \cos(\theta)
- \sin(n\theta)\sin(\theta)
\intertext{and}
\cos((n-1)\theta) &= \cos(n\theta-\theta) = \cos(n\theta) \cos(\theta)
+\sin(n\theta)\sin(\theta) \ .
\end{align*}
Hence
\[
\cos((n+1)\theta) + \cos((n-1)\theta) = 2 \cos(n\theta) \cos(\theta) \ .
\]
If we substitute $\theta = \arccos(x)$ in this last equation, we get the
recurrence relation
\begin{equation} \label{appr_cheb_rec}
T_{n+1}(x) = 2x T_n(x) - T_{n-1}(x)
\end{equation}
for $n = 1$, $2$, $3$, \ldots used to defined the Chebyshev
polynomials.  Thus (\ref{appr_Cheb_defn}) is another way to define the
Chebyshev polynomials.

We know from Theorem~\ref{orthpoly} that the Chebyshev polynomials are
orthogonal on $L^2[-1,1]$, where the weight function is
$\displaystyle w(x) = \frac{1}{\sqrt{1-x^2}}$.  This can be directly proved
from (\ref{appr_Cheb_defn}).  Using the substitution
$\theta= \arccos(x)$ for $-1<x<1$ and
$\displaystyle \dx{\theta} = \frac{-1}{\sqrt{1-x^2}}\dx{x}$, we get
\begin{align*}
\int_{-1}^1 \frac{T_i(x)T_j(x)}{\sqrt{1-x^2}} \dx{x} &
= \int_{-1}^1 \frac{\cos(i\arccos(x))\cos(j\arccos(x))}{\sqrt{1-x^2}} \dx{x}
= - \int_{\pi}^0 \cos(i\theta)\cos(j\theta) \dx{\theta} \\
&= \int_0^\pi \left(\frac{1}{2}\cos((i+j)\theta) +
  \frac{1}{2}\cos((i-j)\theta) \right)\dx{\theta} \\
&= \left(\frac{\sin((i+j)\theta)}{2(i+j)} +
  \frac{\sin((i-j)\theta)}{2(i-j)}\right) \bigg|_{\theta=0}^\pi
= 0
\end{align*}
for $i\neq j$.  The Chebyshev polynomials are not of norm one
because, using the substitution for the previous integral, we have
\begin{align*}
\int_{-1}^1 \frac{T_i^2(x)}{\sqrt{1-x^2}} \dx{x}
&= \int_{-1}^1 \frac{\cos^2(i\arccos(x))}{\sqrt{1-x^2}} \dx{x}
= - \int_{\pi}^0 \cos^2(i\theta) \dx{\theta} \\
&= \int_0^\pi \frac{1}{2}\left(1+ \cos(2i\theta) \right)\dx{\theta}
= \frac{1}{2} \left(\theta - \frac{1}{2i}\sin(2i\theta)
\right)\bigg|_{\theta=0}^\pi = \frac{\pi}{2}
\end{align*}
for  $i\neq 0$, and
\[
\int_{-1}^1 \frac{T_0^2(x)}{\sqrt{1-x^2}} \dx{x}
= \int_{-1}^1 \frac{1}{\sqrt{1-x^2}} \dx{x} = - \int_\pi^0 \dx{x}
= \int_0^\pi \dx{x} = \pi \  .
\]

\begin{prop}
For $n\geq 1$, $T_n$ has $n$ distinct roots and they are in the interval
$[-1,1]$.  These roots are
\[
r_i = \cos\left( \frac{(2i-1)\pi}{2n}\right) \quad , \quad
i=1,2,3,\ldots,n \ .
\]
\end{prop}

\begin{proof}
It is easy to verify by substituting in (\ref{appr_Cheb_defn}) that the
$r_i$'s are $n$ distinct roots of $T_n$.  The $r_i$'s are in the interval
$[-1,1]$ since $-1<\cos(\theta)<1$ for all $\theta \neq n\pi$.  Since $T_n$
is a polynomial of degree $n$, it has no more roots according to the
fundamental theorem of algebra.
\end{proof}

\begin{prop}
For $n>0$, $T_n$ reaches its absolute extrema in the interval $[-1,1]$ at the
points
\[
s_i = \cos\left(\frac{i\pi}{n}\right) \quad , \quad i=0,1,2,\ldots, n \ .
\]
Moreover, $T_n(s_i) = (-1)^i$.
\end{prop}

\begin{proof}
We have $\displaystyle T_n'(x) = \frac{n\sin(n\arccos(x))}{\sqrt{1-x^2}}$.
Since $T_n'(s_i) = 0$ for $0<i<n$, the $s_i$'s for $0<i<n$ are critical
points of $T_n$.  Since $T_n'$ is a polynomial of degree $n-1$, it has at
most $n-1$ roots.  Thus $T_n$ has exactly $n-1$ critical points in $]-1,1[$
given by $s_i$ for $0<i<n$.  These critical points are the only points in the
interval $]-1,1[$ where $T_n$ reaches its extrema.  A direct computation
shows that $T_n(s_i) = \cos(i\pi) = (-1)^i$.

The other two possible points where $T_n$ may reach its extrema are at the
endpoints $-1$ and $1$.  Since $T_n(-1) = \cos(n\pi) = (-1)^n$ and
$T_n(1) = \cos(0) = 1$, the endpoints are also two points where $T_n$
reaches its extrema.
\end{proof}

\begin{defn}
The {\bfseries monic Chebyshev polynomials}\index{Monic Chebyshev
Polynomials} $\tilde{T}_n$ for $n\geq 0$ are 
defined by $\displaystyle \tilde{T}_0(x) = 1$ and
$\displaystyle \tilde{T}_n(x) = \frac{1}{2^{n-1}} T_n(x)$ for $n >0$.
\end{defn}

\begin{rmk}
The recurrence relation (\ref{appr_cheb_rec}) becomes
$\displaystyle \tilde{T}_2(x) = x \tilde{T}_1(x) - \frac{1}{2} \tilde{T}_0(x)$
and
$\displaystyle \tilde{T}_{n+1}(x) = x \tilde{T}_n(x) - \frac{1}{4}
\tilde{T}_{n-1}(x)$ for $n>1$.  Using these relations, it follows by
induction on the degree of the polynomials that the coefficient of
$x^n$ in $\tilde{T}_n(x)$ is $1$, hence the name monic given to the
polynomials $\tilde{T}_n$.

The roots of $\tilde{T}_n$ are the roots $r_i$ of $T_n$.
$\tilde{T}_n$ and $T_n$ reach their extrema at the same points;
namely, at $s_i$ for $0\leq i \leq n$.  However,
$\displaystyle \tilde{T}_n(s_i) = \frac{(-1)^i}{2^{n-1}}$ for
$0\leq i \leq n$ and $n>0$.
\end{rmk}

\begin{prop}
Let $\tilde{\Pi}_n$ be the set of all monic polynomials of degree exactly
$n$.  We have
\[
\frac{1}{2^{n-1}} = \max_{-1\leq x \leq 1} \left| \tilde{T}_n(x)\right| \leq
\max_{-1\leq x \leq 1} \left| p(x) \right| \quad , \quad p \in \tilde{\Pi}_n
\ .
\]
We have equality for $p = \tilde{T}_n$.
\label{appr_nCheb_max}
\end{prop}

\begin{proof}
Suppose that $p\in \tilde{\Pi}_n$ satisfies
\begin{equation} \label{appr_cheb_min}
\max_{-1\leq x \leq 1} \left|p(x)\right| \leq \frac{1}{2^{n-1}} \  .
\end{equation}
Since $p$ and $\tilde{T}_n$ are both monic of degree $n$, we have that
$q = \tilde{T}_n-p$ is a polynomial of degree at most $n-1$.  Moreover,
for $0 \leq i \leq n$, we have
\[
  q(s_i) = \frac{(-1)^i}{2^{n-1}} - p(s_i) \leq 0
\]
if $i$ is odd and
\[
  q(s_i) = \frac{(-1)^i}{2^{n-1}} - p(s_i) \geq 0
\]
if $i$ is even because of (\ref{appr_cheb_min}).  By the Intermediate
Value Theorem, $q$ has 
at least one root between $s_i$ and $s_{i+1}$ for $0\leq i < n$.  Hence, $q$
is a polynomial of degree $n-1$ with at least $n$ roots.  The only
possibility is if $q(x) = 0$ for all $x \in [-1,1]$.
\end{proof}

In Item~\ref{appr_interp_rmk_lbl} of Remark~\ref{appr_interp_rmk}, we said
that Chebyshev points adjusted to the interval $[a,b]$ were the ``best''
choice of interpolatory points for Lagrange interpolation on the interval
$[a,b]$.  We now justify this statement.

Without loss of generality, we may assume that $[a,b]=[-1,1]$.
If $q$ is the Lagrange interpolating polynomial of a sufficiently
differentiable function $f$ on the interval $[-1,1]$ and
$0\leq x_0 < x_1 < \ldots < x_n\leq 1$ are the interpolatory points, then the
error is given by
\[
\left| f(x) - q(x) \right| = \frac{1}{(n+1)!}\  f^{(n+1)}(\xi(x))
\prod_{i=0}^n (x-x_i)
\]
for $-1 \leq x \leq 1$, where $\xi:[-1,1]\rightarrow [-1,1]$.  If we
assume that $f^{(n+1)}$ is (almost) constant on the interval $[-1,1]$,
then we have to minimize
$\displaystyle p(x) = \prod_{i=0}^n (x-x_i)$ for $-1 \leq x \leq 1$
to minimize the error.  Note that $p$ is a monic polynomial of degree
$n+1$, hence
\[
\frac{1}{2^n} = \max_{-1\leq x \leq 1} \left| \tilde{T}_{n+1}(x)\right| \leq
\max_{-1\leq x \leq 1} \left| p(x) \right| \quad , \quad p \in
\tilde{\Pi}_{n+1} \  ,
\]
according to the previous proposition.  We have equality when
$p=\tilde{T}_{n+1}$; namely, when
$\displaystyle x_i = \cos\left(\frac{(2i-1)\pi}{2(n+1)}\right)$
for $1\leq i \leq n+1$.  These are the roots of $T_{n+1}$ which were
called Chebychev points in Remark~\ref{appr_interp_rmk}.

\begin{prop}
Let $f$ be a sufficiently differentiable function $f$ defined on the interval
$[-1,1]$.  If $q$ is the Lagrange interpolating polynomial of $f$ at the
Chebyshev points
$\displaystyle x_i = \cos\left(\frac{(2i-1)\pi}{2(n+1)}\right)$
for $1\leq i \leq n+1$, then
\[
\max_{-1\leq x \leq 1} \left| f(x) - q(x) \right| \leq
\frac{1}{2^n (n+1)!}\max_{-1\leq x \leq 1} \left| f^{(n+1)}(x) \right| \  .
\]
\label{ChebyshevTrancMax}
\end{prop}

\begin{proof}
We have
\[
f(x) - q(x) = \frac{1}{(n+1)!}\, f^{(n+1)}(\xi(x)) \,
\prod_{i=0}^n (x-x_i) = \frac{1}{(n+1)!}\, f^{(n+1)}(\xi(x))
\tilde{T}_{n+1}(x) \  .
\]
Hence,
\begin{align*}
\max_{-1\leq x \leq 1} \left| f(x) - q(x) \right| &= \frac{1}{(n+1)!}\, 
\max_{-1\leq x \leq 1} \left| f^{(n+1)}(\xi(x)) \tilde{T}_{n+1}(x) \right| \\
&\leq \frac{1}{(n+1)!}\, \max_{-1\leq x \leq 1} \left| f^{(n+1)}(\xi(x))\right|
\max_{-1\leq x \leq 1} \left| \tilde{T}_{n+1}(x) \right| \\
&= \frac{1}{2^n(n+1)!} \max_{-1\leq x \leq 1} \left| f^{(n+1)}(\xi(x))\right|
\ ,
\end{align*}
where the last equality comes from Proposition~\ref{appr_nCheb_max}.
\end{proof}

\subsection{How to reduce the Degree of an Interpolating
Polynomial with a  Minimal Loss of Accuracy} \label{bestInterPol}

Suppose that $\displaystyle q(x) = \sum_{j=0}^n a_j x^j$, where
$a_n \neq 0$, is an interpolating polynomial of a fonction $f$ on the
interval $[-1,1]$.  The goal is to find a polynomial $p$ of degree less than
$n$ such that
$\displaystyle \max_{-1\leq x \leq 1} \left| p(x) - q(x) \right|$ is as small
as possible.  If $p$ is of degree less than $n$, then
$\displaystyle (q-p)/a_n$ is a monic polynomial of degree $n$.  Hence,
\[
\max_{-1\leq x \leq 1} \left| q(x)-p(x)\right| =
|a_n| \max_{-1\leq x \leq 1} \left| \frac{q(x)-p(x)}{a_n} \right|
\geq |a_n| \max_{-1\leq x \leq 1} \left| \tilde{T}_n(x) \right|
= \frac{|a_n|}{2^{n-1}} \ .
\]
We have equality when
$\displaystyle \frac{q(x)-p(x)}{a_n} = \tilde{T}_n(x)$.  We should
therefore take $p = q - a_n \tilde{T}_n$.  For this choice,
\[
\max_{-1\leq x \leq 1} \left| q(x)-p(x)\right| = \frac{|a_n|}{2^{n-1}} \ .
\]

\section{Exercises}

\begin{question}
Consider $f(x) = x -\sin(x)$.  Find a small value of $n$ such that the
truncation error of the Taylor polynomial $p_n(x)$ of degree $n$ of
$f$ about the origin for does not exceeding $10^{-9}$ for $|x|<1$.
\label{approxBQ1}
\end{question}

\begin{question}
Consider $\displaystyle f(x) = \frac{1}{1-x}$.  Give the Taylor
polynomial $p_n$ of degree $n$ of $f$ about the origin as well as the
truncation formula for this polynomial.  Find a small value of $n$
such that $p_n$ uniformly approximates $f$ to within $10^{-6}$ on the
interval $[0,1/4]$.
\label{approxBQ2}
\end{question}

\begin{question}
Find the Taylor polynomial $p_2(x)$ of degree two about the origin for
the function $f(x)=e^x\cos(x)$.  Approximate $f(0.5)$ using
$p_2(x)$.  Find an upper bound on the error $|f(0,5)-p_2(0.5)|$ using
the truncation formula for the Taylor polynomial of degree two.
Compare this bound with the real error.
\label{approxBQ3}
\end{question}

%%% Local Variables:
%%% mode: latex
%%% TeX-master: "notes"
%%% End:
