\nonumsection{Chapter~\ref{chaptDiffInt} : Numerical Differentiation and Integration}

\solution{\SOL}{\ref{diffQ1}}{
The polynomial interpolation of degree at most $2$ at the points
$x_0$, $x_1$ and $x_2$ is
\begin{align*}
f(x) &= f[x_0] + f[x_0,x_1]\,(x-x_0) + f[x_0,x_1,x_2]\,(x-x_0)(x-x_1) \\
&+ f[x_0,x_1,x_2,x]\,(x-x_0)(x-x_1)(x-x_2) \ .
\end{align*}
If we derive, we get
\begin{align*}
f'(x) &= f[x_0,x_1] + f[x_0,x_1,x_2]\,\big((x-x_0)+(x-x_1)\big) \\
&\qquad + f[x_0,x_1,x_2,x]\,\big((x-x_1)(x-x_2) + (x-x_0)(x-x_2) +
(x-x_0)(x-x_1)\big) \\
&\qquad + f[x_0,x_1,x_2,x,x]\,(x-x_0)(x-x_1)(x-x_2) \ ,
\end{align*}
where we have used the formula
$\displaystyle \dfdx{f[x_0,x_1,x_2,x]}{x} = f[x_0,x_1,x_2,x,x]$.
Since
$\displaystyle f[x_0,x_1,x_2,x] = \frac{1}{3!}\dydxn{f}{x}{3}(\xi)$
and
$\displaystyle f[x_0,x_1,x_2,x,x] = \frac{1}{4!}\dydxn{f}{x}{4}(\eta)$
for some $\xi$ and $\eta$ in the smallest interval containing $x_0$,
$x_1$, $x_2$ and $x$, we get
\begin{align*}
f'(x) &= f[x_0,x_1] + f[x_0,x_1,x_2]\,\big((x-x_0)+(x-x_1)\big) \\
&\quad
+ \frac{1}{3!}\dydxn{f}{x}{3}(\xi)\,\big((x-x_1)(x-x_2) + (x-x_0)(x-x_2)
+ (x-x_0)(x-x_1)\big) \\
&\quad + \frac{1}{4!}\dydxn{f}{x}{4}(\eta)\,(x-x_0)(x-x_1)(x-x_2) \ .
\end{align*}
Each of the $x_i$ must be replaced by one of $a$, $a+h$ and $a+2h$ but
there is no obligation to have $x_0 < x_1 < x_2$.  We take
$x_0=a+2h$, $x_1 = a+h$ and $x_2 = a$.  Hence, for $x= a$, the previous
equation becomes
\begin{align*}
f'(a) &= f[a+2h,a+h] + f[a+2h,a+h,a]\,\big((-2h)+(-h)\big)
+ \frac{1}{3!}\dydxn{f}{x}{3}(\xi)\,\big((-2h)(-h)\big) \\
&= \frac{f(a+h)-f(a+2h)}{-h} +
\left(\frac{\displaystyle \frac{f(a)-f(a+h)}{-h}
- \frac{f(a+h)-f(a+2h)}{-h}}{-2h}\right)(-3h) \\
&\qquad + \frac{1}{3}\dydxn{f}{x}{3}(\xi)\,h^2 \\
&=\frac{-f(a+2h)+4f(a+h)-3f(a)}{2h} 
+ \frac{1}{3}\dydxn{f}{x}{3}(\xi)\,h^2 \ .
\end{align*}
We get (\ref{dfapproxA}) with the truncation error
$\displaystyle \frac{1}{3}\dydxn{f}{x}{3}(\xi)\,h^2$.
}

\solution{\SOL}{\ref{diffQ2}}{
The polynomial interpolation of degree at most $2$ of $f$ at the three
points $x_0$, $x_1$ and $x_2$ is
\begin{align*}
f(x) &= f[x_0] + f[x_0,x_1]\,(x-x_0) + f[x_0,x_1,x_2]\,(x-x_0)(x-x_1) \\
& \quad + f[x_0,x_1,x_2,x]\,(x-x_0)(x-x_1)(x-x_2) \ .
\end{align*}
If we derive once, we get
\begin{align*}
f'(x) &= f[x_0,x_1] + f[x_0,x_1,x_2]\,\big((x-x_0)+(x-x_1)\big) \\
&\quad + f[x_0,x_1,x_2,x]\,\big((x-x_1)(x-x_2) + (x-x_0)(x-x_2) +
(x-x_0)(x-x_1)\big) \\
&\quad + f[x_0,x_1,x_2,x,x]\,(x-x_0)(x-x_1)(x-x_2) \ ,
\end{align*}
where we have used the formula
\begin{equation}\label{fddderf}
\dfdx{f[x_0,x_1,x_2,x]}{x} = f[x_0,x_1,x_2,x,x] \ .
\end{equation}
If we derive a second time, we get
\begin{align*}
f''(x) &= 2\,f[x_0,x_1,x_2]
+ 2\, f[x_0,x_1,x_2,x]\,\big((x-x_0) + (x-x_1) + (x-x_2)\big) \\
&\quad + 2\,f[x_0,x_1,x_2,x,x]\,\big((x-x_1)(x-x_2) + (x-x_0)(x-x_2) +
(x-x_0)(x-x_1)\big) \\
&\quad + 2\,f[x_0,x_1,x_2,x,x,x]\,(x-x_0)(x-x_1)(x-x_2) \ ,
\end{align*}
where we have used (\ref{fddderf}) and the formula
$\displaystyle \dfdx{f[x_0,x_1,x_2,x,x]}{x} = 2\,f[x_0,x_1,x_2,x,x,x]$.
Since
$\displaystyle f[x_0,x_1,x_2,x] = \frac{1}{3!}\dydxn{f}{x}{3}(\xi)$,
$\displaystyle f[x_0,x_1,x_2,x,x] = \frac{1}{4!}\dydxn{f}{x}{4}(\eta)$
and
$\displaystyle f[x_0,x_1,x_2,x,x,x] = \frac{1}{5!}\dydxn{f}{x}{5}(\nu)$
for some $\xi$, $\eta$ and $\nu$ in the smallest interval containing
$x_0$, $x_1$, $x_2$ and $x$, we get
\begin{align*}
f''(x) &= 2\,f[x_0,x_1,x_2]
+ \frac{2}{3!}\dydxn{f}{x}{3}(\xi)\,\big((x-x_0) + (x-x_1) + (x-x_2)\big) \\
&\quad + \frac{2}{4!}\dydxn{f}{x}{4}(\eta)\,
\big((x-x_1)(x-x_2) + (x-x_0)(x-x_2) + (x-x_0)(x-x_1)\big) \\
&\quad + \frac{2}{5!}\dydxn{f}{x}{5}(\nu)\,(x-x_0)(x-x_1)(x-x_2) \ .
\end{align*}
We take $x_0=a$, $x_1 = a+h$ and $x_2 = a+2h$.  Hence, for $x=a$,  we get
\begin{align*}
f''(a) &= 2\,f[a,a+h,a+2h]
+ \frac{2}{3!}\dydxn{f}{x}{3}(\xi)\,\big((-h) + (-2h)\big)
+ \frac{2}{4!}\dydxn{f}{x}{4}(\eta)\,(-h)(-2h) \\
&= 2\,\left(\frac{\displaystyle\frac{f(a+2h)-f(a+h)}{h} -
\frac{f(a+h)-f(a)}{h}}{2h}\right) - \dydxn{f}{x}{3}(\xi)\,h
+\frac{1}{3!}\dydxn{f}{x}{4}(\eta)\,h^2 \\
&= \frac{ f(a)-2f(a+h)+f(a+2h)}{h^2} -\dydxn{f}{x}{3}(\xi)\,h
+\frac{1}{3!}\dydxn{f}{x}{4}(\eta)\,h^2 \ .
\end{align*}
We get (\ref{ddfapprox}) with the truncation error
$\displaystyle
-\dydxn{f}{x}{3}(\xi)\,h+\frac{1}{3!}\dydxn{f}{x}{4}(\eta)\,h^2$.
}

\solution{\SOL}{\ref{diffQ3}}{
Let $L^0_h(f) = L_h(f)$.  We prove by induction on $n$ that
\begin{equation}\label{Binduct1}
L(f) = L_{h/2^k}^n(f) + \sum_{j=n+1}^\infty
\frac{(2^{2n-1}-2^{2j-1}) \ldots (2^3-2^{2j-1})(2-2^{2j-1})}
{(2^{2n-1}-1)\ldots(2^3-1)(2-1)}\, a_j\, \left(\frac{h}{2^k}\right)^{2j-1} \ ,
\end{equation}
where
\[
L_{h/2^k}^n(f) = \frac{2^{2n-1}L_{h/2^k}^{n-1}(f)-L_{h/2^{k-1}}^{n-1}(f)}
{2^{2n-1}-1}
\]
and $k \geq n > 0$.

\stage{$\mathbf{n=1}$}
If we replace $h$ by $h/2$ in (\ref{questB1}), we get
\begin{equation}\label{questB2}
L(f) = L_{h/2}(f)+ \sum_{j=1}^\infty a_j\,\left(\frac{h}{2}\right)^{2j-1} \ .
\end{equation}
If we subtract (\ref{questB1}) from $2$ times (\ref{questB2}) and divide by
$2-1$, we get
\begin{align*}
L(f) &= L_{h/2}^1(f) + 2 \sum_{j=1}^\infty a_j\,\left(\frac{h}{2}\right)^{2j-1}
- \sum_{j=1}^\infty a_j\,h^{2j-1} \\
&= L_{h/2}^1(f) + \sum_{j=1}^\infty \left( 2
a_j\,\left(\frac{h}{2}\right)^{2j-1}
- 2^{2j-1} a_j\,\left(\frac{h}{2}\right)^{2j-1}\right)
= L_{h/2}^1(f) + \sum_{j=2}^\infty \frac{2 - 2^{2j-1}}{2-1}\, a_j\,
\left(\frac{h}{2}\right)^{2j-1}  \ ,
\end{align*}
where
\[
L_{h/2}^1(f) = \frac{2L_{h/2}(f) -L_h(f)}{2-1} \ .
\]
So (\ref{Binduct1}) is true for $n=1$ and $k=1$.  Replacing $h$ by
$h/2$ as many times as we want, we get that (\ref{Binduct1}) is true
for $n=1$ and $k\geq 1$.

\stage{$\mathbf{n=m}$} We suppose that (\ref{Binduct1}) is true for $n=m$.

\stage{$\mathbf{n=m+1}$} By induction, we have
\begin{align}
L(f) &= L_{h/2^k}^m(f) + \sum_{j=m+1}^\infty
\frac{(2^{2m-1}-2^{2j-1}) \ldots (2^3-2^{2j-1})(2-2^{2j-1})}
{(2^{2m-1}-1)\ldots(2^3-1)(2-1)}\, a_j\, \left(\frac{h}{2^k}\right)^{2j-1}
\label{questB3}
\intertext{and}
L(f) &= L_{h/2^{k+1}}^m(f) + \sum_{j=m+1}^\infty
\frac{(2^{2m-1}-2^{2j-1}) \ldots (2^3-2^{2j-1})(2-2^{2j-1})}
{(2^{2m-1}-1)\ldots(2^3-1)(2-1)}\, a_j\, \left(\frac{h}{2^{k+1}}\right)^{2j-1}
\label{questB4}
\end{align}
for $k\geq m$.

If we subtract (\ref{questB3}) from $2^{2m+1}$ times (\ref{questB4})
and divide by $2^{2m+1}-1$, we get
\begin{align*}
&L(f) = L_{h/2^{k+1}}^{m+1}(f) \\
&\quad + \frac{1}{2^{2m+1}-1}
\bigg( 2^{2m+1} \sum_{j=m+1}^\infty
\frac{(2^{2m-1}-2^{2j-1}) \ldots (2^3-2^{2j-1})(2-2^{2j-1})}
{(2^{2m-1}-1)\ldots(2^3-1)(2-1)}\, a_j\, \left(\frac{h}{2^{k+1}}\right)^{2j-1} \\
& \quad -\sum_{j=m+1}^\infty
\frac{(2^{2m-1}-2^{2j-1}) \ldots (2^3-2^{2j-1})(2-2^{2j-1})}
{(2^{2m-1}-1)\ldots(2^3-1)(2-1)}\, a_j\, \left(\frac{h}{2^k}\right)^{2j-1}
\bigg) \\
&= L_{h/2^{k+1}}^{m+1}(f) + \frac{1}{2^{2m+1}-1}
\bigg( 2^{2m+1} \sum_{j=m+1}^\infty
\frac{(2^{2m-1}-2^{2j-1}) \ldots (2^3-2^{2j-1})(2-2^{2j-1})}
{(2^{2m-1}-1)\ldots(2^3-1)(2-1)}\, a_j\, \left(\frac{h}{2^{k+1}}\right)^{2j-1} \\
& \quad -\sum_{j=m+1}^\infty
2^{2j-1} \,\frac{(2^{2m-1}-2^{2j-1}) \ldots (2^3-2^{2j-1})(2-2^{2j-1})}
{(2^{2m-1}-1)\ldots(2^3-1)(2-1)}\, a_j\, \left(\frac{h}{2^{k+1}}\right)^{2j-1}
\bigg) \\
&= L_{h/2^{k+1}}^{m+1}(f) + \sum_{j=m+1}^\infty
\frac{(2^{2m+1} - 2^{2j-1})(2^{2m-1}-2^{2j-1}) \ldots (2^3-2^{2j-1})(2-2^{2j-1})}
{(2^{2m+1}-1)(2^{2m-1}-1)\ldots(2^3-1)(2-1)}\,
a_j\, \left(\frac{h}{2^{k+1}}\right)^{2j-1} \\
&= L_{h/2^{k+1}}^{m+1}(f) + \sum_{j=m+2}^\infty
\frac{(2^{2m+1} - 2^{2j-1})(2^{2m-1}-2^{2j-1}) \ldots (2^3-2^{2j-1})(2-2^{2j-1})}
{(2^{2m+1}-1)(2^{2m-1}-1)\ldots(2^3-1)(2-1)}\,
 a_j\, \left(\frac{h}{2^{k+1}}\right)^{2j-1}\ ,
\end{align*}
where
\[
L_{h/2^{k+1}}^{m+1}(f) = \frac{2^{2m+1}L_{h/2^{k+1}}^m(f)-L_{h/2^k}^m(f)}
{2^{2m+1}-1} \ .
\]
This is (\ref{Binduct1}) for $n=m+1$ if we substitute
$k \geq n$ by $k\geq m$.

The general formula is
\[
L_{h/2^k}^n(f) = \frac{2^{2n-1} L_{h/2^k}^{n-1}(f) -
  L_{h/2^{k-1}}^{n-1}(f)}{2^{2n-1} -1}
\]
for $k \geq n > 0$ with a truncation error of $O(h^{2n+1})$.
}

\solution{\SOL}{\ref{diffQ4}}{
Let $L^0_h(f) = L_h(f)$.  We prove by induction on $n$ that
\begin{equation}\label{Cinduct1}
L(f) = L_{h/2^k}^n(f) + \sum_{j=n+1}^\infty
\frac{(2^{3n}-2^{3j}) \ldots (2^6-2^{3})(2^3-2^{3j})}
{(2^{3n}-1)\ldots(2^6-1)(2^3-1)}\, a_j\, \left(\frac{h}{2^k}\right)^{3j} \ ,
\end{equation}
where
\[
L_{h/2^k}^n(f) = \frac{2^{3n}L_{h/2^k}^{n-1}(f)-L_{h/2^{k-1}}^{n-1}(f)}
{2^{3n}-1}
\]
and $k \geq n > 0$.

\stage{$\mathbf{n=1}$}
If we replace $h$ by $h/2$ in (\ref{questC1}), we get
\begin{equation}\label{questC2}
L(f) = L_{h/2}(f)+ \sum_{j=1}^\infty a_j\,\left(\frac{h}{2}\right)^{3j} \ .
\end{equation}
If we subtract (\ref{questC1}) from $2^3$ times (\ref{questC2}) and divide by
$2^3-1$, we get
\begin{align*}
  L(f) &= L_{h/2}^1(f) + \frac{1}{2^3-1}\left(
2^3 \sum_{j=1}^\infty a_j\,\left(\frac{h}{2}\right)^{3j}
- \sum_{j=1}^\infty a_j\,h^{3j}\right) \\
&= L_{h/2}^1(f) + \frac{1}{2^3-1} \sum_{j=1}^\infty \left( 2^3
  a_j\,\left(\frac{h}{2}\right)^{3j}
- 2^{3j} a_j\,\left(\frac{h}{2}\right)^{3j}\right)
= L_{h/2}^1(f) + \sum_{j=2}^\infty \frac{2^3 - 2^{3j}}{2^3-1}\, a_j\,
\left(\frac{h}{2}\right)^{3j}  \ ,
\end{align*}
where
\[
L_{h/2}^1(f) = \frac{2^3 L_{h/2}(f) -L_h(f)}{2^3-1} \ .
\]
So (\ref{Cinduct1}) is true for $n=1$ and $k=1$.  Replacing $h$ by
$h/2$ as many times as we want, we get that (\ref{Cinduct1}) is true
for $n=1$ and $k\geq 1$.

\stage{$\mathbf{n=m}$} We suppose that (\ref{Cinduct1}) is true for $n=m$.

\stage{$\mathbf{n=m+1}$} By induction, we have
\begin{align}
L(f) &= L_{h/2^k}^m(f) + \sum_{j=m+1}^\infty
\frac{(2^{2m}-2^{3j}) \ldots (2^6-2^{3j})(2^3-2^{3j})}
{(2^{3m}-1)\ldots(2^6-1)(2^3-1)}\, a_j\, \left(\frac{h}{2^k}\right)^{3j}
\label{questC3}
\intertext{and}
L(f) &= L_{h/2^{k+1}}^m(f) + \sum_{j=m+1}^\infty
\frac{(2^{3m}-2^{3j}) \ldots (2^6-2^{3j})(2^3-2^{3j})}
{(2^{3m}-1)\ldots(2^6-1)(2^3-1)}\, a_j\, \left(\frac{h}{2^{k+1}}\right)^{3j}
\label{questC4}
\end{align}
for $k\geq m$.

If we subtract (\ref{questC3}) from $2^{3m+3}$ times (\ref{questC4})
and divide by $2^{3m+3}-1$, we get
\begin{align*}
&L(f) = L_{h/2^{k+1}}^{m+1}(f) \\
&\quad + \frac{1}{2^{3m+3}-1}
\bigg( 2^{3m+3} \sum_{j=m+1}^\infty
\frac{(2^{3m}-2^{3j}) \ldots (2^6-2^{3j})(2^3-2^{3j})}
{(2^{3m}-1)\ldots(2^6-1)(2^3-1)}\, a_j\, \left(\frac{h}{2^{k+1}}\right)^{3j} \\
&\quad -\sum_{j=m+1}^\infty
\frac{(2^{3m}-2^{3j}) \ldots (2^6-2^{3j})(2^3-2^{3j})}
{(2^{3m}-1)\ldots(2^6-1)(2^3-1)}\, a_j\, \left(\frac{h}{2^k}\right)^{3j}
\bigg) \\
&= L_{h/2^{k+1}}^{m+1}(f) + \frac{1}{2^{3m+3}-1}
\bigg( 2^{3m+3} \sum_{j=m+1}^\infty
\frac{(2^{3m}-2^{3j}) \ldots (2^6-2^{3j})(2^3-2^{3j})}
{(2^{3m}-1)\ldots(2^6-1)(2^3-1)}\, a_j\, \left(\frac{h}{2^{k+1}}\right)^{3j} \\
& \quad -\sum_{j=m+1}^\infty
2^{3j} \,\frac{(2^{3m}-2^{3j}) \ldots (2^6-2^{3j})(2^3-2^{3j})}
{(2^{3m}-1)\ldots(2^6-1)(2^3-1)}\, a_j\, \left(\frac{h}{2^{k+1}}\right)^{3j}
\bigg) \\
&= L_{h/2^{k+1}}^{m+1}(f) + \sum_{j=m+1}^\infty
\frac{(2^{3m+3} - 2^{3j})(2^{3m}-2^{3j}) \ldots (2^6-2^{3j})(2^3-2^{3j})}
{(2^{3m+3}-1)(2^{3m}-1)\ldots(2^6-1)(2^3-1)}\,
a_j\, \left(\frac{h}{2^{k+1}}\right)^{3j} \\
&= L_{h/2^{k+1}}^{m+1}(f) + \sum_{j=m+2}^\infty
\frac{(2^{3m+3} - 2^{3j})(2^{3m}-2^{3j}) \ldots (2^6-2^{3j})(2^3-2^{3j})}
{(2^{3m+3}-1)(2^{3m}-1)\ldots(2^6-1)(2^3-1)}\,
 a_j\, \left(\frac{h}{2^{k+1}}\right)^{3j} \ ,
\end{align*}
where
\[
L_{h/2^{k+1}}^{m+1}(f) = \frac{2^{3m+3}L_{h/2^{k+1}}^m(f)-L_{h/2^k}^m(f)}
{2^{3m+3}-1} \ .
\]
This is (\ref{Cinduct1}) for $n=m+1$ if we substitute
$k \geq n$ by $k\geq m$.

The general formula is
\[
L_{h/2^k}^n(f) = \frac{2^{3n} L_{h/2^k}^{n-1}(f) -
  L_{h/2^{k-1}}^{n-1}(f)}{2^{3n} -1}
\]
for $k \geq n > 0$ with a truncation error of $O(h^{3n+3})$.
}

\solution{\SOL}{\ref{diffQ5}}{
With $L_h(f) = (f(3+h)-f(3-h))/(2h)$, we get the following table.
{
\small
\[
\begin{array}{c|cccccc}
\hline
h & L_h(f) & & L_h^1(f) & & L_h^2(f) \\ 
\hline
0.8 & 0.16441388 &&&& \\ 
0.4 & 0.15472628 & \fbox{4.13687170} & 0.15149708 && \\ 
0.2 & 0.15238451 & \fbox{4.03339716} & 0.15160392 & \fbox{16.53008728}
    & 0.15161104  \\ 
0.1 & 0.15180391 & \fbox{4.00829909} & 0.15161038 & \fbox{16.13017524}
    & 0.15161081 \\ 
0.05 & 0.15165906 && 0.15161078 && 0.15161081 \\ 
\hline
\multicolumn{1}{c}{} & & & & & \\
\cline{2-5}
\multicolumn{1}{c}{} & & L_h^3(f) & L_h^4(f) & |L_h^i(f)-L_{2h}^{i-1}(f)| & \\ 
\cline{2-5}
\multicolumn{1}{c}{} &&&&& \\
\multicolumn{1}{c}{} &&&& -0.0129169 & \\ 
\multicolumn{1}{c}{} &&&& 0.11396344 \times 10^{-03} & \\ 
\multicolumn{1}{c}{} & \fbox{65.68377022} & 0.15161081 &
& -0.23203700\times 10^{-06} & \\ 
\multicolumn{1}{c}{} & & 0.15161081 & 0.15161081 &
0.93304309 \times 10^{-10} & \\
\cline{2-5}
\end{array}
\]
}
All the values in the table have been rounded.
We stop the procedure as soon as $|L_h^i(f) - L_{2h}^{i-1}(f)|$ gets
smaller than $10^{-7}$ and take $L_h^i(f)$ as our approximation of
$f'(3)$.  We have also included the ratios defined in
(\ref{checkRich}) to ensure that the approximating values $L_h^i(h)$
can be trusted.  We have that
$f'(3) \approx L_{0.05}^4(f) \approx 0.15161081$ meets our criterion
of accuracy.
}

\solution{\SOL}{\ref{diffQ7}}{
Let $a=1$, $b=3$, $h=(b-a)/2m = 1/m$ and
$x_i = 1 + i\,h$ for $i=0$, $1$, $2$, \ldots, $n=2m$.

The local truncation error for the composite midpoint rule is
$\displaystyle - \frac{f''(\xi)(b-a)}{6}\,h^2$ for some
$\xi \in [a,b]$.  We seek a small $m$ for which this 
truncation error will be smaller in absolute value than $10^{-5}$.
We have
\[
f''(x) = \frac{1}{x} + \frac{x}{4} -10 \ .
\]
We use the Extremum Theorem to find the maximum of $f''(x)|$ on
$[1,3]$.  We have that $x=2$ is the only critical point of
$f^{(3)}(x) = -1/x^2 + 1/4$ in the interval $[1,3]$.  Since
$f''(2) = -9 < f''(3) = -107/12 < f''(1) = -35/4$, we have that
$-9 \leq f''(x) \leq -35/4$ for $1 \leq x \leq 3$.  Thus,
$|f''(x)| \leq 9$ for $1\leq x \leq 3$.  Hence,
\[
\left| - \frac{f''(\xi)(b-a)}{6} \, h^2\right|
= \frac{|f''(\xi)|}{3m^2} \leq \frac{3}{m^2}
\]
because $1 \leq \xi \leq 3$.  We chose $m$ that satisfies
$\displaystyle \frac{3}{m^2} < 10^{-5}$; namely,
$m > \sqrt{ 3 \times 10^5} \approx 547.72$.

With $m=548$, we get
\[
\int_1^3 \left( x \ln(x) + \frac{x^3}{24} - 5x^2\right) \dx{x} \approx
\frac{1}{274} \, \sum_{i=1}^{548} 
\left( x_{2i-1} \ln(x_{2i-1}) + \frac{x_{2i-1}^3}{24} - 5x_{2i-1}^2 \right)
\approx -39.5562347658 \ .
\]
}

\solution{\SOL}{\ref{diffQ8}}{
Let $a=2$, $b=4$, $h=(b-a)/2m = 1/m$ and
$x_i = 2 + i\,h$ for $i=0$, $1$, $2$, \ldots, $n=2m$.

The local truncation error for the composite Simpson rule is
$\displaystyle - \frac{h^4(b-a)}{180}\,f^{(4)}(\xi)$
for some $\xi \in [a,b]$.  We seek a small $m$ for which this
truncation error will be smaller in absolute value than $10^{-5}$.
We have
\[
f^{(4)}(x) = -\frac{80}{81} \, (x+1)^{-11/3} \ .
\]
Hence,
\[
\left| \frac{h^4(b-a)}{180}\,f^{(4)}(\xi) \right|
= \left(\frac{1}{90\,m^4}\right) \left(\frac{80}{81}\right) (\xi+1)^{-11/3}
\leq \frac{8}{3^6\,m^4} \, 3^{-11/3} = \frac{8}{3^{29/3}\,m^4}
\]
because $2 \leq \xi \leq 4$.  We chose $m$ that satisfies
$\displaystyle \frac{8}{3^{29/3}\,m^4} < 10^{-5}$; namely,
$\displaystyle m > \left( \frac{8}{3^{29/3}}\, 10^{5}\right)^{1/4}
\approx 2.102468339$\ .

With $m=3$, we get
\begin{align*}
\int_2^4\, (x+2)^{1/3} \dx{x} &\approx
\frac{1}{9} \bigg( (3)^{1/3}
+ 2 \sum_{i=1}^2 \left(1+ \left(2+\frac{2i}{3}\right)\right)^{1/3}
+ 4 \sum_{i=0}^2 \left(1+\left(2+\frac{2i+1}{3}\right)\right)^{1/3} 
+ (5)^{1/3} \bigg) \\
& \approx 3.16734727452 \ .
\end{align*}
}

\solution{\SOL}{\ref{diffQ9}}{
Before answering this question, we note that $f'(x) = 2x\ln(x) + x$,
$f''(x) = 2\ln(x)+3$, $f^{(3)}(x) = 2/x$ and $f^{(4)}(x) = -2/x^2$.

Moreover, a simple integration by parts gives
\[
\int_1^3 x^2\,\ln(x)\dx{x} = 9 \ln(3) - \frac{26}{9} \approx 6.99862170912 \ .
\]

We have $a=1$ and $b=3$ in the formulae for the truncation errors of
the composite methods.

\subQ{a} For the midpoint rule, we choose $n = 2m$ and
$h = (b-a)/n = 1/m$ such that the truncation error
$\displaystyle \left| \frac{f''(\eta)(b-a)}{6} \, h^2  \right|$ for
some $\eta \in [1,3]$ satisfies
\[
\left| \frac{f''(\eta)(b-a)}{6} \, h^2  \right|
= \frac{1}{3} \left(\frac{1}{m}\right)^2\, 
\left| 2\ln(\eta)+3\right|
\leq \frac{1}{3m^2} \left| 2\ln(3)+3\right| < 10^{-5} \ .
\]
Thus,
\[
  m > \left(\frac{10^5}{3}\left(2\ln(3)+3\right)\right)^{1/2} \approx 416.22 \ .
\]
We take $m=417$.  It follows that $h=1/417$ and
\[
\int_1^3 x^2\,\ln(x)\dx{x} \approx \frac{2}{417}
\sum_{j=1}^{417} f(x_{2j-1}) \approx 6.99861347 \ ,
\]
where $x_j = 1 + j\,h = 1 + j/417$.  The absolute error is
about $0.82348266 \times 10^{-5}$.

\subQ{b} For the trapezoidal rule, we choose $n$ and
$h = (b-a)/n = 2/n$ such that the truncation error
$\displaystyle \left| \frac{f''(\eta)(b-a)}{12} \, h^2 \right|$ for
some $\eta \in [1,3]$ satisfies
\[
\left| \frac{f''(\eta)(b-a)}{12} \, h^2 \right|
= \frac{1}{6} \,\left(\frac{2}{n}\right)^2\, 
\left| 2\ln(\eta)+3\right|
\leq \frac{2}{3n^2} \left| 2\ln(3)+3\right| < 10^{-5} \ .
\]
Thus,
\[
n > \left( \frac{2\times 10^5}{3}\left(2\ln(3)+3\right) \right)^{1/2}
\approx 588.6269 \ .
\]
We take $n= 589$.  It follows that $h=2/589$ and
\[
\int_1^3 x^2\,\ln(x)\dx{x} \approx \frac{1}{589}
\left( f(x_0) + 2 \sum_{j=1}^{588} f(x_j) + f(x_{589}) \right)
\approx 6.99862996 \ ,
\]
where $x_j = 1 + j\,h = 1 + 2j/589$.  The absolute error is
about $0.82551686 \times 10^{-5}$.

\subQ{c} For the Simpson rule, we choose $n = 2m$ and
$h = (b-a)/n = 1/m$ such that the truncation error
$\displaystyle \left| \frac{f^{(4)}(\eta)(b-a)}{180} \, h^4  \right|$
for some $\eta \in [1,3]$ satisfies
\[
\left| \frac{f^{(4)}(\eta)(b-a)}{180} \, h^4  \right|
= \frac{1}{90} \,\left(\frac{1}{m}\right)^4\, 
\left| \frac{-2}{\eta^2} \right|
\leq \frac{1}{45\,m^4}  < 10^{-5} \ .
\]
Thus,
\[
  m > \left( \frac{10^5}{45}\right)^{1/4} \approx 6.86589 \ .
\]
We take $m=7$.  It follows that $h=1/7$ and
\[
\int_1^3 x^2\,\ln(x)\dx{x} \approx \frac{1}{21}
\left( f(x_0) + 2 \sum_{j=1}^6 f(x_{2j}) +4 \sum_{j=0}^6 f(x_{2j+1})
+ f(x_{14}) \right) \approx 6.99861865 \ ,
\]
where $x_j = 1 + j\,h = 1 + j/7$.    The absolute error is
about $0.30640240 \times 10^{-5}$.
}

\solution{\SOL}{\ref{diffQ10}}{
There are two ways to answer this question.  We could use the formula
\[
\int_a^b  f(x) \dx{x} = \frac{f(a) + 4f((a+b)/2) +f(b)}{6}\, (b-a)
-\frac{f^{(4)}(\xi)\,(b-a)^5}{2880}
\]
for some $\xi$ between $a$ and $b$.  The truncation error will be $0$
for all polynomials of degree less than $4$ becuase $f^{(4)}(x)=0$ for
all polynomials $f$ of degree less than $4$.

The other way to answer the question is to proceed directly.  By
linearity of the integral, it is enough to show that Simpson's rule is
exact for $x^i$ with $i=0$, $1$, $2$ and $3$.  For $f(x) = x^i$,
Simpson's rule becomes
\begin{equation} \label{SimpsonQ10}
\int_a^b x^i \dx{x} \approx \frac{b-a}{6} \left(a^i + 4
\left(\frac{a+b}{2}\right)^i + b^i\right) \ .
\end{equation}
For $i=0$, the right hand side of (\ref{SimpsonQ10}) is
\[
\frac{b-a}{6} \left(1 + 4 + 1\right)
= b-a = \int_a^b \, \dx{x} \ .
\]
For $i=1$, the right hand side of (\ref{SimpsonQ10}) is
\[
\frac{b-a}{6} \left(a + 4
\left(\frac{a+b}{2}\right) + b\right)
= \frac{b-a}{6}\,(3 a + 3 b)
= \frac{1}{2} \left( b^2 - a^2 \right)
= \int_a^b x \dx{x} \ .
\]
For $i=2$, the right hand side of (\ref{SimpsonQ10}) is
\[
\frac{b-a}{6} \left(a^2 + 4
\left(\frac{a+b}{2}\right)^2 + b^2\right)
=\frac{b-a}{6} \left(2 a^2 + 2 ab + 2 b^2\right)
= \frac{1}{3} \left( b^3-a^3\right) = \int_a^b x^2\dx{x} \ .
\]
Finally, for $i=3$, the right hand side of (\ref{SimpsonQ10}) is
\begin{align*}
\frac{b-a}{6} \left(a^3 + 4
\left(\frac{a+b}{2}\right)^3 + b^3\right)
&= \frac{b-a}{6} \left(a^3 + \frac{1}{2}
\left(a^3 + 3 a^2b + 3 ab^3 + b^3\right) + b^3\right) \\
&= \frac{b-a}{12}\left(3 a^3+3 a^2b+3 ab^2+3 b^3\right)
= \frac{1}{4} \left( b^4 - a^4 \right) = \int_a^b x^3\dx{x} \ .
\end{align*}
Thus, Simpson's rule is exact for polynomial of degree up to three.

However, for $i=4$, the right hand side of (\ref{SimpsonQ10}) is
\begin{align*}
\frac{b-a}{6} \left(a^4 + 4
\left(\frac{a+b}{2}\right)^4 + b^4\right)
&= \frac{b-a}{6} \left(a^4 + \frac{1}{4}
\left(a^4 + 4 a^3b + 6 a^2b^2 + 4 ab^4 + b^4\right) + b^4\right) \\
&= \frac{b-a}{24}\left(5 a^4+4 a^3b+6 a^2b^2+4 ab^3+5 b^4\right) \\
&= \frac{1}{24} \left( 5 b^5- ab^4+2 a^2b^3 -2 a^3b^2 + a^4b -5 a^5\right)\\
&\neq \frac{1}{5}\left(b^5-a^5\right) = \int_a^b x^4\dx{x} 
\end{align*}
for almost all values of $a$ and $b$.
}

\solution{\SOL}{\ref{diffQ11}}{
We use the composite trapezoidal rule to generate the first column of
the table for Romberg integration.

\noindent For $n=2$, we have $h=(4-2)/2=1$ and $x_j=2+j\,h = 2+j$ for
$0\leq j \leq 2$.  Hence,
\[
\int_2^4 (1+x)^{1/3} \dx{x} \approx
\frac{1}{2} \left( 3^{1/3} + 2 \, (4^{1/3}) + 5^{1/3} \right)
\approx 3.163513810460252 \ .
\]
For $n=4$, we have $h=(4-2)/4=1/2$ and $x_j=2+j\,h = 2+j/2$ for
$0\leq j \leq 4$.  Hence,
\[
\int_2^4 (1+x)^{1/3} \dx{x} \approx
\frac{1}{4} \left( 3^{1/3} + 2 \left( 3.5^{1/3} + 4^{1/3} +
4.5^{1/3}\right) + 5^{1/3} \right)
\approx 3.166385960422699 \ .
\]
For $n=8$, we have $h=(4-2)/8=1/4$ and $x_j=2+j\,h = 2+j/4$ for
$0\leq j \leq 8$.  Hence,
\[
\int_2^4 (1+x)^{1/3} \dx{x} \approx
\frac{1}{8} \left( 3^{1/3} + 2 \sum_{j=1}^7 (1+2+j/4)^{1/3} + 5^{1/3} \right)
\approx 3.167107453016058 \ .
\]

All values in the table below have been rounded to nine digits.
{
\small
\[
\begin{array}{cc|cccc}
n & h & L_h^0(f) & & L^1_h(f) & L^2_h(f) \\
\hline
2 & 1 & 3.16351381 & & & \\
4 & 1/2 & 3.16638596 & \fbox{3.98084469} & 3.16734334 & \\
8 & 1/4 & 3.16710745 & & 3.16734795 & 3.16734826  
\end{array}
\]
}
Note that
$\displaystyle L_{1/2}^1(f) = (4L_{1/2}^0(f)-L_1^0(f))/(4-1)$,
$\displaystyle L_{1/4}^1(f) = (4L_{1/4}^0(f)-L_{1/2}^0(f))/(4-1)$
and
$\displaystyle L_{1/4}^2(f) = (4^2L_{1/4}^1(f)-L_{1/2}^1(f))/(4^2-1)$.

The requested approximation is $3.16734826$ because
$\displaystyle \left| L_{1/2}^1(f) - L_1^0(f) \right| \approx 0.0038295
> 10^{-5}$ and
$\displaystyle \left| L_{1/4}^2(f) - L_{1/2}^1(f) \right| \approx
0,49139 \times 10^{-5}  < 10^{-5}$.
}

\solution{\SOL}{\ref{diffQ12}}{
We have used the composite trapezoidal rule to generate the first
column of the table for Romberg integration; namely, we have used
\[
L_h^0(f) = \frac{h}{2}\left( f(x_0) + 2 \sum_{j=1}^{n-1} f(x_j) + f(x_n)
\right) \ ,
\]
where $x_i= 1+ih$ and $h=2/n$.
{
\small
\[
\begin{array}{c|ccccc}
\hline
h & L_h^0(f) & & L_h^1(f) & & L_h^2(f) \\
\hline
2 & 2.31607401 & & & & \\ 
1 & 2.34724412 & 3.69506105 & 2.35763416 &  & \\ 
0.5 & 2.35567974 & 3.88972370 & 2.35849161 & 10.75596733 & 2.35854877  \\ 
0.25 & 2.35784843 & \fbox{3.96766871} & 2.35857133 & 13.53301479
& 2.35857664 \\ 
0.125 & 2.35839502 &  & 2.35857722 &  & 2.35857761 \\ 
\hline
\multicolumn{1}{c}{} & & & & & \\
\cline{2-5}
\multicolumn{1}{c}{} & & L_h^3(f) &  L_h^4(f) & |L_h^i(f) - L_{2h}^{i-1}(f)| & \\
\cline{2-5}
\multicolumn{1}{c}{} & & & & & \\
\multicolumn{1}{c}{} & & & & 0.0415601 & \\
\multicolumn{1}{c}{} & & & & 0.914612 \times 10^{-3} & \\
\multicolumn{1}{c}{} & 28.76692226 & 2.35857708 & &
0.283121 \times 10^{-4} & \\ 
\multicolumn{1}{c}{} & & 2.35857762 & 2.35857763 & 0.543938 \times 10^{-6} & \\ 
\cline{2-5}
\end{array}
\]
}
All the values in the table have been rounded.

We stopped the procedure when $|L_h^i(f) - L_{2h}^{i-1}(f)|$ got
smaller than $10^{-5}$ and took $L_h^i(f)$ as our approximation of
the integral.  We have that
\[
\int_3^5 (x-2)^{1/4} \dx{x} \approx L_{0.125}^4 \approx 2.35857763
\]
with the required accuracy.

We have also included the ratios defined in (\ref{checkRich}) to check
if the values of $L_h^i(h)$ can be trusted.  However, we have ignored
this information.
}

\solution{\SOL}{\ref{diffQ13}}{
The first column of table used for Romberg integration is produced by
the composite trapezoidal rule
\[
L_h^0(f) = \frac{h}{2}\left( f(x_0) + 2 \sum_{j=1}^{n-1} f(x_j) + f(x_n)
\right) \ ,
\]
where $x_i= 1+ih$ and $h=2/n$.  The table is given below.
{
\small
\[
\begin{array}{c|ccccc}
\hline
h & L_h^0(f) & & L_h^1(f) & & L_h^2(f) \\ 
\hline
2 & 9.88751060 & & & & \\ 
1 & 7.71634402 & \fbox{4.03101596} & 6.99262183 & & \\ 
0.5 & 7.17772879 & \fbox{4.00899805} & 6.99819039 & 13.81887982 & 6.99856162 \\ 
0.25 & 7.04337721 & \fbox{4.00237347} & 6.99859335 & \fbox{15.17336693}
& 6.99862022 \\ 
0.125 & 7.00980924 & & 6.99861991 & & 6.99862168 \\ 
\hline
\multicolumn{1}{c}{} & & & & & \\
\cline{2-5}
\multicolumn{1}{c}{} & & L_h^3(f) & L_h^4(f) & |L_h^i(f)-L_{2h}^{i-1}(f)| &\\ 
\cline{2-5}
\multicolumn{1}{c}{} &&&&& \\
\multicolumn{1}{c}{} &&&& 2.8948888 & \\
\multicolumn{1}{c}{} &&&& 0.005939793 & \\  
\multicolumn{1}{c}{} & 40.03582483 & 6.99862115 & &
0.59524745 \times 10^{-4} & \\ 
\multicolumn{1}{c}{} && 6.99862170 & 6.99862171 & 0.55889607 \times 10^{-5} & \\
\cline{2-5}
\end{array}
\]
}
All the values in the table have been rounded.
We stopped the procedure when $|L_h^i(f) - L_{2h}^{i-1}(f)|$ got
smaller than $10^{-7}$ and took $L_h^i(f)$ as our approximation of
the integral.  We have that
\[
\int_1^3 x^2\,\ln(x) \dx{x} \approx L_{0.125}^4 \approx 6.99862171
\]
meet our criteria of accuracy.

In question~\ref{diffQ9}, we found that
\[
\int_1^3 x^2\,\ln(x)\dx{x} = 9 \ln(3) - \frac{26}{9} =
6.99862170912\ldots
\]
So, the absolute error of our approximation $6.99862171$ is about
$-0.88 \times 10^{-9}$.  This is better than expected.

We have also included the ratios defined in (\ref{checkRich}) to check
if the values of $L_h^i(h)$ can be trusted.  However, we have ignored
this information.  The approximation of the value of the integral
suggested by the ratios defined in (\ref{checkRich}) is
$L_{0.25}^2(f) \approx 6.99862022$ associated to the ratio
$15.17336693$.  The absolute error for this approximation is about
$0.14891 \times 10^{-5}$.  Despite the fact that the ratios were not
respecting the rule of $4^i$, we did the right thing by proceeding
with the computations.  The problem with the computations of the
ratios defined in (\ref{checkRich}) is that they have to be done with
very high precision because we are dividing by values closed to zero.
}

\solution{\SOL}{\ref{diffQ14}}{
The first column of the table associated to Romberg integration is
given by the composite trapezoidal rule
\[
L_h(f) = \frac{h}{2} \left( f(x_0) + 2\,\sum_{j=1}^{n-1}\,f(x_j)
+ f(x_n)\right) \ ,
\]
where $h=(b-a)/n$ and $x_i = a+i\,h$ for $i=1$, $2$, $3$, \ldots, $n$.

The second column of this table is
\begin{align*}
L_{h/2}^1(f) &= \frac{4L_{h/2}(f) - L_h(f)}{4-1}
= \frac{1}{3} \bigg(
h\, \left( f(x_0) + 2\,\sum_{j=1}^{n-1}\,f(x_j)
+ 2\,\sum_{j=0}^{n-1}\,f(x_j+h/2) + f(x_n)\right) \\
&\qquad - \frac{h}{2} \left( f(x_0) + 2\,\sum_{j=1}^{n-1}\,f(x_{j})
+ f(x_n)\right) \bigg) \\
&= \frac{h}{6} \left(f(x_0) + 2\,\sum_{j=1}^{n-1}\,f(x_j)
+ 4\,\sum_{j=0}^{n-1}\,f(x_{j}+h/2) + f(x_n) \right) \\
&= \frac{\tilde{h}}{3} \left(
f(\tilde{x}_0) + 2\,\sum_{j=1}^{n-1}\,f(\tilde{x}_{2j})
+ 4\,\sum_{j=0}^{n-1}\,f(\tilde{x}_{2j+1}) + f(\tilde{x}_n) \right) \ ,
\end{align*}
where $\tilde{h} = (b-a)/(2n)$ and $\tilde{x}_j = a + j\tilde{h}$ for
$0 \leq j \leq 2n$.  This is the composite Simpson rule with $m$
replaced by $n$, $h$ by $\tilde{h}$ and $x_j$ by $\tilde{x}_j$.
}

\solution{\SOL}{\ref{diffQ15}}{
Use the adaptive quadrature method presented in
Section~\ref{SectAdaptQMethod} to approximate
\[
\int_3^5 (x-2)^{1/4} \dx{x}
\]
with an accuracy of $10^{-5}$.  For this purpose and to simplify the
discussion, let $S(a,b,h)$ be the result of the composite Simpson's Rule,
Theorems~\ref{SimpsonRule} and \ref{CSR}, for
$\displaystyle \int_a^b (x-2)^{1/4} \dx{x}$ with $m = (b-a)/(2h)$.
The values displayed in the following computations have been rounded
to $12$ significant digits though the computations have been
done with as many digits as possible.

\subI{level 0}
\[
\begin{array}{c|ccccc}
i & 1 & 2 & 3 & 4 & 5 \\
\hline
x_i & 3 & 7/2 & 4 & 9/2 & 5
\end{array}
\]
$h=1$, $T=10^{-5}$, $S_{[3,5]} = S(3,5,1) \approx 2.35763415765$,\\
$S_1 = S(3,4,0.5) \approx 1.10265579897$,
$S_2 = S(4,5,0.5) \approx 1.25583580778$,\\
$\tilde{S}_{[3,5]} = S(3,5,0.5) \approx S_1+S_2 = 2.35849160675$ and\\
$\displaystyle \tilde{R}_{[3,5]} \approx
\frac{1}{15}\left| \tilde{S}_{[3,5]} - S_{[3,5]}\right|
\approx 0.571633 \times 10^{-4} \not< 10^{-5}$.

\subI{Level 1}
\[
\begin{array}{c|ccccc}
i & 1 & 2 & 3 & 4 & 5 \\
\hline
x_i & 3 & 13/4 & 7/2 & 15/4 & 4
\end{array}
\]
$h=0.5$, $T=0.5\times 10^{-5}$ (store for $[4, 5]$),
$S_{[3,4]} = S(3,4,0.5) \approx 1.10265579897$,\\
$S_1 = S(3,3.5,0.25) \approx 0.528013914455$,
$S_2 = S(3.5,4,0.25) \approx 0.574711858524$,\\
$\tilde{S}_{[3,4]} = S(3,4,0.25) = S_1+S_2 \approx 1.10272577298$ and\\
$\displaystyle \tilde{R}_{[3,4]} \approx
\frac{1}{15}\left| \tilde{S}_{[3,4]} - S_{[3,4]}\right|
\approx 0.466493 \times 10^{-5} < 0.5 \times 10^{-5}$.\\
We accept $\tilde{S}_{[3,4]} \approx 1.10272577298$ as an approximation of
$\displaystyle \int_3^4 (x-2)^{1/4} \dx{x}$.

\subI{Level 1}
\[
\begin{array}{c|ccccc}
i & 1 & 2 & 3 & 4 & 5 \\
\hline
x_i & 4 & 17/4 & 9/2 & 19/4 & 5
\end{array}
\]
$h=0.5$, $T=0.5\times 10^{-5}$,
$S_{[4,5]} = S(4,5,0.5) \approx 1.25583580778$,\\
$S_1 = S(4,4.5,0.25) \approx 0.612135002521$,
$S_2 = S(4.5, 5, 0.25) \approx 0.643710549703$,\\
$\tilde{S}_{[4,5]} = S(4,5,0.25) = S_1+S_2 \approx 1.25584555222$ and\\
$\displaystyle \tilde{R}_{[4,5]} \approx
\frac{1}{15}\left| \tilde{S}_{[4,5]} - S_{[4,5]}\right|
\approx 0.649630 \times 10^{-6} < 0.5 \times 10^{-5}$.\\
We accept $\tilde{S}_{[4,5]}\approx 1.25584555222$ as an approximation of
$\displaystyle \int_4^5 (x-2)^{1/4} \dx{x}$.

\subI{Level 0} We are done.  We have found that
\begin{align*}
\int_3^5 (x-2)^{1/4} \dx{x} &=
\int_3^4 (x-2)^{1/4} + \int_4^5 (x-2)^{1/4} \\
&\approx 1.10272577298 + 1.25584555222 = 2.35857132520 \ .
\end{align*}
This approximation is meeting the required accuracy.
}

\solution{\SOL}{\ref{diffQ16}}{
Since (\ref{NF01}) is a linear functional with respect to $f$ and
since all polynomial of degree at most $4$ are linear combinations of
the monomials $x^i$ for $0 \leq i \leq 4$, it is enough to show
that (\ref{NF01}) is true for $f(x) = x^i$ with $0\leq i \leq 4$ to
prove that (\ref{NF01}) is true for all polynomials of degree less
than or equal to $4$.  For instance, for $i=4$, we get
\[
\int_0^1 x^4 \dx{x} = \frac{x^5}{5}\bigg|_{x=0}^1 = \frac{1}{5}
\]
and
\[
\frac{1}{90}\left(7 (0^4) +32 \left(\frac{1}{4}\right)^4 +
12 \left(\frac{1}{2}\right)^4 + 32 \left(\frac{3}{4}\right)^4
+ 7 (1^4)\right) = \frac{1}{5} \ .
\]
We leave to the reader the task of verifying the equality for the
other monomials.

If we substitute $y = (b-a)x + a$ in
$\displaystyle \int_a^b\,f(y)\dx{y}$, we get
\begin{align*}
\int_a^b\,f(y)\dx{y} &= \int_0^1 f\big( (b-a)x+a\big)\, (b-a)\dx{x} \\
&= \frac{b-a}{90}\bigg(7\,f(a)+32\,f\left(\frac{3a+b}{4}\right) +
12\,f\left(\frac{a+b}{2}\right)
+ 32\,f\left(\frac{a+3b}{4}\right) + 7\,f(b)\bigg) \ .
\end{align*}
}

\solution{\SOL}{\ref{diffQ17}}{
For $f(x) = x^i$, the formula is
\[
\int_0^1 x^i \dx{x} \approx A\, \left( x_0^i + x_1^i \right) \ .
\]
For $i=0$, we have $\displaystyle 1 = \int_0^1 \dx{x} = 2\,A$.
Thus, $A = 1/2$.  For $i=1$, we have
$\displaystyle
\frac{1}{2} = \int_0^1 x\dx{x} = A \,(x_0+x_1) =
\frac{1}{2}(x_0-x_1)$.  Thus
\begin{equation} \label{Q17equ1}
  1 = x_0+x_1 \ .
\end{equation}
For $i=2$, we have
$\displaystyle
\frac{1}{3} = \int_0^1 x^2 \dx{x} = A\, (x_0^2+x_1^2)
= \frac{1}{2} (x_0^2+x_1^2)$.  Thus,
\begin{equation} \label{Q17equ2}
\frac{2}{3} = x_0^2 + x_1^2 \ .
\end{equation}
If we solve the system of equations given by (\ref{Q17equ1}) and
(\ref{Q17equ2}) for $x_0$ and $x_1$, we get two solutions
$\displaystyle x_0= \frac{1}{2} + \frac{1}{2\sqrt{3}}$ and
$\displaystyle x_1= \frac{1}{2} - \frac{1}{2\sqrt{3}}$, and
$\displaystyle x_0= \frac{1}{2} - \frac{1}{2\sqrt{3}}$ and
$\displaystyle x_1= \frac{1}{2} + \frac{1}{2\sqrt{3}}$.
Hence, we find that the formula
\[
\int_0^1 x^i \dx{x} \approx \frac{1}{2}\left(
f\left(\frac{1}{2} + \frac{1}{2\sqrt{3}}\right)
+f\left(\frac{1}{2} - \frac{1}{2\sqrt{3}}\right)\right)
\]
is exact for polynomials of degree less or equal to $2$.
}

\solution{\SOL}{\ref{diffQ18}}{
For $f(x) = x^i$, the formula is
\[
\int_0^2 x^{i+1} \dx{x} \approx A\,x^i\big|_{x=0} + B\,x^i\big|_{x=1}
+ C\,x^i\big|_{x=2}
\]
for $i \geq 0$.  For $i=0$, we get
\begin{equation} \label{Q18equ1}
  2 = \int_0^2 x\dx{x} = A + B + C \ .
\end{equation}
For $i=1$, we get
\begin{equation} \label{Q18equ2}
 \frac{8}{3} = \int_0^2 x^2\dx{x} = B + 2\,C \ .
\end{equation}
For $i=2$, we get
\begin{equation} \label{Q18equ3}
  4 = \int_0^2 x^3 \dx{x} = B + 4\,C \ .
\end{equation}
The system of linear equations formed of (\ref{Q18equ1}),
(\ref{Q18equ2}) and (\ref{Q18equ3}) has a unique solution given by
$A=0$, $B=4/3$ and $C=2/3$.

For $i=3$, we have
\[
\frac{32}{5} = \int_0^2 x^4\dx{x} = B + 8\,C \ .
\]
Since this equation is not satisfied with $B=4/3$ and $C=2/3$, the
formula is not valid for $i=3$.  Therefore, we can get a formula which
is exact for polynomials of degree up to two with $A=0$, $B=4/3$ and
$C=2/3$.  It is not possible to do better.
}

\solution{\SOL}{\ref{diffQ19}}{
We get $\displaystyle A + B = \int_0^{2\pi} \dx{x} = 2\pi$ when $k=0$
and
$\displaystyle A \cos(0) + B \cos(\pi) = \int_0^{2\pi} \cos(x) \dx{x}
= \sin(x)\Big|_{x=0}^{2\pi} = 0$ when $k=1$.
Thus $A+B=2\pi$ and $A-B=0$.  We find $A=B=\pi$.
The requested formula is
\begin{equation}\label{QuestFourierForm}
\int_0^{2\pi} f(x) \dx{x} \approx \pi\,f(0) + \pi\,f(\pi) \ .
\end{equation}

Since
\[
\int_0^{2\pi} \cos((2k+1)x) \dx{x} =
\frac{1}{2k+1}\sin((2k+1)x)\bigg|_{x=0}^{2\pi} = 0
\]
and
\[
\pi \cos((2k+1)x)\bigg|_{x=0} + \pi\cos((2k+1)x)\bigg|_{x=\pi}
= \pi + \pi\cos((2k-1)\pi) = \pi - \pi = 0
\]
for all $k\geq 0$, (\ref{QuestFourierForm}) is exact for
$f(x) = \cos((2k+1)x)$ with $k\geq 0$.  Moreover, since
\[
\int_0^{2\pi} \sin(kx) \dx{x} = -\frac{1}{k}\cos(kx)\bigg|_{x=0}^{2\pi}
= -\frac{1}{k}\left(\cos(2\pi k)-1\right) = 0
\]
and
\[
\pi \sin(kx)\bigg|_{x=0} + \pi\sin(kx)\bigg|_{x=\pi}
= 0 + \pi\sin(k\pi) = 0 - 0 = 0
\]
for $k>0$, (\ref{QuestFourierForm}) is exact for $f(x) = \sin(kx)$
with $k>0$.  It is also true for $f(x)=\sin(kx)$ with 
$k=0$ because $f(x)=0$ for all $x$ in this case.

By linearity of the integral, (\ref{QuestFourierForm}) is exact for
expressions of the form (\ref{QuestFourierType}).  It is not a really
interesting relation because the integral of (\ref{QuestFourierType})
in (\ref{QuestFourierForm}) is null and therefore does not require the
formula $\pi f(0) + \pi f(\pi)$ to be computed.

\noindent Note: Formula (\ref{QuestFourierForm}) is not true for
$f(x) = \cos(2k\,x)$ with $k\geq 1$.  This can be shown directly.
Another way to prove this is by contradiction.  Suppose that the
formula is true for $f(x) = \cos(2k\,x)$ with $k\geq 1$.  Since all
continuous functions on $[0,2\pi]$ can be expressed as a Fourier
series of $\cos(kx)$ and $\sin(kx)$ for $k\geq 0$,
(\ref{QuestFourierForm}) will then be true for all continuous 
functions on $[0,2\pi]$.  But the formula is not true for $f(x) =x$.
}

\solution{\SOL}{\ref{diffQ20}}{
The polynomial interpolation of $f$ at the points
$x_0=a + (b-a)/3=(2a+b)/3$ and $x_1 = a + 2(b-a)/3 = (a+2b)/3$ is
given by
\[
f(x) = f[x_0] + f[x_0,x_1]\left(x-x_0\right)
+f[x_0,x_1,x]\left(x-x_0\right)\left(x-x_1\right) \ .
\]
Hence,
\[
\int_a^b f(x) \dx{x}
= \int_a^b \left( f[x_0] + f[x_0,x_1]\left(x-x_0\right) \right)\dx{x}
+ \int_a^b f[x_0,x_1,x] \left(x-x_0\right)\left(x-x_1\right) \dx{x} \ .
\]

Since $\displaystyle f[x_0]= f(x_0) = f\left(\frac{2a+b}{2}\right)$
and
\[
f[x_0,x_1] = 
\frac{f\big((a+2b)/3\big) - f\big((2a+b)/3\big)}{(a+2b)/3 - (2a+b)/3}
=
\frac{3}{b-a} \left(f\left(\frac{a+2b}{3}\right)
- f\left(\frac{2a+b}{3}\right) \right) \ ,
\]
we get the formula
\begin{align*}
&\int_a^b f(x) \dx{x} \approx
\int_a^b \left( f[x_0] + f[x_0,x_1]\left(x-x_0\right)\right)\dx{x}
= \left( f(x_0)\, x + \frac{1}{2}\, f[x_),x_1]\,
\left(x-x_0\right)^2 \right)\bigg|_{x=a}^b \\
& \quad = f\left(\frac{2a+b}{3}\right) (b-a)
+ \frac{3}{2(b-a)} \left(f\left(\frac{a+2b}{3}\right)
- f\left(\frac{2a+b}{3}\right) \right)
\left( \left(b -\frac{2a+b}{3}\right)^2
-\left(a-\frac{2a+b}{3}\right)^2\right) \\
& \quad = \frac{b-a}{2}\,
\left(f\left(\frac{a+2b}{3}\right)+f\left(\frac{2a+b}{3}\right)\right) \ .
\end{align*}
We choose $A = B = (b-a)/2$.

For each $x\in[a,b]$, there exists $\xi \in [a,b]$ such that
\[
\left| f[x_0,x_1,x] \right| =
\left| \frac{1}{2}\,f''(\xi) \right| < \frac{M}{2} \ .
\]
Thus,
\begin{align*}
\left| \int_a^b f[x_0,x_1,x]
\left(x-x_0\right)\left(x-x_1\right) \dx{x} \right|
&\leq \int_a^b \left| f[x_0,x_1,x] \right|\,
\left| \left(x-x_0\right)\left(x-x_1\right) \right|
\dx{x} \\
&\leq \frac{M}{2} \int_a^b \left|
\left(x-x_0\right)\left(x-x_1\right) \right| \dx{x} \ .
\end{align*}
To compute this integral, we split the interval of integration in three
subintervals such that the sign of the integrand is constant on each
subinterval.  We can then eliminate the absolute value.
To simplify the computation, we use integration by parts to get
\begin{align*}
\int (x-x_0)(x-x_1) \dx{x}
&= \frac{(x-x_0)(x-x_1)^2}{2}- \frac{(x-x_1)^3}{6} + C_1
\intertext{and}
\int (x-x_0)(x-x_1) \dx{x}
&= \frac{(x-x_1)(x-x_0)^2}{2}- \frac{(x-x_0)^3}{6} + C_2
\end{align*}
for some constants $C_1$ and $C_2$.  It is interesting to note that if
we subtract the second integral from the first integral, we find
that $C_2 - C_1 = (x_1-x_0)^3/6$.
\begin{align*}
&\int_a^b \left| \left(x-x_0\right)\left(x-x_1\right) \right| \dx{x} \\
&\quad = \int_a^{x_0} \left(x-x_0\right)\left(x-x_1\right) \dx{x}
- \int_{x_0}^{x_1} \left(x-x_0\right)\left(x-x_1\right) \dx{x}
+ \int_{x_1}^b \left(x-x_0\right)\left(x-x_1\right) \dx{x} \\
&\quad
= \left(\frac{(x-x_1)(x-x_0)^2}{2}- \frac{(x-x_0)^3}{6}\right)\bigg|_{x=a}^{x_0}
-\left(\frac{(x-x_1)(x-x_0)^2}{2}- \frac{(x-x_0)^3}{6}\right)\bigg|_{x=x_0}^{x_1}
\\
&\quad \qquad
+\left(\frac{(x-x_0)(x-x_1)^2}{2}- \frac{(x-x_1)^3}{6}\right)\bigg|_{x=x_1}^{b}
\\
&\quad = -\left(\frac{(a-x_1)(a-x_0)^2}{2}- \frac{(a-x_0)^3}{6}\right)
+\frac{(x_1-x_0)^3}{6}
+\left(\frac{(b-x_0)(b-x_1)^2}{2}- \frac{(b-x_1)^3}{6}\right) \\
&\quad = -\frac{1}{2}\left(\frac{2(a-b)}{3}\right)
\left(\frac{a-b}{3}\right)^2+\frac{1}{6}\left(\frac{a-b}{3}\right)^3
+\frac{1}{6} \left(\frac{b-a}{3}\right)^3
+ \frac{1}{2} \left(\frac{2(b-a)}{3}\right)\left(\frac{b-a}{3}\right)^2
-\frac{1}{6} \left(\frac{b-a}{3}\right)^3 \\
&\quad = \frac{11(b-a)^3}{162} \ .
\end{align*}
Hence,
\[
\left| \int_a^b f[x_0,x_1,x]
\left(x-x_0\right)\left(x-x_1\right) \dx{x} \right|
\leq \frac{11 M(b-a)^3}{162} \ .
\]
}

\solution{\SOL}{\ref{diffQ21}}{
The polynomial interpolation of $f$ (of degree at most $2$) at the
points $x_0=a + (b-a)/4 = (3a+b)/4$,
$x_1=a+(b-a)/2 = (a+b)/2$ and $x_2=a+3(b-a)/4=(a+3b)/4$ is given by
\begin{align*}
f(x) &= f[x_0] + f[x_0,x_1]\,(x-x_0) + f[x_0,x_1,x_2]\,(x-x_0)(x-x_1) \\
&+ f[x_0,x_1,x_2,x]\,(x-x_0)(x-x_1)(x-x_2) \ .
\end{align*}
Hence,
\begin{equation}\label{QuestNIForm}
\begin{split}
\int_a^b f(x) \dx{x} &= \int_a^b \left(
f[x_0] + f[x_0,x_1]\,(x-x_0) + f[x_0,x_1,x_2]\,(x-x_0)(x-x_1)
\right)\dx{x} \\
&\qquad + \int_a^b f[x_0,x_1,x_2,x]\,(x-x_0)(x-x_1)(x-x_2)\dx{x} \ .
\end{split}
\end{equation}

The integration formula is given by the first integral on the right
side of (\ref{QuestNIForm}).  Since
\begin{align*}
f[x_0] &= f\left(\frac{3a+b}{4}\right) \ , \\
f[x_0,x_1] &= \frac{f(x_1)-f(x_0)}{x_1-x_0} =
\left(\frac{1}{(b+a)/2 - (3a+b)/4}\right)
\left(f\left(\frac{b+a}{2}\right) - f\left(\frac{3a+b}{4}\right)\right) \\
& = \frac{4}{b-a} \,\left(f\left(\frac{b+a}{2}\right)
- f\left(\frac{3a+b}{4}\right)\right) \ , \\
f[x_1,x_2] &= \frac{f(x_2)-f(x_1)}{x_2-x_1} =
\left(\frac{1}{(a+3b)/4 - (b+a)/2}\right)
\left(f\left(\frac{a+3b}{4}\right)-f\left(\frac{b+a}{2}\right)\right) \\
&= \frac{4}{b-a} \,\left(f\left(\frac{a+3b}{4}\right)
- f\left(\frac{b+a}{2}\right)\right)
\intertext{and}
f[x_0,x_1,x_2] &= \frac{f[x_1,x_2]-f[x_0,x_1]}{x_2-x_0} \\
&= \left(\frac{1}{(a+3b)/4 - (3a+b)/4}\right) \bigg(
\frac{4}{b-a} \,\left(f\left(\frac{a+3b}{4}\right)
- f\left(\frac{b+a}{2}\right)\right) \\
&\qquad - \frac{4}{b-a} \,\left(f\left(\frac{b+a}{2}\right)
- f\left(\frac{3a+b}{4}\right)\right) \bigg)\\
&= \frac{8}{(b-a)^2}
\left(f\left(\frac{a+3b}{4}\right)
- 2\,f\left(\frac{b+a}{2}\right)
+ f\left(\frac{3a+b}{4}\right)\right) \ ,
\end{align*}
we get
\begin{align*}
&\int_a^b f(x) \dx{x} \approx \int_a^b \big(
f[x_0] + f[x_0,x_1]\,(x-x_0) + f[x_0,x_1,x_2]\,(x-x_0)(x-x_1) \big)\dx{x}\\
&\quad = f\left(\frac{3a+b}{4}\right) \int_a^b \dx{x} +
\frac{4}{b-a} \,\left(f\left(\frac{b+a}{2}\right)
- f\left(\frac{3a+b}{4}\right)\right)
\int_a^b \left(x-\frac{3a+b}{4}\right)\dx{x} \\
&\qquad  + \frac{8}{(b-a)^2}
\left(f\left(\frac{a+3b}{4}\right) - 2\,f\left(\frac{b+a}{2}\right)
+ f\left(\frac{3a+b}{4}\right)\right)
\int_a^b \left(x-\frac{3a+b}{4}\right)\left(x-\frac{a+b}{2}\right)\dx{x} \ .
\end{align*}
Moreover, since
\begin{align*}
\int_a^b \left(x-\frac{3a+b}{4}\right)\dx{x} &=
\frac{1}{2} \left(x-\frac{3a+b}{4}\right)^2\bigg|_{x=a}^b
= \frac{1}{4}\,\left(b-a\right)^2
\intertext{and}
\int_a^b \left(x-\frac{3a+b}{4}\right)\left(x-\frac{a+b}{2}\right)\dx{x} &=
\int_a^b \left( x^2-\frac{5a+3b}{4}\, x + \frac{3a^2+4ab+b^2}{8}
\right)\dx{x}\\
&= \left(\frac{x^3}{3} -\frac{5a+3b}{8}\, x^2 + \frac{3a^2+4ab+b^2}{8}\, x
\right)\bigg|_{x=a}^b
=\frac{1}{12}\,\left(b-a\right)^3 \ ,
\end{align*}
we finally get
\begin{align*}
\int_a^b f(x) \dx{x} & \approx \int_a^b \big(
f[x_0] + f[x_0,x_1]\,(x-x_0) + f[x_0,x_1,x_2]\,(x-x_0)(x-x_1) \big)\dx{x} \\
&= f\left(\frac{3a+b}{4}\right) \, (b-a)
+ \left(f\left(\frac{b+a}{2}\right) - f\left(\frac{3a+b}{4}\right)\right)
\, \left(b-a\right) \\
&\qquad + \frac{2}{3}
\left(f\left(\frac{a+3b}{4}\right)
- 2\,f\left(\frac{b+a}{2}\right)
+ f\left(\frac{3a+b}{4}\right)\right)
\,\left(b-a\right) \\
&= \left(\frac{2}{3}\, f\left(\frac{a+3b}{4}\right)
- \frac{1}{3}\,f\left(\frac{b+a}{2}\right)
+ \frac{2}{3}\,f\left(\frac{3a+b}{4}\right)\right)
\,\left(b-a\right) \ .
\end{align*}

The truncation error is given by the second integral on the right hand
side of (\ref{QuestNIForm}). Since, for each $x\in [a,b]$ there exists
$\xi\in [a,b]$ such that
\[
\left| f[x_0,x_1,x_2,x] \right| =
\left| \frac{1}{3!}\,f^{(3)}(\xi) \right| < \frac{M}{3!} \ ,
\]
we get
\begin{align*}
&\left| \int_a^b \left(f[x_0,x_1,x_2,x]\,(x-x_0)(x-x_1)(x-x_2)
  \right)\dx{x} \right| \\
&\quad \leq \int_a^b \left|f[x_0,x_1,x_2,x]\right|\,
\left|(x-x_0)(x-x_1)(x-x_2) \right|\dx{x}
\leq \frac{M}{6} \, \int_a^b
\left|(x-x_0)(x-x_1)(x-x_2) \right|\dx{x} \ .
\end{align*}
Because of the symmetry of the graph of
$p(x)=|(x-x_0)(x-x_1)(x-x_2)|$ with respect to the vertical line
$x=(a+b)/2$, we have
\begin{align*}
&\int_a^b \left|(x-x_0)(x-x_1)(x-x_2) \right|\dx{x} 
= 2\int_{x_1}^b \left|(x-x_0)(x-x_1)(x-x_2) \right|\dx{x} \\
&\qquad = - 2\int_{x_1}^{x_2} (x-x_0)(x-x_1)(x-x_2) \dx{x}
+ 2\int_{x_2}^b (x-x_0)(x-x_1)(x-x_2) \dx{x} \ .
\end{align*}
Using integration by parts with $u(x) = (x-x_0)(x-x_1)$ and
$v'(x) = x-x_2$, we get $u'(x) = (x-x_0) + (x-x_1)$,
$v(x) = (x-x_2)^2/2$ and
\begin{align*}
&\int (x-x_0)(x-x_1)(x-x_2) \dx{x}
= u(x)v(x) - \int u'(x) v(x) \dx{x} \\  
&\qquad = \frac{1}{2}(x-x_0)(x-x_1)(x-x_2)^2
- \frac{1}{2} \int (x-x_0)(x-x_2)^2\dx{x}
- \frac{1}{2} \int (x-x_1)(x-x_2)^2\dx{x} \ .
\end{align*}
Again, using integration by parts with $u(x) = (x-x_0)$ and
$v'(x) = (x-x_2)^2$ for the first integral on the right hand side
of the equation above, and $u(x) = (x-x_0)$ and $v'(x) = (x-x_2)^2$
for the second integral on the right hand side of the equation above,
we get
\begin{align*}
& \int (x-x_0)(x-x_1)(x-x_2) \dx{x} \\
&\quad = \frac{1}{2}(x-x_0)(x-x_1)(x-x_2)^2 - \frac{1}{2}
\left( \frac{1}{3}(x-x_0)(x-x_2)^3 - \frac{1}{12}(x-x_2)^4\right) \\
&\quad\qquad - \frac{1}{2}
\left( \frac{1}{3}(x-x_1)(x-x_2)^3 - \frac{1}{12}(x-x_2)^4\right) + C \\
&\quad= \frac{1}{2}(x-x_0)(x-x_1)(x-x_2)^2 - \frac{1}{6}(x-x_0)(x-x_2)^3
- \frac{1}{6}(x-x_1)(x-x_2)^3
+ \frac{1}{12}(x-x_2)^4 + C \ .
\end{align*}
Hence
\begin{align*}
&\int_a^b \left|(x-x_0)(x-x_1)(x-x_2) \right|\dx{x}
= - 2\, \bigg( \frac{1}{2}(x-x_0)(x-x_1)(x-x_2)^2 -
\frac{1}{6}(x-x_0)(x-x_2)^3 \\
&\quad \qquad - \frac{1}{6}(x-x_1)(x-x_2)^3
+ \frac{1}{12}(x-x_2)^4 \bigg)\bigg|_{x=x_1}^{x_2}
+2\, \bigg( \frac{1}{2}(x-x_0)(x-x_1)(x-x_2)^2 \\
&\quad \qquad - \frac{1}{6}(x-x_0)(x-x_2)^3
- \frac{1}{6}(x-x_1)(x-x_2)^3
+ \frac{1}{12}(x-x_2)^4 \bigg)\bigg|_{x=x_2}^b \\
&\quad =  2\, \bigg( -\frac{1}{6}(x_1-x_0)(x_1-x_2)^3
+ \frac{1}{12}(x_1-x_2)^4 \bigg)
+2\, \bigg( \frac{1}{2}(b-x_0)(b-x_1)(b-x_2)^2  \\
&\quad\qquad  - \frac{1}{6}(b-x_0)(b-x_2)^3
- \frac{1}{6}(b-x_1)(b-x_2)^3 + \frac{1}{12}(b-x_2)^4 \bigg) \\
&\quad =  2\left( \frac{1}{4} \left(\frac{b-a}{4}\right)^4 \right)
  +2 \left( \frac{9}{4} \left(\frac{b-a}{4}\right)^4 \right)
=  5 \left(\frac{b-a}{4}\right)^4  \ .
\end{align*}
Therefore,
\[
\left| \int_a^b \left(f[x_0,x_1,x_2,x]\,(x-x_0)(x-x_1)(x-x_2)
  \right)\dx{x} \right| \leq \frac{5 M}{6} \left(\frac{b-a}{4}\right)^4 \ .
\]
}

\solution{\SOL}{\ref{diffQ22}}{
\subQ{a} The polynomial interpolation of $f$ at the points $x_0=-2h$,
$x_1=-h$ and $x_2 = 0$ is given by
\begin{align*}
f(x) &= f[x_0] + f[x_0,x_1](x-x_0)+f[x_0,x_1,x_2](x-x_0)(x-x_1) \\
&\qquad +f[x_0,x_1,x_2,x](x-x_0)(x-x_1)(x-x_2) \ .
\end{align*}
Hence,
\begin{align*}
&\int_0^h f(x) \dx{x}
= \int_0^h \big( f[x_0] + f[x_0,x_1](x-x_0)
+f[x_0,x_1,x_2](x-x_0)(x-x_1) \big)\dx{x} \\
&\qquad\qquad\qquad + \int_0^h f[x_0,x_1,x_2,x]
(x-x_0)(x-x_1)(x-x_2) \dx{x} \\
&\quad= \int_0^h \big( f[-2h] + f[-2h,-h](x+2h)
+f[-2h,-h,0](x+2h)(x+h) \big)\dx{x} \\
&\quad\qquad + \int_0^h f[-2h,-h,0,x] (x+2h)(x+h)x \dx{x} \\
&\quad=  f[-2h]\,x\bigg|_0^h + f[-2h,-h]\,
\left( \frac{x^2}{2} +2hx\right)\bigg|_0^h 
+f[-2h,-h,0]\left( \frac{x^3}{3} + \frac{3hx^2}{2} + 2h^2x\right)\bigg|_0^h \\
&\quad\qquad + \int_0^h f[-2h,-h,0,x] (x+2h)(x+h)x \dx{x} \\
&\quad = f[-2h]\,h + \frac{5}{2} f[-2h,-h]\,h^2
+ \frac{23}{6}f[-2h,-h,0]\,h^3 \\
&\quad\qquad + \int_0^h f[-2h,-h,0,x] (x+2h)(x+h)x \dx{x} \ .
\end{align*}

The formula to approximate the integral is
\begin{align*}
& \int_0^1 f(x)\dx{x} \approx
f[-2h]\,h + \frac{5}{2} f[-2h,-h]\,h^2 + \frac{23}{6}f[-2h,-h,0]\,h^3  \\
&\qquad = f(-2h)\,h + \frac{5}{2} \left(\frac{f(-h)-f(-2h)}{-h-(-2h)}\right) h^2
+ \frac{23}{6}
\left( \frac{\displaystyle \frac{f(0)-f(-h)}{0-(-h)}
 - \frac{f(-h)-f(-2h)}{-h-(-2h)}}{0-(-2h)}\right)h^3 \\
&\qquad = \left(\frac{23}{12}\,f(0) - \frac{4}{3} \, f(-h) + \frac{5}{12} \,
f(-2h)\right) h \ .
\end{align*}

Since $(x+2h)(x+h)x \geq 0$ for all $x\in[0,h]$, we may use the Mean
Value Theorem for Integrals to get $\eta \in [0,h]$ such that
\[
\int_0^h f[-2,-1,0,x] (x+2h)(x+h)x \dx{x}
 = f[-2h,-h,0,\eta] \int_0^h(x+2h)(x+h)x \dx{x} \ .
\]
Moreover, from the theory on divided difference formulas, there exists
$\xi \in [-2h,h]$ such that
\begin{align*}
f[-2h,-h,0,\eta] \int_0^h (x+2h)(x+h)x \dx{x}
&= \frac{1}{3!}f^{(3)}(\xi)
\int_0^h (x+2h)\,(x+h)\,x \dx{x} \\
&= \frac{1}{3!}f^{(3)}(\xi) \,\left(\frac{9}{4}\right)\, h^4
= \frac{3}{8}\,f^{(3)}(\xi)\,h^4 \ .
\end{align*}

Hence
\begin{equation}\label{formula1}
\int_0^h f(x) \dx{x}
= \left(\frac{23}{12}\,f(0) - \frac{4}{3}\,f(-h) +
\frac{5}{12}\,f(-2h)\right)\,h + \frac{3}{8}\,f^{(3)}(\xi)\,h^4
\end{equation}
for some $\xi \in [-2h,h]$.

\subQ{b} To obtain a formula for $\displaystyle \int_a^b f(x)\dx{x}$ from
(\ref{formula1}), we use the substitution $x= a + t$ with $h=b-a$ to get
\begin{align*}
\int_a^b f(x)\dx{x} &= \int_0^h f(a+t) \dx{t}
= \left(\frac{23}{12}\,f(a) - \frac{4}{3}\,f(a-h) +
\frac{5}{12}\,f(a-2h)\right)\,h
+ \frac{3}{8}\,\dfdxn{f(a+t)}{t}{3}\bigg|_{t=\xi} \, h^4 \\
&= \left(\frac{23}{12}\,f(a) - \frac{4}{3}\,f(a-h) +
\frac{5}{12}\,f(a-2h)\right)
+\frac{3}{8}\,f^{(3)}(a+\xi)\,h^4 \ .
\end{align*}

\subQ{c} The formula in (b) can be used to approximate the solution of
the initial value problem given.  If $y(a)$, $y(a-h)$ and
$y(a-2h)$ are known, we can used them to approximate $y(a+h)$.
We have
\[
\int_{a}^{a+h} y'(x) \dx{x} = y(x)\bigg|_{x=a}^{a+h}
= y(a+h) - y(a) \ .
\]
We also have
\begin{align*}
\int_{a}^{a+h} y'(x) \dx{x} &= \int_{a}^{a+h} f(x,y(x)) \dx{x} \\
&\approx \left(\frac{23}{12}\,f(a,y(a)) - \frac{4}{3}\,f(a-h,y(a-h)) +
\frac{5}{12}\,f(a-2h,y(a-2h))\right) \ .
\end{align*}
Hence,
\[
y(a+h) \approx y(a) + h\left(\frac{23}{12}\,f(a,y(a)) -
  \frac{4}{3}\,f(a-h,y(a-h)) + \frac{5}{12}\,f(a-2h,y(a-2y))\right) \ .
\]
}

\solution{\SOL}{\ref{diffQ23}}{
\subQ{a} The polynomial interpolating of $f$ at the points $x_0=-2h$,
$x_1=-h$ and $x_2 = 0$ is
\begin{align*}
f(x) &= f[x_0] + f[x_0,x_1](x-x_0)+f[x_0,x_1,x_2](x-x_0)(x-x_1) \\
&\qquad +f[x_0,x_1,x_2,x](x-x_0)(x-x_1)(x-x_2) \ .
\end{align*}
Hence,
\begin{align*}
&\int_{-h}^h f(x) \dx{x} = \int_{-h}^h \big( f[x_0] + f[x_0,x_1](x-x_0)
+f[x_0,x_1,x_2](x-x_0)(x-x_1) \big)\dx{x} \\
&\quad \qquad\qquad + \int_{-h}^h f[x_0,x_1,x_2,x]
(x-x_0)(x-x_1)(x-x_2) \dx{x} \\
&\quad = \int_{-h}^h \big( f[-2h] + f[-2h,-h](x+2h)
+f[-2h,-h,0](x+2h)(x+h) \big)\dx{x} \\
&\quad\qquad + \int_{-h}^h f[-2h,-h,0,x] (x+2h)(x+h)x \dx{x} \\
&\quad =  f[-2h]\,x\bigg|_{-h}^h
+ f[-2h,-h]\,\left(\frac{x^2}{2} + 2hx\right)\bigg|_{-h}^h
+f[-2h,-h,0]\left( \frac{x^3}{3} + \frac{3hx^2}{2}
+ 2h^2x\right)\bigg|_{-h}^h \\
&\quad\qquad + \int_{-h}^h f[-2h,-h,0,x] (x+2h)(x+h)x \dx{x} \\
&\quad =  2\,f[-2h]\,h + 4 f[-2h,-h]\,h^2 +
\frac{14}{3}f[-2h,-h,0]\,h^3 \\
&\quad\qquad + \int_{-h}^h f[-2h,-h,0,x] (x+2h)(x+h)x \dx{x} \ .
\end{align*}

The formula to approximate the integral is
\begin{align*}
&\int_{-h}^h f(x)\dx{x} \approx
2\,f[-2h]\,h + 4 f[-2h,-h]\,h^2 + \frac{14}{3}f[-2h,-h,0]\,h^3  \\
& \quad = 2\,f(-2h)\,h + 4 \left(\frac{f(-h)-f(-2h)}{-h-(-2h)}\right) h^2
 + \frac{14}{3}
\left(\frac{\displaystyle \frac{f(0)-f(-h)}{0-(-h)}
- \frac{f(-h)-f(-2h)}{-h-(-2h)}}{0-(-2h)}\right) h^3\\
&\quad = \left(\frac{7}{3}\,f(0) - \frac{2}{3}\,f(-h)
+ \frac{1}{3}\,f(-2h)\right)h \ .
\end{align*}

Since $(x+2h)(x+h)x$ changes sign at $x=0$ on the interval $[-h,h]$, we
may not directly use the Mean Value Theorem for Integrals to simplify
the truncation error.  However, from
\[
f[x_0,x_1,x_2,x_2,x] = \frac{f[x_0,x_1,x_2,x] - f[x_0,x_1,x_2,x_2]}{x-x_2} \ ,
\]
we get
\[
f[-2h,-h,0,x] = f[-2h,-h,0,0,x] x + f[-2h,-h,0,0] \ .
\]
Hence,
\begin{align*}
&\int_{-h}^h f[-2h,-h,0,x] (x+2h)(x+h)x \dx{x} \\
&\quad = \int_{-h}^h
\left( f[-2h,-h,0,0,x] x + f[-2h,-h,0,0] \right)
(x+2h)(x+h)x \dx{x} \\
&\quad = \int_{-h}^h f[-2h,-h,0,0,x] (x+2h)(x+h)x^2 \dx{x}
+f[-2h,-h,0,0] \int_{-h}^h (x+2h)(x+h)x \dx{x} \ .
\end{align*}
The second integral can be easily computed.  For the first integral,
Since $(x+2h)(x+h)x^2$ is non-negative on the interval $[-h,h]$, we may
use the Mean Value Theorem for Integrals to get $\eta \in [-h,h]$ such
that
\[
\int_{-h}^h f[-2h,-h,0,0,x] (x+2h)(x+h)x^2 \dx{x}
= f[-2h,-h,0,0,\eta] \int_{-h}^h(x+2h)(x+h)x^2 \dx{x} \ .
\]
Hence,
\begin{align*}
&\int_{-h}^h f[-2h,-h,0,x] (x+2h)(x+h)x \dx{x} \\
&\quad = f[-2h,-h,0,0,\eta] \int_{-h}^h (x+2h)(x+h)x^2 \dx{x}
 + f[-2h,-h,0,0] \int_{-h}^h (x+2h)(x+h)x \dx{x} \\
&\quad = \frac{26}{15} f[-2h,-h,0,0,\eta]\,h^5 + 2 f[-2h,-h,0,0]\,h^4 \ .
\end{align*}
Moreover, from the theory on divided difference formulas, there exists
$\xi_1, \xi_2 \in [-2h,h]$ such that
$\displaystyle f[-2h,-h,0,0,\eta] = \frac{1}{4!} f^{(4)}(\xi_1)$
and
$\displaystyle f[-2h,-h,0,0] = \frac{1}{3!} f^{(3)}(\xi_2)$.
The truncation error is therefore
\begin{align*}
\int_{-h}^h f[-2h,-h,0,x] (x+2h)(x+h)x \dx{x} 
&= \left(\frac{26}{15}\right) \frac{1}{4!} f^{(4)}(\xi_1)\, h^5
+ 2 \,\frac{1}{3!} f^{(3)}(\xi_2)\, h^4 \\
&= \frac{13}{180} \, f^{(4)}(\xi_1)\,h^5 + \frac{1}{3} f^{(3)}(\xi_2)\,h^4 \ .
\end{align*}
Hence,
\begin{equation}\label{Questformula1}
\int_{-h}^h f(x) \dx{x}
= \left(\frac{7}{3}\,f(0) - \frac{2}{3}\,f(-h) +
\frac{1}{3}\,f(-2h)\right)\,h
+ \frac{13}{180} \, f^{(4)}(\xi_1)\,h^5 + \frac{1}{3} f^{(3)}(\xi_2)\,h^4 \ .
\end{equation}

\subQ{b} To obtain a formula for $\displaystyle \int_a^b f(x)\dx{x}$ from
(\ref{Questformula1}), we use the substitution
$\displaystyle x= \frac{b+a}{2} + t$ with $\displaystyle h = \frac{b-a}{2}$
to get
\begin{align*}
&\int_a^b f(x)\dx{x} = \int_{-h}^h
f\left(\frac{b+a}{2} + t\right) \dx{t} \\
&\quad = \left(\frac{7}{3}\,f\left(\frac{a+b}{2}\right)
- \frac{2}{3}\,f(a)
+ \frac{1}{3}\,f\left(\frac{3a-b}{2}\right)\right)
\left(\frac{b-a}{2}\right) \\
&\quad\qquad + \frac{13}{180}\,
\dfdxn{f\left(\frac{b+a}{2}+t\right)}{t}{4}\bigg|_{t=\xi_1}
\left(\frac{b-a}{2}\right)^5 + \frac{1}{3} \,
\dfdxn{f\left(\frac{b+a}{2}+t\right)}{t}{3}\bigg|_{t=\xi_2}
\left(\frac{b-a}{2}\right)^4 \\  
&\quad= \left(\frac{7}{6}\,f\left(\frac{b+a}{3}\right)
- \frac{1}{3}\,f(a) + \frac{1}{6}\,f\left(\frac{3a-b}{2}\right)\right)
(b-a) \\
&\quad\qquad + \frac{13}{5960}\,
f^{(4)}\left(\frac{b+a}{2}+\xi_1\right) (b-a)^5 + \frac{1}{48} \,
f^{(3)}\left(\frac{b+a}{2}+\xi_2\right) (b-a)^4 \ .
\end{align*}
}

\solution{\SOL}{\ref{diffQ24}}{
It suffices to show that there exist $c_1$, $c_2$, \ldots, $c_k$ such
that (\ref{G_quadr}) is true for $p(x) = x^m$ with $0 \leq m <k$.  Namely,
\[
\int_a^b x^m\,w(x)\dx{x} = \sum_{j=1}^k\,c_j\,x_j^m
\]
for $0\leq m < k$.  This can be rewritten as the
system of linear equations  $A\VEC{c} = \VEC{d}$, where
\[
A = \begin{pmatrix}
1 & 1 & \ldots & 1 \\
x_1 & x_2 & \ldots & x_k \\
\vdots & \vdots & \ddots & \vdots \\
x_1^{k-1} & x_2^{k-1} & \ldots & x_k^{k-1} \\
\end{pmatrix}
\ , \quad
\VEC{c} = \begin{pmatrix} c_1 \\ c_2 \\ \ldots \\ c_k \end{pmatrix}
\quad \text{and} \quad
\VEC{d} = \begin{pmatrix} \int_a^b w(x)\dx{x} \\[0.5em]
\int_a^b x\,w(t)\dx{x} \\ \vdots \\
\int_a^b x^{k-1} w(x)\dx{x} \end{pmatrix} \ .
\]
This system has a unique solution because $A$ is an invertible
Vandermonde matrix.  In fact, one can show that the determinant of the
matrix $A$ above is $\displaystyle \prod_{0<i<j\leq k} (x_j - x_i)$.
}

\solution{\SOL}{\ref{diffQ25}}{
Let $p$ be a polynomial of degree less than $k+m$.  By the Euclidean
division algorithm, we get $p=P q + r$, where $q$ and $r$ are
polynomials of degree less than $m$.  Hence,
\begin{equation} \label{ord_2k_1}
\int_a^b p(x)\,w(x)\dx{x} =
\int_a^b P(x)\,q(x)\,w(x)\dx{x} +
\int_a^b r(x)\,w(x)\dx{x} =  \int_a^b r(x)\,w(x)\dx{x}
\end{equation}
because $\ps{P}{q} = 0$ since $q$ is of degree less than $m$.
Moreover,
\begin{equation} \label{ord_2k_2}
\sum_{j=1}^k c_j\,p(x_j) = \sum_{j=1}^k c_j\,P(x_j)\,q(x_j) +
\sum_{j=1}^k c_j\,r(x_j) = \sum_{j=1}^k c_j\,r(x_j)
\end{equation}
because $x_1$, $x_2$, \ldots, $x_k$ are the roots of $P$ by
hypothesis.
Finally, because (\ref{G_quadr}) is true for polynomials of degree
less than $k$ if the coefficients $c_i$ are defined by
(\ref{GquadrCoeff}), we have
\begin{equation} \label{ord_2k_3}
\int_a^b r(x)\,w(x)\dx{x} = \sum_{j=1}^k c_j\,r(x_j)
\end{equation}
since $r$ is of degree less than $m \leq k$.
It follows from (\ref{ord_2k_1}), (\ref{ord_2k_2}) and
(\ref{ord_2k_3}) that
\[
\int_a^b p(t)\,w(t)\dx{x} = \sum_{j=1}^k c_j\,p(x_j)
\]
for any polynomial of degree less than $k+m$.

We now prove that the quadrature formula is not true for all
polynomial of degree $k+m$.  Let $p(x) = x^m P(x) + r(x)$, where $r$
is a polynomial of degree less than $m$.  The equations
(\ref{ord_2k_2}) and (\ref{ord_2k_3}) are still true if we replace
$q(x)$ by $x^m$.  So
\[
  \int_a^b r(x)\,w(x)\dx{x} = \sum_{j=1}^k c_j\,p(x_j) \ .
\]
However,
\begin{align*}
\int_a^b p(x)\,w(x)\dx{x} &=
\underbrace{\int_a^b x^m P(x)\,w(x)\dx{x}}_{\neq 0} +
\int_a^b r(x)\,w(x)\dx{x}
\neq  \int_a^b r(x)\,w(x)\dx{x}
= \sum_{j=1}^k c_j\,p(x_j) \ .
\end{align*}
}

\solution{\SOL}{\ref{diffQ26}}{
If $f$ is $q$-time continuously differentiable on an open interval
containing $[a,b]$, it follows from Taylor's Theorem,
Theorem~\ref{TaylorTheo}, that $f(x) = p(x) + r(x)$, where
\[
p(x) = \sum_{j=0}^{q-1} \frac{f^{(j)}(a)}{j!} \, (x-a)^j \quad
\text{and} \quad r(x) = \frac{f^{(q)}(\xi(x))}{q!} \, (x-a)^q
\]
for some $\xi(x)$ between $a$ in $x$.  Thus,
\[
\int_a^b f(x) w(x) \dx{x} = \int_a^b p(x) w(x) \dx{x} + \int_a^b r(x)
w(x) \dx{x} 
= \sum_{j=1}^k b_j p(x_j) + \int_a^b r(x) w(x) \dx{x}
\]
because $p$ is a polynomial of degree at most $q-1$.  Moreover
\begin{align*}
\int_a^b r(x) w(x) \dx{x}
&= \frac{1}{q!} \int_a^b f^{(q)}(\xi(x))\, (x-a)^q w(x) \dx{x}
= \frac{1}{q!} f^{(q)}(\xi(c)) \int_a^b (x-a)^q w(x) \dx{x} \\
&= \frac{1}{q!} f^{(q)}(\xi(c))\,(b-a)^{q+1} \int_0^1 s^q w(a+(b-a)s)
\dx{s}
\end{align*}
for $c \in [a,b]$, where we have used the Mean Value Theorem for
Integrals, Theorem~\ref{Th4}, to get the second equality and the
substitution $x=a+(b-a)s$ to get the last one.  Finally, we also have
\[
\sum_{j=1}^k\,b_j\,f(c_j)
= \sum_{j=1}^k\,b_j\,p(c_j) + \sum_{j=1}^k\,b_j\,r(c_j)
= \sum_{j=1}^k\,b_j\,p(c_j) + \frac{1}{q!}\sum_{j=1}^k\,b_j\,
f^{(q)}(\xi(c_j)) \, (c_j-a)^q \ .
\]
Therefore,
\begin{align*}
&\bigg| \int_a^b \; f(x)\,w(x)\dx{x} - \sum_{j=1}^k\,b_j\,f(c_j)
\bigg| \\
&= \bigg|
\frac{1}{q!} f^{(q)}(\psi(x))\,(b-a)^{q+1} \int_0^1 s^q w(a+(b-a)s)
\dx{s}
- \frac{1}{q!}\sum_{j=1}^k\,b_j\,
f^{(q)}(\xi(c_j)) \, (c_j-a)^q \bigg| \\
&\leq \frac{1}{q!} \max_{a\leq x \leq b} |f^{(q)}(x)|\,(b-a)^{q+1} 
\int_0^1 s^q w(a+(b-a)s) \dx{s} 
+ \frac{1}{q!} \max_{a\leq x \leq b}|f^{(q)}(x)| \, (b-a)^q
\sum_{j=1}^k\,|b_j| \\
&= K (b-a)^q \max_{a\leq x \leq b}|f^{(q)}(x)| \ ,
\end{align*}
where
\[
K = \frac{1}{q!} \left( (b-a) \int_0^1 s^q w(a+(b-a)s) \dx{s} 
+ \sum_{j=1}^k\,|b_j| \right) \ .
\]
}

\solution{\SOL}{\ref{diffQ28}}{
To use Gauss-Legendre quadrature, we need to transform the intergral
between $1$ and $3$ into an integral between $-1$ and $1$.  For this
purpose, we use the change of variable $y = (x-M)/L$, where $M=2$ is
the middle of the interval $[1,3]$ and $L=1$ is half the length of the
interval $[1,3]$.   Thus $y=x-2$ and, if we solve for $x$, we get
$x=y+2$.  The integral (\ref{gauss_quadr}) becomes
\[
\int_1^3 x^2 \ln(x) \dx{x} = \int_{-1}^1 (y+2)^2 \ln(y+2) \dx{y}
\approx \sum_{j=1}^5\,c_j (y_j+2)^2\,\ln(y_j+2)
\approx 2.4040942246 \ ,
\]
where $y_i$ and $c_i$ are given in the following table.
\[
\begin{array}{lrr}
\hline
n & \text{roots } y_j & \text{coefficients } c_j \\
\hline
5 & -0.9061798459 & 0.2369268851 \\
  & -0.5384693101 & 0.4786286705 \\
  & 0.0 & 0.5688888889 \\
  & 0.5384693101 & 0.4786286705 \\
  & 0.9061798459 & 0.2369268851 \\
\hline
\end{array}
\]
}

\solution{\SOL}{\ref{diffQ29}}{
Because of the factor $\sqrt{(x-2)(3-x)}$ in the denominator of the
integrand, Gauss-Chebyshev quadrature is a possible choice.

To transform the integral from an integral between $2$ and $3$ to an
integral between $-1$ and $1$, we use the substitution
$t= (x-5/2)/(1/2)$.   So $x = t/2 + 5/2$, $\dx{x} = (1/2) \dx{t}$ and
\begin{align*}
\int_2^3 \frac{\sin(x)}{\sqrt{(x-2)(3-x)}} \dx{x}
&= \int_{-1}^1 \frac{\sin(t/2+5/2)}{\sqrt{1-t^2}} \dx{t}
\approx  \frac{\pi}{3} \sum_{j=1}^3 \sin\left(\frac{1}{2}
\cos\left(\frac{(2j-1)\pi}{6}\right) +\frac{5}{2}\right) \\
&\approx 1.7644706129 \ .
\end{align*}
}

\solution{\SOL}{\ref{diffQ32}}{
Because of the factor $\sqrt{x(1-x)}$ in the denominator of the
integrand, Gauss-Chebyshev quadrature is a possible choice.

To transform the integral from an integral between $0$ and $1$ to an
integral between $-1$ and $1$, we use the substitution $x=(t+1)/2$.
We get
\[
\int_0^1  \frac{e^x}{\sqrt{x(1-x)}} \dx{x}
= \int_{-1}^1 \frac{e^{(t+1)/2}}{\sqrt{1-t^2}} \dx{t}
\approx \frac{\pi}{3} \sum_{i=1}^3\, e^{ (\cos((2i-1)\pi/6) + 1)/2 }
\approx 5.50842622975 \ .
\]
}

\solution{\SOL}{\ref{diffQ34}}{
Because of the factor $\sqrt{(1-x)(3+x)}$ in the denominator of the
integrand, Gauss-Chebyshev quadrature is a possible choice.

Since we want the exact answer and the numerator of the integrand is a
polynomial of degree $4$, we have to take $n$ equal to at least $3$ as
suggested by Theorem~\ref{Gaussquadr}.

To transform the integral from an integral between $-3$ and $1$ to an
integral between $-1$ and $1$, we use the substitution $t = (x+1)/2$.
We get $x = 2t-1$ and
\begin{align*}
\int_{-3}^1 \frac{(1+x)^4}{\sqrt{(1-x)(3+x)}} \dx{x} &=
2 \int_{-1}^1 \frac{(2t)^4}{\sqrt{(2-2t)(2+2t)}} \dx{x}
= 16\, \int_{-1}^1 \frac{t^4}{\sqrt{1-t^2}} \dx{x} \\
&= \frac{16 \pi}{3} \sum_{j=1}^3 \cos^4\left(\frac{(2j-1)\pi}{6}\right)
= \frac{16\pi}{3}\, \left( \left(-\frac{\sqrt{2}}{3}\right)^4
+ \left(\frac{\sqrt{2}}{3}\right)^4\right)
= \frac{128\pi}{243} \ .
\end{align*}
}

\solution{\SOL}{\ref{diffQ36}}{
We have to find polynomials $P_k$ of degree exactly $k$ such that the
set $\{P_0, P_1, P_2, \ldots \}$ is an orthogonal set with respect to
the scalar product
\[
\ps{g}{h} = \int_0^1 g(x)\,h(x)\,x\dx{x} \ .
\]
To generate the family of orthogonal polynomials, we use
Theorem~\ref{orthpoly} with $\alpha_{k,k}=1$ for all $k$ and $w(x) =x$ for
$0\leq x\leq 1$.  Let $P_0(x) = 1$ for all $x$.  Since $A_0=1$ and
$\displaystyle
B_0 = \frac{\int_0^1 x^2 \dx{x}}{\int_0^1 x \dx{x}} = \frac{2}{3} \ ,
$
we get
\[
P_1(x) = (x-B_0)P_0(x) = x - \frac{2}{3} \ .
\]
Since $A_1=1$,
$\displaystyle
B_1 = \frac{\int_0^1 x^2(x-2/3)^2 \dx{x}}{\int_0^1 x(x-2/3)^2\dx{x}}
= \frac{8}{15}$ and
$\displaystyle
C_1 = \frac{\int_0^1 x(x-2/3)^2 \dx{x}}{\int_0^1 x\dx{x}}
=\frac{1}{18}$, we get
\[
P_2(x) = (x-B_1)P_1(x)-C_1P_0(x)
= \left(x- \frac{8}{15}\right)\left(x-\frac{2}{3}\right)
-\frac{1}{18} = x^2 - \frac{6}{5} x + \frac{3}{10} \ .
\]
According to Theorem~\ref{Gaussquadr}, we do not need higher
degree polynomials since we want a Gaussian quadrature
formula which is exact for polynomials of degree up to $3$ only.

The roots of $P_2$ are $x_1 = (6-\sqrt{6})/10$ and
$x_2 = (6+\sqrt{6})/10$.

The coefficients of (\ref{quadFormQuest1}) are
\begin{align*}
A &= \int_0^1 \left(\frac{x-x_2}{x_1-x_2}\right) x\dx{x}
= \frac{x^3/3 - x_2x^2/2}{x_1-x_2}\bigg|_{x=0}^1 =
\frac{9-\sqrt{6}}{36}
\intertext{and}
B &= \int_0^1 \left(\frac{x-x_1}{x_2-x_1}\right) x\dx{x}
= \frac{x^3/3-x_1x^2/2}{x_2-x_1}\bigg|_{x=0}^1 =
\frac{9+\sqrt{6}}{36} \ .
\end{align*}

It follows from Theorem~\ref{Gaussquadr} that (\ref{quadFormQuest1})
with the choice of $x_1$, $x_2$, $A$ and $B$ above is exact for
polynomials of degree up to $3$.
}

\solution{\SOL}{\ref{diffQ37}}{
We have to find polynomials $P_k$ of degree exactly $k$ such that the
set $\{P_0, P_1, P_2, \ldots \}$ is an orthogonal set with respect to
the scalar product
\[
\ps{g}{h} = \int_0^1 g(x)\,h(x)\,x^2\dx{x} .
\]
To generate the family of orthogonal polynomials, we use
Theorem~\ref{orthpoly} with $\alpha_{k,k}=1$ for all $k$ and $w(x) =x^2$ for
$0\leq x\leq 1$.  Let $P_0(x) = 1$ for all $x$.  Since $A_0=1$ and
$\displaystyle
B_0 = \frac{\int_{-1}^1 x^3 \dx{x}}{\int_{-1}^1 x^2 \dx{x}} = 0$,
we get
\[
P_1(x) = (x-B_0)P_0(x) = x \ .
\]
Since $A_1=1$,
$\displaystyle B_1 = \frac{\int_{-1}^1 x^5 \dx{x}}{\int_{-1}^1 x^4 \dx{x}}
= 0$ and
$\displaystyle C_1 = \frac{\int_{-1}^1 x^4 \dx{x}}{\int_{-1}^1 x^2\dx{x}}
=\frac{3}{5}$, we get
\[
P_2(x) = (x-B_1)P_1(x)-C_1P_0(x) = x^2 - \frac{3}{5} \ .
\]
According to Theorem~\ref{Gaussquadr}, we do not need higher degree
polynomials since we want a Gaussian quadrature formula
which is exact for polynomials of degree up to $3$ only.

The roots of $P_2$ are $x_1 = -\sqrt{3/5}$ and $x_2 = \sqrt{3/5}$.

The coefficients of (\ref{quadFormQuest2}) are
\begin{align*}
A &= \int_{-1}^1 \frac{x-x_2}{x_1-x_2}\, x^2\dx{x}
= \frac{x^4/4-x_2x^3/3}{x_1-x_2}\bigg|_{x=-1}^1 =
\frac{1}{3}
\intertext{and}
B &= \int_{-1}^1 \frac{x-x_1}{x_2-x_1}\, x^2\dx{x}
= \frac{x^4/4-x_1x^3/3}{x_2-x_1}\bigg|_{x=-1}^1 =
\frac{1}{3} \ .
\end{align*}

It follows from Theorem~\ref{Gaussquadr} that (\ref{quadFormQuest1})
with the choice of $x_1$, $x_2$, $A$ and $B$ above is exact for
polynomials of degree up to $3$.
}

\solution{\SOL}{\ref{diffQ38}}{
Recall that the Gauss-Chebyshev quadrature with $n>0$ is given by the
the formula
\begin{equation}\label{GCquestA}
\int_a^b f(x) \dx{x} \approx \int_a^b p(x) \dx{x} \ ,
\end{equation}
where $p$ is the interpolating polynomial of degree $n$ of $f$ at the
$n+1$ Chebishev points adjusted to the interval $[a,b]$; namely, at
the points
\[
x_j = \frac{a+b}{2} + \frac{b-a}{2}\cos\left(\frac{(2j-1)\pi}{2(n+1)}\right)
\]
for $j=1$, $2$, \ldots , $n+1$.

If we substitute $\displaystyle x= \frac{a+b}{2} + \frac{b-a}{2}\,t$
in the integrals in (\ref{GCquestA}), we get
\begin{align*}
\int_a^b f(x) \dx{x} &=
\frac{b-a}{2} \int_{-1}^1 f\left(\frac{a+b}{2} + \frac{b-a}{2}\,t\right) \dx{t}
\intertext{and}
\int_a^b p(x) \dx{x} &=
\frac{b-a}{2}  \int_{-1}^1 p\left(\frac{a+b}{2} + \frac{b-a}{2}\,t\right)
\dx{t} \ .
\end{align*}
$\displaystyle p\left(\frac{a+b}{2} + \frac{b-a}{2}\,t\right)$ is the
interpolating polynomial of
$\displaystyle f\left(\frac{a+b}{2} + \frac{b-a}{2}\,t\right)$ at the
Chebyshev points
$\displaystyle t_i = \cos\left( \frac{(2i-1)\pi}{2(n+1)}\right)$
for $1\leq i \leq n+1$.
From Proposition~\ref{ChebyshevTrancMax}, we have that
\begin{align*}
&\left| f\left(\frac{a+b}{2} + \frac{b-a}{2}\,t\right) -
p\left(\frac{a+b}{2} + \frac{b-a}{2}\,t\right) \right|
\leq \frac{1}{2^n(n+1)!}
\max_{-1\leq t \leq 1}\left| \dfdxn{
f\left(\frac{a+b}{2} + \frac{b-a}{2}\,t\right)}{t}{(n+1)} \right| \\
&\qquad
\leq \frac{1}{2^n(n+1)!}
\max_{-1\leq t \leq 1}\left|
f^{(n+1)}\left(\frac{a+b}{2} + \frac{b-a}{2}\,t\right)\right|
\left(\frac{b-a}{2}\right)^{n+1} \\
&\qquad
\leq \frac{1}{2^n(n+1)!} \max_{a\leq x \leq b}\left| f^{(n+1)}(x)\right|
\left(\frac{b-a}{2}\right)^{n+1}
\leq \frac{M(b-a)^{n+1}}{2^{2n+1}(n+1)!}
\end{align*}
for $-1\leq x \leq 1$.  Hence
\begin{align*}
&\left| \int_a^b f(x) \dx{x} - \int_a^b p(x) \dx{x} \right|
= \left| \frac{b-a}{2}
\int_{-1}^1 \left( f\left(\frac{a+b}{2} + \frac{b-a}{2}\,t\right) 
- p\left(\frac{a+b}{2} + \frac{b-a}{2}\,t\right)\right) \dx{t} \right| \\
&\qquad
\leq \frac{b-a}{2}
\int_{-1}^1 \left| f\left(\frac{a+b}{2} + \frac{b-a}{2}\,t\right) -
p\left(\frac{a+b}{2} + \frac{b-a}{2}\,t\right) \right| \dx{t} \\
& \qquad \leq
\frac{b-a}{2} \int_{-1}^1 \frac{M(b-a)^{n+1}}{2^{2n+1}(n+1)!} \dx{t}
= \frac{M(b-a)^{n+2}}{2^{2n+1}(n+1)!} \dx{t} \ .
\end{align*}
}

\solution{\SOL}{\ref{diffQ39}}{
Let $q$ be any polynomial of degree less than $n$, then
$\displaystyle f(x) = q(x)\,\prod_{j=1}^n (x-x_j)$
is a polynomial of degree less than $2n$.  By hypothesis, we then have
\[
\int_a^b f(x)\,w(x)\dx{x} = \sum_{i=1}^n a_i \,f(x_i)
= \sum_{i=1}^n \left( a_i \,q(x_i)\,\prod_{j=1}^n
\underbrace{(x_i-x_j)}_{=0\text{ for }j=i} \right) = 0 \ .
\]
}

\solution{\SOL}{\ref{diffQ40}}{
Consider the polynomial of degree $2n$ defined by
$\displaystyle f(x) = \prod_{i=1}^n (x-x_i)^2$.  If
(\ref{2nQuadrQuestEqu}) is exact for polynomials of degree $2n$, we
must have
\[
\int_a^b f(x)\,w(x) \dx{x} = \sum_{j=1}^n c_j \,f(x_j)
= \sum_{j=1}^n \left( c_j \prod_{i=1}^n
\underbrace{(x_j-x_i)^2}_{=0\text{ for }j=i} \right) = 0 \ .
\]
But this is not possible because the integral cannot be null since $f$
is a continuous function such that $f(x)>0$ for all
$x \in [a,b]\setminus \{x_1,x_2,\ldots,x_n\}$ and $w(x)>0$ almost everywhere.
}

%%% Local Variables:
%%% mode: latex
%%% TeX-master: "notes"
%%% End:
