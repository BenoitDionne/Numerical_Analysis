\chapter{Splines}\label{chaptInterB}

In the previous chapter, we showed how to generate a polynomial
whose graph traverses a set of points $(x_i,f(x_i))$ for $i=0$, $1$,
$2$, \ldots, $n$.  This polynomial could be of high degree and not be a
very good fit for the function $f$ that produced the points.  In the
present chapter, we describe several methods to generate a piecewise
polynomial function $p$ that may provide a good fit for the function $f$.
For some methods, the piecewise polynomial function $p$ may traverse
all the points $(x_i,f(x_i))$ but this is not a necessity.  Because
$p$ is a piecewise functions, it is possible to impose some conditions
on $p$ at the points $(x_i,f(x_i))$ (using what is called ``control
points'' ) to provide a good fit for the function $f$.

Some of the methods presented in this chapter could be use to generate
a piecewise parametric curve that traverses some points $(x_i,y_i)$ for
$0 \leq i \leq n$, and satisfies some conditions at these points by
adding some ``control points''.

\section{Cubic Spline Interpolation} \label{CSI}

Let $f:[a,b]\rightarrow \RR$ be a continuously differentiable
function and $a = x_0 < x_1 < \ldots < x_n = b$.  We have seen in
Remark~\ref{matlab} that MATLAB uses a piecewise linear function
through the points $(x_i,f(x_i))$ for $0 \leq i \leq n$ to sketch
the graph of $f$; to be precise, we should say that MATLAB plot a
piecewise linear curve that looks like the graph of $f$.  Instead of
using linear interpolation to join the points $(x_{i-1},f(x_{i-1}))$
and $(x_i,f(x_i))$, we now propose to use cubic polynomial
interpolation on the intervals $[x_{i-1},x_i]$.  The function 
$p$ that we get is called a
{\bfseries piecewise cubic polynomial}\index{Splines!Piecewise Cubic
Polynomial}.
Using cubic polynomials, we can impose a better fit between $f$ and
$p$ than with linear polynomials.  Cubic spline interpolation is ideal
to approximate function with discontinuous derivatives.

\begin{defn}
Let $f:[a,b]\rightarrow \RR$ be a continuously differentiable
function and $a = x_0 < x_1 < \ldots < x_n = b$.  A
{\bfseries free or natural spline interpolant}\index{Splines!Free or Natural
Spline Interpolant} for the function $f$ on
the {\bfseries nodes}\index{Splines!Nodes} $x_0$, $x_1$,\ldots, $x_n$ is a
piecewise cubic polynomial $p$ defined as follows.
\[
p(x) = p_i(x) \quad \text{ if } \quad  x_i \leq x \leq x_{i+1} \ ,
\]
where the $p_i$ are polynomials of degree three that satisfy
\begin{enumerate}
\item $p_i(x_i) = f(x_i)$ for $i=0$, $1$, \ldots , $n-1$,
\item $p_i(x_{i+1}) = f(x_{i+1})$ for $i=0$, $1$, \ldots, $n-1$,
\item $p'_i(x_{i+1}) = p'_{i+1}(x_{i+1})$ for $i=0$, $1$, \ldots, $n-2$,
\item $p''_i(x_{i+1}) = p''_{i+1}(x_{i+1})$ for $i=0$, $1$, \ldots, $n-2$,
\item $p''_0(x_0) = p''_{n-1}(x_n) = 0$.
\end{enumerate}
\label{cubicspline}
\end{defn}

\begin{defn}
If the fifth condition in Definition~\ref{cubicspline} is replaced
by
\begin{enumerate}
\setcounter{enumi}{4}
\item $p'_0(x_0) = f'(x_0)$ and $p'_{n-1}(x_n) = f'(x_n)$,
\end{enumerate}
then $p$ is called a
{\bfseries clamped spline interpolant}\index{Splines!Clamped Spline Interpolant}
for the function $f$ on the nodes $x_0$, $x_1$, \ldots, $x_n$.

\noindent If the third, fourth and fifth conditions in
Definition~\ref{cubicspline} are replaced by
\begin{enumerate}
\renewcommand{\labelenumii}{\arabic{enumii}.}
\setcounter{enumii}{2}
\item $p'_i(x_i) = f'(x_i)$ for $i=0$, $1$, \ldots, $n-1$,
\item $p'_i(x_{i+1}) = f'(x_{i+1})$ for $i=0$, $1$, \ldots, $n-1$,
\end{enumerate}
Then $p$ is called a
{\bfseries piecewise cubic Hermite interpolant}\index{Splines!Piecewise
Cubic Hermite Interpolant} for the function $f$ on the nodes $x_0$,
$x_1$,\ldots, $x_n$.
\end{defn}

We now describe how to find the cubic polynomials $p_i$ needed for the 
spline interpolant.  Let $z_i = p''(x_i)$ for $i=0$, $1$, \ldots, $n$.  
We need to find the values of the $z_i$'s to satisfy the natural or
clamped cubic splines.

Since we assume that the $p_i$'s are cubic polynomials, $p''_i$ is a
linear function through the points $(x_i,z_i)$ and
$(x_{i+1}, z_{i+1})$.  Recall that $\dtx{x_i} = x_{i+1} - x_i$.  Hence, 
\begin{align*}
p''_i(x) &= \frac{z_{i+1}-z_i}{\dtx{x_i}} x +
\frac{x_{i+1}z_i-x_i z_{i+1}}{\dtx{x_i}}
= \left(\frac{z_i}{\dtx{x_i}}\right) (x_{i+1}-x)
+ \left(\frac{z_{i+1}}{\dtx{x_i}}\right) (x-x_i) \ .
\end{align*}
Integrating twice gives
\begin{align}
p_i(x) &= \left(\frac{z_i}{6 \dtx{x_i}}\right) (x_{i+1}-x)^3
+ \left(\frac{z_{i+1}}{6 \dtx{x_i}}\right) (x-x_i)^3
+ A_i x + B_i \nonumber \\
&= \left(\frac{z_i}{6 \dtx{x_i}}\right) (x_{i+1}-x)^3
+ \left(\frac{z_{i+1}}{6 \dtx{x_i}}\right) (x-x_i)^3
+ C_i (x - x_i) + D_i (x_{i+1}-x) \ , \label{ncs1}
\end{align}
where $C_i-D_i=A_i$ and $D_i x_{i+1} - C_i x_i = B_i$.

From $p_i(x_i) = f(x_i)$ and $p_i(x_{i+1}) = f(x_{i+1})$, we get
\[
f(x_i) = \left(\frac{z_i}{6}\right) (\dtx{x_i})^2
+ D_i \dtx{x_i}
\quad \text{and} \quad
f(x_{i+1}) = \left(\frac{z_{i+1}}{6}\right) (\dtx{x_i})^2
+ C_i \dtx{x_i} \ .
\]
Solving for $C_i$ and $D_i$, we get
\[
C_i = \frac{f(x_{i+1})}{\dtx{x_i}} - \frac{z_{i+1}\dtx{x_i}}{6}
\quad \text{and} \quad
D_i = \frac{f(x_i)}{\dtx{x_i}}-\frac{z_i \dtx{x_i}}{6} \ .
\]
If we substitute these values of $C_i$ and $D_i$ in (\ref{ncs1}), we
get
\begin{equation} \label{ncs2}
\begin{split}
p_i(x) &= \left(\frac{z_i}{6 \dtx{x_i}}\right) (x_{i+1}-x)^3
+ \left(\frac{z_{i+1}}{6 \dtx{x_i}}\right) (x-x_i)^3 \\
&\qquad + \left(\frac{f(x_{i+1})}{\dtx{x_i}}
-\frac{z_{i+1} \dtx{x_i}}{6}\right)(x-x_i)
+ \left( \frac{f(x_i)}{\dtx{x_i}}-\frac{z_i \dtx{x_i}}{6}\right)
(x_{i+1}-x) \ .
\end{split}
\end{equation}

To determine the values of the $z_i$'s, we will used the property that
$p'_i(x_i) = p'_{i-1}(x_i)$ for $1\leq i \leq n-1$.  This gives $n-1$
equations to determine the $n+1$ variables $z_i$ for $i=0$, $1$,
\ldots, $n$.

\subsection{Natural Spline}

For the natural spline interpolant, we set $z_0 = z_n = 0$ and
determine the values of the other $z_i$'s using 
$p'_i(x_i) = p'_{i-1}(x_i)$ for $1\leq i \leq n-1$.

From (\ref{ncs2}), we get
\begin{equation} \label{ncs4}
\begin{split} 
p'_i(x) &= -\left(\frac{z_i}{2 \dtx{x_i}}\right) (x_{i+1}-x)^2
+ \left(\frac{z_{i+1}}{2 \dtx{x_i}}\right) (x-x_i)^2 \\
&\qquad + \left(\frac{f(x_{i+1})}{\dtx{x_i}}
-\frac{z_{i+1} \dtx{x_i}}{6}\right)
- \left( \frac{f(x_i)}{\dtx{x_i}}-\frac{z_i \dtx{x_i}}{6}\right)
\end{split}
\end{equation}
for $i=0$, $1$, \ldots, $n-1$.  Hence,
\begin{align*}
p'_i(x_i) &= -\left(\frac{z_i}{2}\right) \dtx{x_i}
+ \left(\frac{f(x_{i+1})}{\dtx{x_i}}-\frac{z_{i+1} \dtx{x_i}}{6}\right)
- \left( \frac{f(x_i)}{\dtx{x_i}}-\frac{z_i \dtx{x_i}}{6}\right) \\
&= -\frac{z_{i+1} \dtx{x_i}}{6} - \frac{z_i \dtx{x_i}}{3}
+ \frac{f(x_{i+1})-f(x_i)}{\dtx{x_i}} \ .
\end{align*}
Similarly,
\begin{align*}
p'_{i-1}(x) &= -\left(\frac{z_{i-1}}{2 \dtx{x_{i-1}}}\right) (x_i-x)^2
+ \left(\frac{z_i}{2 \dtx{x_{i-1}}}\right) (x-x_{i-1})^2 \\
&\qquad  + \left(\frac{f(x_i)}{\dtx{x_{i-1}}}
-\frac{z_i \dtx{x_{i-1}}}{6}\right)
- \left( \frac{f(x_{i-1})}{\dtx{x_{i-1}}}
- \frac{z_{i-1} \dtx{x_{i-1}}}{6}\right)
\end{align*}
for $i=1$, $2$, \ldots, $n$.  Hence,
\begin{align*}
p'_{i-1}(x_i) &= \left(\frac{z_i \dtx{x_{i-1}}}{2}\right)
+ \left(\frac{f(x_i)}{\dtx{x_{i-1}}}-\frac{z_i \dtx{x_{i-1}}}{6}\right)
- \left( \frac{f(x_{i-1})}{\dtx{x_{i-1}}}
- \frac{z_{i-1} \dtx{x_{i-1}}}{6}\right) \\
&= \frac{z_i \dtx{x_{i-1}}}{3} + \frac{z_{i-1} \dtx{x_{i-1}}}{6}
+ \frac{f(x_i)-f(x_{i-1})}{\dtx{x_{i-1}}} \ .
\end{align*}

The relation $p'_i(x_i) = p'_{i-1}(x_i)$ yields
\begin{equation} \label{ncs5}
\begin{split}
&z_{i+1} \dtx{x_i}  +2 z_i \left( \dtx{x_i} + \dtx{x_{i-1}}\right)
+ z_{i-1} \dtx{x_{i-1}} \\
&\qquad = \frac{6}{\dtx{x_i}}\left(f(x_{i+1}) -f(x_i) \right)
- \frac{6}{\dtx{x_{i-1}}} \left(f(x_i) - f(x_{i-1}) \right)
\end{split}
\end{equation}
for $i=1$, $2$, \ldots, $n-1$.  We conclude that the $z_i$'s for
$1\leq i \leq n-1$ are given by the solution of the
$n-1$ dimensional linear system $A \VEC{z} = \VEC{b}$,
where
\begin{equation} \label{ncs6}
A = \left( \begin{array}{ccccccccc}
d_1  & u_1 & 0 & \ldots & \ldots & \ldots & \ldots & \ldots & \ldots \\ 
l_2 & d_2  & u_2 & 0 & \ldots & \ldots & \ldots & \ldots & \ldots \\ 
0 & l_3 & d_3 & u_3 & \ldots & \ldots & \ldots & \ldots & \ldots \\ 
\vdots & 0 & l_4 & d_4 & \ldots & \ldots & \ldots & \ldots & \ldots \\ 
\vdots & \vdots & 0 & l_5 & \ldots & \ldots & \ldots & \ldots & \ldots \\ 
\vdots & \vdots & \vdots & \vdots & \ddots & \ldots & \ldots &
\ldots & \ldots \\
\vdots & \vdots & \vdots & \vdots & \vdots & u_{n-5} & 0 & \ldots & \ldots \\ 
\vdots & \vdots & \vdots & \vdots & \vdots & d_{n-4} &  u_{n-4} & 0 & \ldots \\ 
\vdots & \vdots & \vdots & \vdots & \vdots & l_{n-3} &  d_{n-3} & u_{d-3} & 0 \\ 
\vdots & \vdots & \vdots & \vdots & \vdots & 0 & l_{n-2} & d_{n-2} & u_{n-2} \\
\vdots & \vdots & \vdots & \vdots & \vdots & \vdots & 0 & l_{n-1} & d_{n-1} 
\end{array} \right)
\end{equation}
and
\[
\VEC{b} = \begin{pmatrix}
\displaystyle - z_0 \dtx{x_0} + \frac{6}{\dtx{x_1}}\left(f(x_2)
  -f(x_1) \right) - \frac{6}{\dtx{x_0}} \left(f(x_1) - f(x_0) \right) \\[0.8em]
\displaystyle \frac{6}{\dtx{x_2}}\left(f(x_3) -f(x_2) \right)
- \frac{6}{\dtx{x_1}} \left(f(x_2) - f(x_1) \right) \\[0.8em]
\vdots \\
\displaystyle \frac{6}{\dtx{x_{n-2}}}\left(f(x_{n-1}) -f(x_{n-2}) \right)
- \frac{6}{\dtx{x_{n-3}}} \left(f(x_{n-2}) - f(x_{n-3}) \right) \\[0.8em]
\displaystyle - z_n \dtx{x_{n-1}}
+ \frac{6}{\dtx{x_{n-1}}}\left(f(x_n) -f(x_{n-1}) \right)
- \frac{6}{\dtx{x_{n-2}}}\left(f(x_{n-1}) - f(x_{n-2}) \right)
\end{pmatrix}
\]
with $d_i = 2(\dtx{x_{i-1}} + \dtx{x_i})$, $u_i = \dtx{x_i}$ and
$l_i = \dtx{x_{i-1}}$.

To evaluate the polynomial $p_i$ defined in (\ref{ncs2}), we rewrite
it in nested form.  If we expand $p_i$ around $x = x_i$, we get
\begin{align*}
p_i(x) &= \frac{z_{i+1}-z_i}{6 \dtx{x_i}} (x-x_i)^3
+ \frac{z_i}{2} (x-x_i)^2 \\
&+ \left(-\frac{z_i \dtx{x_i}}{3} - \frac{z_{i+1} \dtx{x_i}}{6} +
\frac{f(x_{i+1})-f(x_i)}{\dtx{x_i}}\right) (x-x_i) + f(x_i) \ .
\end{align*}
Thus,
\begin{equation}\label{ncs3}
p_i(x) = \left(\left(\alpha_i(x-x_i) + \beta_i\right)(x-x_i)
+\gamma_i\right)(x-x_i) + \delta_i \ ,
\end{equation}
where
\begin{equation}\label{ncs7}
\begin{split}
\delta_i &= f(x_i) \ ,\\
\gamma_i &= -\frac{z_i \dtx{x_i}}{3} - \frac{z_{i+1} \dtx{x_i}}{6} +
\frac{f(x_{i+1})-f(x_i)}{\dtx{x_i}} \ ,\\
\beta_i &= \frac{z_i}{2} \ ,\\
\alpha_i &= \frac{z_{i+1}-z_i}{6 \dtx{x_i}} \ .
\end{split}
\end{equation}

\begin{egg}
Using the information in the table below, construct the natural spline
interpolant for $f$ on the nodes $0$, $1$, $2$, $3$ and $5$.
\[
\begin{array}{c|c}
x & f(x) \\
\hline
0 & 1 \\
1 & 0.540302305868140 \\
2 & -0.416146836547142 \\
3 & -0.989992496600445 \\
5 & 0.283662185463226
\end{array}
\]
All the numerical results displayed below will be rounded to $10$
digits.  The computations are done with more precision.

We have
\[
p_i(x) = \left(\left(\alpha_i(x-x_i) + \beta_i\right)(x-x_i)
+\gamma_i\right)(x-x_i) + \delta_i
\]
on $[x_i,x_{i+1}]$ for $0\leq i \leq 3$, where $x_0=0$, $x_1=1$,
$x_2=2$, $x_3=3$ and $x_4=5$.

Let $\VEC{w} \in \RR^3$ be the solution of $A\VEC{w} = \VEC{b}$, where
\[
A = \begin{pmatrix}
2 (x_2-x_0) & x_2-x_1 & 0 \\
x_2-x_1 & 2(x_3-x_1) & x_3-x_2 \\
0 & x_3-x_2 & 2(x_4-x_2)
\end{pmatrix}
= \begin{pmatrix}
4 & 1 & 0 \\
1 & 4 & 1 \\
0 & 1 & 6
\end{pmatrix}
\]
and
\[
\VEC{b} = \begin{pmatrix}
\displaystyle 6\,\frac{f(x_2) - f(x_1)}{x_2-x_1} -
6\,\frac{f(x_1) - f(x_0)}{x_1-x_0} \\[0.8em]
\displaystyle 6\,\frac{f(x_3) - f(x_2)}{x_3-x_2} -
6\,\frac{f(x_2) - f(x_1)}{x_2-x_1} \\[0.8em]
\displaystyle 6\,\frac{f(x_4) - f(x_3)}{x_4-x_3} -
6\,\frac{f(x_3) - f(x_2)}{x_3-x_2}
\end{pmatrix}
= \begin{pmatrix}
-2.9805086897 \\
2.29562089 \\
7.26403801
\end{pmatrix} \ .
\]
We find
\[
\VEC{w} = \begin{pmatrix}
-0.8728068282 \\ 0.5107186229 \\ 1.125553231
\end{pmatrix} \ .
\]

Let
\[
\VEC{z} = \begin{pmatrix}
0 \\ -0.8728068282 \\ 0.5107186229 \\ 1.125553231 \\ 0
\end{pmatrix} \ .
\]
The coefficients of $p_i$ are given by
\begin{align*}
\delta_i &= f(x_i)  \ , \\
\gamma_i &= -\frac{z_i (x_{i+1}-x_i)}{3} - \frac{z_{i+1}(x_{i+1}-x_i)}{6} +
\frac{f(x_{i+1})-f(x_i)}{x_{i+1} - x_i}  \ , \\
\beta_i &= \frac{z_i}{2}
\intertext{and}
\alpha_i &= \frac{z_{i+1}-z_i}{6 (x_{i+1}- x_i)}
\end{align*}
for $i=0$, $1$, $2$ and $3$.

The following table gives the values of the coefficients of
$p_i$.
\[
\begin{array}{c|cccc}
i & \alpha_i & \beta_i & \gamma_i & \delta_i \\
\hline
0 & -0.1454678047 & 0 & -0.3142298894 & 1 \\
1 & 0.2305875752 & -0.4364034141 & -0.7506333035 & 0.5403023059 \\
2 & 0.1024724346 & 0.2553593115 & -0.9316774061 & -0.4161468365 \\
3 & -0.09379610255 & 0.5627766153 & -0.1135414794 & -0.9899924966
\end{array}
\]
\end{egg}

\subsection{Clamped Spline}

For the clamped spline interpolant, $z_0$ and $z_n$ are free but
we have the additional constraints $p'(x_0) = p_0'(x_0) = f'(x_0)$ and
$p'(x_n) = p_{n-1}'(x_n) = f'(x_n)$.  Using (\ref{ncs4}), we get
\begin{align*}
f'(x_0) &= p_0'(x_0) = - \frac{z_0 \dtx{x_0}}{3} - \frac{z_1 \dtx{x_0}}{6}
+ \frac{f(x_1)-f(x_0)}{\dtx{x_0}}
\intertext{and}
f'(x_n) &= p_{n-1}'(x_n) =
\frac{z_n \dtx{x_{n-1}}}{3} + \frac{z_{n-1} \dtx{x_{n-1}}}{6}
+ \frac{f(x_n)-f(x_{n-1})}{\dtx{x_{n-1}}} \ .
\end{align*}

Hence
\begin{align*}
2 z_0  \dtx{x_0} + z_1 \dtx{x_0} &= - 6f'(x_0)
+ \frac{6}{\dtx{x_0}}\left(f(x_1)-f(x_0)\right)
\intertext{and}
z_{n-1} \dtx{x_{n-1}} + 2 z_n \dtx{x_{n-1}}  &= 6f'(x_n)
- \frac{6}{\dtx{x_{n-1}}}\left(f(x_n)-f(x_{n-1})\right) \ .
\end{align*}
These two equations give two linear equations that we may add to the
$(n-1)$ linear equations given in (\ref{ncs5}).

If we define $\dtx{x_{-1}} = 0$ and $\dtx{x_n} = 0$,
the $z_i$'s for $0\leq i \leq n$ are given by the solution of the
$n+1$ dimensional linear system $A \VEC{z} = \VEC{b}$,
where
\begin{equation} \label{ccs1}
A = \left( \begin{array}{ccccccccc}
d_0  & u_0 & 0 & \ldots & \ldots & \ldots & \ldots & \ldots & \ldots \\ 
l_1 & d_1  & u_1 & 0 & \ldots & \ldots & \ldots & \ldots & \ldots \\ 
0 & l_2 & d_2 & u_2 & \ldots & \ldots & \ldots & \ldots & \ldots \\ 
\vdots & 0 & l_3 & d_3 & \ldots & \ldots & \ldots & \ldots & \ldots \\ 
\vdots & \vdots & 0 & l_4 & \ldots & \ldots & \ldots & \ldots & \ldots \\ 
\vdots & \vdots & \vdots & \vdots & \ddots & \ldots & \ldots &
\ldots & \ldots \\ 
\vdots & \vdots & \vdots & \vdots & \vdots & u_{n-4} & 0 & \ldots & \ldots \\ 
\vdots & \vdots & \vdots & \vdots & \vdots & d_{n-3} &  u_{n-3} & 0 & \ldots \\ 
\vdots & \vdots & \vdots & \vdots & \vdots & l_{n-2} &  d_{n-2} & u_{d-2} & 0 \\ 
\vdots & \vdots & \vdots & \vdots & \vdots & 0 & l_{n-1} & d_{n-1} & u_{n-1} \\
\vdots & \vdots & \vdots & \vdots & \vdots & \vdots & 0 & l_n & d_n 
\end{array} \right)
\end{equation}
and
\[
\VEC{b} = \begin{pmatrix}
\displaystyle - 6f'(x_0) + \frac{6}{\dtx{x_0}}\left(f(x_1)-f(x_0)\right)
\\[0.8em]
\displaystyle \frac{6}{\dtx{x_1}}\left(f(x_2) -f(x_1) \right)
- \frac{6}{\dtx{x_0}} \left(f(x_1) - f(x_0) \right) \\[0.8em]
\displaystyle \frac{6}{\dtx{x_2}}\left(f(x_3) -f(x_2) \right)
- \frac{6}{\dtx{x_1}} \left(f(x_2) - f(x_1) \right) \\[0.8em]
\vdots \\
\displaystyle \frac{6}{\dtx{x_{n-2}}}\left(f(x_{n-1}) -f(x_{n-2}) \right)
- \frac{6}{\dtx{x_{n-3}}} \left(f(x_{n-2}) - f(x_{n-3}) \right) \\[0.8em]
\displaystyle \frac{6}{\dtx{x_{n-1}}}\left(f(x_n) -f(x_{n-1}) \right)
- \frac{6}{\dtx{x_{n-2}}} \left(f(x_{n-1}) - f(x_{n-2}) \right) \\[0.8em]
\displaystyle 6f'(x_n) - \frac{6}{\dtx{x_{n-1}}}\left(f(x_n)-f(x_{n-1})\right)
\end{pmatrix}
\]
with $d_i$, $u_i$ and $l_i$ for $0 \leq i \leq n$ defined as for
the natural spline before.

The expression for $p_i$ given in (\ref{ncs3}) and (\ref{ncs7}) is
still valid for the clamped cubic spline interpolant.

\begin{egg}
Using the information in the table below, construct the clamped spline
interpolant for $f$ on the nodes $0$, $0.3$ and $1$.
\[
\begin{array}{c|ccc}
x & 0 & 0.3 & 1 \\
\hline
f(x) & 1 & 0.548811636094027 & 0.135335283236613 \\
f'(x) & -2 & & -0.270670566473225
\end{array}
\]
All the numerical results displayed will be rounded to $10$
digits.  The computations are done with more precision.

We have
\[
p_i(x) = \left(\left(\alpha_i(x-x_i) + \beta_i\right)(x-x_i)
+\gamma_i\right)(x-x_i) + \delta_i
\]
on $[x_i,x_{i+1}]$ for $i=0$ and $1$, where $x_0 = 0$, $x_1 = 0.3$ and
$x_2=1$.

Let $\VEC{z} \in \RR^3$ be the solution of $A\VEC{z} = \VEC{b}$, where
\[
A = \begin{pmatrix}
2 (x_1-x_0) & x_1-x_0 & 0 \\
x_1-x_0 & 2(x_2-x_0) & x_2-x_1 \\
0 & x_2-x_1 & 2(x_2-x_1)
\end{pmatrix}
= \begin{pmatrix}
0.6 & 0.3 & 0 \\
0.3 & 2 & 0.7 \\
0 & 0.7 & 1.4
\end{pmatrix}
\]
and
\[
\VEC{b} = \begin{pmatrix}
\displaystyle - 6f'(x_0) + 6\, \frac{f(x_1) - f(x_0)}{x_1-x_0}\\[0.8em]
\displaystyle 6\,\frac{f(x_2) -f(x_1)}{x_2-x_1}
- 6\,\frac{f(x_1) - f(x_0)}{x_1-x_0}\\[0.8em]
\displaystyle 6f'(x_2) - 6\,\frac{f(x_2)-f(x_1)}{x_2-x_1}
\end{pmatrix}
= \begin{pmatrix}
2.976232722 \\
5.479684254 \\
1.920059626    
\end{pmatrix} \ .
\]
We find
\[
\VEC{z} = \begin{pmatrix}
3.949875177 \\ 2.021025387 \\ 0.3609584679
\end{pmatrix} \ .
\]

The coefficients of $p_i$ are given by
\begin{align*}
\delta_i &= f(x_i)  \ , \\
\gamma_i &= -\frac{z_i \dtx{x_i}}{3} - \frac{z_{i+1} \dtx{x_i}}{6} +
\frac{f(x_{i+1})-f(x_i)}{\dtx{x_i}}  \ , \\
\beta_i &= \frac{z_i}{2}
\intertext{and}
\alpha_i &= \frac{z_{i+1}-z_i}{6 \dtx{x_i}}
\end{align*}
for $i=0$ and $1$.

The following table gives the values of the coefficients of
$p_i$.
\[
\begin{array}{c|cccc}
i & \alpha_i & \beta_i & \gamma_i & \delta_i \\
\hline
0 & -1.071583217 & 1.974937588 & -2 & 1 \\
1 & -0.3952540283 & 1.010512693 & -1.104364916 & 0.5488116361
\end{array}
\]
\label{egg_clamped}
\end{egg}

We give below a code to find the clamped cubic spline interpolant.
We leave the task of writing a code to find the natural cubic spline
interpolant to the reader.

\begin{code}[Clamped Cubic Spline Interpolant - System] \label{codeCCSI}
This program computes the tridiagonal matrix $A$ and the right hand
side $\VEC{b}$ associated to the clamped cubic spline interpolant.\\
\subI{Input} The nodes $x_i$ for $0 \leq i \leq n$ (x(i+1) in the code
below).\\
The values $f(x_i)$ for $0 \leq i \leq n$ (f(i+1) in the code
below).\\
The values $f'(x_0)$ and $f'(x_n)$ (fx(1) and fx(2) respectively
in the code below).\\
\subI{Output} The lower diagonal $L$, the diagonal $D$ and the upper
diagonal $U$ of the tridiagonal matrix $A$.\\
The right hand side $\VEC{b}$ of $A\VEC{x} = \VEC{b}$.
\small
\begin{verbatim}
%  [L,D,U,b] = clampedsplinematrix(f,fx,x)

function [L,D,U,b] = campledsplinematrix(f,fx,x)
  N = length(x);
  L = repmat(NaN,1,N-1);
  U = repmat(NaN,1,N-1);
  D = repmat(NaN,1,N);
  b = repmat(NaN,1,N);

  dx = x(2)-x(1);
  if (dx == 0)
    return;
  end
  ratio = (f(2)-f(1))/dx;
  D(1) = 2*dx;
  U(1) = dx;
  b(1) = 6*(ratio - fx(1));

  for n=2:N-1
    prevdx = dx;
    dx = x(n+1)-x(n);
    if (dx == 0)
      return;
    end
    prevratio = ratio;
    ratio = (f(n+1)-f(n))/dx;
    L(n-1) = prevdx;
    D(n) = 2*(dx+prevdx);
    U(n) = dx;
    b(n) = 6*(ratio - prevratio);
  end

  L(N-1) = dx;
  D(N) = 2*dx;
  b(N) = 6*(fx(2) - ratio);
end
\end{verbatim}
\end{code}

\begin{code}[Tridiagonal Matrix]
To solve a system of the form $A\VEC{x} = \VEC{b}$, where $A$ is a
tridiagonal matrix.\\
\subI{Input} The lower diagonal $L$, the diagonal $D$ and the upper
diagonal $U$ of the tridiagonal matrix $A$.  None of the components of
the diagonal $D$ can be null.\\
The right hand side $\VEC{b}$.\\
\subI{Output} The solution if the system can be solved.
\small
\begin{verbatim}
%  z = tridmatrix(L,D,U,b)

function z = tridmatrix(L,D,U,b)
  m = length(D);
  z = repmat(NaN,1,m);

  for n=2:m
    if (D(n-1) == 0)
      return;
    end
    q = L(n-1)/D(n-1);
    D(n) = D(n)-q*U(n-1);
    b(n) = b(n)-q*b(n-1);
  end

  if (D(m) == 0)
    return;
  end

  % Backward substitution
  z(m) = b(m)/D(m);
  for n=(m-1):-1:1
    z(n)=(b(n)-U(n)*z(n+1))/D(n);
  end
end
\end{verbatim}
\end{code}

\begin{code}[Cubic Spline Interpolant - Polynomial]
To evaluate a cubic spline interpolant defined by
\[
  p(x) = (\alpha_i(x-x_i) + \beta_i)*(x-x_i) + \gamma_i)*(x-x_i) + \delta_i
\]
for $x_i < x \leq x_{i+1}$.\\
\subI{Input} The points $x_i$ for $0 \leq i \leq n$ (x(i+1) in the
code below).\\
The values $f(x_i)$ for $0 \leq i \leq n$ (f(i+1) in the code
below).\\
The solution $\VEC{z}$ of the system $A \VEC{z} = \VEC{b}$ associated
to the cubic spline used.\\
The values of $x$ where the cubic spline interpolant must
be evaluated (X in the code below).\\
\subI{Output} The value of the cubic spline interpolant at all the
given values of $x$.\\
The coefficients for each polynomials
\[
  p_i(x) = ((c_{i,1} (x-x_i) + c_{i,2})(x-x_i) + c_{i,3})(x-x_i) + c_{i,4}
\]
for $i=1$ ,$2$, \ldots, $n-1$ (the matrix coeffs in the code below).
\small
\begin{verbatim}
function [y, coeffs] = splinepoly(z,f,x,X)
  npoints = length(x);
  N = length(X);
  y = repmat(NaN,1,N);

  for m=1:1:npoints-1
    coeffs(m,4) = f(m);
    dx = x(m+1)-x(m);
    df = f(m+1)-f(m);
    coeffs(m,3) = -(2*z(m)+z(m+1))*dx/6 + df/dx;
    coeffs(m,2) = z(m)/2;
    coeffs(m,1) = (z(m+1)-z(m))/(6*dx);
  end

  for n=1:1:N
    J = 0;
    if ( X(n) >= x(1) && X(n) <= x(npoints) )
      for m = 2:1:npoints
        if ( X(n) <= x(m) )
          J = m-1;
          break;
        end
      end
      dx = X(n) - x(J);
      y(n) = ((coeffs(J,1)*dx + coeffs(J,2))*dx + coeffs(J,3))*dx ...
           + coeffs(J,4);
    end
  end
end
\end{verbatim}
\end{code}

\subsection{Existence of Interpolants}

The are a few questions that come naturally after the presentation of
the natural and clamped spline interpolants.  First, do the linear
systems of the form $A\VEC{z} = \VEC{b}$ used to find the natural and
clamped spline interpolants have always a solution?  If so, it is
unique?  How good are the natural and clamped spline interpolant?
We answer these questions below.

\begin{prop}
If $B$ is a strictly diagonally dominant \nn matrix, then
$B$ is invertible.
\end{prop}

\begin{proof}
Suppose that $\VEC{x} \neq \VEC{0}$ satisfies $B\VEC{x} = \VEC{0}$.
Let $k$ be an index such that
\[
|x_k| = \|\VEC{x}\|_\infty = \max_{1\leq i \leq n} |x_i| \ .
\]
We have $|x_k|>0$ because $\VEC{x}\neq \VEC{0}$.

From $\displaystyle \sum_{j=1}^n b_{k,j}x_j = 0$, we get
\[
b_{k,k} = \sum_{\substack{j=0\\j\neq k}}^n b_{k,j}
\left(\frac{x_j}{x_k}\right)  \ .
\]
Hence
\[
|b_{k,k}| \leq \sum_{\substack{j=0\\j\neq k}}^n |b_{k,j}|
\left|\frac{x_j}{x_k}\right|
\leq \sum_{\substack{j=0\\j\neq k}}^n |b_{k,j}| \  .
\]
This contradict that $B$ is strictly diagonally dominant.
\end{proof}

\begin{theorem}
Let $f:[a,b]\rightarrow \RR$ be a continuously differentiable
function and $a = x_0 < x_1 < \ldots < x_n = b$.  There exists a
unique natural cubic spline interpolant for $f$ on the nodes $x_0$,
$x_1$, \ldots, $x_n$.  Similarly, There exists a unique clamped cubic
spline interpolant for $f$ on the nodes $x_0$, $x_1$, \ldots, $x_n$.
\end{theorem}

\begin{proof}
Any natural cubic spline on the nodes $x_0$, $x_2$, \ldots,
$x_n$ has to satisfy the system $A\VEC{z} = \VEC{b}$ for $A$
given in (\ref{ncs6}).  Since $A$ is strictly diagonally dominant, it
follows from the previous proposition that $A$ is invertible.  Thus,
the solution of $A\VEC{z}=\VEC{b}$ is unique.  The same reasoning is
true for clamped cubic splines with $A$ given in (\ref{ccs1}).
\end{proof}

\begin{theorem}
If $p$ is the natural cubic spline interpolant for a function $f$ of
class $C^2$ on the nodes $a = x_0 < x_1 < \ldots < x_n=b$, then
\[
\int_a^b (p''(x))^2 \dx{x} \leq \int_a^b (f''(x))^2 \dx{x} \ .
\]
\end{theorem}

\begin{proof}
Let $g = f - p$.  We have
\begin{align*}
\int_a^b (f''(x))^2\dx{x} &= \int_a^b (g''(x)+p''(x))^2 \dx{x} \\
&= \int_a^b (g''(x))^2\dx{x} + \int_a^b (p''(x))^2 \dx{x}
+ 2\int_a^b g''(x) p''(x)\dx{x} \ .
\end{align*}
To prove the theorem, we show that
$\displaystyle \int_a^b g''(x) p''(x)\dx{x} = 0$.

Using integration by parts and $p''_{n-1}(x_n)=p''_0(x_0) = 0$ for
the natural cubic spline, we get
\begin{align*}
\int_a^b g''(x) p''(x)\dx{x} &= \sum_{j=0}^{n-1}
\int_{x_j}^{x_{j+1}} g''(x) p''_j(x)\dx{x} \\
&= \sum_{j=0}^{n-1} \left(g'(x_{j+1})p''_j(x_{j+1})
- g'(x_j)p''_j(x_j) \right) -
\sum_{j=0}^{n-1} \int_{x_j}^{x_{j+1}} g'(x) p'''_j(x)\dx{x} \\
&= g'(x_n)p''_{n-1}(x_n) - g'(x_0)p''_0(x_0) -
\sum_{j=0}^{n-1} \int_{x_j}^{x_{j+1}} g'(x)
\left( \frac{z_{i+1}-z_i}{\dtx{x_i}} \right) \dx{x} \\
&= - \sum_{j=0}^{n-1} \left( \frac{z_{i+1}-z_i}{\dtx{x_i}} \right)
\left( g(x_{j+1})-g(x_j) \right) = 0 \ .
\end{align*}
The last equality comes from $g(x_j)=0$ for all $j$ because
$p(x_j)=f(x_j)$ for all $j$.
\end{proof}

Using the approach presented in the next section, it is possible to
prove the following theorem.

\begin{theorem}
Let $f:[a,b]\rightarrow \RR$ be a four times continuously
differentiable function and suppose that
\[
\max_{x \in [a,b]}\,|f^{(4)}(x)| < M \ .
\]
If $p$ is the clamped cubic spline interpolant for $f$ on $x_0$,
$x_1$,\ldots, $x_n$ in $]a,b[$, then
\begin{align*}
\max_{x \in [a,b]}\,|f(x)-p(x)| &\leq \frac{5M}{384}\,
\max_{0\leq i<n-1}\,|\dtx{x_i}|^4 \\
\intertext{and}
\max_{x \in [a,b]}\,|f'(x)-p'(x)| &\leq \frac{M}{24}\,
\max_{0\leq i<n-1}\,|\dtx{x_i}|^3 \ .
\end{align*}
\end{theorem}

It follows from the previous theorem that the clamped cubic spline
polynomial $p$ can be a good fit for a function $f$ if
$\displaystyle \max_{0\leq i < n} |\Delta x_i|$ is small enough.

\subsection{Another Approach}

The presentation of the cubic spline that follows is based on
\cite{CdB}.

Suppose that $p$ is a cubic spline defined in
Definition~\ref{cubicspline}.  If we express $p_i$ as
\[
p_i(x) = p(x_i) + p[x_i,x_i] (x-x_i) + p[x_i,x_i,x_{i+1}](x-x_i)^2
+ p[x_i,x_i,x_{i+1},x_{i+1}](x-x_i)^2(x-x_{i+1})
\]
and substitute $(x-x_{i+1}) = (x-x_i) + (x_i-x_{i+1})$, we get
\begin{align*}
p_i(x) &= p(x_i) + p[x_i,x_i] (x-x_i) +
\left(p[x_i,x_i,x_{i+1}] - (x_{i+1}-x_i)p[x_i,x_i,x_{i+1},x_{i+1}]\right)
(x-x_i)^2 \\
&+ p[x_i,x_i,x_{i+1},x_{i+1}](x-x_i)^3 \ .
\end{align*}
Hence, we have
\[
p_i(x) = a_i + b_i\,(x-x_i) + c_i\,(x-x_i)^2 + d_i\,(x-x_i)^3 \  ,
\]
where
\begin{equation}\label{boor1}
\begin{split}
a_i &= p(x_i)  \ , \\
b_i &= p[x_i,x_i] = p'(x_i) \ , \\
d_i &= p[x_i,x_i,x_{i+1},x_{i+1}]
= \frac{p[x_i,x_{i+1},x_{i+1}] - p[x_i,x_i,x_{i+1}]}{\dtx{x_i}} \\
&= \frac{p[x_{i+1},x_{i+1}] - 2p[x_i,x_{i+1}] + p[x_i,x_i]}{(\dtx{x_i})^2} \\
& = \frac{b_{i+1} - 2 f[x_i,x_{i+1}] + b_i}{(\dtx{x_i})^2} \ ,\\
c_i &= p[x_i,x_i,x_{i+1}] - (x_{i+1}-x_i)p[x_i,x_i,x_{i+1},x_{i+1}] \\
&= \frac{p[x_i,x_{i+1}]-p[x_i,x_i]}{\dtx{x_i}}
- p[x_i,x_i,x_{i+1},x_{i+1}] \dtx{x_i}
= \frac{f[x_i,x_{i+1}]- b_i}{\dtx{x_i}} -  d_i \dtx{x_i} \\
&= \frac{-2b_i - b_{i+1} + 3f[x_i,x_{i+1}]}{\dtx{x_i}}
\end{split}
\end{equation}
for $i=0$, $1$, \ldots, $n-1$.  We have that
$p[x_i,x_{i+1}] = f[x_i,x_{i+1}]$ because $p(x_i) = f(x_i)$ for all
$i$.  We also have that $p[x_i,x_i] = p'(x_i)$ for $0\leq i \leq n$.

There are only $n+1$ unknowns in (\ref{boor1}); namely,
$b_i$ for $i=0$, $1$, \ldots, $n$.

The conditions $p''_{i-1}(x_i)= p''_i(x_i)$ for $1\leq i \leq n-1$
imply that
\[
2c_{i-1} + 6d_{i-1} \dtx{x_{i-1}} = 2 c_i
\]
for $1\leq i \leq n-1$.  Using the definitions of $c_i$ and $d_i$ in 
(\ref{boor1}), we get
\[
(\dtx{x_i}) b_{i-1}  + 2 (\dtx{x_i} + \dtx{x_{i-1}}) b_i
+ (\dtx{x_{i-1}}) b_{i+1} 
= 3 \left( f[x_{i-1},x_i] \dtx{x_i} + f[x_i,x_{i+1}] \dtx{x_{i-1}}
\right)
\]
for $1 \leq i \leq n-1$.  Since we have $n+1$ unknowns and $n-1$
equations, we have two free variables.  It is natural to take
$b_0$ and $b_n$ as free variables.

For the clamped cubic spline interpolant, we require
$p_0'(x_0) = f'(x_0)$ and $p_{n-1}'(x_n) = f'(x_n)$.  Since
$p_0'(x_0) = b_0$ and
\begin{align*}
p_{n-1}'(x_n) &= b_{n-1} + 2 c_{n-1} (x_n - x_{n-1})
+ 3 d_{n-1} ( x_n - x_{n-1})^2 \\
& = b_{n-1} +2
\left(\frac{-2b_{n-1} - b_n + 3f[x_{n-1},x_n]}{\dtx{x_{n-1}}}\right) 
\dtx{x_{n-1}} \\
&\qquad
+ 3 \left(\frac{b_n - 2 f[x_{n-1},x_n] + b_{n-1}}{(\dtx{x_{n-1}})^2}\right)
(\dtx{x_{n-1}})^2  = b_n \ ,
\end{align*}
we have $b_0 = f'(x_0)$ and $b_n = f'(x_n)$.  The
other $b_i$'s are given by the solution of $n-1$ dimensional linear system
$A \VEC{b} = \VEC{q}$ where
\[
A = \left( \begin{array}{ccccccccc}
d_1  & u_0 & 0 & \ldots & \ldots & \ldots & \ldots & \ldots & \ldots \\ 
l_2 & d_2  & u_1 & 0 & \ldots & \ldots & \ldots & \ldots & \ldots \\ 
0 & l_3 & d_3 & u_2 & \ldots & \ldots & \ldots & \ldots & \ldots \\ 
\vdots & 0 & l_4 & d_4 & \ldots & \ldots & \ldots & \ldots & \ldots \\ 
\vdots & \vdots & 0 & l_5 & \ldots & \ldots & \ldots & \ldots & \ldots \\ 
\vdots & \vdots & \vdots & \vdots & \ddots & \ldots & \ldots &
\ldots & \ldots \\ 
\vdots & \vdots & \vdots & \vdots & \vdots & u_{n-6} & 0 & \ldots & \ldots \\ 
\vdots & \vdots & \vdots & \vdots & \vdots & d_{n-4} &  u_{n-5} & 0 & \ldots \\ 
\vdots & \vdots & \vdots & \vdots & \vdots & l_{n-3} &  d_{n-3} & u_{d-4} & 0 \\ 
\vdots & \vdots & \vdots & \vdots & \vdots & 0 & l_{n-2} & d_{n-2} & u_{n-3} \\
\vdots & \vdots & \vdots & \vdots & \vdots & \vdots & 0 & l_{n-1} & d_{n-1} 
\end{array} \right)
\]
and
\[
\VEC{q} = \begin{pmatrix}
- b_0 \dtx{x_1} + 3 \left( f[x_0,x_1]\dtx{x_1} + f[x_1,x_2]
\dtx{x_0} \right) \\[0.4em]
3 \left( f[x_1,x_2]\dtx{x_2} + f[x_2,x_3] \dtx{x_1} \right) \\
\vdots \\
3 \left( f[x_{n-3},x_{n-2}]\dtx{x_{n-2}}
+ f[x_{n-2},x_{n-1}] \dtx{x_{n-3}} \right) \\[0.4em]
- b_n \dtx{x_{n-2}} + 3 \left( f[x_{n-2},x_{n-1}]\dtx{x_{n-1}}
+ f[x_{n-1},x_n] \dtx{x_{n-2}} \right)
\end{pmatrix}
\]
with $d_i = 2(\dtx{x_{i-1}} + \dtx{x_i})$, $u_i = \dtx{x_i}$ and
$l_i = \dtx{x_{i-1}}$.

This gives us another formulation for the clamped cubic spline.

\begin{rmk}
To find the piecewise cubic Hermite interpolant $p$ for a function $f$
on the nodes $x_0$, $x_1$,\ldots, $x_n$, one uses the formulas above
with $b_i = f[x_i,x_i] = f'(x_i)$ for $i=0$, $1$, \ldots, $n$.  The
piecewise cubic Hermite interpolant gives a good approximation of $f$
but requires almost twice as much information about $f$ than the
clamped or free spline interpolant. We need to know $f'(x_i)$ for
$i=0$, $1$, \ldots, $n$.
\end{rmk}

\begin{egg}
Using the information in the table below and the approach developed
in this subsection, construct the clamped spline interpolant for $f$ on
the nodes $0$, $0.3$ and $1$.
\[
\begin{array}{c|ccc}
x & 0 & 0.3 & 1 \\
\hline
f(x) & 1 & 0.548811636094027 & 0.135335283236613 \\
f'(x) & -2 & & -0.270670566473225
\end{array}
\]
All the numerical results displayed will be rounded to $10$
digits.  The computations are done with more precision.

We have
\[
p_i(x) = a_i + b_i\,(x-x_i) + c_i\,(x-x_i)^2 + d_i\,(x-x_i)^3
\]
on $[x_i,x_{i+1}]$ for $i=0$ and $1$, where $x_0 = 0$, $x_1 = 0.3$ and
$x_2=1$.

The coefficients of $p_i$ are given by
\begin{align*}
a_i &= p(x_i)  \ , \quad
b_i = p'(x_i) \ , \quad
d_i = \frac{b_{i+1} - 2 f[x_i,x_{i+1}] + b_i}{(\dtx{x_i})^2}
\intertext{and}
c_i &= \frac{-2b_i - b_{i+1} + 3f[x_i,x_{i+1}]}{\dtx{x_i}}
\end{align*}
for $i=0$ and $1$.  We have $b_0 = f'(0)$, $b_2 = f'(1)$ and
$b_1$ is the solution of
\[
(\dtx{x_1})b_0 + 2(\dtx{x_0} + \dtx{x_1})b_1 + (\dtx{x_0})b_2 =
3\left( f[x_0,x_1]\dtx{x_1} + f[x_1,x_2]\dtx{x_0} \right) \ ;
\]
namely,
\[
0.7f'(0) + 2 b_1 + 0.3 f'(1)
= 3\left( \frac{0.7}{0.3} \left(f(0.3) - f(0)\right)
 + \frac{0.3}{0.7} \left(f(1) - f(0.3)\right)\right) \ .
\]
Thus $b_1 = -1.104364916$\ .

The following table gives the values of the coefficients of
$p_i$.
\[
\begin{array}{c|cccc}
i & d_i & c_i & b_i & a_i \\
\hline
0 & -1.071583217 & 1.974937588 & -2 & 1 \\
1 & -0.3952540283 & 1.010512693 & -1.104364916 & 0.5488116361
\end{array}
\]
As expected, we find the same clamped cubic spline as in
Example~\ref{egg_clamped}.
\end{egg}

\section{Parametric Curves: Bézier Curves} \label{BEZIER}

A general curves $C$ in the plane is the image of a vector valued
function $\phi:[a,b] \rightarrow \RR^2$.  The function $\phi$ is
called a
{\bfseries parametric representation}\index{Bézier Curves!Parametric
Representation} of the curve $C$.  The parametric representation of a
curve is not unique.

It is not always possible to describe a curve $C$ by the graph of a
function $y=f(x)$ for $a\leq x \leq b$.  When it is possible,
$\phi:[a,b]\to \RR^2$ defined by $\phi(x) = (x,f(x))$ is a parametric
representation of the curve $C$.

\begin{egg}
The circle $C$ of radius $1$ centred at the origin has the following
well known parametric representation.
\[
\phi(\theta) = (\cos(\theta),\sin(\theta))
\]
for $0 \leq \theta \leq 2 \pi$.  It is impossible to describe the full
circle as the graph of a function $y=f(x)$.
\end{egg}

Given $n+1$ points $\VEC{p}_0 = (x_0,y_0)$, $\VEC{p}_1 = (x_1,y_1)$,
\ldots, $\VEC{p}_n = (x_n,y_n)$, the goal is to find polynomial
maps of degree three $\phi_i:[0,1] \rightarrow \RR^2$ such
that $\phi_i(0) = \VEC{p}_i$ and $\phi_i(1) = \VEC{p}_{i+1}$ for $0\leq i <n$.
By pasting all the mappings $\phi_i$ together, we hope to get a
nice parametric representation of a curve.

The curves that we are going to describe are called
{\bfseries cubic Bézier curves}\index{Bézier Curves!Cubic Bézier Curves}.
The mapping $\phi_i:[0,1]\to \RR^2$ between
the points $\VEC{p}_i = (x_i,y_i)$ and $\VEC{p}_{i+1} = (x_{i+1},y_{i+1})$,
is defined by
\begin{enumerate}
\item $\phi_i(0) = \VEC{p}_i$,
\item $\phi_i(1) = \VEC{p}_{i+1}$,
\item $\phi_i'(0) = 3(\alpha_i,\beta_i)$ and
\item $\phi_i'(1) = 3(\alpha_{i+1}, \beta_{i+1})$,
\end{enumerate}
where the $\alpha_i$'s and $\beta_i$'s are parameters to be described
later.

Let $\VEC{\check{q}}_i=(x_i+\alpha_i,y_i+\beta_i)$ and
$\VEC{\hat{q}}_{i+1}=(x_{i+1}-\alpha_{i+1},y_{i+1}-\beta_{i+1})$.
It is easy to see that
\begin{equation}\label{Bez_form1}
\VEC{\phi}_i(t) = (1-t)^3 \VEC{p}_i + 3t(1-t)^2 \VEC{\check{q}}_i
+ 3t^2(1-t) \VEC{\hat{q}}_{i+1} + t^3 \VEC{p}_{i+1}
\end{equation}
satisfies the four conditions above.

The points $\VEC{\check{q}}_i$ and $\VEC{\hat{q}}_{i+1}$ are called
the {\bfseries control points}\index{Bézier Curves!Control Points}
of the Bézier curve with endpoints $\VEC{p}_i$ and $\VEC{p}_{i+1}$.

For the parametric representation between $\VEC{p}_i$ and $\VEC{p}_{i+1}$,
\[
\pdydx{y}{x}\bigg|_{x=\VEC{p}_i} = \frac{\beta_i}{\alpha_i}\ .
\]
As long as the ratio $\beta_i/\alpha_i$ is
constant, the parametric representation has the same slope at
$\VEC{p}_i$.  By taking $\alpha_i$ and $\beta_i$ very large, we
flatten the image of the representation near $\VEC{p}_i$.

We illustrate graphically the meaning
of the parameters $\alpha_i$, $\beta_i$. $\alpha_{i+1}$ and
$\beta_{i+1}$ in Figure~\ref{Bez_fig}.

\pdfF{interpolation_B/bezier}{Piece of a Bézier curve}{Piece of a
Bézier curve}{Bez_fig}

(\ref{Bez_form1}) can be rewritten as
\begin{align}
\VEC{\phi}_i &= (-t^3+3t^2-3t+1) \VEC{p}_i + (3t^3-6t^2+3t) \VEC{\check{q}}_i +
(-3t^3+3t^2) \VEC{\hat{q}}_{i+1} + t^3 \VEC{p}_{i+1} \nonumber \\
&= \left(-\VEC{p}_i +3\VEC{\check{q}}_i - 3 \VEC{\hat{q}}_{i+1} +
\VEC{p}_{i+1}\right) t^3 + \left(3 \VEC{p}_i -6 \VEC{\check{q}}_i
 + 3 \VEC{\hat{q}}_{i+1}\right) t^2 \nonumber \\
&\qquad + \left(-3 \VEC{p}_i + 3  \VEC{\check{q}}_i\right) t + \VEC{p}_i \ .
\label{Bez_form2}
\end{align}
It is (\ref{Bez_form2}) instead of (\ref{Bez_form1}) that is used in
computer codes to draw Bézier curves.  The coefficients are computed
once and the nested form of (\ref{Bez_form2}) is used.  The number of
arithmetic operations is minimal.

\pdfF{interpolation_B/bezier_2}{Construction of a Bézier curve}
{Construction of a Bézier curve}{Bez_2_fig}

\begin{rmk}
There is a nice geometric interpretation of the Bézier curve.

Let $\VEC{a}$ be the middle point of the line from $\VEC{p}_i$ to
$\VEC{\check{q}}_i$, $\VEC{b}$ be the middle point of the line from
$\VEC{p}_{i+1}$ to $\VEC{\hat{q}}_{i+1}$, $\VEC{c}$ be the middle point of
the line from $\VEC{\check{q}}_i$ to $\VEC{\hat{q}}_{i+1}$, $\VEC{d}$
be the middle point of the line from $\VEC{a}$ to $\VEC{c}$, $\VEC{e}$
be the middle point of the line from $\VEC{b}$ to $\VEC{c}$, and
$\VEC{f}$ be the middle point of the line from $\VEC{d}$ to $\VEC{e}$.

The point $\VEC{f}$ is on the Bézier curve $\Gamma$ with endpoints
$\VEC{p}_i$, $\VEC{p}_{i+1}$ and control points $\VEC{\check{q}}_i$,
$\VEC{\hat{q}}_{i+1}$.  Moreover, the Bézier curve $\Gamma$ is the
pasting of the Bézier curve with endpoints $\VEC{p}_i$, $\VEC{f}$ and
control points $\VEC{a}$, $\VEC{d}$, and the Bézier curve with
endpoints $\VEC{f}$, $\VEC{p}_{i+1}$ and control points $\VEC{e}$,
$\VEC{b}$.

We can apply the previous construction to the Bézier curve with
endpoints $\VEC{p}_i$, $\VEC{f}$ and control points $\VEC{a}$,
$\VEC{d}$ to find another point $\VEC{f}\,'$ on the the Bézier curve
$\Gamma$.  Similarly, the Bézier curve with endpoints $\VEC{f}$,
$\VEC{p}_{i+1}$ and control points $\VEC{e}$, $\VEC{b}$ gives another
point $\VEC{f}\,''$ on the Bézier curve $\Gamma$.  Repeating this
construction on smaller and smaller portion of the Bézier curve
$\Gamma$ gives a sequence of points on the Bézier curve
$\Gamma$.  To draw the Bézier curve $\Gamma$, one may draw straight
lines between the points of $\Gamma$ that have been found when the
distance between them is smaller than a given small value.  It is not
suggested however to use this method to draw Bézier curves because
of the large number of operations needed to draw the curve.

We first show that $\VEC{f}$ is on the Bézier curve with endpoints
$\VEC{p}_i$, $\VEC{p}_{i+1}$ and control points $\VEC{\check{q}}_i$,
$\VEC{\hat{q}}_{i+1}$.  We have that
\begin{align*}
\VEC{f} & = \frac{1}{2} \left(\VEC{d}+\VEC{e}\right)
= \frac{1}{2} \left( \frac{1}{2}\left(\VEC{a}+\VEC{c}\right) +
\frac{1}{2}\left(\VEC{c}+\VEC{b}\right) \right)
= \frac{1}{4} \VEC{a} + \frac{1}{2} \VEC{c} + \frac{1}{4} \VEC{b} \\
 & = \frac{1}{4} \left(\frac{1}{2}\left(\VEC{p}_i
+\VEC{\check{q}}_i\right)\right)
+ \frac{1}{2} \left(\frac{1}{2}\left(\VEC{\check{q}}_i
+\VEC{\hat{q}}_{i+1}\right)\right)
+ \frac{1}{4} \left(\frac{1}{2}\left(\VEC{\hat{q}}_{i+1}
+\VEC{p}_{i+1}\right)\right) \\
& = \frac{1}{8} \left( \VEC{p}_i + 3 \VEC{\check{q}}_i
+ 3 \VEC{\hat{q}}_{i+1} + \VEC{p}_{i+1} \right)
= \VEC{\phi}_i\left(\frac{1}{2}\right)\ .
\end{align*}

We now show that the first half of the Bézier curve with endpoints
$\VEC{p}_i$, $\VEC{p}_{i+1}$ and control points $\VEC{\check{q}}_i$,
$\VEC{\hat{q}}_{i+1}$, namely $\VEC{\phi}_i(t)$ for
$0\leq t \leq 1/2$, is the Bézier curve with endpoints
$\VEC{p}_i$, $\VEC{f}$ and control points $\VEC{a}$, $\VEC{d}$.
We leave to the reader the proof that the second half of the Bézier
curve with endpoints $\VEC{p}_i$, $\VEC{p}_{i+1}$ and control points
$\VEC{\check{q}}_i$, $\VEC{\hat{q}}_{i+1}$, namely $\VEC{\phi}_i(t)$
for $1/2 \leq t \leq 1$, is the Bézier curve with endpoints
$\VEC{f}$, $\VEC{p}_{i+1}$ and control points $\VEC{e}$, $\VEC{b}$.

The parametric representation of the Bézier curve with endpoints
$\VEC{p}_i$, $\VEC{f}$ and control points $\VEC{a}$, $\VEC{d}$ is
given by
\[
\VEC{\psi}(s) = (1-s)^3 \VEC{p}_i +3s(1-s)^2 \VEC{a} +
3s^2(1-s)\VEC{d} +s^3 \VEC{f}
\]
for $0 \leq s \leq 1$.  Hence,
\begin{align*}
&\VEC{\psi}(s) =
(1-s)^3 \VEC{p}_i +3s(1-s)^2 \VEC{a} + 3s^2(1-s)\VEC{d} +
s^3 \frac{1}{2}\left(\VEC{d}+\VEC{e}\right) \\
&\quad = (1-s)^3 \VEC{p}_i +3s(1-s)^2 \VEC{a} +
\left( 3s^2(1-s) + \frac{1}{2} s^3\right) \VEC{d}
+ \frac{1}{2} s^3 \VEC{e} \\
&\quad = (1-s)^3 \VEC{p}_i +3s(1-s)^2 \VEC{a} +
\left( 3s^2(1-s) + \frac{1}{2} s^3\right) \left(
\frac{1}{2}\left(\VEC{a}+\VEC{c}\right)\right)
+ \frac{1}{2} s^3 \left( \frac{1}{2}\left(\VEC{c}+\VEC{b}\right)\right) \\
&\quad = (1-s)^3 \VEC{p}_i +
\left(3s(1-s)^2 + \frac{3}{2}s^2(1-s) + \frac{1}{4} s^3\right) \VEC{a}
+ \left( \frac{3}{2}s^2(1-s) + \frac{1}{2} s^3 \right) \VEC{c}
+ \frac{1}{4} s^3 \VEC{b} \\
&\quad = (1-s)^3 \VEC{p}_i +
\left(3s(1-s)^2 + \frac{3}{2} s^2(1-s) + \frac{1}{4} s^3\right)
\left( \frac{1}{2}\left(\VEC{p}_i+\VEC{\check{q}}_i\right) \right) \\
&\qquad + \left( \frac{3}{2} s^2(1-s) + \frac{1}{2} s^3 \right)
\left( \frac{1}{2}\left(\VEC{\check{q}}_i+\VEC{\hat{q}}_{i+1}\right)\right)
+ \frac{1}{4} s^3 \left(
\frac{1}{2}\left(\VEC{p}_{i+1}+\VEC{\hat{q}}_{i+1}\right) \right) \\
&\quad = \left( (1-s)^3 + \frac{3}{2}s(1-s)^2 + \frac{3}{4} s^2(1-s)
+ \frac{1}{8} s^3 \right) \VEC{p}_i \\
&\qquad + \left(\frac{3}{2} s(1-s)^2 +
\frac{3}{2} s^2(1-s) + \frac{3}{8} s^3\right) \VEC{\check{q}}_i
+ \left( \frac{3}{4} s^2(1-s) + \frac{3}{8} s^3 \right) \VEC{\hat{q}}_{i+1}
+ \frac{1}{8} s^3 \VEC{p}_{i+1} \\
&\quad = \left( \frac{s}{2}+(1-s) \right)^3 \VEC{p}_i +
\frac{3s}{2}\left(\frac{s}{2} + (1-s) \right)^2 \VEC{\check{q}}_i +
3 \left( \frac{s}{2} \right)^2\left(\frac{s}{2}+(1-s) \right)
\VEC{\hat{q}}_{i+1} + \left( \frac{s}{2} \right)^3 \VEC{p}_{i+1} \\
&\quad = \left( 1 -\frac{s}{2}\right)^3 \VEC{p}_i +
\frac{3s}{2}\left( 1- \frac{s}{2}\right)^2 \VEC{\check{q}}_i +
3 \left( \frac{s}{2} \right)^2\left(1 - \frac{s}{2}\right) \VEC{\hat{q}}_{i+1}
+ \left( \frac{s}{2} \right)^3 \VEC{p}_{i+1}
= \VEC{\phi}_i\left(\frac{s}{2}\right)
\end{align*}
for $0 \leq s \leq 1$.
\end{rmk}

\begin{egg}
We want to construct a piecewise cubic Bézier curve that satisfy the
following conditions.
\[
\begin{array}{c|cccc}
i & \VEC{p}_i & \VEC{p}_{i+1} & \VEC{\check{q}}_i & \VEC{\hat{q}}_{i+1} \\
\hline
0 & (0,2) & (1,3) & (1,2) & (0.5,2.5) \\
1 & (1,3) & (3,3) & (1.5,3.5) & (2.5,3.5) \\
2 & (3,3) & (4,2) & (3.5,2.5) & (3.5,2.5) \\  
3 & (4,2) & (5,2) & (4.5,1.5) & (4.5,2.5) \\
4 & (5,2) & (5.5,1) & (5.5,1.5) & (5,1)
\end{array}
\]
The pieces of the curve are given by
\begin{align*}
\VEC{\phi}_i(t) &=
  \left(-\VEC{p}_i +3\VEC{\check{q}}_i - 3 \VEC{\hat{q}}_{i+1} +
\VEC{p}_{i+1}\right) t^3 + \left(3 \VEC{p}_i -6 \VEC{\check{q}}_i
+ 3 \VEC{\hat{q}}_{i+1}\right) t^2 \\
&\qquad + \left(-3 \VEC{p}_i + 3  \VEC{\check{q}}_i\right) t + \VEC{p}_i \\
&= \begin{cases}
\begin{pmatrix} 2.5 \\ -0.5 \end{pmatrix} t^3
+ \begin{pmatrix} -4.5 \\ 1.5 \end{pmatrix} t^2
+ \begin{pmatrix} 3.0 \\ 0.0 \end{pmatrix} t
+ \begin{pmatrix} 0.0 \\ 2.0 \end{pmatrix} & \quad \text{if}
\quad i = 0 \\[1em]
\begin{pmatrix} -1.0 \\ 0.0 \end{pmatrix} t^3
+ \begin{pmatrix} 1.5 \\ -1.5 \end{pmatrix} t^2
+ \begin{pmatrix} 1.5 \\ 1.5 \end{pmatrix} t
+ \begin{pmatrix} 1.0 \\ 3.0 \end{pmatrix} & \quad \text{if}
\quad i = 1 \\[1em]
\begin{pmatrix} 1.0 \\ -1.0 \end{pmatrix} t^3
+ \begin{pmatrix} -1.5 \\ 1.5 \end{pmatrix} t^2
+ \begin{pmatrix} 1.5 \\ -1.5 \end{pmatrix} t
+ \begin{pmatrix} 3.0 \\ 3.0 \end{pmatrix} & \quad \text{if}
\quad i = 2 \\[1em]
\begin{pmatrix} 1.0 \\ -3.0 \end{pmatrix} t^3
+ \begin{pmatrix} -1.5 \\ 4.5 \end{pmatrix} t^2
+ \begin{pmatrix} 1.5 \\ -1.5 \end{pmatrix} t
+ \begin{pmatrix} 4.0 \\ 2.0 \end{pmatrix} & \quad \text{if}
\quad i = 3 \\[1em]
\begin{pmatrix} 2.0 \\ 0.5 \end{pmatrix} t^3
+ \begin{pmatrix} -3.0 \\ 0.0 \end{pmatrix} t^2
+ \begin{pmatrix} 1.5 \\ -1.5 \end{pmatrix} t
+ \begin{pmatrix} 5.0 \\ 2.0 \end{pmatrix} & \quad \text{if}
\quad i = 4
\end{cases}
\end{align*}
for $0 \leq t \leq 1$.  The graph of the piecewise cubic Bézier
curve is given below.
\figbox{interpolation_B/bezier_ex1}{8cm}
\end{egg}

\begin{rmk}
The {\bfseries Bernstein polynomial}\index{Bernstein Polynomial} of
degree $m \in \NN^+$ for a function $f:[0,1]\rightarrow \RR$ is the
polynomial
\begin{equation} \label{bernstein1}
B_m(t; f) = \sum_{k=0}^m f\left(\frac{k}{m}\right) \binom{m}{k}
t^k(1-t)^{m-k} \ .
\end{equation}
One can prove that $B_m(\cdot ; f) \rightarrow f$ uniformely on $[0,1]$ as
$m\rightarrow \infty$ if $f$ is continuous on $[0,1]$.  One of the
proofs of the Stone-Weierstrass Theorem, Theorem~\ref{SWtheorem}, is
effectively based on the Bernstein polynomials.

The Bézier curve (\ref{Bez_form1}) can be written as the Bernstein
polynomial
\[
\phi_i(t) = \sum_{k=0}^3 \VEC{c}_{i,k} \binom{3}{k} t^k(1-t)^{3-k}
\]
for $0 \leq i < n$, where $\VEC{c}_{i,k} \in \RR^2$ satisfies
\[
\binom{3}{k} \VEC{c}_{i,k} =
\begin{cases}
\VEC{p}_i & \quad \text{if} \quad k = 0 \\
\VEC{\check{q}}_i & \quad \text{if} \quad k = 1 \\
\VEC{\hat{q}}_{i+1} & \quad \text{if} \quad k = 2 \\
\VEC{p}_{i+1} & \quad \text{if} \quad k = 3
\end{cases}
\]
for $0 \leq i < n$.

Similarly, we may generalize Bézier curves to more
than two control points.  For each $m \geq 3$, we define the Bézier
curve with $m-1$ control points as the curve defined by
\begin{equation} \label{bernstein2}
\phi_i(t) = \sum_{k=0}^m \VEC{c}_{i,k} \binom{m}{k} t^k(1-t)^{m-k} \ .
\end{equation}
The control points are
$\displaystyle \binom{m}{k} \VEC{c}_{i,k}$ for $k=1$, $2$,
\ldots, $m-1$.  In particular, if we assume that
\[
\binom{m}{k} \VEC{c}_{i,k} =
\begin{cases}
\VEC{p}_i & \quad \text{if} \quad k = 0 \\
\VEC{\check{q}}_i & \quad \text{if} \quad k = 1 \\
\VEC{\hat{q}}_{i+1} & \quad \text{if} \quad k = m-1 \\
\VEC{p}_{i+1} & \quad \text{if} \quad k = m
\end{cases}
\]
then $\phi_i(0) = \VEC{p}_i$ and
$\phi_i(1) = \VEC{p}_{i+1}$ for $0\leq i < n$.  Moreover, since
\[
\phi_i'(t) = m
\sum_{k=0}^{m-1} \left( \VEC{c}_{i,k+1} - \VEC{c}_{i,k} \right)
\binom{m-1}{k} t^k(1-t)^{m-1-k} \ ,
\]
we get
\[
  \phi_i'(0) = m\left( \VEC{c}_{i,1} - \VEC{c}_{i,0} \right)
  = m\left(\VEC{\check{q}}_i - \VEC{p}_i\right) = m (\alpha_i , \beta_i )
\]
and
\[
\phi_i'(1) = m\left( \VEC{c}_{i,m} - \VEC{c}_{i,m-1} \right)
= m\left(\VEC{p}_{i+1} - \VEC{\hat{q}}_{i+1}\right)
= m (\alpha_{i+1} , \beta_{i+1} ) \ .
\]
We still get a curve which is tangent to $(\alpha_i,\beta_i)$ at
$\VEC{p}_i$ and tangent to $(\alpha_{i+1},\beta_{i+1})$ at
$\VEC{p}_{i+1}$.
\end{rmk}

\section{B-Spline Interpolation} \label{BSI}

In this section, we consider an infinite sequence of knots
\[
\ldots < t_{-2} < t_{-1} < t_0 < t_1 < t_2 < \ldots
\]
such that $\displaystyle \lim_{i\rightarrow -\infty} t_i = -\infty$ and
$\displaystyle \lim_{i\rightarrow \infty} t_i = +\infty$.

\begin{defn}
The {\bfseries B-splines of degree $0$}\index{B-Splines of Degree $0$}
are defined by
\[
B_i^0(t) = \begin{cases}
1 & \qquad \text{if} \quad t_i \leq t < t_{i+1} \\
0 & \qquad \text{otherwise}
\end{cases}
\]
for $i\in \ZZ$.\\
The {\bfseries B-splines of degree $k>0$}\index{B-Splines of Degree $k>0$}
are defined by the recurrence relation
\begin{equation}\label{Bsplinerecurence}
B_i^k(t) = v_i^k(t) B_i^{k-1}(t)
+ \left( 1-v_{i+1}^k(t) \right) B_{i+1}^{k-1}(t)
\end{equation}
for $i\in \ZZ$, where
$\displaystyle v_i^k(t) = \frac{t-t_i}{t_{i+k}-t_i}$.
\end{defn}

We sketch in Figure~\ref{Bsplines} a B-spline of degree $0$ and a
B-spline of degree $1$.

\pdfF{interpolation_B/b-splines}{The B-spline of degree $0$, $1$ and $2$.}
{Clockwise from the top left corner: $B_i^0$, $B_i^1$ and $B_i^2$.}{Bsplines}

The next propositions state some of the properties of the B-splines.
We give some of the proofs and refer the reader to \cite{KC} for the
missing proofs and more information.

\begin{prop}
The B-splines of degree $0$ are piecewise constant functions which are
continuous from the right.  For $k>0$, the B-splines of degree $k$ are
piecewise polynomials of degree $k$ and class $C^{k-1}$.
\label{BSplineCkm1}
\end{prop}

That the B-splines of degree $k>0$ are piecewise polynomials of degree
$k$ is proved by induction using (\ref{Bsplinerecurence}).   To proof
that they are of class $C^{k-1}$ requires induction and tedious
computations to show that
\begin{equation}\label{derBspline}
  \dfdx{B_i^k(t)}{t} =
  \left( \frac{k}{t_{i+k}-t_i} \right) B_i^{k-1}(t)
- \left( \frac{k}{t_{i+k+1}-t_{i+1}} \right) B_{i+1}^{k-1}(t)
\end{equation}
for $k>1$.  This formula is also true for $k=1$ as long as $t\neq t_j$
for $j=i$, $i+1$ and $i+2$.

A simple proof by induction based on (\ref{Bsplinerecurence}) gives
the following result.

\begin{prop}
$B_i^0(t) > 0$ for $t_i \leq t < t_{i+1}$ and $B_i^0(t) = 0$ otherwise.
For $k>0$, $B_i^k(t) > 0$ for $t_i < t < t_{i+k+1}$ and $B_i^k(t) = 0$
otherwise.
\label{Bsplienonneg}
\end{prop}

The next result is quite useful to evaluate B-splines.

\begin{prop}
Suppose that
\[
  p(t) = \sum_{i=-\infty}^\infty C_i^k(t) B_i^k(t) \quad .
\]
Given $t \in [t_j ,t_{j+1}[$, if we use the relation
\begin{equation}\label{table_spline}
\begin{split}
  C_i^{r-1}(t) &= C_i^r(t) v_i^r(t) + C_{i-1}^r \left( 1-   v_i^r(t)\right) \\
&= \frac{1}{t_{r+i}-t_i} \left( (t-t_i) C_i^r(t) +
  (t_{r+i}-t) C_{i-1}^r(t) \right)
\end{split}
\end{equation}
for $r=k$, $k-1$, \ldots, $0$ to generate the table
\[
\begin{array}{*{4}{c@{\hspace{1em}}}c}
C_j^k(t) & C_j^{k-1}(t) & \ldots & C_j^1(t) & C_j^0(t) \\[0.5em]
C_{j-1}^k(t) & C_{j-1}^{k-1}(t) & \ldots & C_{j-1}^1(t) & \\ 
\vdots & \vdots & \udots &  & \\
C_{j-k+1}^k(t) & C_{j-k+1}^{k-1}(t) & & & \\[0.5em]
C_{j-k}^k(t) & & & & 
\end{array}
\]
then $p(t) = C_j^0(t)$.
\end{prop}

\begin{proof}
The proof of the previous proposition is based on the relation
\begin{align*}
\sum_{i=-\infty}^\infty C_i^r(t) B_i^r(t)
&= \sum_{i=-\infty}^\infty C_i^r(t) \left( v_i^r(t) B_i^{r-1}(t)
+ \left( 1-v_{i+1}^r(t) \right) B_{i+1}^{r-1}\right) \\
&= \sum_{i=-\infty}^\infty C_i^r(t) v_i^r(t) B_i^{r-1}(t)
+ \sum_{i=-\infty}^\infty C_i^r(t) \left( 1-v_{i+1}^r(t) \right) B_{i+1}^{r-1} \\
&= \sum_{i=-\infty}^\infty C_i^r(t) v_i^r(t) B_i^{r-1}(t)
+ \sum_{i=-\infty}^\infty C_{i-1}^r(t) \left( 1-v_i^r(t) \right) B_i^{r-1} \\
&= \sum_{i=-\infty}^\infty \big( C_i^r(t) v_i^r(t) 
+  C_{i-1}^r(t) \left( 1-v_i^r(t) \right) \big) B_i^{r-1} \\
&= \sum_{i=-\infty}^\infty C_i^{r-1}(t) B_i^{r-1}(t)
\end{align*}
and a simple proof by induction to get
\[
p(t) = \sum_{i=-\infty}^\infty C_i^k(t) B_i^k(t)
= \sum_{i=-\infty}^\infty C_i^0(t) B_i^0(t) \ .
\]
Don't forget that, for $t$ given, all sums are finite.
\end{proof}

\begin{rmk}
The spline interpolant that we will present later will be
of the form
\[
  p(t) = \sum_{i=-\infty}^\infty  c_i^k B_i^k(t)
\]
for some constants $c_i^k$.  If we set $C_i^r(t) = c_i^r$, we can use
the method presented in the proposition above to compute $p(t)$.
\end{rmk}

\begin{prop}
$\displaystyle \sum_{i=-\infty}^{+\infty} B_i^k(t) = 1$ for all $t \in \RR$
and $k\geq 0$.
\label{Bsplinesum1}
\end{prop}

\begin{proof}
We use the previous proposition with $C_i^k(t) = 1$ for all $i$.  We
have
\[
C_i^{k-1}(t) = C_i^k(t) v_i^k(t) + C_{i-1}^k \left( 1- v_i^k(t)\right)
=  v_i^k(t) + \left( 1-  v_i^k(t)\right) = 1
\]
for all $i$.   A simple proof by induction shows that
$C_i^r(t) = 1$ for all $i$ and all $r$ with $0 \leq r \leq k$.  Thus,
\[
  \sum_{i=-\infty}^\infty B_i^k(t) = \sum_{i=-\infty}^\infty B_i^0(t) \ .
\]
For $t\in [t_j,t_{j+1}[$, we get
\[
  \sum_{i=-\infty}^\infty B_i^k(t) = \sum_{i=-\infty}^\infty B_i^0(t)
  = B_j^0(t) = 1 \ .  \qedhere
\]
\end{proof}

\begin{prop}
The set
$\displaystyle \left\{ B_j^k, B_{j+1}^k, \ldots , B_{j+k}^k \right\}$
is linearly independent on $]t_{j+k},t_{j+k+1}[$.
\end{prop}

We note that the only B-splines $B_i^k$ that are not trivially null on
$]t_{j+k},t_{j+k+1}[$ are those for $j \leq i \leq j+k$.  The proof of
this proposition is by induction and requires the formula for the
derivative of the B-splines $B_i^k$ that was required for the proof
of Proposition~\ref{BSplineCkm1}.

\begin{prop}
The set of B-splines
$\displaystyle \left\{ B_{-k}^k, B_{-k+1}^k, \ldots , B_{n-1}^k \right\}$
is a basis for the space $S_n^k$ of functions $p$ of class $C^{k-1}$ on
$[t_0,t_n]$ such that
$\displaystyle p\big|_{[t_i,t_{i+1}]}$ is a polynomials of degree at most
$k$ for $0\leq i < n$.
\label{BSplineLinInd}
\end{prop}

\begin{proof}
We note that the only B-splines $B_i^k$ that are not trivially null on
$]t_0,t_n[$ are those for $-k \leq i \leq n-1$.

\subI{Linear Independence}
Suppose that $\displaystyle \sum_{i=-k}^{n-1} c_i B_i^k = 0$ on $[t_0,t_n]$.
We therefore also have that
\[
  \sum_{i=-k}^0 c_i B_i^k = \sum_{i=-k}^{n-1} c_i B_i^k = 0
\]
on $]t_0,t_1[$.  It follows from the previous proposition that
$c_i = 0$ for $-k \leq i \leq 0$.

Suppose that $j<n$ is the smallest index such that $c_j\neq 0$.  From
the previous discussion, we have $j>0$.  Hence, for
$t \in [t_j,t_j+1[$, we have
\[
 0 = \sum_{i=-k}^{n-1} c_i B_i^k(t) = \sum_{i=j}^{n-1} c_i B_i^k(t) =
  c_j B_j^k(t) \ .
\]
Since $B_j^k(t) > 0$, we get $c_j = 0$.  This is a contradiction that
$c_j \neq 0$.  So, there is no such $j$ between $0$ and $n$ such
$c_j \neq 0$.

In particular, this proves that $S_n^k$ is at least of dimension
$n+k$. 

\subI{Generating set}
We prove that all functions $p \in S_n^k$ can be expressed as a linear
combination over $\RR$ of the following $n+k$ functions in $S_n^k$:
$t^j$ for $0\leq j \leq k$ and $H(t-t_j)(t-t_j)^k$ for
$1\leq j \leq n-1$, where $H$ is the Heavyside function defined by
\[
  H(x) = \begin{cases} 1 & \quad \text{if} \quad x \geq 0 \\
    0 & \quad \text{if} \quad x < 0
  \end{cases}
\]
This will prove that $S_n^k$ is at most of dimension $n+k$.  Combined
with what we proved in the first part of the proof, this shows that
$S_n^k$ is of dimension $n+k$ and that\\
$\displaystyle \left\{ B_{-k}^k, B_{-k+1}^k, \ldots , B_{n-1}^k \right\}$
is a basis of $S_n^k$.

Given $p \in S_n^k$, we have that $p_0 = p\big|_{[t_0,t_1]}$ is a polynomial
of degree at most $k$.  So
$\displaystyle p_0(t) = \sum_{i=0}^k a_i t^i$ for some $a_i \in \RR$.

We prove by induction that there exist constants $a_{k+i}$ such that
\begin{equation}\label{Snkbasis}
  p(t) =  \sum_{i=0}^k a_i t^i + \sum_{i=1}^j a_{k+i}H(t-t_i)(t-t_i)^k
\end{equation}
for $t_0 \leq t \leq t_{j+1}$ with $1 \leq j\ < n$.

We have that $P_1 = p\big|_{[t_1,t_2]}$ is a polynomial of degree at
most $k$.  Since $p$ is of class $C^{k-1}$, we have that
\[
  \dfdxn{(p_1 - p_0)(t_1)}{t}{m} = 0
\]
for $0 \leq m < k$.  Since $p_1-p_0$ is a
polynomial of degree at most $k$, it follows from
Lemma~\ref{FactPolyRoot} that
$(p_1-p_0)(t) = a_{k+1} (t- t_1)^k$ for some $a_{k+1} \in \RR$.
Thus
\[
  p(t) =  \sum_{i=0}^k a_i t^i + a_{k+1}H(t-t_1)(t-t_1)^k
\]
for $t_0 \leq t \leq t_2$.  This proves (\ref{Snkbasis}) for $j=1$.

Suppose that (\ref{Snkbasis}) is true for $j$.
We have that $p_{j+1} = p\big|_{[t_{j+1},t_{j+2}]}$ is a polynomial
of degree at most $k$.  Moreover,
\[
p_j = p\big|_{[t_j,t_{j+1}]} = \left( \sum_{i=0}^k a_i t^i
+ \sum_{i=1}^j a_{k+i}H(t-t_i)(t-t_i)^k\right)\bigg|_{[t_j,t_{j+1}]}
\]
is a polynomial of degree at most $k$.  Since $p$ is of class
$C^{k-1}$, we have that
\[
  \dfdxn{(p_{j+1} - p_j)}{t}{m}(t_{j+1}) = 0
\]
for $0 \leq m < k$.  Since $p_{j+1}-p_j$ is a polynomial of degree at
most $k$, it again follows from
Lemma~\ref{FactPolyRoot} that
$(p_{j+1}-p_j)(t) = a_{k+j+1} (t- t_{j+1})^k$ for some $a_{k+j+1} \in \RR$.
thus
\[
  p(t) =  \sum_{i=0}^k a_i t^i + \sum_{i=1}^{j+1} a_{k+i}H(t-t_i)(t-t_i)^k
\]
for $t_0 \leq t \leq t_{j+2}$.  This proves (\ref{Snkbasis}) for $j$
replaced by $j+1$.

(\ref{Snkbasis}) with $j=n-1$ shows that $f$ is a linear combination
of the $n+k$ functions $t^j$ for $0\leq j \leq k$ and
$H(t-t_j)(t-t_j)^k$ for $1\leq j \leq n-1$ as claimed.
\end{proof}

Our interpolation problem is as follows.   Given points
$(x_1,y_1)$, $(x_2,y_2)$, \ldots, $(x_{n+k},y_{n+k})$
such that $\displaystyle x_j < x_{j+1}$ for $1\leq j < n+k$, and
$ x_j \in[t_0,t_n]$ for $1\leq j \leq n+k$,
can we find constants $c_i$ with $-k \leq i \leq n-1$ such that
\begin{equation}\label{BsplineFormCoeff}
\sum_{i=-k}^{n-1} c_i B_i^k(x_j) = y_j
\end{equation}
for $1\leq j \leq n+ k$?   If such $c_i$ exist, then
\begin{equation}\label{BsplineForm}
p(x) = \sum_{i=-k}^{n-1} c_i B_i^k(x)
\end{equation}
is a
{\bfseries spline interpolant}\index{B-Splines!Spline Interpolant} on
the nodes $x_1$, $x_2$, \ldots, $x_{n+k}$.

The answer to this question is a consequence of the following result.

\begin{theorem}[Schoemberg-Whitney]
Given $q \in \ZZ$ and $x_1 < x_2 < \ldots < x_m$, consider the
\nm{m}{m} matrix $Q$ with the entries $Q_{j,i} = B_{i+q}^k(x_j)$ for
$1 \leq i,j \leq m$.  Then, $Q$ is invertible if and only if
$Q_{j,j} \neq 0$ for $1 \leq j \leq m$.
\end{theorem}

To find the $c_i^k$ required to satisfy (\ref{BsplineFormCoeff}), we
have to solve the linear system $Q \VEC{z} = \VEC{y}$, where
$Q_{j,i} = B_{i-k-1}^k(x_j)$ and $z_i = c_{i-k-1}$ 
for $1\leq i,j \leq n+k$.  We get the following result from
Schoemberg-Whitney Theorem and Proposition~\ref{Bsplienonneg}.

\begin{prop}
The system $Q\VEC{z} = \VEC{y}$ defined above has a solution if and
only if $Q_{j,j} = B_{j-k-1}^k(x_j) \neq 0$ for $1\leq j \leq n+k$;
namely, if $t_{j-k-1} < x_j < t_j$ for $1\leq j \leq n+k$.
\end{prop}

If we consider the set of B-splines
$\displaystyle \left\{ B_{-k}^k, B_{-k+1}^k, \ldots , B_{n-1}^k \right\}$,
then Schoemberg-Whitney Theorem not only gives a condition for the
existence of a solution to $Q\VEC{z} = \VEC{y}$ but it also shows that
this solution is unique.  However, there is no obligation to specify
all the $n+k$ points $(x_1,y_1)$, $(x_2,y_2)$, \ldots,
$(x_{n+k},y_{n+k})$.   Namely, we do not have to use all the equations
(\ref{BsplineFormCoeff}) for $1\leq j \leq n+ k$ to determine the
$c_i$.  We may use other conditions to determine some of the
$c_i$.  We will do just that in an example below.

\begin{rmk}
We will need the following information for the next example.  Let
\[
  p(t) = \sum_{-\infty}^\infty c_i B_i^k(t) \ .
\]
If we derive $f$ using (\ref{derBspline}), we get
\[
p'(t) = k\sum_{-\infty}^\infty \left(\frac{c_i - c_{i-1}}{t_{i+k}-t_i}\right)
B_i^{k-1}(t)
\]
for $k>1$.  This formula is also true for $k=1$ as long as $t\neq t_i$
for all $i$.  Again, if we derive $f'$ using (\ref{derBspline}), we get
\[
p''(t) = k(k-1)
\sum_{-\infty}^\infty\left(\frac{1}{t_{i+k-1}-t_i}\right)
\left(\frac{c_i - c_{i-1}}{t_{i+k}-t_i}
- \frac{c_{i-1} - c_{i-2}}{t_{i+k-1}-t_{i-1}}
\right) B_i^{k-2}(t)
\]
for $k>2$.  This formula is also true for $k=2$ as long as $t\neq t_i$
for all $i$.
\end{rmk}

\begin{egg}
Suppose that $f:\RR \rightarrow \RR$.  We will give a natural cubic
interpolant of $f$ on the nodes $x_1 < x_2 < \ldots < x_n$. 

We have $k=3$ and we select the knots $t_i$ such that $x_j = t_{j-1}$ for
$1\leq j \leq n$.   The spline interpolant $p$ that we are
looking for will be determined by the points
$(x_j,y_j) = (t_{j-1},f(t_{j-1}))$ for $1\leq j \leq n$.
We need $j-4 \leq i < j$ to possibly have that $B_i^3(x_j)$ is non null.

The spline interpolant $p$ is of the form
\[
p(x) = \sum_{i=-3}^{n-2} c_i B_i^3(x) \ ,
\]
where
\begin{equation}\label{BsplieCubiccoeff}
p(x_j) = \sum_{i=-3}^{n-2} c_i B_i^3(x_j)
= \sum_{i=j-3}^{j-2} c_i B_i^3(x_j) = y_j
\end{equation}
for $1\leq j \leq n$.  We have dropped the term for
$i\not\in \{-3,-2,\ldots,n-2\}$ from the
summation because $B_i^3(x_j) = B_i^3(t_{j-1}) = 0$ for these values of
$i$.  There are $n$ equations with $n+2$ variables $c_i$ for
$-3\leq i \leq n-2$.  We use the two extra variables to
satisfy the conditions for a natural cubic spline interpolant; namely,
$p''(x_1) = p''(x_n) = 0$.

From the result stated in the previous remark, we have
\begin{align*}
p''(x_1) &= p''(t_0) = 6 \sum_{-\infty}^\infty\left(\frac{1}{t_{i+2}-t_i}\right)
\left(\frac{c_i - c_{i-1}}{t_{i+3}-t_i}
- \frac{c_{i-1} - c_{i-2}}{t_{i+2}-t_{i-1}} \right) B_i^1(t_0) \\
&= 6 \left(\frac{1}{t_1-t_{-1}}\right) \left(\frac{c_{-1} - c_{-2}}{t_2-t_{-1}}
- \frac{c_{-2} - c_{-3}}{t_1-t_{-2}}\right)  = 0
\end{align*}
and
\[
p''(x_n) = p''(t_{n-1}) = 6
\left(\frac{1}{t_n-t_{n-2}}\right)\left(\frac{c_{n-2} - c_{n-3}}{t_{n+1}-t_{n-2}}
- \frac{c_{n-3} - c_{n-4}}{t_n-t_{n-3}}\right) = 0  \ .
\]
These give two extra equations,
\begin{align*}
&(t_1-t_{-2})c_{-1} -(t_2 + t_1- t_{-1} - t_{-2}) c_{-2}
  + ( t_2 - t_{-1}) c_{-3} = 0
\intertext{and}
&(t_n-t_{n-3}) c_{n-2} - ( t_{n+1}+t_n - t_{n-2} - t_{n-3})c_{n-3}
+ (t_{n+1}-t_{n-2})c_{n-4} = 0 \ ,
\end{align*}
to combine with the $n$ equations in (\ref{BsplieCubiccoeff}) to
determine the $n+2$ variables $c_i^3$ for $-3 \leq i \leq n-2$.
\end{egg}

Consider a function $f:I \to \RR$, where $I$ is a sub-interval
of $\RR$, and $\delta >0$.  The
{\bfseries modulus of continuity}\index{B-Splines!Modulus of Continuity} of
$f$ on $I$ is defined by
\[
  \omega(f; \delta, I) = \sup \{ |f(x)-f(y)| : x,y\in I \text{ and }
  |x-y|\leq \delta \} \quad .
\]
For a uniformly continuous function $f$ on $I$, we can have
$\omega(f; \delta,I)$ as small as we want by taking $\delta$ small
enough.

\begin{theorem}
Let $\displaystyle q(t) = \sum_{i=-\infty}^\infty f(t_{i+2}) B_i^k(t)$ for
$t\in \RR$ and $k\geq 2$.  If $f:[t_{-k},t_{n+1}] \to \RR$, then 
\[
  \sup_{t_0 \leq t\leq t_n} |f(t) - q(t)|
  \leq k \omega(f; \delta, [t_{-k},t_{n+1}])
\]
for $\displaystyle \delta = \max_{-k\leq i \leq n+1} |t_i - t_{i-1}|$.
\label{SplineApprox}
\end{theorem}

\begin{proof}
Using Propositions~\ref{Bsplienonneg} and \ref{Bsplinesum1}, we may write
\begin{align*}
|f(t) - q(t)| &= \left| f(t) \sum_{i=-\infty}^\infty B_i^k(t)
- \sum_{i=-\infty}^\infty f(t_{i+2})B_i^k(t) \right|
= \left| \sum_{i=-\infty}^\infty \left( f(t) - f(t_{i+2})\right) B_i^k(t)
\right| \\
&\leq \sum_{i=-\infty}^\infty \left| f(t) - f(t_{i+2})\right| B_i^k(t)
= \sum_{i=j-k}^j \left| f(t) - f(t_{i+2})\right| B_i^k(t)
\end{align*}
for $t \in [t_j,t_{j+1}]$ and $0 \leq j \leq n-1$.  Hence,
\[
|f(t) - q(t)|
\leq \max_{j-k \leq i \leq j} \left| f(t) - f(t_{i+2})\right|
\  \underbrace{\sum_{i=j-k}^j B_i^k(t)}_{\leq 1}
\leq \max_{j-k \leq i \leq j}  \left| f(t) - f(t_{i+2})\right|
\]
for $t \in [t_j,t_{j+1}]$.
For $i = j$, we have
\[
\left| f(t) - f(t_{i+2})\right|
= \left| f(t) - f(t_{j+2})\right|
\leq \left| f(t) - f(t_{j+1})\right| + \left| f(t_{j+1}) - f(t_{j+2})\right|
\leq 2 \omega(f, \delta, [t_{-k},t_{n+1}])
\]
for $t \in [t_j,t_{j+1}]$.  For $i = j-1$, we have
\[
\left| f(t) - f(t_{i+2})\right|
= \left| f(t) - f(t_{j+1})\right|
\leq \omega(f, \delta, [t_{-k},t_{n+1}])
\]
for $t \in [t_j,t_{j+1}]$.  For $i = j-2$, we have
\[
\left| f(t) - f(t_{i+2})\right|
= \left| f(t) - f(t_j)\right| \leq \omega(f, \delta, [t_{-k},t_{n+1}])
\]
for $t \in [t_j,t_{j+1}]$.  For $i = j-s$ with $3\leq s \leq k$, we have
\begin{align*}
&\left| f(t) - f(t_{i+2})\right|
= \left| f(t) - f(t_{j-s+2})\right| \\
& \quad \leq \left| f(t) - f(t_j)\right| + \left| f(t_j) - f(t_{j-1})\right|
+ \ldots + \left| f(t_{j-s+3}) - f(t_{j-s+2})\right|
\leq (s-1) \omega(f, \delta, [t_{-k},t_{n-1}])
\end{align*}
for $t \in [t_j,t_{j+1}]$.  In all cases, we have
$\left| f(t) - f(t_{i+2})\right| \leq k \omega(f, \delta, [t_{-k},t_{n-1}])$
for $t \in [t_j,t_{j+1}]$.  The conclusion of the theorem follows
since this is true for all $j$ such that $0 \leq j < n$.

\noindent Note: As the proof shows, we could have used only the
interval $[t_{-k+2},t_{n+1}]$ instead of $[t_{-k},t_{n+1}]$ in the
statement of the theorem.  We have used the second one because the
statement was nicer.
\end{proof}

Since every element of $S_n^k$ is a linear combination of
$B_{j}^k$ for $-k \leq j <n$, we get the following result from the
previous theorem.

\begin{cor}
We have that
\[
  \dist{f}{S_n^k} \leq k \omega(f;\delta, [t_{-k},t_{n+1}])
\]
for all $f:[t_{-k},t_{n+1}] \to \RR$.
\end{cor}

If $f$ is continuous on $[t_{-k},t_{n-1}]$, and so uniformly
continuous on $[t_{-k},t_{n-1}]$, we have that 
$\omega(f;\delta, [t_{-k},t_{n+1}] \to 0$ as $\delta \to 0$.
Therefore, to theoretically improve the accuracy of the interpolation of a
function $f$ on a given interval, we may increase the number of knots
$t_i$ in the interval (increase $n$) while decreasing the
distance between them (decreasing $\delta$).

\section{Other Spline Methods}

Consider $n+1$ points $\VEC{p}_i = (x_i,y_i)$ for $i=0$, $1$,
\ldots, $n$.  We now define a piecewise polynomial, parametric
representation of a curve that shadows the points $\VEC{p}_i$
but does not include the points $\VEC{p}_i$.

First, we add the points $\VEC{p}_{-2} = \VEC{p}_{-1} = \VEC{p}_0$ and
$\VEC{p}_{n+2} = \VEC{p}_{n+1} = \VEC{p}_n$.  There are other
approaches to handle the end points $\VEC{p}_0$ and $\VEC{p}_n$.  With
this approach, $\VEC{p}_0$ and $\VEC{p}_n$ are on the spline.

For $-1 \leq i \leq n$, we define the curve with the parametric
representation
\begin{equation} \label{Bspline}
\VEC{\phi}_i(t) = \sum_{j=-1}^2 b_j(t) \VEC{p}_{i+j} \  ,
\end{equation}
where
$\displaystyle
b_{-1}(t) = -\frac{t^3}{6} + \frac{t^2}{2} - \frac{t}{2} + \frac{1}{6}$,
$\displaystyle b_0(t) = \frac{t^3}{2} - t^2 + \frac{2}{3}$,
$\displaystyle
b_1(t) = -\frac{t^3}{2} + \frac{t^2}{2} + \frac{t}{2} + \frac{1}{6}$
and
$\displaystyle b_2(t) = \frac{t^3}{6}$
for $0 \leq t \leq 1$.  Each component of the parametric representation
is a polynomial of degree three in $t$.

The small curve defined by the parametric representation
$\VEC{\phi}_i(t)$ with $0\leq t \leq 1$ is in the convex hull of the 
points $\VEC{p}_j$ for $j=i-1$, $i$, $i+1$ and $i+2$.  The coordinates
of $\VEC{\phi}_i(t)$ are the weighted sums of the coordinates of
$\VEC{p}_j$ for $j=i-1$, $i$, $i+1$ and $i+2$ because
$\displaystyle \sum_{j=-1}^2 b_j(t) = 1 $ for all $t$.

The parametric representations $\VEC{\phi}_i(t)$ satisfy the following
properties:
\[
\VEC{\phi}_i(1) = \VEC{\phi}_{i+1}(0) \ ,\quad
\VEC{\phi}'_i(1) = \VEC{\phi}'_{i+1}(0) \quad
\text{and} \quad
\VEC{\phi}''_i(1) = \VEC{\phi}''_{i+1}(0)
\]
for $i=-1$, $0$, $1$, \ldots, $n$.

The parametric representation $\VEC{\phi}_i(t)$ given in (\ref{Bspline})
can be rewritten as
\begin{align}
\VEC{\phi}_i(t) &= \frac{1}{6} \left(-\VEC{p}_{i-1} + 3 \VEC{p}_i
- 3 \VEC{p}_{i+1} + \VEC{p}_{i+2} \right) t^3
+ \frac{1}{6} \left(3 \VEC{p}_{i-1} - 6 \VEC{p}_i
+ 3 \VEC{p}_{i+1} \right) t^2
\nonumber \\
&\quad + \frac{1}{6} \left(-3 \VEC{p}_{i-1} + 3 \VEC{p}_{i+1} \right) t
+ \frac{1}{6} \left(\VEC{p}_{i-1} + 4 \VEC{p}_i + \VEC{p}_{i+1} \right)\ .
\label{Bspline_comp}
\end{align}

Note the resemblance between (\ref{Bspline_comp}) and the definition
of Bézier curves in Section~\ref{BEZIER}.

\section{Exercises}

\begin{question}
Construct the clamped cubic spline interpolant to $f$ associated to
the data of the following table.
\[
\begin{array}{c|cccccc}
x & 0 & 1 & 3 & 4 & 5 & 5.5 \\
\hline
f(x) & 2 & 3 & 3 & 2 & 2 & 1 \\
f'(x) & 0 & & & & & -2 
\end{array}
\]
Plot the graph of this cubic spline for $0 \leq x \leq 5.5$.
\label{intBQ1}
\end{question}

\begin{question}
Write a code similar to Code~\ref{codeCCSI} for the natural cubic spline
interpolation and use it to draw the natural cubic spline interpolant
to $f$ associated to the data of the following table.
\[
\begin{array}{c|cccccc}
x & 0 & 1 & 3 & 4 & 5 & 5.5 \\
\hline
f(x) & 2 & 3 & 3 & 2 & 2 & 1
\end{array}
\]
\label{intBQ2}
\end{question}

%%% Local Variables: 
%%% mode: latex
%%% TeX-master: "notes"
%%% End: 
